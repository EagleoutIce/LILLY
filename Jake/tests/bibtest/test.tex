\documentclass[Typ=Mitschrieb,Vorlesung=LAII]{Lilly} 

%%PREHEAD
%\def\LILLYxBOXxNoBoxBemerkung{TRUE}
%\providecommand\LILLYxBOXxMODE{ALTERNATE}
% \def\LILLYxNOxPACKAGEC{}
\usepackage{lipsum}

%MACH EINEN FEHLER

%%% \dataInput{_LILLY_FONTS.tex} %% ESCAPE TODO


\begin{document}
    \chapter{Allgemeine Grundlagen ein sehr langes Kapitelnamen gedöns}\elable{jmp}
    \TitleSUB{Dies ist ein sehr wichtiges Kapitel}
    Docname: \LILLYxDOCUMENTNAME\newline
    Boxmode: \LILLYxBOXxMODE\newline
    %%\MauMau
    Wannabe: \LILLYxWANNABExERROR\newline
    Demo Fontenc: \LILLYxDEMOxFONTENC\newline
    PFAD : \LILLYxCLSPATH!\newline

    OFFICIAL(Lill-root: LILLYxgetCLSPATH): \LILLYxgetCLSPATH!\newline
    DOC-ROOT: $\LILLYxgetDOCPATH$!\newline
    \includegraphics[height=16em]{\getSemester{2}{Graphics/titleimageANA1.pdf}}

    %% This will fail, was used for auto root-detection :D
    %\includegraphics[height=16em]{\getSemester{2}{titleimagePVS.pdf}}

    Hallo \cite{Beispiel}


    %Wrapfig tester:
    %\begin{wrapfigure}{c}{4cm}
    %    Hallo Mama, na wie gehts?
    %    \caption{Das ist toll :D}
    %\end{wrapfigure}

    \section{IntroMintro a}
        \begin{tabularx}{\linewidth}{lX}
            h & b \\
            c & d
        \end{tabularx}

        \begin{definition*}
            Important NoName DEF
        \end{definition*}

        \begin{satz}[Übrigens]
            Übrigens - ich mag Züge
        \end{satz}

        \begin{definition*}[Important]
            Important AName DEF
        \end{definition*}

        \begin{definition}[Hallo Mama]
            Dies ist ein sehr tolles und interessantes Definitiönchen
        \end{definition}
        \begin{bemerkung}[Hallo Lama]
            Dies ist ein sehr tolles und interessantes Bemerkungli
        \end{bemerkung}

        \inputUB{Tolles Übungsblatt}{13}{UBdata.tex}

        FF\LILLYxMODE FF\LILLYxDEBUG AA\lipsum[2]
        \DEF{Matching}{
            Ein Matching (auch genannt: \say{Matsching} tihihi) in einem Graphen ist eine Teilmenge der Form: \(M \subseteq E\) der Kanten, so dass keine zwei Kanten einen Endknoten gemeinsam haben. ein Matching \(M\) heißt \textbf{perfekt}, falls durch die Kanten in \(M\) alle Knoten des Graphen erfasst werden. \\
    Das bedeutet: \[M \text{ ist perfekt} \Leftrightarrow |M| = \frac{|V|}{2}\]
    Allgemein gilt: \[|M| \leq \left\lfloor \frac{|V|}{2} \right\rfloor\]
        }
        \lipsum[12-16]
        \BEM{Echt wichtig}{
            Dies ist ein \jmark[Link!!!!!]{jmp}
        }
        \jmark[Link!!!!!]{jmp}
        \lipsum[13]
        \SAT{Ich heiße Marvin}{
            Halt die Klappe Markus!!!\\
            {\raggedleft Aber Marvin???}
            \lipsum[2-5]
            \jmark[Link!!!!!]{jmp}
            \BEW{Lemmataparadoxon}{
                Ich nutze meine Lemmas ja eher zum einschlafen! \\
                \lipsum[9-10]\jmark[Link!!!!!]{jmp}
                \LEM{Wichtig}{
                    Hier ist nichts \\ gar \\\jmark[Link!!!!!]{jmp}
                }
            }
            \jmark[Link!!!!!]{jmp}
        }
        \clearpage
        \section{BINITRINTO}
        \renewcommand{\tabularxcolumn}[1]{m{#1}}
        \BEI{Binärbäume}{
            Im folgenden seiein hier wichtige Beispiele präsentiert:
            \begin{tabularx}{\textwidth}{|>{\centering\vspace{0.14cm}\arraybackslash}X|>{\small\vspace{0.05cm}}X|>{\centering\vspace{0.14cm}\arraybackslash}X|>{\small\vspace{0.05cm}\arraybackslash}X|}
                \hline
                \begin{tikzpicture}
                    \oragraphdot{(0,0)}{w}{a};
                    %\node at(0,-1) {};
                \end{tikzpicture} & \ding{51}, da ein einzelner Knoten bereits ein gültiger Binärbaum ist.& \begin{tikzternal}
                    \oragraphdot{(0,0)}{a}{a};
                    \oragraphdot{(-1,-1)}{b}{b};
                    \node at(1,-1) {};
                    \node at(0,0.75) {};
                    \draw (a) -- (b);
                \end{tikzternal}\vspace{0.14cm} & \ding{55}, da jeder Knoten entweder \(2\) oder \(0\) Kindknoten haben muss. \\\hline
                \begin{tikzternal}
                    \oragraphdot{(0,0)}{a}{a};
                    \oragraphdot{(-1,-1)}{b}{b};
                    \oragraphdot{(1,-1)}{c}{c};
                    \node at(0,0.75) {};
                    \draw (a) -- (b) (a) -- (c) (b) -- (c) ;
                \end{tikzternal}\vspace{0.14cm} & \ding{55}, da es keine \say{Ringe} geben darf. & \begin{tikzternal}[scale=0.5, every node/.style={transform shape}]
                    \oragraphdot{(0,0)}{a}{a};
                    \oragraphdot{(-1,-1)}{b}{b};
                    \oragraphdot{(1,-1)}{c}{c};
                    \oragraphdot{(0,-2)}{d}{d};
                    \oragraphdot{(2,-2)}{e}{e};
                    \draw (a) -- (b) (a) -- (c) (c) -- (d) (c) -- (e);
                \end{tikzternal} & \ding{51}, da jeder Knoten \(0\) oder \(2\) Kindknoten besitzt\\\hline
            \end{tabularx}
        }
        \renewcommand{\tabularxcolumn}[1]{p{#1}}
        \begin{aufgabe}[Hallo Mama][1]
            Diese Aufgabe ist wichtig !
        \end{aufgabe}
        \begin{aufgabe}[Hallo Mama][42]
            Diese ist noch viel wichtiger sie gibt 42 mal so viel Punkte!
        \end{aufgabe}

\chapter{Spezifische supertests}

\section{Graphentests - wrapper }
\subsection{nonfloating - default}
    \begin{graph}
        \plotline{\x*\x}
    \end{graph}\lipsum[6-8]

\subsection{as custom wrap}
\begin{wrapfigure}{l}{0pt}
    \begin{graph}
        \plotline{\x*\x}
    \end{graph}
\end{wrapfigure}\lipsum[6-8]

\subsection{as wrapper}

        \begin{wgraph}{l}
            \plotline[purple]{\x*\x}
        \end{wgraph}\lipsum[6-8]

\subsection{as wrapper with enforced wrongwidth}
        \begin{wgraph}{l}[][][1cm]
            \plotline[Ao]{\x*\x}
        \end{wgraph}\lipsum[6-8]

\subsection{Beispiel mit externem caption}
        \begin{wgraph}{l}[][][0pt][\caption{Wichtiger Graph}]
            \plotline[Azure]{\x*\x}
        \end{wgraph}\lipsum[6-8]

\clearpage
\section{Graphentests - floats}
    \subsection{as float}
    \begin{wgraph}{L}[][]
        \plotline[Veronica]{\x*\x}
    \end{wgraph}\lipsum[6-8]

\subsection{as float with enforced wrongwidth}
    \begin{wgraph}{L}[][][1cm]
        \plotline[Amber]{\x*\x}
    \end{wgraph}\lipsum[6-8]

\subsection{Beispiel mit externem caption}
    \begin{wgraph}{L}[][][0pt][\caption{Wichtiger Graph}]
        \plotline[Aureolin]{\x*\x}
    \end{wgraph}\lipsum[6-8]
\end{document}
