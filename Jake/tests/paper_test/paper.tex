%% Von Jake erstelltes Lilly-Texfile :D
%% TODO: implement structures
\providecommand{\TITLE}{Clusteranalyse \& Klassifikation}
\providecommand{\BRIEF}{Informatik Student,\\Universität Ulm -- \heute}


\begin{filecontents}{books.bib}
@inbook{ Kriesel2007NeuralNetworks, 
    author      = { David Kriesel }, 
    title       = { Ein kleiner \"{U}berblick \"{u}ber Neuronale Netze },
    year        = { 2007 },
    chapter     = A,
    pages       = { 179-189 },
    publisher   = {Erh\"{a}ltlich auf \url{http://www.dkriesel.com}},
  }
\end{filecontents}


\documentclass[paper]{Lilly}

\usepackage{lipsum}

%% TODO CHANGE HEADER AND FOOTER

\begin{document}

\printHeader

\section{Motivation}
Diese Aufgabe war sehr interessant, deswegen mache ich sie. Sie ist toll und das macht mir Spaß. Because I'm happi düldü düdüdü dümm\ldots\newline 
Die Informationen basieren zum Großteil auf \cite{Kriesel2007NeuralNetworks}
\subsection{Das Ziel}
\lipsum[1]
\section{Grundlagen}
Die Basics: Definitionen, was sind zahlen, wie gehen sie, was wollen sie?
\subsection{Die Mathematik dahinter}
\lipsum[2-3]
\begin{definition}[Metrik]
    Das ist toll
\end{definition}
\lipsum[4]
\subsection{Bezeichner}
Rofl \lipsum[5]
\clearpage
\section{Clusteranalyse}
Diese Algorithmen, blah blah blah
\subsection{k-Means Clustering}
\lipsum[1-2]

\vspace{2em}
\section{Anhang}
\printbib
\end{document}
