%\def\LILLYxBOXxNoBoxBemerkung{TRUE} 
%\providecommand\LILLYxBOXxMODE{alternate}
\documentclass[Typ=Mitschrieb,Vorlesung=LAII]{Lilly} 
\usepackage{lipsum}  

\begin{document}  
    \chapter{Allgemeine Grundlagen ein sehr langes Kapitelnamen gedöns}\elable{jmp} 
    \TitleSUB{Dies ist ein sehr wichtiges Kapitel}
\section{IntroMintro}
        \begin{tabularx}{\linewidth}{lX}
            h & b \\
            c & d
        \end{tabularx}

        \begin{definition*}
            Important NoName DEF
        \end{definition*}

        \begin{satz}[Übrigens]
            Übrigens - ich mag Züge
        \end{satz}

        \begin{definition*}[Important]
            Important AName DEF
        \end{definition*}

        \begin{definition}[Hallo Mama]
            Dies ist ein sehr tolles und interessantes Definitiönchen
        \end{definition}
        \begin{bemerkung}[Hallo Lama]
            Dies ist ein sehr tolles und interessantes Bemerkungli
        \end{bemerkung}

        \inputUB{Tolles Übungsblatt}{13}{UBdata.tex}

        FF\LILLYxMODE FF\LILLYxDEBUG AA\lipsum[2]
        \DEF{Matching}{
            Ein Matching (auch genannt: \say{Matsching} tihihi) in einem Graphen ist eine Teilmenge der Form: \(M \subseteq E\) der Kanten, so dass keine zwei Kanten einen Endknoten gemeinsam haben. ein Matching \(M\) heißt \textbf{perfekt}, falls durch die Kanten in \(M\) alle Knoten des Graphen erfasst werden. \\
    Das bedeutet: \[M \text{ ist perfekt} \Leftrightarrow |M| = \frac{|V|}{2}\]
    Allgemein gilt: \[|M| \leq \left\lfloor \frac{|V|}{2} \right\rfloor\]
        }
        \lipsum[12-16]
        \BEM{Echt wichtig}{
            Dies ist ein \jmark[Link!!!!!]{jmp}
        }
        \jmark[Link!!!!!]{jmp}
        \lipsum[13]
        \SAT{Ich heiße Marvin}{
            Halt die Klappe Markus!!!\\
            {\raggedleft Aber Marvin???}
            \lipsum[2-5]
            \jmark[Link!!!!!]{jmp}
            \BEW{Lemmataparadoxon}{
                Ich nutze meine Lemmas ja eher zum einschlafen! \\
                \lipsum[9-10]\jmark[Link!!!!!]{jmp}
                \LEM{Wichtig}{
                    Hier ist nichts \\ gar \\\jmark[Link!!!!!]{jmp}
                }
            }
            \jmark[Link!!!!!]{jmp}
        }
        \newpage
        \section{BINITRINTO}
        \renewcommand{\tabularxcolumn}[1]{m{#1}}
        \BEI{Binärbäume}{
            Im folgenden seiein hier wichtige Beispiele präsentiert:
            \begin{tabularx}{\textwidth}{|>{\centering\vspace{0.14cm}\arraybackslash}X|>{\small\vspace{0.05cm}}X|>{\centering\vspace{0.14cm}\arraybackslash}X|>{\small\vspace{0.05cm}\arraybackslash}X|}
                \hline
                \begin{tikzpicture}
                    \oragraphdot{(0,0)}{w}{a}; 
                    %\node at(0,-1) {};
                \end{tikzpicture} & \ding{51}, da ein einzelner Knoten bereits ein gültiger Binärbaum ist.& \begin{tikzpicture}
                    \oragraphdot{(0,0)}{a}{a}; 
                    \oragraphdot{(-1,-1)}{b}{b}; 
                    \node at(1,-1) {};
                    \node at(0,0.75) {};
                    \draw (a) -- (b); 
                \end{tikzpicture}\vspace{0.14cm} & \ding{55}, da jeder Knoten entweder \(2\) oder \(0\) Kindknoten haben muss. \\\hline
                \begin{tikzpicture}
                    \oragraphdot{(0,0)}{a}{a}; 
                    \oragraphdot{(-1,-1)}{b}{b}; 
                    \oragraphdot{(1,-1)}{c}{c};
                    \node at(0,0.75) {};
                    \draw (a) -- (b) (a) -- (c) (b) -- (c) ; 
                \end{tikzpicture}\vspace{0.14cm} & \ding{55}, da es keine \say{Ringe} geben darf. & \begin{tikzpicture}[scale=0.5, every node/.style={transform shape}]
                    \oragraphdot{(0,0)}{a}{a}; 
                    \oragraphdot{(-1,-1)}{b}{b}; 
                    \oragraphdot{(1,-1)}{c}{c};
                    \oragraphdot{(0,-2)}{d}{d}; 
                    \oragraphdot{(2,-2)}{e}{e};
                    \draw (a) -- (b) (a) -- (c) (c) -- (d) (c) -- (e); 
                \end{tikzpicture} & \ding{51}, da jeder Knoten \(0\) oder \(2\) Kindknoten besitzt\\\hline    
            \end{tabularx}
        }
        \renewcommand{\tabularxcolumn}[1]{p{#1}}
        \begin{aufgabe}[Hallo Mama][1]
            Diese Aufgabe ist wichtig !    
        \end{aufgabe}
        \begin{aufgabe}[Hallo Mama][42]
            Diese ist noch viel wichtiger sie gibt 42 mal so viel Punkte!
        \end{aufgabe}

\end{document}  