\errorcontextlines 100000

\documentclass{article}

\usepackage{LILLYxTABLES}
\usepackage{LILLYxORNAMENTS}
\usepackage{LILLYxLISTINGS}
\usepackage{colortbl}
\begin{document}

\begin{mltable}{ll}
    Hallo & Welt \\
    \rowcolor{purple} Na wie & geht es \\
    dir & denn
\end{mltable}


\begin{tabular}{lll}

    Hi & Na du? \\
    \rowcolor{purple} Na wie & geht es \\
    dir & denn

\end{tabular}
\bigskip\newline
% \begin{mtabular}{ll}
%     Hi & Na du? \\
%     Na wie & geht es \\
%     Hi & Na du? \\
%     Na wie & geht es \\
%     Hi & Na du? \\
%     Na wie & geht es \\
%     Hi & Na du? \\
%     Na wie & geht es \\
%     dir & denn
% \end{mtabular}

\begin{longtable}{tl}
    Bezeichner & Evaluiert zu\\ % On every Page
    TEXFILE & Expandiert zum vollen Bezeichner TeX-Datei \\
    BASENAME & Expandiert zum Namen der TeX-Datei ohne Endung \\
    FINALNAME & Expandiert zum Namen nach der Generierung (nur sofern im Kontext klar vorhanden) \\
    LOGFILE & Expandiert zum Pfad der Logdatei \\
    PDFFILE & Expandiert zum Namen der PDF-Datei \\
    LATEXARGS & Expandiert zu den Latex-Argumenten (\T{-shell-escape}, \ldots) \\
    OUTPUTDIR & Expandiert zum Ausgabeordner \\
    INPUTDIR & Expandiert zum Quellordner \\
    BOXMODES & Expandiert zu den Boxmodi \\
    CLEANTARGETS & Expandiert zu den zu löschenden Endungen \\
    SIGNATURECOL & Expandiert zur Signaturfarbe \\
    AUTHOR & Expandiert zum Author \\
    AUTHORMAIL & Expandiert zur Email-Adresse des Autors \\
    NAMEPREFIX & Expandiert zum Namenspräfix \\
    SEMESTER & Expandiert zur Semesterzahl \\
    VORLESUNG & Expandiert zur Vorlesung \\
    LILLY\_CONFIGS\_PATH & Expandiert zum Pfad der Konfigurationen \\
    LILLY\_DATA\_PATH & Expandiert zum Pfad der Daten \\
    N & Expandiert zur Übungsblattnummer \\
    JOBCOUNT & Expandiert zur Maximalen Jobanzahl \\
    \_LILLYARGS & Expandiert zu den Argumenten für Lilly \\
    \_C & Expandiert zu einem wundervollen Komma \\%\Smiley \\
    HOME & Expandiert zum Homeverzeichnis \\
    TRUE & Expandiert zu \say{\emph{true}} \\
    FALSE & Expandiert zu \say{\emph{false}} \\
    S\_TRUE & Expandiert zur Jake-Definition von \emph{true} (\say{\emph{true}}) \\
    S\_FALSE & Expandiert zur Jake-Definition von \emph{false} (\say{\emph{false}})\\
    PDFFILE & Expandiert zum Namen der PDF-Datei \\
    LATEXARGS & Expandiert zu den Latex-Argumenten (\T{-shell-escape}, \ldots) \\
    OUTPUTDIR & Expandiert zum Ausgabeordner \\
    INPUTDIR & Expandiert zum Quellordner \\
    BOXMODES & Expandiert zu den Boxmodi \\
    CLEANTARGETS & Expandiert zu den zu löschenden Endungen \\
    SIGNATURECOL & Expandiert zur Signaturfarbe \\
    AUTHOR & Expandiert zum Author \\
    AUTHORMAIL & Expandiert zur Email-Adresse des Autors \\
    NAMEPREFIX & Expandiert zum Namenspräfix \\
    SEMESTER & Expandiert zur Semesterzahl \\
    VORLESUNG & Expandiert zur Vorlesung \\
    LILLY\_CONFIGS\_PATH & Expandiert zum Pfad der Konfigurationen \\
    LILLY\_DATA\_PATH & Expandiert zum Pfad der Daten \\
    N & Expandiert zur Übungsblattnummer \\
    JOBCOUNT & Expandiert zur Maximalen Jobanzahl \\
    \_LILLYARGS & Expandiert zu den Argumenten für Lilly \\
    \_C & Expandiert zu einem wundervollen Komma \\%\Smiley \\
    HOME & Expandiert zum Homeverzeichnis \\
    TRUE & Expandiert zu \say{\emph{true}} \\
    FALSE & Expandiert zu \say{\emph{false}} \\
    S\_TRUE & Expandiert zur Jake-Definition von \emph{true} (\say{\emph{true}}) \\
    S\_FALSE & Expandiert zur Jake-Definition von \emph{false} (\say{\emph{false}})
\end{longtable}



\constructList[,]{Dieter}

\pushList{Dieter}{Hallo}
\theDieter
\pushList{Dieter}{Hallo2}
\theDieter
\pushList{Dieter}{Hallo3}
\pushList{Dieter}{Hallo4}
\pushList{Dieter}{Hallo5}

\theDieter


\end{document}