\documentclass{article}

\usepackage{LILLYxCONTROLLERxBOX}
\usepackage{LILLYxLISTINGS}
\usepackage{LILLYxFORMATxCONTROL}

\newcommand*{\blankcmd}[1]{\textcolor{Orchid}{\LILLYxlstTypeWriter\textbackslash #1}}

\begin{document}
\begin{bemerkung}[Was gibts?]
    Hier die neuen Freuden des Pakets: \LILLYxLIBRARY{LILLYxFORMATxCONTROL}, es ermöglicht
    Iteratoren zu kreieren, die den ersten Buchstaben eines jeden Wortes modifiziert.
\end{bemerkung}

\paragraph{Acronym}~\\[-\baselineskip]
\begin{latex}
\Acronym{hallo, ich mag Züge, toll, Oder?}
\end{latex}
Ergibt:
\begin{center}
    \Acronym{hallo, ich mag Züge, toll, Oder?}
\end{center}
Definiert ist dieser Iterator sehr einfach durch:
\begin{latex}
\def!**!\Acronym!**!#1{%
    \gdef!**!\LILLY@FORMATTER@CURRENT{\TextBfFormat}%
    \def!**!\@Acronym{\lilly@format@iter!**!#1 \@nil}{\@Acronym}%
}
\end{latex}
Wobei \blankcmd{TextBfFormat} einfach nur eine Definition von \blankcmd{textbf} ist. Die Hauptarbeit übernimmt also:
\blankcmd{lilly@format@iter}.

\begin{latex}
\PoliteWords{hallo, ich mag Züge, toll, Oder?}
\end{latex}
Ergibt:
\begin{center}
    \PoliteWords{hallo, ich mag Züge, toll, Oder?}
\end{center}

\begin{latex}
\ColerfulWords{hallo, ich mag Züge, toll, Oder?}
\end{latex}
Ergibt:
\begin{center}
    \ColerfulWords{hallo, ich mag Züge, toll, Oder?}
\end{center}

\clearpage

Dieses Dokument wurde erstellt durch folgenden Code:
\ilatex{./formatdemo.tex}

\end{document}