\documentclass{article}
\usepackage[textwidth=19cm]{geometry}

\usepackage{LILLYxRUNTIMES}
\usepackage{LILLYxLISTINGS}
\usepackage{LILLYxLIST}

\usepackage{tikz}

\begin{document}
\noindent\previewBash{echo Hallo}
\noindent\rbash{echo na du? | ls}
\cbash{echo na du? | ls}

\clatex{:bs:documentclass\{latex\}}

\noindent\rhaskell{5+23}

Konstruiere Liste Peter mit Separator Komma: \constructList[,]{Peter}
\begin{latex}
\constructList[,]{Peter}
\end{latex}
Füge Elemente hinzu: \pushList{Peter}{2}\pushList{Peter}{55}\pushList{Peter}{3}\pushList{Peter}{26}
\begin{latex}
\pushList{Peter}{2}\pushList{Peter}{55}
\pushList{Peter}{3}\pushList{Peter}{26}
\end{latex}
Peter enthält nun: \thePeter
\begin{latex}
\thePeter
\end{latex}
Über eine Liste kann man iterieren:
\begin{latex}
\getPeter
\foreach \x in \lillyxlist {
    X:\x
}
\end{latex}
Ergebnis: \getPeter % erfrage Liste Peter in \lillyxlist
\foreach \x in \lillyxlist {
    X:\x
}\newline
Konstruiere eine neue Liste '{Waffel}':
\begin{latex}
\constructList[, ]{Waffel}
\end{latex}
\constructList[, ]{Waffel}
Füge Elemente hinzu:
\begin{latex}
    \pushList{Waffel}{A}\pushList{Waffel}{\thePeter}
    \pushList{Waffel}{X}\pushList{Waffel}{21}
\end{latex}
\pushList{Waffel}{A}\pushList{Waffel}{\thePeter}
\pushList{Waffel}{X}\pushList{Waffel}{21}
Ausgabe von \clatex{:bs:theWaffel}: \theWaffel
Weiter ergibt:
\begin{latex}
\getWaffel
\foreach \x in \lillyxlist {
    => \x
}
\end{latex}
Ergebnis: \getWaffel % erfrage Liste Peter in \lillyxlist
\foreach \x in \lillyxlist {
    => \x
}\newline
Listen werden also vereinigt und nicht durch einzelne Elemente separiert. Wir können testen ob ein Element in einer Liste enthalten ist, mit:
\begin{latex}
\containsList{Waffel}{A}
\containsList{Waffel}{410021}
\end{latex}
A: \containsList{Waffel}{A}\\
410021: \containsList{Waffel}{410021} \\

\makeatletter

\getPeter
\expandafter\lilly@iter@commalist\lillyxlist!\\

\lillyxlist\\
%\expandafter\forlist\lillyxlist

\end{document}



















\iffalse
\clearpage
%\isLanguageLoaded{sql/lSql} \isLanguageLoaded{quatschopotl}
Die erste Seite wurde wie folgt erstellt (jaay recursive Code:) )
\begin{latex}
\documentclass{article}
\usepackage[textwidth=19cm]{geometry}

\usepackage{LILLYxLISTINGS}
\usepackage{LILLYxLIST}

\usepackage{tikz}

\begin{document}
Konstruiere Liste Peter mit Separator Komma: \constructList[,]{Peter}
:bs:begin{latex}
\constructList[,]{Peter}
:bs:end{latex}
Füge Elemente hinzu: \pushList{Peter}{2}\pushList{Peter}{55}\pushList{Peter}{3}\pushList{Peter}{26}
:bs:begin{latex}
\pushList{Peter}{2}\pushList{Peter}{55}
\pushList{Peter}{3}\pushList{Peter}{26}
:bs:end{latex}
Peter enthält nun: \thePeter
:bs:begin{latex}
\thePeter
:bs:end{latex}
Über eine Liste kann man iterieren:
:bs:begin{latex}
\getPeter
\foreach \x in \lillyxlist {
    X:\x
}
:bs:end{latex}
Ergebnis: \getPeter % erfrage Liste Peter in \lillyxlist
\foreach \x in \lillyxlist {
    X:\x
}\newline
Konstruiere eine neue Liste '{Waffel}':
:bs:begin{latex}
\constructList[, ]{Waffel}
:bs:end{latex}
\constructList[, ]{Waffel}
Füge Elemente hinzu:
:bs:begin{latex}
    \pushList{Waffel}{A}\pushList{Waffel}{\thePeter}
    \pushList{Waffel}{X}\pushList{Waffel}{21}
:bs:end{latex}
\pushList{Waffel}{A}\pushList{Waffel}{\thePeter}
\pushList{Waffel}{X}\pushList{Waffel}{21}
Ausgabe von \clatex{:bs:theWaffel}: \theWaffel
Weiter ergibt:
:bs:begin{latex}
\getWaffel
\foreach \x in \lillyxlist {
    => \x
}
:bs:end{latex}
Ergebnis: \getWaffel % erfrage Liste Peter in \lillyxlist
\foreach \x in \lillyxlist {
    => \x
}\newline
Listen werden also als ein Element eingefügt, ungeachtet des Separators. Um die Listen zu vereinigen, wird ein identischer Separator benötigt und eine weitere Expandierung. Wir können testen ob ein Element in einer Liste enthalten ist, mit:
:bs:begin{latex}
\containsList{Waffel}{A}
\containsList{Waffel}{410021}
:bs:end{latex}
A: \containsList{Waffel}{A}\\
410021: \containsList{Waffel}{410021}
\end{document}
\end{latex}
\fi