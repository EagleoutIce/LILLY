\errorcontextlines 999999
\documentclass[a4paper,landscape]{scrartcl}
\usepackage[right=1cm, left=.2cm, top=1.5cm, bottom=1.5cm,noheadfoot]{geometry}

\usepackage{LILLYxTIMETABLESxUNIVERSITY}
\usepackage{LILLYxLISTINGS}

\begin{document}

% % IntToWeekday Test:\newline
% % \foreach \x in {0,...,6}{
% %     I\IntToWeekday{\x}Waffel
% % }
% % \bigskip\newline
% % if day end time is not reachable in steps of two, it will be enlarged by one
% \NewTimeTable[week end day=4, day start time=8]{MeinStundenplan}

% % \theMeinStundenplan\bigskip\newline

% \RawTimeTableEvent{MeinStundenplan}
% \RawTimeTableEvent[title=Waffel,Tuesday,bgcolor=DebianRed!15]{MeinStundenplan}
% \RawTimeTableEvent[Freitag,8 uhr]{MeinStundenplan}
% \RawTimeTableEvent[Mittwoch,9 uhr]{MeinStundenplan}
% \RawTimeTableEvent[Freitag,10 uhr,title={Wau}]{MeinStundenplan}
% \RawTimeTableEvent[Mittwoch,11 uhr,title={Wau}]{MeinStundenplan}

% \RawTimeTableEvent[title=Einführung in die Informatik,Do,bgcolor=DebianRed!15]{MeinStundenplan}


% \hbox{}\vfill\begin{center}
% \resizebox{0.75\textwidth}{!}{%
%     \DrawTimeTable{MeinStundenplan}%
% }
% \end{center}
% \vfill\hbox{}
% \clearpage

% \NewTimeTable[week end day=4, day start time=8]{testPlan}

% \NewTimeTableEvent{eidi}%
%     {Einführung in die Informatik}% Titel
%     {ChromeYellow!25}% Farbe
%     [Vorlesung] % Extra 1
%     []  % Extra 2, kann überschrieben werden :D
%     [Prof. Dr. Thom Frühwirth] % Extra 3

% \neweidi[extra 2=H1]{testPlan}{Dienstag}{14 uhr}

% \hbox{}\vfill\begin{center}
%     \resizebox{0.75\textwidth}{!}{%
%         \DrawTimeTable{testPlan}%
%     }
%     \end{center}
%     \vfill\hbox{}

% \clearpage








% University check :D

\NewLectureSeries[%
    vl length = 2 hours, % default
    vl where  = O28 - H22,
    %
    exercise instructor = Marcus Müller,
    ub length = 2 hours, % default
    ub where  = O28 - H22,
    %
    tutor = Lukas Schmid,
    tu length = 2 hours,
    tu where = N24 - 104/101
]{anaI}{Analysis für Inf. und Ing.}{Dr. Liebezeit}

\NewLectureSeries[%
    short title=PvS,
    vl length = 2 hours, % default
    vl where  = O28 - H22,
    %
    exercise instructor = Stefan Götz,
    ub enabled = false,
    %
    % No tutorium, we will ignore :D
    tutor = Jakob Meyer-Hilberg,
    tu length = 1 hour,
    tu where = O28 - 1001
]{pvs}{Programmierung von Systemen}{Prof. Dr. Tichy}

\NewLectureSeries[%
    short title=GdBS,
    vl length = 2 hours, % default
    vl where  = N25 - H4/5,
    %
    % We will set the übung to be the labor :d
    exercise instructor = Dipl.-Ing Siedenburg,% David Mödinger,
    ub length = 2 hours,
    ub where = O27 - 2205,
    ub title = Labor,
    ub signature = lb, % will be lbgdbs not ubgdbs!
    ub bgcolor = Veronica!25, % different bg color
    %
    % No tutorium, we will ignore :D
    tutor = Der Vergessene,
    tu length = 1 hour,
    tu where = N25 - 2102
]{gdbs}{Grundlagen der Betriebssysteme}{Prof. Dr.-Ing. Hauck}

\NewLectureSeries[%
    vl length = 2 hours, % default
    vl where  = N25 - H4/5,
    %
    exercise instructor = Sabrina Böhm,
    ub enabled = false,
    %
    % No tutorium, we will ignore :D
    tutor = Pascal Weber,
    tu length = 2 hours,
    tu where = O27 - 2202
]{pdp}{Paradigmen der Programmierung}{Dr. Raschke, Prof. Dr. Frühwirth}

\NewLectureSeries[%
    docent = Dr. Friedhelm Schwenker,
    vl length = 2 hours, % default
    vl where  = O27 - 429,
    vl signature = ps, % command will be \psknn NOT \vlknn
    vl bgcolor = DebianRed!25, % different color :D
    vl title = Proseminar,
    %
    ub enabled = false,
    tu enabled = false
]{knn}{Künstliche Neuronale Netze}{}

\NewTimeTable[title=Stundenplan SoSe 19/20]{stundenplan}

% ANA
\ubanaI{stundenplan}{Dienstag}{14 uhr}% day, starttime, length constructed above :D
\vlanaI{stundenplan}{Donnerstag}{12 uhr}
\vlanaI{stundenplan}{Freitag}{8 uhr}
\tuanaI{stundenplan}{Freitag}{10 uhr}

% PVS
\vlpvs{stundenplan}{Montag}{14 uhr}
\vlpvs{stundenplan}{Mittwoch}{12 uhr} % maybe add option for cct and ct?
\tupvs{stundenplan}{Freitag}{13 uhr}

% GDBS
\vlgdbs{stundenplan}{Montag}{16 uhr}
\vlgdbs[extra 2 = O28 - H22]{stundenplan}{Donnerstag}{16 uhr}
\lbgdbs{stundenplan}{Montag}{12 uhr}
%\tugdbs{stundenplan}{Freitag}{11 uhr}

% PDP
\vlpdp{stundenplan}{Dienstag}{16 uhr}
\tupdp{stundenplan}{Donnerstag}{10 uhr}

% Knn
\psknn{stundenplan}{Mittwoch}{8 uhr}
\PresentTimeTable{stundenplan}

\clearpage
\section{How To}
How to create your own Timetable:
\begin{latex}[style=LSTHI,morekeywords={[5]{\\ubanaI,\\vlanaI,\\tuanaI,\\vlpvs,\\tupvs,\\vlgdbs,\\lbgdbs,\\vlpdp,\\tupdp,\\psknn}}]
% \documentclass[a4paper,landscape]{scrartcl}
% \usepackage[right=1cm, left=.2cm, top=1.5cm, bottom=1.5cm,noheadfoot]{geometry}
\usepackage{LILLYxTIMETABLESxUNIVERSITY} % University extension

\NewTimeTable[title=Stundenplan SoSe 19]{stundenplan}
% ANA
\ubanaI{stundenplan}{Dienstag}{14 uhr}% Tuesday or Di
\vlanaI{stundenplan}{Donnerstag}{12 uhr}
\vlanaI{stundenplan}{Freitag}{8 uhr}% Friday or Fr
\tuanaI{stundenplan}{Freitag}{10 uhr}
% PVS
\vlpvs{stundenplan}{Montag}{14 uhr}
\vlpvs{stundenplan}{Mittwoch}{12 uhr}
\tupvs{stundenplan}{Freitag}{13 uhr}
% GDBS
\vlgdbs{stundenplan}{Montag}{16 uhr}
\vlgdbs[extra 2 = |plain|O28 - H22|plain|]{stundenplan}{Donnerstag}{16 uhr}
\lbgdbs{stundenplan}{Montag}{12 uhr}
% PDP
\vlpdp{stundenplan}{Dienstag}{16 uhr}
\tupdp{stundenplan}{Donnerstag}{10 uhr}
% Knn
\psknn{stundenplan}{Mittwoch}{8 uhr}

\PresentTimeTable{stundenplan}
\end{latex}
The Lectures are defined as:
\begin{latex}[style=LSTHI]
\NewLectureSeries[%
    vl length = 2 hours, % default
    vl where  = |plain|O28 - H22|plain|,
    % ...
]{anaI}{Analysis für Inf. und Ing.}{Dr. Liebezeit}
\end{latex}

\end{document}