\documentclass{article}

\usepackage{LILLYxLISTINGS}
\usepackage{LILLYxCONTROLLERxBOX}

\begin{document}

\begin{bemerkung}[Kompilierung]
    Dieses Dokument wurde kompiliert durch: \begin{center}
        \bbash{pdflatex test-lst.tex}
    \end{center}
\end{bemerkung}

\section{Einfache Benutzung zum Code-Highlighting, etlicher Sprachen}
Java:
\begin{java}
public static void main(String[] args) {
    int i = 42*3;
    System.out.println("Hallo Welt");
}
\end{java}
C++:
\begin{cpp}
#include <iostream>
public static int main(int argc, char** argv) {
    std::cout << "Hallo Welt" << std::endl;
    std::cout << "Die Äntwört ist: " << 21*2+14-(2*7) << std::endl;
}
\end{cpp}
Die Existenz einer Sprache kann dank \LILLYxNOTExLibrary{LILLYxLIST} leicht geprüft werden:
\begin{latex}
\isLanguageLoaded{java/lJava}   % :yields: 'TRUE' (\true)
\isLanguageLoaded{super/lSuper} % :yields: 'FALSE' (\false)
\end{latex}
\begin{bemerkung}[Notation]
    Soll dies auf nur '\verb|java|' geändert werden?
\end{bemerkung}
Alle geladenen Sprachen finden sich in der Liste \verb|RegisteredLanguages|.
\section{Besonderheiten}
Durch die Syntax '\verb|:<Keyword>:|' können Sonderzeichen und andere tolle Features in listings eingeführt werden, so wurden die oberen Pfeile durch '\verb|:yields:|' erzeugt. Weiter interessant sind vielleicht noch '\verb|:lan:|' (left angle: '\bjava{:lan:}') und '\verb|:ran:|' (right angle: '\bjava{:ran:}')

\section{Befehle}
Für jede Sprache wie '\verb|Java|' existiert ein entsprechend kleingeschriebenes Environment\footnote{C++ ist 'cpp'} was das Highlighting übernimmt (hier: '\verb|java|'), soll die Sprache ohne Firlefanz gesetzt werden, so existiert \verb|lstplain|: \begin{latex}
\begin{lstplain}[language=lJava]
public static void main(String[] args) {
    int i = 42*3;
    System.out.println("Hallo Welt");
}
\end{lstplain}
\end{latex}
Ergibt:
\begin{lstplain}[language=lJava]
public static void main(String[] args) {
    int i = 42*3;
    System.out.println("Hallo Welt");
}
\end{lstplain}
Für das inline-Code-Highlighting, existieren für jede Sprache \verb|c<Sprache>| (code) und \verb|b<Sprache>| (blank):
\begin{latex}
\cpython{print('hallo')}, \bpython{print('hallo')}
\end{latex}
\cpython{print("hallo")}, \bpython{print('hallo')}\newline
Ersterer kann aufgrund der Implementation des Hintegrund nicht umgebrochen werden, Listings versucht sich allerdings in einem normalen Maße der bad-paragraph (überlauf Problematik) anzunehmen. Letzterer kann ganz normal umgebrochen werden. Hier ein Beispiel:\newline
\cpython{print('hallo welt, na wie geht es dir? dies ist sehr langer Text, ein bisschen unangenehm oder? aber irgendwie muss ich die Zeile sprengen tut mir wirklich leid.')}\newline
Dahingegegen:\newline
\bpython{print('hallo welt, na wie geht es dir? dies ist sehr langer Text, ein bisschen unangenehm oder? aber irgendwie muss ich die Zeile sprengen tut mir wirklich leid.')}\medskip\newline
Weiter existiert noch '\verb|i<Sprache>|', diesem Befehl kann eine Datei übergeben werden der diese dann in der jeweiligen Sprache hervorhebt. Beispiel mit diesem Dokument:
\begin{latex}
\ilatex{test-lst.tex}
\end{latex}
Da es sich hierbei auch um das Dokument handelt, findet sich das Ergebnis auf der nächsten Seite.
\clearpage
\section{Dokumentcode}
\ilatex{test-lst.tex}
\end{document}
