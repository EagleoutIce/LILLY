\begin{poem}[Geschrieben auf einem Elternabend der Klasse meiner Schwester.]{Winter}{2013}
Ein kleiner Fluss entflieht der Klaue,
Ein Igel versteckt sich flink im Laube.
Die letzten Blätter verfliegen der Luft,
Schwebend und fatternd, stets auf der Flucht.

Der Herbst hat das Handtuch längst geworfen,
Des Eises Kälte regieret nun.
Eine weiße Flocke um die andere,
Segelt auf das Feld hinab,
Und verziert herbstes Blättergrab.

Kein Erbarmen kennt der Winter,
Sehet wie alles keucht.
Er, er wird nie, niemals besinnter,
Selbst wenn ihm alles entfleucht.

So zittert alles und wartet flehend,
Auf den Krieger der bald kommt.
Möge der Winter doch bald gehen,
Hätt' er es doch nur gekonnt\ldots
\end{poem}