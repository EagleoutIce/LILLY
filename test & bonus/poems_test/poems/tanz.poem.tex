\begin{poem}[][ornafoot]{Tanz}{16.04.2017}
Mit dem Klang der ersten Note wirkst du so grazil,
Als Tänzer zwischen Licht und Schatten; vertieft im Liebesspiel.
Du drehst und springst im Takt der Geige, die dich durch die Bühne trägt,
Man merkt, wie mit jeder Saite, dein Herz wie das der Harfe schlägt.

Du schwimmst durch das Notenmeer, das den ganzen Saal erfüllt,
Und, immer mehr und mehr, den Hörer aus der Fassung spült.
Denn du, du bist die Tänzerin, die die Zeit verhüllt; und,
all den Kummer und die Plagen eines Lebens stillt.

Du, du spannst die Energie, die den Hörer mit dir reißt,
Du machst aus Klängen Harmonie, während der Hörer eine Welt bereist,
Die nur so vor Farben sprudelt, in der all die Freude brodelt,
Die sonst der Alltag unterdrückt; nun hält sie nichts mehr zurück.

Du blickst in ein Meer aus Menschen, die in deinen Wellen treiben,
Wie sie, voller Liebe, den Blick in deine Welt beschreiben.
Zusammen mit der Partitur öffnest du die Tür,
Die in das innerste eines jeden Herzens führt.

Da tanzen sie: Arme und Reiche, Schwarze und Weiße, Frau und Mann,
Im Alltag vielleicht aufs Blut verfeindet, hier tanzen sie zusammen.
Du bist ein Stück lang Ankerpunkt in einer Welt die übertaktet,
Und verhinderst, dass der Takt dem Ruhesehen nachgibt.

Und auch wenn der Hörer nur ich bin,
Und all die klänge aus dem Hörer sind,
Bist du der Ankerpunkt in meiner Welt die übertaktet,
Und verhinderst, Tag für Tag, dass sie dem Ruhedehnen nachgibt.

Deine Bühne ist das Notenmeer, ohne Wann und wo,
Denn in all den großen Sälen spielst du sowieso.
Mit dem Klang der ersten Note, wirkst du so grazil,
Und deine Welt hältst du, hältst mich am Leben,
Auch wenn der Saalvorhang schon lange fiel.

Und ich danke dir dafür\ldots
Danke
\end{poem}