\begin{poem}{Von den Raben}{23.04.2017}
\say{Die Raben kommen}
schallt es von den Dächern,
\say{Habt ihrs vernommen?}
Sie ernten nur Gelächter.
Die einen fliehen.

Die anderen bleiben.
Die einen schrien,
Die anderen schweigen.
Während die Raben weiterziehen
und sie vertreiben
oder sich an ihnen weiden.

\say{Die Raben kommen}
schallt es von den Dächern,
\say{Habt ihrs vernommen?}
Sie ernten nur Gelächter.

Ein Flügelschlag
Ein Krächzen,
Ein Menschen-ächzen,
Neue Stadt. Neuer Tag.
Scheints, keiner weiß was kommen mag.
Ein Rabenflügelschlag.

\say{Die Raben kommen}
schallt es von den Dächern,
\say{Habt ihrs vernommen?}
Sie ernten nur Gelächter.
Ein Krächzen ein Augenschlag,
Schwarze Wolke rotes Grab,
Neue Stadt und neuer Tag.

Die Zeitungen sind voll davon,
Dass doch bald die Raben kommen,
Nur wenn sie kommen,
bleibt es unvernommen,
Weil die, die die von den Dächern rufen,
Schnell das Weite suchen,
Um die Raben zu besuchen.
\end{poem}