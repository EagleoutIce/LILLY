\begin{poem}[][ornahat]{Sie}{21.03.2017}
Eine schwarze Silhouette im Sonnenuntergang,
Wie ein Anker für die Welt schwingt sich ihre Bahn.
Die Augen, wie aus Rosenquarz, im Glitzerspiel mit Feuer
Und als das Licht dem Stern entsagt ist es sie, die es erneuert.

Ihre Haare wehn‘ im Wind – seiden-gleich dem Wasserfall,
Wie als kleines Kind wird der Moment zum endlos Intervall.
Mit dem Antlitz einer Göttin bildet sie den Hafen,
In dem sich schon vor langer Zeit die Kinder Amors‘ trafen.

Der Duft ihrer Gestalt gleicht dem eines Blumenbeets
Das sich, Jahr für Jahr, nach dem Frühlings-morgen sehnt.
Ihr Mund der so Geschwungen dem Tor, einem Gemälde, gleicht
Und sich heimlich, tief, in die Gedanken des Betrachters schleicht.

Eine Silhouette, die vom Mondlicht so wundervoll erquickt,
Wie ein Engel die Boten der Liebe zu dir schickt.
Eine Silhouette, die vom Mondlicht gemalt und belebt
So zart, als ein Geschöpf des Himmels, so sanft über dem Boden schwebt.

Die Haut getränkt im weißen Schein, wie Marmor, makellos.
Im kühlen Nachtwind stellt sie die Gefühle bloß,
Die Masken, Stress, die Abgaslichter, sie weichen ihrer Wärme
Sie nimmt mich an der Hand und wir ziehen in die Ferne\ldots
\end{poem}