%%%% ULTIMATIVES PACKETLADEGEDÖNS!
%% JAKE SOLL DIE HIERRIN ENTHALTENEN DATEIEN DYNAMISCH LADEN KÖNNEN UND SO BEI DER INSTALLATION ODER BEI EINEM UPDATE DYNAMISCH NEUE PAKETE HINZUFÜGEN KÖNNEN

\providecommand{\LILLYxWANNABExERROR}{\false} %% GENERAL WHOOPS

%% \LILLYxUSure testet ob ein Paket geladen ist. Signatur: \LILLYxUSure[Fehlermeldung]{Paket}
\makeatletter
\newcommand{\LILLYxUSure}[2][Du hast mich verraten! Ein Paket sollte da sein, ist es aber nicht!]{%
    \@ifpackageloaded{#2}
        {}
        {\ClassError{Lilly}{Assertion for #2 failed: #1        }{This cannot be recovered}}%
}

\newcommand\LILLYxPoliteKnock[4][.sty]{%
\IfFileExists{#2#1}{#3}{#4}%
}

%% [Packargs]{Info}{Paket}{Fehlerm}{Fehlerb}{Call-Cmd}{type}{true}
\newcommand\LILLY@L@adP@ck@ge[9][]{
\ClassInfo{Lilly}{#6: #2               }\relax{}%%Infotext ist optional, wird nicht angezeigt wenn leer
%% \IfFileExists{} %% check
\LILLYxPoliteKnock{#3}{%true
\relax\RequirePackage[#1]{#3}\relax
\protect#9%%
}{%false
\csname #7\endcsname{Lilly}{Paket #3 (genauer: #3.sty) nicht gefunden! Fehlermeldung: #4}{}%%return false %% Optional Error help with Demand - not implemented
\ClassInfo{Lilly}{#8             }% - Vielleicht spezifisch machen, bringt jabei error nichts
\protect#5%%
}
}

\@ifundefined{LILLYxNOxPACKAGEC}{%
%% Wenn Lilly's Packet-Handling verwendet werden soll, so soll auch ein Fehler erhoben werden!
%% Signtatur: {Name des Pakets}{Info}{Fehler-Meldung}{Fehler-Behandlung}{OPTIONAL ARGUMENTS FOR PACKAGE}{true}
%% DEMAND VARIANT: CRITICAL LOAD
\newcommand\LILLYxDemandPackage[5]{\LILLY@L@adP@ck@ge[#4]{#2}{#1}{#3}{}{LILLYxDemandPackage}{ClassError}{Dieser Fehler wird nicht behoben!}{#5}}% consumes erhandling mjammi
%%Load VARIANT: WARNING LOAD
\newcommand\LILLYxLoadPackage[6]{\LILLY@L@adP@ck@ge[#5]{#2}{#1}{#3}{\renewcommand{\LILLYxWANNABExERROR}{#1}#4}{LILLYxLoadPackage}{ClassWarning}{Versuche es mit einer Fehlerbehandlung, vielleicht bringt es ja etwas!}{#6}}%
%%%%%%%%%
}{ % Es ist nicht gewünscht autodetect-Features zu verwenden
\ClassInfo{Lilly}{Lilly's intelligentes Packet-Handling wurde explizit mithilfe der Definition des Makros LILLYxNOxPACKAGEC deaktiviert!}%
\providecommand\LILLYxLoadPackage[6]{\ClassInfo{Lilly}{LILLYxLoadPackage: #2}\fi\relax\RequirePackage[#5]{#1}\relax\protect#6}%
\providecommand\LILLYxDemandPackage[5]{\LILLXxLoadPackage{#1}{#2}{#3}{}{#4}{#5}}{}%
}
