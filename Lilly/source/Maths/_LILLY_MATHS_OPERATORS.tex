
%% Bessere Formatierungen mancher Formatierungen in der Matheumgebung

\providecommand{\overbar}[1]{\mkern 1.5mu\overline{\mkern-1.5mu#1\mkern-1.5mu}\mkern 1.5mu} %Anpassung overlapping GDRA

\RequirePackage{letltxmacro}


\let\oldr@@t\r@@t
\def\r@@t#1#2{%
\setbox0=\hbox{$\oldr@@t#1{#2\,}$}\dimen0=\ht0
\advance\dimen0-0.2\ht0
\setbox2=\hbox{\vrule height\ht0 depth -\dimen0}%
{\box0\lower0.4pt\box2}}
\LetLtxMacro{\oldsqrt}{\sqrt}
\renewcommand*{\sqrt}[2][\ ]{\oldsqrt[#1]{#2} }      %%Schönere Wurzel

\providecommand{\das}
    {\vcentcolon\nolinebreak\mkern-1mu\nolinebreak=}
\providecommand{\daseq}
    {\vcentcolon\nolinebreak\mkern-1mu\nolinebreak\Leftrightarrow}
\providecommand{\sad}
    {=\nolinebreak\mkern-1mu\nolinebreak\vcentcolon}
\providecommand{\qesad}
    {\Leftrightarrow\nolinebreak\mkern-1mu\nolinebreak\vcentcolon}
\providecommand{\shouldeq}{\stackrel{!}{=}}

\providecommand{\crectat}[2]{\fill[#2] #1++(-0.2,-0.2) rectangle ++(0.4,0.4)}
\providecommand{\rectat}[1]{\fill[candypink] #1++(-0.2,-0.2) rectangle ++(0.4,0.4)}
\renewcommand{\det}{\text{det}~}
\providecommand{\adj}{\text{adj}~}
\providecommand{\inf}{\text{inf}~}
\providecommand{\sup}{\text{sup}~}

\providecommand{\max}{\text{max}~}
\providecommand{\min}{\text{min}~}

\LILLYcommand{\emptyset}{\ensuremath{\o}}

\providecommand\z{\ensuremath{z}~}
\providecommand\x{\ensuremath{x}~}
\providecommand\y{\ensuremath{y}~}
\providecommand{\LH}{\ensuremath{{\cal LH}}}
\providecommand{\eig}{\text{Eig}\tab[0.04cm]}
\providecommand{\Dim}{{\text{dim}\tab[0.08cm]}}

\providecommand{\sel}{\textsc{sel}}
\providecommand{\sign}{\text{sign}~}
\LILLYcommand\diag{\text{diag}\tab[0.04cm]}
\renewcommand{\mod}{{\tab[0.1cm]\textsc{mod}\tab[0.1cm]}}
\renewcommand{\Im}{\mathfrak{Im}}
\renewcommand{\Re}{\mathfrak{Re}}
\providecommand\LK{LK}
\providecommand\rg{{\textsl{rg}\tab[0.04cm]}}
\providecommand\KER{{\textsl{ker}\tab[0.04cm]}}
\providecommand\Eig{{\textsl{Eig}\tab[0.04cm]}}

\renewcommand*\env@matrix[1][*\c@MaxMatrixCols c]{%
\hskip -\arraycolsep
\let\@ifnextchar\new@ifnextchar
\array{#1}}
