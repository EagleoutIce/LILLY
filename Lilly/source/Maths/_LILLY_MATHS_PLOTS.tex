%%% Implementation des in der Dokumentation gewünschten LILLY-PLOT environments!

%% Dieses Feature ist seinerseits mit dem Wechsel in Version 1.0.8 angsetzt

%% \RequirePackage{pgfkeys}
\LILLYxDemandPackage{pgfkeys}{Damit wir angenehme kv-pairs haben}
                    {Leider ist die graph-Umgebung auf dieses Paket angewiesen}
                    {}{}

\LILLYxDemandPackage{wrapfig}{Falls wir es floating haben möchten :D}
                    {Leider ist die graph-Umgebung auf dieses Paket angewiesen}
                    {}{}

\makeatletter
\LILLYxLoadPackage{placeins}{Halte Floatings auf (sub)section level konsistent}
                    {Dieses Paket ist für Lilly nicht Lebensnotwendig}
                    {} %% Wir müssen keine Befehle faken damit die Welt schön aussieht
                    {section}{\AtBeginDocument{%
                    \expandafter\renewcommand\expandafter\subsection\expandafter
                      {\expandafter\@fb@subsecFB\subsection}%
                    \newcommand\@fb@subsecFB{\FloatBarrier
                    \gdef\@fb@afterHsHook{\@fb@topbarrier \gdef\@fb@afterHsHook{}}}
                    \g@addto@macro\@afterheading{\@fb@afterHsHook}
                    \gdef\@fb@afterHsHook{} %% currently not working as excepted :P
                 }}

\pgfkeys { %
    /lillyxPlot/.is family, /lillyxPlot,
    defaults/.style = { %
        scale=1, minX=-2, maxX=2,
        minY=0, maxY=4, offset=0.4,
        loffset=0.1, labelX=$x$, labelY=$y$,
        samples=250,
    },
    scale/.estore in = \lillyxPlotxScale,
    minX/.estore in = \lillyxPlotxMinX,
    maxX/.estore in = \lillyxPlotxMaxX,
    minY/.estore in = \lillyxPlotxMinY,
    maxY/.estore in = \lillyxPlotxMaxY,
    offset/.estore in =\lillyxPlotxOffset,
    loffset/.estore in =\lillyxPlotxLoffset,
    labelX/.estore in = \lillyxPlotxLabelX,
    labelY/.estore in = \lillyxPlotxLabelY,
    samples/.estore in = \lillyxPlotxSamples,
}

\NewDocumentCommand{\plotline}{ O{Ao} O{\x} m }{\draw[domain=\lillyxPlotxMinX:\lillyxPlotxMaxX, smooth, variable=#2,#1] plot[samples=\lillyxPlotxSamples] ({#2}, {#3});}

%% color variable function max min circlesize
\NewDocumentCommand{\plotseq}{ O{DebianRed} O{\x} m O{\lillyxPlotxMaxX} O{1} O{1pt}}{\foreach #2 in {#5,...,#4}{\pgfmathparse{#3}\filldraw[#1](#2,\pgfmathresult) circle (#6);}}

%%% [KEYS][TIKZPICTURE]
\DeclareDocumentEnvironment{graph}{ O{} O{}}{
\pgfkeys{/lillyxPlot, defaults, #1} %% Hier köznnte graph wrapfigure support erhalten mit ifthenelse{\equal{#3}{}}{}{\wrapfigure{#3}{#4}} oder so
\begingroup\begin{tikzpicture}[scale=\lillyxPlotxScale,#2]
    \draw[help lines, color=gray!30, densely dashed] (\lillyxPlotxMinX-\lillyxPlotxOffset, \lillyxPlotxMinY-\lillyxPlotxOffset) 
            grid (\lillyxPlotxMaxX+\lillyxPlotxOffset, \lillyxPlotxMaxY+\lillyxPlotxOffset);
            \draw[->, thick] (\lillyxPlotxMinX-\lillyxPlotxOffset-\lillyxPlotxLoffset, 0) 
            -- (\lillyxPlotxMaxX+\lillyxPlotxOffset+\lillyxPlotxLoffset, 0)
            node[right] {\lillyxPlotxLabelX};
            \draw[->, thick] (0, \lillyxPlotxMinY-\lillyxPlotxOffset-\lillyxPlotxLoffset) 
            -- (0, \lillyxPlotxMaxY+\lillyxPlotxOffset+\lillyxPlotxLoffset)
            node[above] {\lillyxPlotxLabelY};
        }{\end{tikzpicture}\endgroup}

%% {POS}[PGFKEYS][TIKZKEYS][WIDTH][WRAPFIGOFF]
\DeclareDocumentEnvironment{wgraph}{ m O{} O{}  O{0pt} O{}} {%
\wrapfigure{#1}{#4}% as \begin{wrapfigure} will enforce break
    \begin{graph}[#2][#3]}{%
    \end{graph}#5%
\endwrapfigure%
}
