%% https://tex.stackexchange.com/questions/23647/drawing-a-directory-listing-a-la-the-tree-command-in-tikz#34268

% disable ", otherwise forest will fail
\catcode`\"=12
\usepackage[edges]{forest}
\catcode`\"=13

\newcount\dirtree@lvl
\newcount\dirtree@plvl
\newcount\dirtree@clvl
\def\dirtree@growth{%
  \ifnum\tikznumberofcurrentchild=1\relax
  \global\advance\dirtree@plvl by 1
  \expandafter\xdef\csname dirtree@p@\the\dirtree@plvl\endcsname{\the\dirtree@lvl}
  \fi
  \global\advance\dirtree@lvl by 1\relax
  \dirtree@clvl=\dirtree@lvl
  \advance\dirtree@clvl by -\csname dirtree@p@\the\dirtree@plvl\endcsname
  \pgf@xa=1cm\relax
  \pgf@ya=-1cm\relax
  \pgf@ya=\dirtree@clvl\pgf@ya
  \pgftransformshift{\pgfqpoint{\the\pgf@xa}{\the\pgf@ya}}%
  \ifnum\tikznumberofcurrentchild=\tikznumberofchildren
  \global\advance\dirtree@plvl by -1
  \fi
}

\tikzset{
  dirtree/.style={
    growth function=\dirtree@growth,
    every node/.style={anchor=north,font=\LILLYxlstTypeWriter,rectangle,draw,rounded corners=2.75pt,inner sep=3pt, minimum height=0.65cm, minimum width=2em, align=center,transform shape}, %% 3em
    every child node/.style={anchor=west},
    edge from parent path={(\tikzparentnode\tikzparentanchor) |- (\tikzchildnode\tikzchildanchor)}
  }
}

\newenvironment{directory}[1][]{%
    \begin{tikzpicture}[dirtree, #1]
    }{;\end{tikzpicture}}




\colorlet{folderbg}{bondiBlue}
\colorlet{folderbgdark}{bondiBlue!85!black}
\colorlet{ifolderbg}{DebianRed}
\colorlet{ifolderbgdark}{DebianRed!85!black}
\newlength\len@folder@size
\setlength\len@folder@size{4pt}

% #1 light color
% #2 darker color
% #3 white color
\def\@lilly@td@folder#1#2#3{%
  \fill[rounded corners=0.5pt, color=#2]% peak
  (-1*\len@folder@size,-\len@folder@size) rectangle (1.85*\len@folder@size,\len@folder@size+2pt);
\fill[rounded corners=0.5pt, color=#2]% back
  (-1*\len@folder@size,-\len@folder@size) rectangle (0.35*\len@folder@size,\len@folder@size+3.25pt);
\fill[rounded corners=0.5pt, color=#3]% doc
  (-0.8*\len@folder@size,-\len@folder@size) rectangle (1.65*\len@folder@size,\len@folder@size+1.25pt);
\filldraw[rounded corners=0.5pt, color=#1, ultra thin]%
  (-1*\len@folder@size,-\len@folder@size) rectangle (1.85*\len@folder@size,\len@folder@size);
}

% #1 light color
% #3 darker color
% #2 white color
\def\@lilly@td@file#1#2#3{%
  \fill[fill=#1!85!#3, rounded corners=0.9pt] (-\len@folder@size,.6*\len@folder@size+2.75pt) coordinate (a) |- (\len@folder@size,-1.2*\len@folder@size) coordinate (b) to[rounded corners=0pt] ++(0,1.8*\len@folder@size) coordinate (c) to[rounded corners=0pt,] ++(-2.75pt,2.75pt) coordinate (d) -- cycle;
  \fill[fill=#3!65!#1, rounded corners=0.9pt] (d) |- (c);
  % Set Text Bars:
  \foreach \y/\fx in {0/1,0.2/1,0.4/1,0.6/0.75} {%
    \fill[#3] (-0.5\len@folder@size,-\y*\len@folder@size) rectangle ++ (\fx*\len@folder@size,0.4pt);
  }
}

\tikzset{%
  folder/.pic={%
    \@lilly@td@folder{folderbg}{folderbgdark}{MudWhite}%
  },
  importantfolder/.pic={%
        \@lilly@td@folder{ifolderbg}{ifolderbgdark}{MudWhite}%
  },
  file/.pic={%
    \@lilly@td@file{folderbg}{folderbgdark}{MudWhite}%
  },
  importantfile/.pic={%
    \@lilly@td@file{ifolderbg}{ifolderbgdark}{MudWhite}%
  },
}
\forestset{%
  declare autowrapped toks={pic me}{},
  declare boolean register={pic root},
  pic root=0,
  pic dir tree/.style={%
    for tree={%
      folder,
      font=\LILLYxlstTypeWriter,
      grow'=0,
    },
    before typesetting nodes={%
      for tree={%
        edge label+/.option={pic me},
      },
      if pic root={
        tikz+={
          \pic at ([xshift=\len@folder@size].west) {folder};
        },
        align={l}
      }{},
    },
  },
  pic me set/.code n args=2{%
    \forestset{%
      #1/.style={%
        inner xsep=2.6\len@folder@size,
        pic me={pic {#2}},
      }
    }
  },
  pic me set={dir}{folder},
  pic me set={idir}{importantfolder},%
  pic me set={file}{file},
  pic me set={ifile}{importantfile},
}

% \DeclareDocumentEnvironment{fancydir}{
%   O{} % forest opts CURRENTLY UNUSED :D
%   o % forest main content, allows to set forest opts if set
% }{%
% \IfValueTF{#2}{%
% %     \ifthenelse{\equal{#1}{}}{
% %       \begingroup
% %     }{
% %       \resizebox{#1}{!}\begingroup
% %     }
%     \forest
%     pic dir tree,%
%     pic root,%
%     for tree={dir,},%
%       [#2]%
%     }{%
%     \forest
%     pic dir tree,%
%     pic root,%
%     for tree={dir,},%
%     [#1]}
% }{\endforest\endgroup}


\environbodyname\STLXTREE% forest uses \BODY too. collision prevention over 9000
\bracketset{action character=@}
\NewEnviron{fancydir}{%
\parbox{1\linewidth}{
\begin{forest}%
    pic dir tree,%
    pic root,%
    for tree={dir,}%
    @\STLXTREE
\end{forest}%
}}%
\environbodyname\BODY% For the others, we don't want to change that