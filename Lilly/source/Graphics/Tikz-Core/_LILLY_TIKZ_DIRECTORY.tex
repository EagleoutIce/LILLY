%% https://tex.stackexchange.com/questions/23647/drawing-a-directory-listing-a-la-the-tree-command-in-tikz#34268

% disable ", otherwise forest will fail
\catcode`\"=12
\usepackage[edges]{forest}
\catcode`\"=13

\newcount\dirtree@lvl
\newcount\dirtree@plvl
\newcount\dirtree@clvl
\def\dirtree@growth{%
  \ifnum\tikznumberofcurrentchild=1\relax
  \global\advance\dirtree@plvl by 1
  \expandafter\xdef\csname dirtree@p@\the\dirtree@plvl\endcsname{\the\dirtree@lvl}
  \fi
  \global\advance\dirtree@lvl by 1\relax
  \dirtree@clvl=\dirtree@lvl
  \advance\dirtree@clvl by -\csname dirtree@p@\the\dirtree@plvl\endcsname
  \pgf@xa=1cm\relax
  \pgf@ya=-1cm\relax
  \pgf@ya=\dirtree@clvl\pgf@ya
  \pgftransformshift{\pgfqpoint{\the\pgf@xa}{\the\pgf@ya}}%
  \ifnum\tikznumberofcurrentchild=\tikznumberofchildren
  \global\advance\dirtree@plvl by -1
  \fi
}

\tikzset{
  dirtree/.style={
    growth function=\dirtree@growth,
    every node/.style={anchor=north,font=\LILLYxlstTypeWriter,rectangle,draw,rounded corners=2.75pt,inner sep=3pt, minimum height=0.65cm, minimum width=2em, align=center,transform shape}, %% 3em
    every child node/.style={anchor=west},
    edge from parent path={(\tikzparentnode\tikzparentanchor) |- (\tikzchildnode\tikzchildanchor)}
  }
}

\newenvironment{directory}[1][]{%
    \begin{tikzpicture}[dirtree, #1]
    }{;\end{tikzpicture}}




\colorlet{folderbg}{bondiBlue}
\colorlet{ifolderbg}{DebianRed}
\colorlet{filebg}{bondiBlue}
\colorlet{ifilebg}{DebianRed}
\newlength\len@folder@size
\setlength\len@folder@size{4pt}

\def\@folderindent{2.6}
\def\@fileindent{1.75}

\def\SetFolderFileSameIndent{%
    \gdef\@fileindent{\@folderindent}
}

% #1 light color
% #2 darker color
% #3 white color
% #4 Icon
\def\@lilly@td@folder#1#2#3#4{%
    \begin{scope}[every node/.style={overlay}]  
  \fill[rounded corners=0.5pt, color=#2]% peak
  (-1*\len@folder@size,-\len@folder@size) rectangle (1.85*\len@folder@size,\len@folder@size+2pt);
\fill[rounded corners=0.5pt, color=#2]% back
  (-1*\len@folder@size,-\len@folder@size) rectangle (0.35*\len@folder@size,\len@folder@size+3.25pt);
\fill[rounded corners=0.5pt, color=#3]% doc
  (-0.8*\len@folder@size,-\len@folder@size) rectangle (1.65*\len@folder@size,\len@folder@size+1.25pt);
\filldraw[rounded corners=0.5pt, color=#1, ultra thin]% front
  (-1*\len@folder@size,-\len@folder@size) rectangle (1.85*\len@folder@size,\len@folder@size) node[midway, centered,#3,font={\tiny\sffamily}] {#4};
  \end{scope}
}

% #1 light color
% #2 darker color
% #3 white color
% #4 logo, if empty => bars
\def\@lilly@td@file#1#2#3#4{%
\begin{scope}[every node/.style={overlay}]
  \fill[fill=#1!85!#3, rounded corners=0.9pt] (-\len@folder@size,.6*\len@folder@size+2.75pt) coordinate (a) |- (\len@folder@size,-1.2*\len@folder@size) coordinate (b) to[rounded corners=0pt] ++(0,1.8*\len@folder@size) coordinate (c) to[rounded corners=0pt,] ++(-2.75pt,2.75pt) coordinate (d) -- cycle;
  \fill[fill=#3!65!#1, rounded corners=0.9pt] (d) |- (c);
  % Set Text Bars:
  \ifthenelse{\equal{#4}{}}{
  \foreach \y/\fx in {0/1,0.2/1,0.4/1,0.6/0.75} {%
    \fill[#3] (-0.5\len@folder@size,-\y*\len@folder@size) rectangle ++ (\fx*\len@folder@size,0.4pt);
  }
  }{\node[#3,font={\tiny\sffamily},overlay] at(0,-0.25\len@folder@size) {#4};} % absatz für den knick
\end{scope}
}

% \def\@SmashText#1\@nil{%
%   \smash{XX#1XX}%
% }

\tikzset{%
  folder/.pic={%
        \@lilly@td@folder{#1}{#1!85}{MudWhite}{}%
  },
  pics/folder/.default={folderbg},
  logofolder/.pic={%
        \@lilly@td@folder{folderbg}{folderbg!85}{MudWhite}{#1}%
  },
  pics/logofolder/.default={$\pi$},
  pics/complexfolder/.style 2 args={%
        code ={\@lilly@td@folder{#1}{#1!85}{MudWhite}{#2}}%
  },
  pics/complexfolder/.default={folderbg}{$\pi$},
  %
  %
  %
  file/.pic={%
        \@lilly@td@file{#1}{#1!85}{MudWhite}{}%
  },
  pics/file/.default={filebg},
  pics/logofile/.default={$\pi$},
  logofile/.pic={%
        \@lilly@td@file{filebg}{filebg!85}{MudWhite}{#1}%
  },
  pics/logofile/.default={$\pi$},
  pics/complexfile/.style 2 args={%
        code ={\@lilly@td@file{#1}{#1!85}{MudWhite}{#2}}%
  },
  pics/complexfile/.default={filebg}{$\pi$}
}
\forestset{%
  declare autowrapped toks={icon pic}{},
  declare boolean register={pic root},
  pic root=0,
  pic dir tree/.style={%
    for tree={%
      folder,
      s sep=3.5pt,
      font={\LILLYxlstTypeWriter},
      grow'=0,
    },
      % execute at end node=\noexpand\@nil,%
    before typesetting nodes={%
      for tree={%
        edge label+/.option={icon pic},
        text depth={1pt},
        execute at begin node={\strut},%
        %node format = {\noexpand\node(\forestoption{name}) [\forestoption{node options}]{\forestoption{content format}};},
      },
      if pic root={
        tikz+={
          \pic at ([xshift=\len@folder@size].west) {folder};
        },
        align={l}
      }{},
    },
  },
  % #1 style name
  % #2 pic name
  % #3 color default
  % #4 offset
  create pic key/.code n args=4{%
    \forestset{%
      #1/.style={%
        inner xsep=#4*\len@folder@size,
        icon pic={pic {#2=##1}},
      },
      #1/.default=#3,
    }
  },
  % #1 style name
  % #2 pic name
  % #3 color default
  % #4 offset
  % #5 default symbol
  create symbol pic key/.code n args=5{%
    \forestset{%
      #1/.style n args=2{%
        inner xsep=#4*\len@folder@size,
        icon pic={pic {#2={##1}{##2}}},
      },
      #1/.default={#3}{#5},
    }
  },
  create pic key={dir}{folder}{folderbg}{\@folderindent}, % normal dir
  create pic key={idir}{folder}{ifolderbg}{\@folderindent}, % important dir
  create pic key={ldir}{logofolder}{$\mathbf{\pi}$}{\@folderindent}, % normal dir with logo
  create symbol pic key={cdir}{complexfolder}{folderbg}{\@folderindent}{$\mathbf{\pi}$}, % colored dir with logo option
  create pic key={file}{file}{filebg}{\@fileindent},
  create pic key={ifile}{file}{ifilebg}{\@fileindent},
  create pic key={lfile}{logofile}{$\mathbf{\pi}$}{\@fileindent}, %complexfile
  create symbol pic key={cfile}{complexfile}{filebg}{\@fileindent}{$\mathbf{\pi}$}, % colored file with logo option
}


\DeclareDocumentCommand{\CreateNewFolderType}{
    o           % Color
    m           % Name
    O{}         % icon
}{%
  \IfValueTF{#1}{
    \forestset{% 
      #2/.style n args=2{%
        inner xsep=\@folderindent*\len@folder@size,%
        icon pic={pic {complexfolder={##1}{##2}}},%
      },%
      #2/.default={#1}{#3},%
    }%
  }{% No #1
    \forestset{% 
      #2/.style = {%
        inner xsep=\@folderindent*\len@folder@size,%
        icon pic={pic {logofolder={##1}}},%
      },%
      #2/.default={#3},%
    }%
  }
}



\DeclareDocumentCommand{\CreateNewFileType}{
    o           % Color
    m           % Name
    O{}         % icon
}{%
  \IfValueTF{#1}{%
    \forestset{% 
      #2/.style n args=2{%
        inner xsep=\@fileindent*\len@folder@size,%
        icon pic={pic {complexfile={##1}{##2}}},%
      },%
      #2/.default={#1}{#3},%
    }
  }{% No #1
    \forestset{% 
      #2/.style = {%
        inner xsep=\@fileindent*\len@folder@size,%
        icon pic={pic {logofile={##1}}},%
      },%
      #2/.default={#3},%
    }
  }
}



\environbodyname\STLXTREE% forest uses \BODY too. collision prevention over 9000
\bracketset{action character=@}
\NewEnviron{fancydir}{%
\parbox{1\linewidth}{
\begin{forest}%
    pic dir tree,%
    pic root,%
    for tree={dir,}%
    @\STLXTREE
\end{forest}%
}}%
\environbodyname\BODY% For the others, we don't want to change that