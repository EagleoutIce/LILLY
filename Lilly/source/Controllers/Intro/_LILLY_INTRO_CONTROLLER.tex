%% Dieser Controller kümmert sich um die Korrekte implementierung des Beginn eines Dokuments er ist der einzige Controller der AtBeginDocument mit Schreibzugriffen initiiert (zT an Singleton-Kinder ausgelagert)

%%DEBUG wird immer die erste Seite sein, sofern dies gewünscht ist, alle anderen Pages werden sich zwangsläufig an dieser positionierung orientieren müssen. Damit ist auch nicht gewährleistet, dass das Titelbild sicher die erste Seite der PDF-Datei erhält!!

\ifx\LILLYxDEBUG\true
    \errorcontextlines 10000
    %% Für die Korrekturphase hartgecoded:
    %\ifx\LILLYxKORREKTUR\true
        %%% Wird nur genutzt wenn die KORREKTUR-Flag gesetzt wurde

%%\RequirePackage{showframe}

\AtBeginDocument{
\clearpage\begin{titlepage} 
    {\centering \LARGE\textbf{KORREKTUR-TITLEPAGE}\par}~\\
    \renewcommand{\arraystretch}{1.3}
    Heyhooooo du hast dich also dazu entschieden \VORLESUNG{} Korrektur zu lesen \smiley, das freut mich!
    es wäre von Vorteil, wenn du alle Dinge, welche dir auffallen direkt in der PDF-Datei kommentierst. Dafür bieten sich verschiedene PDF-Viewer an!:\\[0.25cm]
    \begin{itemize}\setlength{\itemsep}{0pt}
        \item Mobil:     
                    \begin{itemize}\setlength{\itemsep}{0pt}
                        \item Xodo \href{https://play.google.com/store/apps/details?id=com.xodo.pdf.reader}{\faGoogle} \href{https://itunes.apple.com/de/app/xodo-pdf-pro/id805075929}{\faApple}
                    \end{itemize}
        \item Am PC:
                    \begin{itemize}\setlength{\itemsep}{0pt}
                        \item Adobe PDF-Viewer \href{https://get.adobe.com/de/reader/otherversions/}{\faWindows} \href{https://get.adobe.com/de/reader/otherversions/}{\faApple}
                        \item Okular \href{https://okular.kde.org/download.php}{\faLinux} (Ubuntu: \texttt{sudo apt install okular})\\[0.25cm]
                    \end{itemize}
    \end{itemize}
    
    Damit man Texte leichter findet, welche über die Seitenrändern hinaus gehen, habe ich die Seitenränder aktiviert - wenn dich das stört, kann ich sie gerne auch entfernen!
    Neben den klassischen Rechtschräibföhlürn und, Kommasetzungsfehler, bitte ich auch, auf Logikfehler zu weisen hin. Weiter sind folgende Punkte relevant (neben generell alles was auffällt):\\[0.25cm]
    \textbf{TODOs:}
    \begin{itemize}[label=$\square$]\setlength{\itemsep}{0pt}
        \item Kapitel ohne unterüberschrift
        \item Große Weisflächen auf Seiten
        \item Definitionen - die nicht richtig gewichtet sind
        \item fehlende Üungsblätter (die sind grad eh noch nicht vollständig), bzw Fehler in den Übungsblättern, wie zum Beispiel dass die Punkte nicht ganz rechts stehen
        \item zu Lange Boxen
        \item unübersichtliche Erklärungen
        \item Fehlen irgendwo noch Beispiele
        \item Passen Abstände nicht?
        \item alternative Farben?
        \item Namen, Ideen, weiters\ldots\\[0.25cm]
    \end{itemize}
    Wenn du übrigens andere Boxen haben möchtest - sag mir das einfach :D\\[0.5cm]
    {\par\centering\Huge Danke! \Ninja\par}
\end{titlepage} 
\newpage
}

    %\fi
    %%% Wird nur genutzt wenn die DEBUG-Flag gesetzt wurde

\AtBeginDocument{
\clearpage\begin{titlepage} 
    {\centering \LARGE\textbf{DEBUG-TITLEPAGE}\par}~\\
    \renewcommand{\arraystretch}{1.3}
    \textbf{Generelles}:
    \[\begin{tabular}{|>{\bfseries}l<{:}|c|}
        \hline
        Lilly & \LILLYxVERSIONxLONG \\\hline
        Datum & \heute\\\hline
        Kompilier-Modus & \LILLYxMODE\\\hline
        Farbprofil & \raisebox{-2pt}{\LILLYxCOLORxRainbow}\\\hline

    \end{tabular}\]  
    \textbf{PDF-Spezifisch:}
    \[\begin{tabular}{|>{\bfseries}l<{:}|c|}
        \hline
        Author & \AUTHOR\\\hline
        Dokument-Name & \LILLYxDOCUMENTNAME\\\hline    
        PDF-Name & \LILLYxPDFNAME\\\hline
    \end{tabular}\]   
    \renewcommand{\arraystretch}{1}
    \textbf{TODOs:}
    \begin{itemize}\setlength{\itemsep}{0pt}
        \item Stichwortverzeichnis für Druckversion
        \item Besseres externalize
        \item Interne Dokumentparameter
        \item Bessere Struktur => LILLY als bibliothek
        \item Packet warnungen, infos, etc.
    \end{itemize}
\end{titlepage} 
\newpage
}

\fi

%% Hier greifen wir auf die Daten des KEYVAL-Prozessors zurück um Meta-Informationen des Dokuments zu erhalten. 

\ifnum\LILLY@Semester>0
\LILLYcommand{\anaI}[1][~]{{\bfseries\large\usefont{T1}{qzc}{m}{it}\selectfont Ana\-ly\-sis~1}#1}
\LILLYcommand{\pdp}[1][~]{{\bfseries\large\usefont{T1}{qzc}{m}{it}\selectfont Pa\-ra\-dig\-men der Pro\-gram\-mie\-rung}#1}
\LILLYcommand{\gdbs}[1][~]{{\bfseries\large\usefont{T1}{qzc}{m}{it}\selectfont Grund\-la\-gen der Be\-triebs\-sys\-tem\-e}#1}
\LILLYcommand{\pvs}[1][~]{{\bfseries\large\usefont{T1}{qzc}{m}{it}\selectfont Pro\-gram\-mier\-rung von Sys\-tem\-en}#1}

%% Seminar
\LILLYcommand{\knn}[1][~]{{\bfseries\large\usefont{T1}{qzc}{m}{it}\selectfont Kuenstliche neuronale Netze}#1}


\LILLYcommand{\LILLYxFlavourText}{Und da waren sie wieder und tanzten im Mondlicht,\\
Während um sie eine Welt voller Hohn ist,\\
Es waren die Schritte, die sie verführten,\\
Sie klammheimlich zueinander entführten\\
Und da schlugen ihre Herzen wie die Füße im Takt\par
{\hspace*{3.5em}- ein Glück, dass das Eis gehalten hat -\hfill {\tiny{\usefont{T1}{qzc}{m}{it}  Florian Sihler, 28.02.2018}\par}}}
\fi
\RequirePackage[yyyymmdd,hhmmss]{datetime}
\providecommand{\heute}{\number\day.~\month@ngerman\space\number\year}

\ifx\LILLY@Typ\LILLY@Typ@Mitschrieb

    %%Lade gewünschtes Titelblatt
    \input{\LILLYxPATHxDATA/Semester/\LILLY@Semester/Definitions/\LILLY@Vorlesung} %%Konstanten etc.

    \ifx\LILLY@Typ\LILLY@Typ@Mitschrieb
\AtBeginDocument{
\clearpage\begin{titlepage}
    {\centering\par\vspace*{7em}\par
    {\bfseries\usefont{T1}{qzc}{m}{it}\fontsize{32pt}{16pt}\selectfont\FULLTITLE\par}
    \vspace{1cm}
    {\scshape \tiny{\LILLY@Typ{} von}\\\large\AUTHOR\par}
    \vspace{0.5cm}
    {\itshape\small Version vom: \\ \heute\par}
    \par\vspace*{9em}\par
    \includegraphics[height=14em]{\LILLYxPATHxDATA/Semester/2/Graphics/titleimage\LILLYxVorlesung.pdf}
    \par\vspace*{1em}\par
    {\fontsize{3pt}{2pt}\selectfont\centering Fassung vom \heute{} um \currenttime Uhr}\hfill\href{https://www.instagram.com/eagleoutice/}{\faInstagram} \href{https://steamcommunity.com/id/EagleoutIce/}{\faSteamSquare} \href{https://www.amazon.de/Niederegger-Marzipan-Klassiker-Variationen-1-075/dp/B00L7UQY2K/ref=sr_1_29?s=grocery&ie=UTF8&qid=1548528851&sr=1-29&keywords=Marzipan}{\faGift}\vspace{-1.1em}
    \noindent\rule{\textwidth}{0.6pt}\\[0.2cm] %% TITLECANDY :D
    \footnotesize{Und da waren sie wieder und tanzten im Mondlicht,\\
        Während um sie eine Welt voller Hohn ist,\\
        Es waren die Schritte, die sie verführten,\\
        Sie klammheimlich zueinander entführten\\
        Und nun schlugen ihre Herzen wie die Füße im Takt\\[0.1cm]
        - ein Glück, dass das Eis gehalten hat -\par
    }
    {\raggedleft\tiny{\usefont{T1}{qzc}{m}{it}  Florian Sihler, 28.02.2019}\\[-1.2em]}
    \noindent\rule{\textwidth}{0.6pt}}
\end{titlepage}
}
\else\ifx\LILLY@Typ\LILLY@Typ@Zusammenfassung
\AtBeginDocument{
\clearpage\begin{titlepage}
    {\centering\par
    \par\vspace*{3em}\par
    {\tab[\LILLYxTITLExOffset] \raisebox{4.5em}{\parbox[0em]{0em}{\includegraphics[height=8em]{\LILLYxPATHxDATA/Semester/2/Graphics/titleimage\LILLYxVorlesung.pdf}}\tab[-\LILLYxTITLExOffset]}}
    {\bfseries\usefont{T1}{qzc}{m}{it}\fontsize{32pt}{16pt}\selectfont\FULLTITLE\par}
    \vspace{1.15cm}
    {\scshape\tiny{Zusammenfassung von}\\\large\AUTHOR\par}
	\vspace{0.5cm}
    {\itshape\small Version vom:\\ \heute\par}
    }
\pagestyle{empty}
\addtocontents{TOPIC}{\protect\thispagestyle{empty}}
{\par\vspace*{4.25em}\par\phantomsection\elable{INHA}
\ifx\LILLYxMODE\LILLYxMODExPRINT
{\bfseries\Large Inhaltsübersicht}\\[-0.25cm]
\else
{\bfseries\Large Inhaltsübersicht\hfill\small\jmark[DEF]{mrk:DEFS}}\\[-0.25cm]
\fi
\noindent\rule{\textwidth}{2pt}\\[0.2cm]
\listofTOPICS
\vfill
\POLITEINTRO
}
\end{titlepage}\cleardoublepage
}
\fi\fi

    %%Generieren Inhaltsverzeichnis

    \LILLYcommand{\contentsname}{\vspace{-1cm}\hfill Inhaltsverzeichnis\hfill\vspace{-0.8cm}} 
    \addto\captionsngerman{%
        \renewcommand{\contentsname}{\hbox{}\relax\hfill Inhaltsverzeichnis\hfill\hbox{}\relax}%
    }
    \LILLYcommand{\cftsubsecfont}{\normalfont\footnotesize\hypersetup{linkcolor=\LILLYxLINKSxMainColorDarker!90!white}}
    %\renewcommand{\cftsecleader}{\hfill}
    \LILLYcommand{\cftsubsecleader}{\tiny{\cftdotfill{\cftsecdotsep}}}
    \LILLYcommand{\cftsubsecpagefont}{\normalfont\scriptsize\hypersetup{linkcolor=\LILLYxLINKSxMainColorDarker!70!white}}

    \AtBeginDocument{
        %%Niemand will irgendwas sehen beim TOC :D
        \newpage\addtocontents{toc}{\protect\pagestyle{empty}}
        \addtocontents{toc}{\protect\thispagestyle{empty}}
        %%Sichergeh :D 
        \thispagestyle{empty}
        \protect\pagestyle{empty}
        \elable{eagleTOC}
        \tableofcontents
        \ifx\LILLYxMODE\LILLYxMODExPRINT%
        \newpage\hbox{}\relax%
        \clearpairofpagestyles%
        \cleardoublepage%
        \else %
        \newpage%
        \clearpairofpagestyles%
        \fi%
    }

    %% Layout für Mitschrieb

    \RequirePackage[headsepline]{scrlayer-scrpage}         %Header konfigurieren
\RequirePackage{caption}
\RequirePackage{lastpage}                                       %Letzte Seite erhalten
\RequirePackage{relsize}


%% https://www.ctan.org/pkg/changepage
\LILLYxDemandPackage{changepage}{Genaue identifikation der Seite}
                    {Fuer die genaue identifikation der Seite noetig!}
                    {}{} %% keine Argumente

%% https://ctan.org/pkg/wasysym
\LILLYxLoadPackage{wasysym}{Existenz des Blitzes :D}%% Package, Info
                    {Sonst geht halt errorEmblem nicht xD}
                    {}{nointegrals}{\let\wasysymLightning\lightning}%% otherwise this would collide with the amssymb :D

%%%%%%%%%%%%%%%%%%%%%%%%%%%%%%%%%%%%%%%%%%%%%%%%%%
% General and Layoutkonfiguration                %
%%%%%%%%%%%%%%%%%%%%%%%%%%%%%%%%%%%%%%%%%%%%%%%%%%


\renewcommand{\sectionmark}[1]{%
    \markright{#1}\renewcommand{\LILLYxRIGHTMARK}{#1}
}

\RequirePackage{titlesec}
% lines above and below, number right
\renewcommand{\chaptermark}[1]{\markboth{\rmdefault #1}{\faBookmark}} %%Chaptermark fix inject

\LILLYcommand{\leftmark}{} %% fix no chapter bug
\LILLYcommand{\rightmark}{}

\def\@lilly@mitschrieb@chfont{\scshape}


\titleformat{\chapter}[display]%
        {\relax}
        {\vspace{-4\baselineskip}\raggedleft{\color{LightGray}\chapterNumber\thechapter} \\ }
        {0pt}%
        {\vspace*{-4\baselineskip}\color{\LILLYxColorxTITLExCOLOR}\huge\vspace*{.9\baselineskip}\raggedright\@lilly@mitschrieb@chfont}% change color and size here
        [\normalcolor\normalsize\vspace*{-.7\baselineskip}\rule{\textwidth}{0.5pt}\vspace*{-1.7\baselineskip}]%

\newcommand*{\TitleSUB}[1]{
    \begingroup\vspace{-1.25\baselineskip}\textsc{\textcolor{\LILLYxColorxTITLExCOLOR}{#1}}~\\[1\baselineskip]\endgroup
}
\newcommand{\LILLYxRIGHTMARK}{\faBookmark}
\let\defChapter\chapter

\DeclareDocumentCommand{\chapter}{
    s   % starred
    O{} % optional short
    m   % name of chapter
}{%
    \IfBooleanTF{#1}{% starred
        \defChapter*{#3}% no opt
    }{% unstarred
        \ifthenelse{\equal{#2}{}}%
        {\defChapter{#3}\def\xchapname{#3}}% use same
        {\defChapter[{#2}]{#3}\def\xchapname{#2}}%
        \getRegisteredBoxLists
        \foreach \ctmplistname in \lillyxlist {%
            \addcontentsline{\ctmplistname}{chapter}{\xchapname}%
        }
        \renewcommand{\leftmark}{\xchapname}\renewcommand{\LILLYxRIGHTMARK}{\faBookmark}%
    }%
}

% \let\defChapter\chapter
% \renewcommand*{\chapter}{
%     \@ifstar{\starchapter}{\@dblarg\nostarchapter}
% }
% \def\starchapter*#1{
%     \defChapter*{#1}
% }
% \def\nostarchapter[#1]#2{\defChapter[{#1}]{#2}\addcontentsline{DEFS}{chapter}{#1}%
% \addcontentsline{SATZE}{chapter}{#1}\addcontentsline{ZSM}{chapter}{#1}\addcontentsline{UB}{chapter}{#1}%
% \addcontentsline{LEMME}{chapter}{#1}\renewcommand{\leftmark}{{#2}}\renewcommand{\LILLYxRIGHTMARK}{\faBookmark}}%

\LILLYcommand{\LILLYxLayoutxClear}{ %
    \pagestyle{empty} %KOMA-FTW!
    \pagenumbering{gobble}
}
\LILLYcommand{\LILLYxLayoutxRestore}{ %
    \pagenumbering{arabic}
    \pagestyle{scrheadings} %KOMA-FTW!
    \renewcommand{\chapterpagestyle}{scrheadings} %Kein Übungsblatt? => Mitschrieb => Konfiguriere Kapitel
    %\def\leftmark{Ruf mich an!}
    %\ifEAGLE@zsfg
    %\lofoot{\tiny{\jmark[Kapitel]{INHA}\textnormal{\guilsinglright}\theTOPIC}}
    %\else
    \ifx\LILLYxMODE\LILLYxMODExPRINT
        %\setheadsepline{0pt}
        \lehead{\textsf{$\mathbf{\thechapter}\mid\,$ \normalfont\textsf{\leftmark}}}
        %%\rohead{{\rightmark}} %% seems to be buggy?
        \rohead{\normalfont\textsf{\LILLYxRIGHTMARK}}
        \cofoot{{\scriptsize{\AUTHOR}}}
        \cefoot{{\scriptsize{\TITLE}}}
        \ofoot{\normalfont\textbf{\thepage}}
        %\ihead{\hspace{-0.1cm}\raisebox{-2em}{\faCaretDown}}
    \else%% \LILLYxFACULTYxCOLOR
        %%\lohead{\hspace*{-1.83cm}\small{\tikz[baseline=1.215ex]{\filldraw[LightGray!50!white,draw, rounded corners=1pt] (0,0) -- ++(1.45,0) -- ++(0.25,0.25) -- ++ (-0.25,0.25) -- ++(-1.45,0) -- cycle (0.75,0.25) node[minimum height=\baselineskip, inner sep=1pt,\LILLYxColorxLINKSxMainColor]{\silentHmark[Kapitel]{eagleTOC}[\LILLYxColorxLINKSxMainColor]};} \leftmark}}
        %\fi
        \lohead{\AUTHOR}
        \lofoot{\silentHmark[\small\textit{Kapitel}]{eagleTOC}[\LILLYxColorxLINKSxMainColor]\textnormal{\guilsinglright}\leftmark}
        \cofoot{}
        \rohead{\heute}
        %\lofoot{\tiny{{\rightmark} \(\rangle\){\leftmark}}}
        \ifx\LILLYxFOOTERxBUTTONS\true
            \rofoot{\raisebox{0.75pt}{\eXButton{Find}{\tiny \faSearch} ~ \eXButton{GoBack}{\tiny \faUndo} ~ \eXButton{GoForward}{\tiny \faRepeat}~ \eXButton{PrevPage}{\(\LHD\)}} \thepage/\pageref{LastPage} \raisebox{0.75pt}{\eXButton{NextPage}{\(\RHD\)}}}
        \else
            \rofoot{\thepage}
        \fi
    \fi
}

%\expandafter\expandafter\expandafter\widthof{\csname the#1 \endcsname}
\makeatletter
% As considered by the impracicability of the margin-layout this is no the use old and can be controlled by the if:
\newif\iflilly@mitschrieb@sectionlines@useold@
\lilly@mitschrieb@sectionlines@useold@false
\renewcommand\sectionlinesformat[4]{%
    \iflilly@mitschrieb@sectionlines@useold@%
        \phantomsection%%
        \protect\if@twoside%%% e.g. LILLYxMODEXPRINT
        \strictpagecheck\checkoddpage\ifoddpage%\protect\ifodd\c@page%% oddpage -> rSide
        {#4\hfill\makebox[0pt][l]{\smaller\hspace{.75\marginparsep}#3}}%
        \else%
        {\makebox[0pt][r]{\smaller#3\hspace{\marginparsep}}#4}%
        \fi%
        \else %% is oneside
        {\makebox[0pt][r]{\textcolor{\LILLYxColorxTITLExCOLOR}{#3}\hspace{\marginparsep}}#4}%
        \fi%
    \else%
	    \phantomsection%%\textcolor{\LILLYxColorxTITLExCOLOR}{
	    {{\smaller #3} #4}%
    \fi
}

\renewcommand{\@seccntformat}[1]{\textcolor{\LILLYxColorxTITLExCOLOR}{\csname the#1\endcsname}}

\def\@lilly@mitschrieb@selfont{\sffamily\bfseries}
% \def\@lilly@mitschrieb@selfont{\useefont\bfseries}

\renewcommand{\section}{\@startsection{section}{1}{\z@}
{-4ex \@plus -1ex \@minus -.4ex}
{1ex \@plus.2ex }
{\normalfont\LARGE\@lilly@mitschrieb@selfont}}
\renewcommand{\subsection}{\@startsection {subsection}{2}{\z@}
{-3ex \@plus -0.1ex \@minus -.4ex}
{0.5ex \@plus.2ex }
{\normalfont\Large\@lilly@mitschrieb@selfont}}
\renewcommand{\subsubsection}{\@startsection {subsubsection}{3}{\z@}
{-2ex \@plus -0.1ex \@minus -.2ex}
{.2ex \@plus.2ex }
{\normalfont\@lilly@mitschrieb@selfont}}
\renewcommand\paragraph{\@startsection{paragraph}{4}{\z@}
{-2ex \@plus-.2ex \@minus .2ex}
{.1ex}
{\normalfont\small\@lilly@mitschrieb@selfont}}

\setlength{\parindent}{0pt}

%%%%%%%%%%%%%%%%%%%%%%%%%%%%%%%%%%%%%%%%%%%%%%%%%%
% Introkonfiguration                             %
%%%%%%%%%%%%%%%%%%%%%%%%%%%%%%%%%%%%%%%%%%%%%%%%%%

%%Lade gewünschtes Titelblatt => LILLYxTITLE
\RequestConfig{\LILLYxPATHxDATA/Semester/Definitions/\LILLYxVorlesung.tex}
%}{}
%%OLD: \ifx\LILLY@Typ\LILLY@Typ@Mitschrieb
\AtBeginDocument{
\clearpage\begin{titlepage}
    {\centering\par\vspace*{7em}\par
    {\bfseries\usefont{T1}{qzc}{m}{it}\fontsize{32pt}{16pt}\selectfont\FULLTITLE\par}
    \vspace{1cm}
    {\scshape \tiny{\LILLY@Typ{} von}\\\large\AUTHOR\par}
    \vspace{0.5cm}
    {\itshape\small Version vom: \\ \heute\par}
    \par\vspace*{9em}\par
    \includegraphics[height=14em]{\LILLYxPATHxDATA/Semester/2/Graphics/titleimage\LILLYxVorlesung.pdf}
    \par\vspace*{1em}\par
    {\fontsize{3pt}{2pt}\selectfont\centering Fassung vom \heute{} um \currenttime Uhr}\hfill\href{https://www.instagram.com/eagleoutice/}{\faInstagram} \href{https://steamcommunity.com/id/EagleoutIce/}{\faSteamSquare} \href{https://www.amazon.de/Niederegger-Marzipan-Klassiker-Variationen-1-075/dp/B00L7UQY2K/ref=sr_1_29?s=grocery&ie=UTF8&qid=1548528851&sr=1-29&keywords=Marzipan}{\faGift}\vspace{-1.1em}
    \noindent\rule{\textwidth}{0.6pt}\\[0.2cm] %% TITLECANDY :D
    \footnotesize{Und da waren sie wieder und tanzten im Mondlicht,\\
        Während um sie eine Welt voller Hohn ist,\\
        Es waren die Schritte, die sie verführten,\\
        Sie klammheimlich zueinander entführten\\
        Und nun schlugen ihre Herzen wie die Füße im Takt\\[0.1cm]
        - ein Glück, dass das Eis gehalten hat -\par
    }
    {\raggedleft\tiny{\usefont{T1}{qzc}{m}{it}  Florian Sihler, 28.02.2019}\\[-1.2em]}
    \noindent\rule{\textwidth}{0.6pt}}
\end{titlepage}
}
\else\ifx\LILLY@Typ\LILLY@Typ@Zusammenfassung
\AtBeginDocument{
\clearpage\begin{titlepage}
    {\centering\par
    \par\vspace*{3em}\par
    {\tab[\LILLYxTITLExOffset] \raisebox{4.5em}{\parbox[0em]{0em}{\includegraphics[height=8em]{\LILLYxPATHxDATA/Semester/2/Graphics/titleimage\LILLYxVorlesung.pdf}}\tab[-\LILLYxTITLExOffset]}}
    {\bfseries\usefont{T1}{qzc}{m}{it}\fontsize{32pt}{16pt}\selectfont\FULLTITLE\par}
    \vspace{1.15cm}
    {\scshape\tiny{Zusammenfassung von}\\\large\AUTHOR\par}
	\vspace{0.5cm}
    {\itshape\small Version vom:\\ \heute\par}
    }
\pagestyle{empty}
\addtocontents{TOPIC}{\protect\thispagestyle{empty}}
{\par\vspace*{4.25em}\par\phantomsection\elable{INHA}
\ifx\LILLYxMODE\LILLYxMODExPRINT
{\bfseries\Large Inhaltsübersicht}\\[-0.25cm]
\else
{\bfseries\Large Inhaltsübersicht\hfill\small\jmark[DEF]{mrk:DEFS}}\\[-0.25cm]
\fi
\noindent\rule{\textwidth}{2pt}\\[0.2cm]
\listofTOPICS
\vfill
\POLITEINTRO
}
\end{titlepage}\cleardoublepage
}
\fi\fi

\AtBeginDocument{\LILLYxTITLE}

%%Generieren Inhaltsverzeichnis

\LILLYcommand{\contentsname}{\vspace{-1cm}\hfill \xtranslate{Inhaltsverzeichnis}\hfill\vspace{-0.8cm}}
\addto\captionsngerman{%
    \renewcommand{\contentsname}{\hbox{}\relax\hfill \xtranslate{Inhaltsverzeichnis}\hfill\hbox{}\relax}%
}


\LILLYcommand{\cftsubsecfont}{\normalfont\footnotesize\hypersetup{linkcolor=\LILLYxColorxLINKSxMainColorDarker!90!white}}
%\renewcommand{\cftsecleader}{\hfill}
\LILLYcommand{\cftsubsecleader}{{\lilly@cftdotfill{\cftsubsecdotsep}}}
\LILLYcommand{\cftsubsecpagefont}{\normalfont\scriptsize\hypersetup{linkcolor=\LILLYxColorxLINKSxMainColorDarker!70!white}}

\AtBeginDocument{
    %%Niemand will irgendwas sehen beim TOC :D
    \newpage\addtocontents{toc}{\protect\pagestyle{empty}}
    \addtocontents{toc}{\protect\thispagestyle{empty}}
    %%Sichergeh :D
    \thispagestyle{empty}%
    \protect\pagestyle{empty}%
    \elable{eagleTOC}
    \tableofcontents
    \ifx\LILLYxMODE\LILLYxMODExPRINT%
        \newpage\hbox{}\relax%
        \clearpairofpagestyles%
        \cleardoublepage%
    \else %
        \newpage%
        \clearpairofpagestyles%
    \fi%
}

\AtBeginDocument{
    \LILLYxLayoutxRestore
}

%%%%%%%%%%%%%%%%%%%%%%%%%%%%%%%%%%%%%%%%%%%%%%%%%%
% Outrokonfiguration                             %
%%%%%%%%%%%%%%%%%%%%%%%%%%%%%%%%%%%%%%%%%%%%%%%%%%

%% Dieser Controller kümmert sich um die korrekte implementierung des OUTROS eines Dokuments er fügt bei Bedarf Listen für Definitionen usw an und bindet auch den Index ein etc.

% there is a clear command i know :P
\def\LILLYxCLEARxHEADFOOT{\lohead{}\cohead{}\rohead{}\lofoot{}\cofoot{}\rofoot{}}

%% Wir haben einen Mitschrieb =>
%% - Definitionen
%% - Sätze
%% - Lemmas
%% - Zusamenfassungen
%% - Übungsblätter
\AtEndDocument{
    \pagestyle{empty}\tocloftpagestyle{plain}
     \ohead{}\chead{}\ihead{} %clear header
     \ofoot{}\cfoot{}\ifoot{} %clear footer => fixes double page issue
    \pagestyle{empty}% \typeout{REDNECK: \n@true\space \LILLYxSEENxDEFINITION}
    \ifx\LILLYxBOXxDefinitionxEnable\true\ifthenelse{\equal{\LILLYxSEENxDEFINITION}{\n@true}}{%
    \clearpage\phantomsection\addcontentsline{toc}{chapter}{Definitionen}
    \listofDEFINITIONS
    \clearpage}{}\fi
    \ifx\LILLYxBOXxSatzxEnable\true\ifthenelse{\equal{\LILLYxSEENxSATZ}{\n@true}}{%
    \phantomsection\addcontentsline{toc}{chapter}{Sätze}
    \listofSATZE
    \clearpage}{}\fi
     \ifx\LILLYxBOXxLemmaxEnable\true\ifthenelse{\equal{\LILLYxSEENxLEMMA}{\n@true}}{%
    \phantomsection\addcontentsline{toc}{chapter}{Lemmata}
    \listofLEMMATA\clearpage}{}\fi
    \ifx\LILLYxBOXxZusammenfassungxEnable\true\ifthenelse{\equal{\LILLYxSEENxZUSAMMENFASSUNG}{\n@true}}{%
    \phantomsection\addcontentsline{toc}{chapter}{Zusammenfassungen}
    \listofZUSAMMENFASSUNGEN\clearpage}{}\fi
     \ifx\LILLYxBOXxUebungsblattxEnable\true\ifthenelse{\equal{\LILLYxSEENxUBS}{\n@true}}{%
    \phantomsection\addcontentsline{toc}{chapter}{Übungsblätter}
    \listofUBS}{}\fi
 }
\else\ifx\LILLY@Typ\LILLY@Typ@Uebungsblatt
    
    \input{\LILLYxPATHxDATA/Semester/\LILLY@Semester/Definitions/\LILLY@Vorlesung} %%Konstanten etc.

    \RequirePackage[yyyymmdd,hhmmss]{datetime}
\RequirePackage[automark,headsepline]{scrlayer-scrpage}         %Header konfigurieren
\RequirePackage{caption}
\RequirePackage{lastpage}                                       %Letzte Seite erhalten


%% Vital configurations
\RequestConfig{\LILLYxPATHxDATA/Semester/\LILLYxSemester/Definitions/\LILLYxVorlesung} %%Konstanten etc.

%%%%%%%%%%%%%%%%%%%%%%%%%%%%%%%%%%%%%%%%%%%%%%%%%%
% General and Layoutkonfiguration                %
%%%%%%%%%%%%%%%%%%%%%%%%%%%%%%%%%%%%%%%%%%%%%%%%%%

\LILLYcommand{\TUTORBOX}{ %Tutorboxen für Blätter die analog abgegeben werden sollen => !Wird automatisch eingebunden. Siehe weiter unten. Komfort! Yay
    \begin{centered}
    \noindent\fbox{
        \begin{minipage}{\dimexpr0.8\textwidth-2\fboxsep-2\fboxrule\relax}
            \vspace{0.1cm}
        \begin{centered}
            \textbf{Tutor}: \\
            \large{\TUTOR}
        \end{centered}
        \vspace{0.1cm}
        \end{minipage}
    }
    \end{centered}
    \vspace{0.5cm}
}

\LILLYcommand{\LILLYxLayoutxClear}{ %
    \pagestyle{empty} %KOMA-FTW!
    \pagenumbering{gobble}
}

\LILLYcommand{\LILLYxLayoutxRestore}{ %
    \pagenumbering{arabic}
    \pagestyle{scrheadings} %KOMA-FTW!
    \lohead{\parbox[b][\headheight][b]{0.4\textwidth}{\AUTHOR}}
    \cohead{\parbox[b][\headheight][b]{0.5\textwidth}{\centering\tiny{\UEBUNGSHEADER}}}
    \rohead{\parbox[b][\headheight][b]{0.3\textwidth}{\raggedleft \heute}} \cofoot{}
    \rofoot{\thepage/\pageref{LastPage}}
    \ifx\LILLYxUBxSHOWTUTOR\true
        \TUTORBOX
    \fi
}

\newcommand{\points}[1]{\hbox{}\hfill{\emph{#1}}}

\AtBeginDocument{
    \LILLYxLayoutxRestore
}

%%%%%%%%%%%%%%%%%%%%%%%%%%%%%%%%%%%%%%%%%%%%%%%%%%
% Outrokonfiguration                             %
%%%%%%%%%%%%%%%%%%%%%%%%%%%%%%%%%%%%%%%%%%%%%%%%%%

%% None %% Einleitung Übungsblatt

\else\ifx\LILLY@Typ\LILLY@Typ@Dokumentation

%% Wir setzen keine Definitionen und definieren ein Standard Layout

\RequirePackage[automark,headsepline]{scrlayer-scrpage}         %Header konfigurieren
\RequirePackage{caption}
\RequirePackage{lastpage}                                       %Letzte Seite erhalten
\RequirePackage{titlesec}
\RequirePackage{marginnote}


%%%%%%%%%%%%%%%%%%%%%%%%%%%%%%%%%%%%%%%%%%%%%%%%%%
% General and Layoutkonfiguration                %
%%%%%%%%%%%%%%%%%%%%%%%%%%%%%%%%%%%%%%%%%%%%%%%%%%


\LILLYcommand{\cftsubsecleader}{\hfill}
\LILLYcommand{\cftsubsecfont}{\pgfsetfillopacity{1.0}\footnotesize}
\LILLYcommand{\cftsecfont}{\pgfsetfillopacity{1.0}\small}
\LILLYcommand{\cftchapfont}{\pgfsetfillopacity{1.0}\bfseries}


\LILLYcommand{\cftsecleader}{\normalfont{\cftdotfill{\cftsecdotsep}}}

\LILLYcommand{\cftsubsecpagefont}{\pgfsetfillopacity{0.0}}

\renewcommand{\sectionmark}[1]{%
    \markright{#1}%
}

% lines above and below, number right
\renewcommand{\chaptermark}[1]{\markboth{#1}{\faBookmark}} %%Chaptermark fix inject


\titleformat{\chapter}[display]%
        {\relax}
        {\vspace{-4\baselineskip}\raggedleft{\color{LightGray}\chapterNumber\thechapter} \\ }
        {0pt}%
        {\vspace*{-4\baselineskip}\color{\LILLYxColorxTITLExCOLOR}\huge\vspace*{.9\baselineskip}\raggedright\scshape}% change color and size here
        [\normalcolor\normalsize\vspace*{-.7\baselineskip}\rule{\textwidth}{0.5pt}\vspace*{-1.7\baselineskip}]%

\newcommand*{\TitleSUB}[1]{
    \begingroup\vspace{-1.25\baselineskip}\textsc{\textcolor{\LILLYxColorxTITLExCOLOR}{#1}}~\\\endgroup
}


\let\defChapter\chapter
\DeclareDocumentCommand{\chapter}{
    s   % starred
    O{} % optional short
    m   % name of chapter
}{%
    \IfBooleanTF{#1}{% starred
        \defChapter*{#3}% no opt
    }{% unstarred
        \ifthenelse{\equal{#2}{}}%
        {\defChapter{#3}\def\xchapname{#3}}% use same
        {\defChapter[{#2}]{#3}\def\xchapname{#3}}% different from Mitschrieb!
        % \getRegisteredBoxLists
        % \foreach \ctmplistname in \lillyxlist {%
        %     \addcontentsline{\ctmplistname}{chapter}{\xchapname}%
        % }
        \addcontentsline{TOP}{chapter}{\thechapter~\xchapname}\gdef\leftmark{\xchapname}%
    }%
}

% \let\defChapter\chapter
% \renewcommand*{\chapter}{
%     \@ifstar{\starchapter}{\@dblarg\nostarchapter}
% }
% \newcommand{\starchapter}[1]{
%     \defChapter*{#1}
% }
%\def\nostarchapter[#1]#2{\defChapter[{#1}]{#2}\addcontentsline{TOP}{chapter}{\thechapter~#2}\renewcommand{\leftmark}{#2}}

\newcommand{\LILLYxFormatxTitle}[2][]{~\\{\bfseries\usefont{T1}{qzc}{m}{it}\fontsize{14pt}{4pt}\selectfont #2\hfill{\normalfont\tiny #1}}\\[-0.3cm]
\noindent\rule{\textwidth}{1.25pt}\\[0.15cm]}

\newlistof{TOPS}{TOP}{\LILLYxFormatxTitle{Inhaltsverzeichnis}\vspace*{-2cm}}


\LILLYcommand{\LILLYxLayoutxClear}{ %
    \pagestyle{empty} %KOMA-FTW!
    \pagenumbering{gobble}
}
\LILLYcommand{\LILLYxLayoutxRestore}{ %
    \pagenumbering{arabic}   %
    \pagestyle{scrheadings} %KOMA-FTW!
    \renewcommand{\chapterpagestyle}{scrheadings} %Kein Übungsblatt? => Mitschrieb => Konfiguriere Kapitel
    \lofoot{\scriptsize{\silentHmark[Kapitel]{eagleTOC}\textnormal{\guilsinglright} \leftmark}} %
    \lohead{\AUTHOR} %
    \cofoot{} %
    \rohead{Dokumentation} %
    %\lofoot{\tiny{{\rightmark} \(\rangle\){\leftmark}}}
    \ifx\LILLYxFOOTERxBUTTONS\true %
        \rofoot{\raisebox{0.75pt}{\eXButton{Find}{\tiny \faSearch} \text{ } \eXButton{GoBack}{\tiny \faUndo} \text{ } \eXButton{GoForward}{\tiny \faRepeat}\text{ } \eXButton{PrevPage}{\(\LHD\)}} \thepage/\pageref{LastPage} \raisebox{0.75pt}{\eXButton{NextPage}{\(\RHD\)}}} %
    \else %
        \rofoot{\thepage/\pageref{LastPage}} %
    \fi %
}

\AtBeginDocument{
    \LILLYxLayoutxRestore
}


\makeatletter
\renewcommand{\@seccntformat}[1]{\llap{\textcolor{\LILLYxColorxDefinition}{\csname the#1\endcsname}\hspace{1em}}}
\renewcommand{\section}{\@startsection{section}{1}{\z@}
{-4ex \@plus -1ex \@minus -.4ex}
{1ex \@plus.2ex }
{\normalfont\LARGE\sffamily\bfseries}}
\renewcommand{\subsection}{\@startsection {subsection}{2}{\z@}
{-3ex \@plus -0.1ex \@minus -.4ex}
{0.5ex \@plus.2ex }
{\normalfont\Large\sffamily\bfseries}}
\renewcommand{\subsubsection}{\@startsection {subsubsection}{3}{\z@}
{-2ex \@plus -0.1ex \@minus -.2ex}
{.2ex \@plus.2ex }
{\normalfont\small\sffamily\bfseries}}
\renewcommand\paragraph{\@startsection{paragraph}{4}{\z@}
{-2ex \@plus-.2ex \@minus .2ex}
{.1ex}
{\normalfont\small\sffamily\bfseries}}


\let\defSection\section
\renewcommand*{\section}{
    \@ifstar{\starsection}{\@dblarg\nostarsection}
}
\newcommand{\starsection}[1]{
    \defSection*{#1}
}
\def\nostarsection[#1]#2{\defSection[{#1}]{#2}\addcontentsline{TOP}{section}{\thesection~#2}\renewcommand{\rightmark}{#2}}

\let\defSubsection\subsection
\renewcommand*{\subsection}{
    \@ifstar{\starsubsection}{\@dblarg\nostarsubsection}
}
\newcommand{\starsubsection}[1]{
    \defSubsection*{#1}
}
\def\nostarsubsection[#1]#2{\defSubsection[{#1}]{#2}\addcontentsline{TOP}{subsection}{\thesubsection~#2}}

\providecommand{\LILLYxBOXxDefinitionxLock}{TRUE}

\setlength{\parindent}{0pt}
\setlength{\parskip}{0pt}

%%%%%%%%%%%%%%%%%%%%%%%%%%%%%%%%%%%%%%%%%%%%%%%%%%
% Introkonfiguration                             %
%%%%%%%%%%%%%%%%%%%%%%%%%%%%%%%%%%%%%%%%%%%%%%%%%%

\let\originalmarginnote\marginnote
\renewcommand{\marginnote}{%
\ifthispageodd{\normalmarginpar}{\reversemarginpar}\originalmarginnote
}

%%%%%%%%%%%%%%%%%%%%%%%%%%%%%%%%%%%%%%%%%%%%%%%%%%
% Outrokonfiguration                             %
%%%%%%%%%%%%%%%%%%%%%%%%%%%%%%%%%%%%%%%%%%%%%%%%%%

%% None

\RequirePackage{marginnote} 

\let\originalmarginnote\marginnote
\renewcommand{\marginnote}{%
  \ifthispageodd{\normalmarginpar}{\reversemarginpar}\originalmarginnote
}

\else\ifx\LILLY@Typ\LILLY@Typ@Zusammenfassung
\input{\LILLYxPATHxDATA/Semester/\LILLY@Semester/Definitions/\LILLY@Vorlesung} %%Konstanten etc.

\RequirePackage[automark,headsepline]{scrlayer-scrpage}         %Header konfigurieren
\RequirePackage{lastpage,titlesec}
\RequestConfig{\LILLYxPATHxDATA/Semester/Definitions/\LILLYxVorlesung} %Konstanten etc.
% General and Layoutkonfiguration
\newif\ifappendix
\appendixfalse

\LILLYcommand\cftsubsecleader{\hfill}
\LILLYcommand\cftsubsecfont{\pgfsetfillopacity{1.0}\footnotesize}
\LILLYcommand\cftsecfont{\pgfsetfillopacity{1.0}\small}
\LILLYcommand\cftchapfont{\pgfsetfillopacity{1.0}\bfseries}
\LILLYcommand\cftsecleader{\normalfont{\cftdotfill{\cftsecdotsep}}}
\LILLYcommand\cftsubsecpagefont{\pgfsetfillopacity{0.0}}

\newcommand\LILLYxFormatxTitle[2][]{~\\{\bfseries\usefont{T1}{qzc}{m}{it}\fontsize{14pt}{4pt}\selectfont #2\hfill{\normalfont\tiny #1}}\\[-0.3cm]
\noindent\rule{\textwidth}{1.25pt}\\[0.15cm]}

\newlistof{TOPS}{TOP}{\LILLYxFormatxTitle{Inhaltsverzeichnis}\vspace*{-2cm}}

\LILLYcommand\LILLYxLayoutxClear{ %
    \pagestyle{empty} %KOMA-FTW!
    \pagenumbering{gobble}
}
\LILLYcommand\LILLYxLayoutxRestore{ %
\pagenumbering{arabic}
\pagestyle{scrheadings} %KOMA-FTW!
\ifx\LILLYxMODE\LILLYxMODExPRINT
    \lehead{\textsf{$\mathbf{\arabic{TOPICS}}\mid\,$ \normalfont\textsf{\leftmark}}}
    \rohead{\normalfont\textsf{\VORLESUNG}}
    \cofoot{{\scriptsize{\AUTHOR}}}
    \cefoot{{\scriptsize{\TITLE}}}
    \ofoot{\normalfont\textbf{\thepage}}
    \chead{}
\else
    \lohead{\AUTHOR}
    \lofoot{\silentHmark[\small\textit{Thema}]{eagleTOC}[\LILLYxColorxLINKSxMainColor]\textnormal{\guilsinglright}\,\leftmark}
    \cfoot{}
    \chead{}
    \rohead{\VORLESUNG}
    \ifx\LILLYxFOOTERxBUTTONS\true
        \rofoot{\raisebox{0.75pt}{\eXButton{Find}{\tiny \faSearch} \text{ } \eXButton{GoBack}{\tiny \faUndo} \text{ } \eXButton{GoForward}{\tiny \faRepeat}\text{ } \eXButton{PrevPage}{\(\LHD\)}} \thepage/\pageref{LastPage} \raisebox{0.75pt}{\eXButton{NextPage}{\(\RHD\)}}} %
    \else
        \rofoot{\thepage/\pageref{LastPage}} %
    \fi
\fi
}
\providecommand\LILLYxBOXxDefinitionxLock{TRUE}

\parindent=\z@
\newcounter{ctr_ADDONS}
\def\curAddonsCounter{ctr_ADDONS}

\newcommand\startExtra[1]{\clearpage\par\phantomsection
\addcontentsline{toc}{section}{#1}\addcontentsline{TOPIC}{TOPICS}{#1}\addcontentsline{DEFS}{chapter}{#1}%
\ifx\LILLYxMODE\LILLYxMODExPRINT\lehead{\textsf{$\mathbf{A}\mid\,$ \normalfont\textsf{#1}}}%
\else\lofoot{\silentHmark[\small\textit{Thema}]{eagleTOC}[\LILLYxColorxLINKSxMainColor]\textnormal{\guilsinglright}\,#1}%
\fi
}

\newcommand\startAppendix{\startExtra{Anhang}\appendixtrue}

\newcommand\theTOPIC{undefined}
\newcommand\listTOPICSname{\leavevmode\\[-4.75cm]}
\newlistof{TOPICS}{TOPIC}{\listTOPICSname}
\newcommand\TOPICS[1]{%
\renewcommand\theTOPIC{#1}%
\refstepcounter{TOPICS}%
% Set in Appendix without a number
\ifappendix\ifnum\value{TOPICS}>0 \addcontentsline{TOPIC}{TOPICS}{#1}\fi\par\else\ifnum\value{TOPICS}>0\addcontentsline{TOPIC}{TOPICS}{\arabic{TOPICS}) #1}\fi\par\fi}

\def\compileidx{\makeindex[title=Stichwortverzeichnis,options=-s \LILLYxPATHxINDEX,columns=2,columnsep=.75cm]\renewcommand\indexname{Stichwortverzeichnis}}

\newcommand\aLink[1]{\ifx\LILLYxMODE\LILLYxMODExPRINT\pageref{#1}%
\else\jmark[\faLeanpub]{#1}\fi}

\newcommand\bLink[1]{%
\ifx\LILLYxMODE\LILLYxMODExPRINT
    \pageref{#1}%
\else
    \jmark[\faPencil]{#1}%
\fi}
\let\formatAddons\arabic
\providecommand\TOP[3][top_default_jmpmrk]{%
    \addvspace{1.5em plus 0.75em minus 0.5em}\ifappendix%
    \refstepcounter{\curAddonsCounter}\addcontentsline{toc}{subsection}{#2}\else%
    \ifnum\value{TOPICS}>-1\addcontentsline{DEFS}{chapter}{#2}\fi\addcontentsline{toc}{section}{#2}%
    \TOPICS{#2}\markleft{#2}\fi\elable{top:#1}%
    \nopagebreak[4]\par\nopagebreak[4]%breakpatch
    \parbox{\linewidth}{{\bfseries\usefont{T1}{qzc}{m}{it}\fontsize{14pt}{4pt}\selectfont \ifappendix\formatAddons{\curAddonsCounter}\else\arabic{TOPICS}\fi) #2} \hfill{\tiny #3}}\vspace{-0.5\baselineskip}\\*%
    \rule{\textwidth}{2pt}\smallskip\\*\ignorespacesafterend%
}
\DeclareRobustCommand\kw[1]{\textbf{#1}\index{#1}}
\DeclareRobustCommand\sw[2][Gruppenname]{#2\index{#1!#2}}
\DeclareRobustCommand\sr[3][Gruppenname]{#3\index{#1!#2!#3}}

% definitions
\def\imp{\ensuremath{\prec}}
\def\<{\ensuremath{\langle}}
\def\>{\ensuremath{\rangle}}
\def\mto{\ensuremath{\to}}
% Register
\def\reg#1{\T{#1}}

\gdef\customex#1{\begingroup\scriptsize\textit{#1}\normalsize\endgroup}
\newlength{\askip} \askip=-.9\baselineskip
\newlength{\bskip} \bskip=-.275\baselineskip
\gdef\negaskip{\vspace*{\askip}}
\gdef\negbskip{\vspace*{\bskip}}
\gdef\TOPskip{\vspace*{-.35cm}}

\def\infot#1{\begingroup\tiny\itshape#1\endgroup}

\newenvironment{smalldesc}{%
\begingroup\negbskip\begin{description}\narrowitems\footnotesize}
{\negbskip\end{description}\endgroup}
% Smaller variant of the ditemize environment
\newenvironment{smalldite}{%
\begingroup\negbskip\begin{ditemize}\narrowitems\footnotesize}
{\negbskip\end{ditemize}\endgroup}

\newlength{\showcaseXwidth}\setlength{\showcaseXwidth}{6cm}

% Will render a tikzternal as a nice info-box
\DeclareDocumentCommand{\showcase}{%
    O{Charcoal}         % Color
    O{scale=1}          % TikZ-Commands
    m                   % Name
    +m                  % Description
    o                   % Bonusnote
    O{}                 % Bottomtag
}{%
\begin{tikzternal}[every node/.style={transform shape,#1},#2]
    \draw[thick,rounded corners=8pt,#1] (0,0.5) rectangle ++(\showcaseXwidth,-3.5);
    \node[rectangle,fill=white,right] at (0.25cm,0.5) {\bfseries #3};
    \IfValueT{#5}{\node[rectangle,fill=white,left] at (\showcaseXwidth-0.25cm,0.5) {#5};}
    \node[below right] at (0.1,0.25) {\parbox[t][3cm]{\showcaseXwidth-0.5cm}{\parbox[t][2.75cm]{\showcaseXwidth-0.5cm}{\small\ignorespaces%
        #4
    }{\relax\hbox{}\hfill\tiny #6}}};
\end{tikzternal} \tab[0.15cm]
}
\compileidx

\AtBeginDocument{\LILLYxLayoutxRestore}

% Introkonfiguration
% Work in Progress new Layout:
\AtBeginDocument{%
    \ifnum\LILLY@n>0\relax
        \LILLYxTITLExBONUS{Zusammenfassung \LILLY@n}%
    \else
        \LILLYxTITLExBONUS{Zusammenfassung}%
    \fi
    \cleardoublepage\POLITEINTRO
}

% TODO: make small arrow in limerence if page breaks
% Outrokonfiguration
\providecommand\LILLYxINDEXxINTERMEDIATExFORMAT{\clearpage}

\AtEndDocument{\makeatletter
    \addtocontents{DEFINITIONS}{\protect\thispagestyle{empty}}
    \protect\pagestyle{empty}\thispagestyle{empty}
    \LILLYxCLEARxHEADFOOT\pagestyle{empty}
    \protect\pagestyle{empty}\TOPICS{Stichwortverzeichnis}
    \printindex
    \LILLYxINDEXxINTERMEDIATExFORMAT
    \ifx\LILLYxBOXxDefinitionxEnable\true\ifthenelse{\equal{\LILLYxSEENxDEFINITION}{\n@true}}{%
    \pagenumbering{gobble}\protect\pagestyle{empty}\phantomsection\addcontentsline{toc}{chapter}{Definitionen}\elable{mrk:DEFS}
    \listofDEFINITIONS}{}\fi
}

\ifx\LILLY@Typ\LILLY@Typ@Mitschrieb
\AtBeginDocument{
\clearpage\begin{titlepage}
    {\centering\par\vspace*{7em}\par
    {\bfseries\usefont{T1}{qzc}{m}{it}\fontsize{32pt}{16pt}\selectfont\FULLTITLE\par}
    \vspace{1cm}
    {\scshape \tiny{\LILLY@Typ{} von}\\\large\AUTHOR\par}
    \vspace{0.5cm}
    {\itshape\small Version vom: \\ \heute\par}
    \par\vspace*{9em}\par
    \includegraphics[height=14em]{\LILLYxPATHxDATA/Semester/2/Graphics/titleimage\LILLYxVorlesung.pdf}
    \par\vspace*{1em}\par
    {\fontsize{3pt}{2pt}\selectfont\centering Fassung vom \heute{} um \currenttime Uhr}\hfill\href{https://www.instagram.com/eagleoutice/}{\faInstagram} \href{https://steamcommunity.com/id/EagleoutIce/}{\faSteamSquare} \href{https://www.amazon.de/Niederegger-Marzipan-Klassiker-Variationen-1-075/dp/B00L7UQY2K/ref=sr_1_29?s=grocery&ie=UTF8&qid=1548528851&sr=1-29&keywords=Marzipan}{\faGift}\vspace{-1.1em}
    \noindent\rule{\textwidth}{0.6pt}\\[0.2cm] %% TITLECANDY :D
    \footnotesize{Und da waren sie wieder und tanzten im Mondlicht,\\
        Während um sie eine Welt voller Hohn ist,\\
        Es waren die Schritte, die sie verführten,\\
        Sie klammheimlich zueinander entführten\\
        Und nun schlugen ihre Herzen wie die Füße im Takt\\[0.1cm]
        - ein Glück, dass das Eis gehalten hat -\par
    }
    {\raggedleft\tiny{\usefont{T1}{qzc}{m}{it}  Florian Sihler, 28.02.2019}\\[-1.2em]}
    \noindent\rule{\textwidth}{0.6pt}}
\end{titlepage}
}
\else\ifx\LILLY@Typ\LILLY@Typ@Zusammenfassung
\AtBeginDocument{
\clearpage\begin{titlepage}
    {\centering\par
    \par\vspace*{3em}\par
    {\tab[\LILLYxTITLExOffset] \raisebox{4.5em}{\parbox[0em]{0em}{\includegraphics[height=8em]{\LILLYxPATHxDATA/Semester/2/Graphics/titleimage\LILLYxVorlesung.pdf}}\tab[-\LILLYxTITLExOffset]}}
    {\bfseries\usefont{T1}{qzc}{m}{it}\fontsize{32pt}{16pt}\selectfont\FULLTITLE\par}
    \vspace{1.15cm}
    {\scshape\tiny{Zusammenfassung von}\\\large\AUTHOR\par}
	\vspace{0.5cm}
    {\itshape\small Version vom:\\ \heute\par}
    }
\pagestyle{empty}
\addtocontents{TOPIC}{\protect\thispagestyle{empty}}
{\par\vspace*{4.25em}\par\phantomsection\elable{INHA}
\ifx\LILLYxMODE\LILLYxMODExPRINT
{\bfseries\Large Inhaltsübersicht}\\[-0.25cm]
\else
{\bfseries\Large Inhaltsübersicht\hfill\small\jmark[DEF]{mrk:DEFS}}\\[-0.25cm]
\fi
\noindent\rule{\textwidth}{2pt}\\[0.2cm]
\listofTOPICS
\vfill
\POLITEINTRO
}
\end{titlepage}\cleardoublepage
}
\fi\fi

\else
\userput{_LILLY_LAYOUT_\LILLY@Typ}{\lillyPathLayout}{\LILLYxPATHxCONTROLLERS/Intro/Layouts}
\fi\fi\fi\fi

