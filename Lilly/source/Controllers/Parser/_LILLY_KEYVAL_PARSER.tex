\RequirePackage{kvoptions}                                  %Erweiterte Optionen

\SetupKeyvalOptions{                                        %Interne Benennung für Steuervariablen
    family=LILLY,
    prefix=LILLY@
}

\DeclareStringOption [-1]{n}[-1]                            %Das wievielte Übungsblatt ist es?
\DeclareStringOption [0]{Semester}[0]                       %Das wievielte Semester ist es?
\DeclareStringOption [GDRA]{Vorlesung}[GDRA]                %Bezeichner der Vorlesung
\DeclareStringOption [PLAIN]{Typ}[PLAIN]
\DeclareBoolOption[false]{Jake}
\DeclareBoolOption[false]{Universe}
\DeclareBoolOption[false]{paper}
\DeclareBoolOption[false]{beamer}
\DeclareBoolOption[true]{beamerKiz}

\DeclareVoidOption{Mitschrieb}{\renewcommand{\LILLY@Typ}{Mitschrieb}}
\DeclareVoidOption{Dokumentation}{\renewcommand{\LILLY@Typ}{Dokumentation}}
\DeclareVoidOption{Zusammenfassung}{\renewcommand{\LILLY@Typ}{Zusammenfassung}}\DeclareVoidOption{zsfg}{\renewcommand{\LILLY@Typ}{Zusammenfassung}}
\DeclareVoidOption{Uebungsblatt}{\renewcommand{\LILLY@Typ}{Uebungsblatt}}\DeclareVoidOption{ub}{\renewcommand{\LILLY@Typ}{Uebungsblatt}}

\DeclareDefaultOption{\renewcommand{\LILLY@Typ}{PLAIN}}

%%\DeclareBoolOption{Vorlesung}                               % Wir wollen einfache keys :D

\ProcessKeyvalOptions*                                      %Verarbeite Paketargumente (Optionen, wie auch immer)
\ifLILLY@Jake
\ClassInfo{Lilly}{Nutze Jake-Unterstuetzung}
\providecommand{\LILLYxSemester}{\LILLY@Semester}
\providecommand{\LILLYxVorlesung}{\LILLY@Vorlesung}
\else
\ClassInfo{Lilly}{Nutze keine Jake-Unterstuetzung}
\renewcommand{\LILLYxSemester}{\LILLY@Semester}
\renewcommand{\LILLYxVorlesung}{\LILLY@Vorlesung}
\fi

\ifLILLY@paper
\ClassInfo{Lilly}{Nutze Paper Layout} %% TODO PRELOADER FOR SOME LAYOUTS COMING BEFORE LOADCLASS
%% \typeout{Moin}
\providecommand{\LILLYxGeneralxFontsize}{8pt}
\providecommand{\LILLYxLayoutxWidth}{5.5in}
\providecommand{\LILLYxPAPER}{twocolumn}
\providecommand{\LILLYxGeometryxExtras}{marginparwidth=6em,marginparsep=3mm}
\providecommand{\LILLYxDocumentxTYPE}{scrartcl}
\providecommand{\LILLYxBOXxDefinitionxBox}{FALSE}
\providecommand{\LILLYxColorxLINKSxMainColor}{black}
\def\LILLY@Typ{PAPER}
\else\providecommand{\LILLYxPAPER}{}\fi
