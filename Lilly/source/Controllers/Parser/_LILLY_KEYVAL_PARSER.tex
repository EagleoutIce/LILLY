\RequirePackage{kvoptions}                                  %Erweiterte Optionen

\SetupKeyvalOptions{                                        %Interne Benennung für Steuervariablen
    family=LILLY,
    prefix=LILLY@
}

\DeclareStringOption [-1]{n}[-1]                            %Das wievielte Übungsblatt ist es?
\DeclareStringOption [1]{Semester}[1]                        %Das wievielte Semester ist es?
\DeclareStringOption [Niemand]{Vorlesung}[Liemann]          %Bezeichner der Vorlesung
\DeclareStringOption [DONNO]{Typ}[ONNO]
\DeclareBoolOption[false]{Jake}



%%\DeclareBoolOption{Vorlesung}                               % Wir wollen einfache keys :D

\ProcessKeyvalOptions*                                      %Verarbeite Paketargumente (Optionen, wie auch immer)
\ifLILLY@Jake
\ClassInfo{Lilly}{Nutze Jake-Unterstuetzung}
\providecommand{\LILLYxSemester}{\LILLY@Semester}
\providecommand{\LILLYxVorlesung}{\LILLY@Vorlesung}
\else
\ClassInfo{Lilly}{Nutze keine Jake-Unterstuetzung}
\renewcommand{\LILLYxSemester}{\LILLY@Semester}
\renewcommand{\LILLYxVorlesung}{\LILLY@Vorlesung}
\fi