%%% Wird nur genutzt wenn die KORREKTUR-Flag gesetzt wurde
\AtBeginDocument{
\clearpage\begin{titlepage}
{\centering \LARGE\textbf{KORREKTUR-TITLEPAGE}\par}~\\
\renewcommand{\arraystretch}{1.3}
Heyhooooo du hast dich also dazu entschieden \VORLESUNG{} Korrektur zu lesen \smiley, das freut mich!
es wäre von Vorteil, wenn du alle Dinge, welche dir auffallen direkt in der PDF-Datei kommentierst. Dafür bieten sich verschiedene PDF-Viewer an!:\\[0.25cm]
\begin{itemize}\setlength{\itemsep}{0pt}
    \item Mobil:
    \begin{itemize}\setlength{\itemsep}{0pt}
        \item Xodo \href{https://play.google.com/store/apps/details?id=com.xodo.pdf.reader}{\faGoogle} \href{https://itunes.apple.com/de/app/xodo-pdf-pro/id805075929}{\faApple}
    \end{itemize}
    \item Am PC:
    \begin{itemize}\setlength{\itemsep}{0pt}
        \item Adobe PDF-Viewer \href{https://get.adobe.com/de/reader/otherversions/}{\faWindows} \href{https://get.adobe.com/de/reader/otherversions/}{\faApple}
        \item Okular \href{https://okular.kde.org/download.php}{\faLinux} (Ubuntu: \texttt{sudo apt install okular})\\[0.25cm]
    \end{itemize}
\end{itemize}

Damit man Texte leichter findet, welche über die Seitenrändern hinaus gehen, habe ich die Seitenränder aktiviert - wenn dich das stört, kann ich sie gerne auch entfernen!
Neben den klassischen Rechtschräibföhlürn und, Kommasetzungsfehler, bitte ich auch, auf Logikfehler zu weisen hin. Weiter sind folgende Punkte relevant (neben generell alles was auffällt):\\[0.25cm]
\textbf{TODOs:}
\begin{itemize}[label=$\square$]\setlength{\itemsep}{0pt}
    \item Kapitel ohne unterüberschrift
    \item Große Weisflächen auf Seiten
    \item Definitionen - die nicht richtig gewichtet sind
    \item fehlende Üungsblätter (die sind grad eh noch nicht vollständig), bzw Fehler in den Übungsblättern, wie zum Beispiel dass die Punkte nicht ganz rechts stehen
    \item zu Lange Boxen
    \item unübersichtliche Erklärungen
    \item Fehlen irgendwo noch Beispiele
    \item Passen Abstände nicht?
    \item alternative Farben?
    \item Namen, Ideen, weiters\ldots\\[0.25cm]
\end{itemize}
Wenn du übrigens andere Boxen haben möchtest - sag mir das einfach :D\\[0.5cm]
{\par\centering\Huge Danke! \Ninja\par}
\end{titlepage}
\newpage
}
