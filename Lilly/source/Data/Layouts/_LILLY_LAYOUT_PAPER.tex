\RequirePackage[automark]{scrlayer-scrpage}         %Header konfigurieren
\RequirePackage{caption}
\RequirePackage{lastpage}                                       %Letzte Seite erhalten
\RequirePackage{relsize}

\RequirePackage[german]{fancyref}
\RequirePackage{lipsum} %% macht vieles einfacher


\RequirePackage{float} %% for H

\LILLYxLoadPackage{subcaption}{Für tolle Untertitel}
                    {Dieses Paket ist oft nicht vorhanden, nicht schlimm, wird dann nicht benutzt!}
                    {}
                    {}{}

\providecommand{\LILLYxBOXxDefinitionxLock}{TRUE}
\providecommand{\LILLYxBOXxBeispielxLock}{TRUE}
\providecommand{\LILLYxBOXxBemerkungxLock}{TRUE}
\providecommand{\LILLYxBOXxSatzxLock}{TRUE}
\providecommand{\LILLYxBOXxBeweisxLock}{TRUE}
\providecommand{\LILLYxBOXxLemmaxLock}{TRUE}
\providecommand{\LILLYxBOXxZusammenfassungxLock}{TRUE}

\@ifpackagelater{scrbase}{2014/12/12}{}{\def\scr@startsection{\@startsection}}%


\renewcommand{\sectionmark}[1]{%
    \markright{#1}\renewcommand{\LILLYxRIGHTMARK}{#1}
}

\RequirePackage{titlesec}

\LILLYcommand{\leftmark}{}
\LILLYcommand{\rightmark}{}

\newcommand{\LILLYxRIGHTMARK}{\faBookmark}


\LILLYcommand{\LILLYxLayoutxClear}{ %
    \pagestyle{empty} %KOMA-FTW!
    \pagenumbering{gobble}
}
\LILLYcommand{\LILLYxLayoutxRestore}{ %
    \pagenumbering{arabic}
    \pagestyle{scrheadings} %KOMA-FTW!

    \lofoot{\textsc{\TITLE}}
    \cofoot{}
    \rofoot{\textbf{\thepage}/\,\begingroup\hypersetup{linkcolor=black}\pageref{LastPage}\endgroup}
}


\providecommand{\ABSTRACT}{Hier könnte ihre Werbung stehen. Ich mag Züge wirklich sehr immer mehr, immer mehr, dies ist das Abstract - oh - Abstract oh jäah! \textbf{Ändere diesen Text durch:
{\T{\textbackslash providecommand\{\textbackslash ABSTRACT\}\{<Dein neuer Titel>\}}} vor {\T{\textbackslash documentclass[paper]\{LILLY\}.}}}}
\providecommand{\TITLE}{Hier könnte ihre Werbung stehen1. (\T{\textbackslash TITLE})}
\providecommand{\BRIEF}{Informatik Student,\\Universit\"{a}t Ulm -- \heute}

\newcommand{\printHeader}{\setlength{\columnsep}{2em}{\twocolumn[
    \begin{@twocolumnfalse}
     {\centering \usefont{T1}{qzc}{m}{it}\fontsize{24pt}{19pt}\selectfont{\TITLE}\par~\newline}
     {\centering \AUTHOR,~\BRIEF\par}~\bigskip\newline
     \rule{\linewidth}{1pt}\vspace{3pt}\newline
     \textbf{Abstract:}\quad {\ABSTRACT}\par
     \rule{\linewidth}{1pt}\bigskip\newline
    \end{@twocolumnfalse}
 ]}}

\newcommand{\printLILLY}{%
\paragraph{Dokumenterstellung}
Diese Ausarbeitung wurde mit \textLilly{} (Ver. \LILLYxVERSION) erstellt. Hierbei handelt es sich um ein Framework für \LaTeX, welches Florian Sihler für seine Studiumsmitschriften (aktuell) entwickelt. Es ist Quelloffen und befindet sich auf GitHub: \url{https://github.com/EagleoutIce/LILLY}.
}



\AtBeginDocument{
    \LILLYxLayoutxRestore
}

\RequirePackage{marginnote}
%% fontsize
\def\XfszXsection{\normalsize}
\def\XfszXsubsection{\footnotesize}
\def\XfszXsubsubsection{\scriptsize}
\def\XfszXparagraph{\footnotesize}
\def\LILLYxsection{section}
\def\LILLYxparagraph{paragraph}
%\expandafter\expandafter\expandafter\widthof{\csname the#1 \endcsname}
\makeatletter
% \renewcommand\sectionlinesformat[4]{%
% \def\st{#1}\ifx\st\LILLYxparagraph\else\marginnote[\csname XfszX#1\endcsname\hbox{}\hfill\sffamily\textcolor{\LILLYxTITLExCOLOR}{\ifx\st\LILLYxsection\else\vspace{-1.4\baselineskip}\fi\vspace{-0.02cm}\newline#3\vphantom{\normalsize I}}]{\csname XfszX#1\endcsname\sffamily\textcolor{\LILLYxTITLExCOLOR}{\ifx\st\LILLYxsection\else\vspace{-1.4\baselineskip}\fi\vspace{-0.02cm}\newline#3\vphantom{\normalsize I}}}\fi\phantomsection\ifx\st\LILLYxsection{\nobreak\tikz{\draw (0,0) |- ++(\linewidth,0.2) -- ++(0,-0.2);}}\vspace{-1.35\baselineskip}\newline\fi\begingroup\parskip=0pt\par\nopagebreak\centering#4\par\noindent\ignorespacesafterend\endgroup\ifx\st\LILLYxsection{\vspace{-1.55\baselineskip}\newline\nobreak\tikz{\draw (0,0) |- ++(\linewidth,-0.2) -- ++(0,0.2);}}\vspace{-0.5\baselineskip}\newline\else\vspace{-1.9\baselineskip}\newline\fi
% }
\addto\captionsngerman{%
    \renewcommand{\listfigurename}{\vspace{-1.5\baselineskip}}%
    \renewcommand{\listtablename}{\vspace{-1.5\baselineskip}}%
    \renewcommand{\figurename}{\textbf{Abb.}~}
    \renewcommand{\tablename}{\textbf{Tbl.}~}
}

\providecommand{\LILLYxPAPERxNUM}{TRUE}

\newcommand{\getNum}[1]{
    \ifx\LILLYxPAPERxNUM\true\Roman{#1}\else\Alph{#1}\fi
}

\newcommand{\startAppendix}{
    \grenewcommand{\LILLYxPAPERxNUM}{FALSE}
    \setcounter{section}{0}
    \section{Anhang}
    \printbib
    \tocloftpagestyle{scrheadings}
}

\newcommand{\intro}[1]{\smallskip\par\textit{#1}\smallskip\par}

\renewcommand{\sectionlinesformat}[4]{
    \begingroup%
        \def\st{#1}\nopagebreak%
        \ifx\st\LILLYxsection%
            \parskip=0pt\par\centering%
            \getNum{#1}.~%
        \else\ifx\st\LILLYxparagraph\else
            \normalsize\getNum{section}.\arabic{subsection}
        \fi\fi%
        #4%
        \par%
    \noindent\ignorespacesafterend\endgroup
}

\renewcommand{\section}{\@startsection{section}{1}{\z@}
{-4ex \@plus -1ex \@minus -.4ex}
{1ex \@plus.2ex }
{\normalfont\LARGE\bfseries}}
\renewcommand{\subsection}{\@startsection {subsection}{2}{\z@}
{-3ex \@plus -0.1ex \@minus -.4ex}
{0.5ex \@plus.2ex }
{\normalfont\normalsize\bfseries}}
\renewcommand{\subsubsection}{\@startsection {subsubsection}{3}{\z@}
{-2ex \@plus -0.1ex \@minus -.2ex}
{.2ex \@plus.2ex }
{\normalfont\small\bfseries}}
\renewcommand\paragraph{\@startsection{paragraph}{4}{\z@}
{-2ex \@plus-.2ex \@minus .2ex}
{.1ex}
{\normalfont\bfseries}}

\setlength{\parindent}{0pt}
