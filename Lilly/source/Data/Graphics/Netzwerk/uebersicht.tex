\gdef\GRAPHICSxUebersicht{% Test
\begin{tikzpicture}[every node/.style={font={\small\sffamily}, transform shape}, router size/.style={scale=.2}, tiny router size/.style={scale=.1},scale=.75,link/.style={}]
% todo: fix nested tikzpic via router, i'm lazy right now
% Regional ISP
\fill[\cA!15] (0,-.85) circle (1.85);
\node[router size] (regisp1) at (0,0) {\router{}};
\node[router size] (regisp2) at ($(regisp1) + (-1,-1)$) {\router{}};
\node[router size] (regisp3) at ($(regisp1) + (1,-1)$) {\router{}};
\node[below] at($(regisp1) + (0,-1.5)$) {regional ISP};
\draw[link] (regisp1) -- (regisp2) -- (regisp3) -- (regisp1); % connect regional isp

% global ISP, seems to be somewhat 'crazy' :P
\coordinate (globcenter) at (3,3.5);
\fill[\cB!15] (globcenter) circle (2.125);
\node[router size] (globisp1) at (2,3) {\router{}};
\node[router size] (globisp2) at ($(globisp1) + (-.25,.8)$) {\router{}};
\node[router size] (globisp3) at ($(globisp1) + (1.75,1.15)$) {\router{}};
\node[router size] (globisp4) at ($(globisp1) + (2.25,.15)$) {\router{}};
\node[above] at ([yshift=1cm]globcenter) {global ISP};
\draw[link] (globisp1) -- (globisp2) -- (globisp3) -- (globisp4) --  (globisp1) -- (globisp3); % connect regional isp
% dotted 'world' connection:
\draw[densely dashed] (globisp4) -- ++(1.25,0);
% connect global and local ISP
\draw[link] (regisp1) -- (globisp1) (regisp3) -- (globisp4);

% home network
% let's make a house
\fill[\cC!15] (-3.5,-.25) -| ++ (-2,1.5) -- ++(-.25,0) -- ++(1.25,.5) -- ++(1.25,-.5) -| cycle;

\node[tiny router size] (homenet1) at (-4,0) {\router{}};
\node[server,scale=.75] (homepc1) at ($(homenet1)+(0,.75)$) {};
\node[server,scale=.75] (homepc2) at ($(homepc1)+(-.85,.125)$) {};

\node[rack switch,scale=.5] (homeroute1) at ($(homenet1)+(-.85,0)$) {};
% "wifi":
\node at([yshift=.3cm,xshift=-.075cm]homeroute1) {\faWifi};
\node[below=.25cm] at ([xshift=-.5cm]homenet1) {home network};
% connect the home to the web :P
\draw[link] (homepc1) -- (homenet1) -- (regisp2) (homenet1) -- (homeroute1);

% institution network, we will do it tiny for 'consistency reasons' :P

\coordinate (instinit) at ($(regisp1)+(6.75,-2.5)$);
\fill[\cC!15] (instinit)++(2,-1.25) rectangle ++(-6,2);
\node[tiny router size] (instrout1) at (instinit) {\router{}};
\node[tiny router size] (instrout2) at ($(instrout1) + (-.95,-.5)$) {\router{}};
\node[tiny router size] (instrout3) at ($(instrout1) + (.5,-.5)$) {\router{}};
\draw[link] (instrout1) -- (instrout2) -- (instrout3) -- (instrout1); % connect regional isp

\node[server,scale=.75] (instserv1) at ($(instrout3)+(1.25,.5)$) {};
\node[server,scale=.75] (instserv2) at ($(instserv1)+(-.285,0)$) {};
\node[server,scale=.75] (instserv3) at ($(instserv2)+(-.285,0)$) {};

\draw[link] (instrout3) -| (instserv1) (instrout3) -| (instserv2) (instrout3) -| (instserv3);

\node[below=.25cm] at ([xshift=-.25cm]instrout3) {institutional network};
% the cheap wifi i provide them (somehow i, well... done something wrong with the naming. only needed twice, nvmd):
\node[rack switch,scale=.5] (instroute1) at ($(instrout1)+(-.85,0)$) {};

\node[server,scale=.75] (instpc1) at ($(instroute1)+(-.75,0)$) {};

\foreach \i [remember=\i as \lasti (initially 1)] in {2,...,4} {
    \node[server,scale=.75] (instpc\i) at ($(instpc\lasti)+(-.6,0)$) {};
}

\node[server,scale=.5] (instpc5) at ($(instpc1)+(-.125,-.75)$) {};
\node[server,scale=.5] (instpc6) at ($(instpc3)+(-.125,-.75)$) {};
\draw[link] (instpc5) -- (instrout2);

% "wifi":
\node at([yshift=.3cm,xshift=-.075cm]instroute1) {\faWifi};

% connect the instituion to the regional isp
\draw[link] (instroute1) -- (instrout1) |- (regisp3);

% mobile world:
\coordinate (mobinit) at (-1,3);
\fill[\cD!15] (mobinit)+(-1,1) circle (2.125);
\draw (mobinit)+(-1,2.5) node {mobile network};
\node[tiny router size] (mobilenet1) at (mobinit) {\router{}};

% well, i have no shortcut for a wifi-station or something similar, so let's create it on the spot:
\begin{scope}[every path/.style={very thick},scale=.25]
    \coordinate (BaseStart) at ($(mobilenet1)+(-3.25,1.5)$);
    %trapezoid base
    \draw (BaseStart) -- ++(1.5,.75) -- ++(1.5,-.75) -- ++(-1.5,-.75) -- cycle;%
    %draw tower:
    \draw (BaseStart) -- ++(1.5,3.5) -- ++(1.5,-3.5) (BaseStart)++(1.5,-.75) -- ++(0,4.25);
    %draw lines:

    \coordinate (BottomPoint) at ($(BaseStart)+(1.5,-.75)$);
    \coordinate (BottomLeftPoint) at ($(BaseStart)+(3,0)$); % weeeeell. it's bottom right, buuuut ... hehehehehehe
    \coordinate (BottomRightPoint) at (BaseStart);
    \coordinate (TopPoint) at ($(BaseStart)+(1.5,3.5)$);

    \foreach \y in {1,...,4}{
        \draw ($(BottomLeftPoint)!\y/6!(TopPoint)$) -- ($(BottomPoint)!\y/6!(TopPoint)$) -- ($(BottomRightPoint)!\y/6!(TopPoint)$);
    }

    % Top Dot:
    \filldraw[\cD,draw=black,semithick] (TopPoint) circle (.25);

    %\draw the Radio Waves
    \foreach \r in {.5,1,...,2.5}{
        \draw[thin] (TopPoint) +(-35:\r cm) arc[radius=\r cm,delta angle=70,start angle=-35];
        \draw[thin] (TopPoint) +(180-35:\r cm) arc[radius=\r cm,delta angle=70,start angle=180-35];
    }
\end{scope}
    % now, lets draw the rest :D

\draw (BottomLeftPoint) -- (mobilenet1) -- (globisp2);

\node[server,scale=.5] (mobserv1) at ($(TopPoint)+(-1.75,.25)$) {};
\node[server,scale=.5] (mobserv2) at ($(TopPoint)+(-1.75,-.75)$) {};

\node at ($(TopPoint)+(-.75,-.75)$) {\Large\faMobile};
\node at ($(TopPoint)+(-1,0)$) {\Large\faMobile};
\node at ($(TopPoint)+(-.25,.75)$) {\Large\faMobile};
\node at ($(TopPoint)+(-.25,-1.5)$) {\Large\faMobile};
\node at ($(TopPoint)+(-1.5,-1.35)$) {\Large\faMobile};
\end{tikzpicture}
}
\LILLYcommand\LILLYxGRAPHICSxSHOW{\GRAPHICSxUebersicht}
