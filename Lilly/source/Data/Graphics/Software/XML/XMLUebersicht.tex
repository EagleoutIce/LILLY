\lillyXgraphicsXproviderXisresizeablefalse%
%\gdef\GRAPHICSxXMLUebersicht{%
\begin{minipage}{14cm}
    \begin{tikzternal}[every node/.style={font=\footnotesize\itshape}]
        \node[below right] at(0,0) {%
        \begin{minipage}{0.7\linewidth}\footnotesize
            \begin{lstplain}[language=lXML,morekeywords={[5]{blaetter,blatt,aufgabe,unteraufgabe,punkte,text,grafik}}]
<?xml version="1.0" encoding="UTF-8"?>

<blaetter> <!-- Alle Blaetter-->
<blatt num="42" abgabe="10.10.2019">

  <aufgabe topic="Java NIO" points="12">
    <unteraufgabe nr="1.1">
      <punkte>4</punkte>
      <text>Bitte ... &amp;</text>
      <grafik/>
    </unteraufgabe>
  </aufgabe>
</blatt>
</blaetter>
\end{lstplain}
\end{minipage}
    };
    \begin{scope}[every path/.style ={latex-}]
        \draw (0.75,-0.25) -- ++(-0.25,0.5);
        \node at(0,0.25) [above right] {Liefert Metadaten zur Datei, \T{xml} ist das einzige reservierte Keyword};
        \draw (0,-1.3) -- ++(-0.5,0) node [left] {Wurzelelement};
        \draw (4.25,-1.1) |- ++(0.5,0.25) node [right] {Kommentar};
        \draw (4.65,-1.85) |- ++(0.75,-0.25) node [right] {Attribut des \say{\T{blatt}} Elements};
        \draw (1.65,-1.85) |- ++(-0.75,-0.25) node [left] {Kind des \say{\T{blaetter}}-Elements};
        \draw (1.95,-3.24) -- ++(-0.5,0) node [left] {\say{\T{punkte}}-Tag};
        \draw (5.6,-3.24) -- ++(0.5,0) node [right] {\say{\T{punkte}}-Endtag};
        \draw (4,-4.025) -- ++(0.5,0) node [right] {Leerer \say{\T{Grafik}}-Tag};
    \end{scope}
    \draw [decorate,decoration={brace,amplitude=10.5pt,raise=4pt},yshift=0pt] (4.5,-4.25) -- ++(0,-1.575) node [black,midway,xshift=0.8cm,right] {Verboser End-Tag-Spaß};
\end{tikzternal}%
\end{minipage}