\gdef\GRAPHICSxUMLStateDiagramExample{%
\begin{tikzternal}[every node/.style={font=\sffamily}]
    \begin{umlstate}[width=14cm,x=0,y=0]{Leben}
%%%%%%%%%%%%%%%%%%%%%
        \umlstateinitial[x=0,y=-10,width=2ex,name=Binit2] 
        \umlbasicstate[x=0,y=0, width=6.6cm,name=happi,entry={Beziehungsstatus = Vergeben},exit={Beziehungsstatus = Supersingle}]{Glücklich} 
        \umlbasicstate[x=0,y=-8, width=6.6cm,name=sad,entry={Kaufe Kuschelteddy},exit={}]{Traurig} 
        \umltrans{Binit2}{sad};
        \umlbasicstate[x=7.25,y=-4, name=tollertyp]{Toller Typ}; 
        \umltrans[recursive=-170|-110|2cm, pos=2, recursive direction=left to bottom]{tollertyp}{tollertyp} 
        \draw[->,thick] (happi.-45) -| (sad.45) node[pos=0.75,above,sloped] (ft2) {macht Schluss [kein Geld]};
        \draw[->,thick] (sad.135) -| (happi.225) node[pos=0.75,above,sloped] (ft1) {Freundin [hat Geld] / Endorphin};    
    \end{umlstate}
    
%%%%%%%%%%%%%%%%%%%%%

    \draw[densely dashed] (5cm,2.25) -- ++(0,-13);

    \providecommand{\descNode}[4][above]{%
        \draw ##2 node[##1]  {\parbox{5cm}{\textbf{##3}:\\\itshape ##4}};
    }

\begin{scope}[every path/.style={latex-}]
    \descNode[above right]{(tollertyp) -- ++(2,2)}{Paralleler Zustand}{Egal was links passiert, man ist immer ein \say{Toller Typ}. Beachte die gestrichelte Linie.}
    \descNode{(happi) -- ++(0,3)}{Zustand}{Beschreibt den Zustand \say{Glücklich}. Wird er Betreten/Verlassen wird der Beziehungsstatus entsprechend gesetzt.}
    \descNode{(Leben) -- ++(6,8)}{Zustand}{Beschreibt den Zustand \say{Leben}.}
    \descNode[left]{(ft1) -- ++(-3,0)}{Transition}{Wenn man eine Freundin bekommt (Event), Geld hat (Guard) stößt man Endorphin aus (Action). Die Komponenten sind alle Optional.}
    \descNode[below]{(Binit2) -- ++(0.5,-1.5)}{Init}{Hier startet alles. Ich habe den Init-Zustand beim tollen Typ weggelassen}
    \descNode[right]{(sad) -- ++(9.25,0)}{Verschachtelter Zustand}{Wie alle anderen ist \say{Traurig ein Subzustand von \say{Leben}, beim Eintreten kauft man einen Kuschelteddy. Beim Austreten geschieht nichts.}};

    \descNode[right]{(ft2) -- ++(2.25,1.75)}{Transition}{Macht sie Schluss und man hat kein Geld, so wird man wieder Traurig}
\end{scope}

\end{tikzternal}}
\LILLYcommand{\LILLYxGRAPHICSxSHOW}{\GRAPHICSxUMLStateDiagramExample} 