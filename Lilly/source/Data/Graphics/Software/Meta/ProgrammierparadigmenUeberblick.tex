\gdef\GRAPHICSxProgrammierparadigmenUeberblick{%
\begin{tikzternal}[scale=0.9, every node/.style={font=\mdseries}]
    \node (a) at (0,0) {Programmieren};%

    \node (1a) at (-3,-1) {{zustandsbasiert}};
    \node (1b) at (3, -1) {{regelbasiert}};

    \node (2a) at (-3,-2) {{imperativ}};
    \node (2b) at(1.5,-2) {{funktional}};
    \node (2c) at(4.5,-2) {{logisch}};

    \node (3a) at (-3,-3) {{objektorientiert}};

    %\begin{scope}[every path/.style={-latex}]
    \draw (a) -- (1a) -- (2a) -- (3a) %
          (a) -- (1b) -- (2b) %
                 (1b) -- (2c);
    %\end{scope}
    \begin{scope}[every node/.style={font=\scriptsize\color{gray}}]
        \draw (3a) ++ (-1.75,0.5) node [above left] (k0) {Klassen};
        \draw (3a) ++ (-1.5,-0.5) node [below left] (k1) {Abstraktion};

        \draw (3a) ++ (+1.25,-0.4) node [below right] (k2) {\parbox{3cm}{Problem:\\ Verifikation}};

        \draw (2b) ++ (-0.75,0.4) node [above left] (k3) {Alles: Funktionen};
        \draw (2b) ++ (-0.85,-0.35) node [below left] (k4) {$\lightning$ Variablen};

        \draw (2c) ++ (+1.65,-0.4) node [below right] (k5) {\parbox{3cm}{Prädikate}};
        \draw (2c) ++ (+1.45,0.2) node [right] (k6) {\parbox{3cm}{Logische\\Schlussfolgerungen}};

        \draw (2c) ++ (-0.25,0.75) node [above right] (k7) {\parbox{3cm}{Backtracking}};
    \end{scope}

    \begin{scope}[every path/.style={densely dashed,gray}]
        \draw (3a) -- (k0) (3a) -- (k1) (3a) -- (k2);
        \draw (2b) -- (k3) (2b) -- (k4);
        \draw (2c) -- (k5) (2c) -- (k6) (2c) -- (k7);
    \end{scope}

    %% TODO MAKE THIS A DAMMIT MACRO
    \begin{scope}[rounded corners=3pt]
        \draw[\Hcolor, fill=\Hcolor, fill opacity=0.1] (2a) ++(-1.5,0.25) rectangle ++(3,-1.5) ++(-1.5,0) node[below, fill opacity=1] {Java};

        \draw[\Hcolor, fill=\Hcolor, fill opacity=0.1] (2b) ++(-1.25,0.25) rectangle ++(2.5,-0.5) ++(-1.25,0) node[below, fill opacity=1] {Haskell};

        \draw[\Hcolor, fill=\Hcolor, fill opacity=0.1] (2c) ++(-1.25,0.25) rectangle ++(2.5,-0.5) ++(-1.25,0) node[below, fill opacity=1] {Prolog};
    \end{scope}


    %\draw [decorate,decoration={brace,amplitude=10pt,raise=4pt,mirror},yshift=0pt] (2b) ++ (-1.5,-0.75) -- ++(6,0) node [black,midway, below,yshift=-0.5cm] {Die behandeln wir.};
\end{tikzternal}%
}
\LILLYcommand{\LILLYxGRAPHICSxSHOW}{\GRAPHICSxProgrammierparadigmenUeberblick}