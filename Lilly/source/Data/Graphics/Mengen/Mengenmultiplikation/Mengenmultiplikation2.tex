\gdef\GRAPHICSxMengenmultiplikation2{%
\begin{tikzternal}
        \node at(0,0) {\(\begin{tabular}{cc}
          & \( \begin{pmatrix}
              3 & 0 \\
              1 & 2 \\
              1 & 1 \\
              1 & 2
          \end{pmatrix}\)\\[2.4em]
          \(\begin{pmatrix}
              1 & 3 & 4 & -1\\
              1 & 1 & 3 & 0
          \end{pmatrix}\) & \(\begin{pmatrix}
            9&\textbf{b}\\
            c&d
          \end{pmatrix}\)
      \end{tabular}\)};
      \draw[dashed,thin, color=lightgray] (-2.5,-1.03) -- ++(5,0);
      \draw[dashed,thin, color=lightgray] (1.56,2) -- ++(0,-4);
      \draw[latex-latex,color=limegreen] (-1.75,-0.5) to[bend left] (1.65,1.4); % node [midway,xshift=-0.8cm,yshift=0.85cm] {\(*\)};
      \draw[latex-latex,color=tealblue] (-1.25,-0.5) to[bend left] (1.65,0.9);
      \draw[latex-latex,color=mint] (-0.7,-0.5) to[bend left] (1.65,0.4);
      \draw[latex-latex,color=purple] (0,-0.5) to[bend left] (1.65,0);
      \draw[thick] (3,2.5) -- ++(0,-5);
      \node at(6.2,0) {\begin{minipage}{0.4\textwidth}
        Für \(b\) ergibt sich also: \\
        \(b = \textcolor{limegreen}{1*0} + \textcolor{tealblue}{3*2} + \textcolor{mint}{4*1} + \textcolor{purple}{(-1)* 2}\)\\
        \(\hphantom{b} = 8\)
      \end{minipage}};
      \end{tikzternal}
}
\LILLYcommand{\LILLYxGRAPHICSxSHOW}{\GRAPHICSxMengenmultiplikation2} 
