\gdef\GRAPHICSxMengenmultiplikation3{%
\begin{tikzternal}
\node at(0,0) {\(\begin{tabular}{cc}
& \( \begin{pmatrix}
    3 & 0 \\
    1 & 2 \\
    1 & 1 \\
    1 & 2
\end{pmatrix}\)\\[2.4em]
\(\begin{pmatrix}
    1 & 3 & 4 & -1\\
    1 & 1 & 3 & 0
\end{pmatrix}\) & \(\begin{pmatrix}
  9&8\\
  \textbf{c}&d
\end{pmatrix}\)
\end{tabular}\)};
\draw[dashed,thin, color=lightgray] (-2.5,-1.03) -- ++(5,0);
\draw[dashed,thin, color=lightgray] (1.56,2) -- ++(0,-4);
\draw[latex-latex,color=limegreen] (-1.75,-1.1) to[bend left] (1.13,1.4);
\draw[latex-latex,color=tealblue] (-1.25,-1.1) to[bend left] (1.13,0.9);
\draw[latex-latex,color=mint] (-0.7,-1.1) to[bend left] (1.13,0.4);
\draw[latex-latex,color=purple] (0,-1.1) to[bend left] (1.13,0);
\draw[thick] (3,2.5) -- ++(0,-5);
\node at(6.2,0) {\begin{minipage}{0.4\textwidth}
Für \(c\) ergibt sich also: \\
\(c = \textcolor{limegreen}{1*3} + \textcolor{tealblue}{1*1} + \textcolor{mint}{3*1} + \textcolor{purple}{0*1}\)\\
\(\hphantom{c} = 7\)
\end{minipage}};
\end{tikzternal}
}
\LILLYcommand\LILLYxGRAPHICSxSHOW{\GRAPHICSxMengenmultiplikation3}
