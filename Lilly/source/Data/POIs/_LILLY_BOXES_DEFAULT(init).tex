%%% TITEL TITLEPREFIX OPTIONAL-CONTROLLERS

\tcbset{LillyxBOXxDesignxDefault/.style={enhanced jigsaw, pad before break*=2mm, pad after break=2mm, %
lines before break=4, before skip=0pt, boxrule = 0mm, toprule=0.5mm, %
bottomtitle=0.5mm,bottomrule=1.2mm, after skip=0pt, enlarge top by=0.2\baselineskip, enlarge bottom by=0.2\baselineskip, sharp corners=south, enforce breakable}}

\providecommand{\LILLYxDEFAULTxTYPESETxTITLE}[2]{\!\relax{}\textbf{\textsc{{#2}}} \hfill \parbox[t]{0.75\linewidth}{\raggedleft\textbf{#1}}\vspace{-0.175cm}}

%%Es scheint nicht m�glich IfValueTF noch andere Optionen in kombination mithilfe von auto counter zu benutzen - dies ist in jeder hinsicht kacke

%%% Die Definition-Box:
\ifx\LILLYxBOXxDefinitionxLock\true
\DeclareTColorBox[auto counter]{LILLYxBOXxDefinition}{ O{} O{Definition \thetcbcounter~} O{drop fuzzy shadow} }{ LillyxBOXxDesignxDefault, %
colback=\LILLYxColorxDefinition!5!white, colframe=\LILLYxColorxDefinition, #3, %
title={\LILLYxDEFAULTxTYPESETxTITLE{#1}{#2}\ifx\LILLYxBOXxDefinitionxLock\true\\\else\\[-0.4\baselineskip]\fi} %
}
\DeclareTColorBox[use counter from=LILLYxBOXxDefinition]{LILLYxBOXxDefinition*}{ O{} O{Definition \thetcbcounter~} O{drop fuzzy shadow} }{ LillyxBOXxDesignxDefault, %
colback=\LILLYxColorxDefinition!5!white, colframe=\LILLYxColorxDefinition, #3, %
title={\LILLYxDEFAULTxTYPESETxTITLE{#1}{#2}\ifx\LILLYxBOXxDefinitionxLock\true\\\else\\[-0.4\baselineskip]\fi}, bookmark={#2 -- #1} %
}
\else
\DeclareTColorBox[auto counter, number within=\LILLYxBOXxDefinitionxLock]{LILLYxBOXxDefinition}{ O{} O{Definition \thetcbcounter~} O{drop fuzzy shadow} }{ LillyxBOXxDesignxDefault, %
colback=\LILLYxColorxDefinition!5!white, colframe=\LILLYxColorxDefinition, #3, %
title={\LILLYxDEFAULTxTYPESETxTITLE{#1}{#2}\ifx\LILLYxBOXxDefinitionxLock\true\\\else\\[-0.4\baselineskip]\fi} %
}
\DeclareTColorBox[use counter from=LILLYxBOXxDefinition]{LILLYxBOXxDefinition*}{ O{} O{Definition \thetcbcounter~} O{drop fuzzy shadow} }{ LillyxBOXxDesignxDefault, %
colback=\LILLYxColorxDefinition!5!white, colframe=\LILLYxColorxDefinition, #3, %
title={\LILLYxDEFAULTxTYPESETxTITLE{#1}{#2}\ifx\LILLYxBOXxDefinitionxLock\true\\\else\\[-0.4\baselineskip]\fi} %
}
\fi


%%% Die Beispiel-Box:
\ifx\LILLYxBOXxBeispielxLock\true
\DeclareTColorBox[auto counter]{LILLYxBOXxBeispiel}{ O{} O{Beispiel \thetcbcounter~} O{} }{ LillyxBOXxDesignxDefault, %
colback=\LILLYxColorxBeispiel!5!white, colframe=\LILLYxColorxBeispiel, #3,%
title={\LILLYxDEFAULTxTYPESETxTITLE{#1}{#2}\ifx\LILLYxBOXxBeispielxLock\true\\\else\\[-0.4\baselineskip]\fi}%
}
\else
\DeclareTColorBox[auto counter, number within=\LILLYxBOXxBeispielxLock]{LILLYxBOXxBeispiel}{ O{} O{Beispiel \thetcbcounter~} O{} }{ LillyxBOXxDesignxDefault, %
colback=\LILLYxColorxBeispiel!5!white, colframe=\LILLYxColorxBeispiel, #3,%
title={\LILLYxDEFAULTxTYPESETxTITLE{#1}{#2}\ifx\LILLYxBOXxBeispielxLock\true\\\else\\[-0.4\baselineskip]\fi}%
}
\fi


%%% Die Bemerkung-Box:
\ifx\LILLYxBOXxBemerkungxLock\true
\DeclareTColorBox[auto counter]{LILLYxBOXxBemerkung}{ O{} O{Bemerkung \thetcbcounter~} O{} }{ LillyxBOXxDesignxDefault, %
colback=\LILLYxColorxBemerkung!5!white, colframe=\LILLYxColorxBemerkung, #3,%
title={\LILLYxDEFAULTxTYPESETxTITLE{#1}{#2}\ifx\LILLYxBOXxBemerkungxLock\true\\\else\\[-0.4\baselineskip]\fi} %
}
\else
\DeclareTColorBox[auto counter, number within=\LILLYxBOXxBemerkungxLock]{LILLYxBOXxBemerkung}{ O{} O{Bemerkung \thetcbcounter~} O{} }{ LillyxBOXxDesignxDefault, %
colback=\LILLYxColorxBemerkung!5!white, colframe=\LILLYxColorxBemerkung, #3,%
title={\LILLYxDEFAULTxTYPESETxTITLE{#1}{#2}\ifx\LILLYxBOXxBemerkungxLock\true\\\else\\[-0.4\baselineskip]\fi} %
}
\fi


%%% Die Satz-Box:
\ifx\LILLYxBOXxSatzxLock\true
\DeclareTColorBox[auto counter]{LILLYxBOXxSatz}{ O{} O{Satz \thetcbcounter~} O{} }{LillyxBOXxDesignxDefault,%
colback=\LILLYxColorxSatz!5!white, colframe=\LILLYxColorxSatz, #3,%
title={\LILLYxDEFAULTxTYPESETxTITLE{#1}{#2}\ifx\LILLYxBOXxSatzxLock\true\\\else\\[-0.4\baselineskip]\fi} %
}
\else
\DeclareTColorBox[auto counter, number within=\LILLYxBOXxSatzxLock]{LILLYxBOXxSatz}{ O{} O{Satz \thetcbcounter~} O{} }{LillyxBOXxDesignxDefault,%
colback=\LILLYxColorxSatz!5!white, colframe=\LILLYxColorxSatz, #3,%
title={\LILLYxDEFAULTxTYPESETxTITLE{#1}{#2}\ifx\LILLYxBOXxSatzxLock\true\\\else\\[-0.4\baselineskip]\fi} %
}
\fi


%%% Die Beweis-Box:
\ifx\LILLYxBOXxBeweisxLock\true
\DeclareTColorBox[auto counter]{LILLYxBOXxBeweis}{ O{} O{Beweis \thetcbcounter~} O{} }{ LillyxBOXxDesignxDefault,%
colback=\LILLYxColorxBeweis!5!white, colframe=\LILLYxColorxBeweis, #3,%
title={\LILLYxDEFAULTxTYPESETxTITLE{#1}{#2}\ifx\LILLYxBOXxBeweisxLock\true\\\else\\[-0.4\baselineskip]\fi} %
}
\else
\DeclareTColorBox[auto counter, number within=\LILLYxBOXxBeweisxLock]{LILLYxBOXxBeweis}{ O{} O{Beweis \thetcbcounter~} O{} }{ LillyxBOXxDesignxDefault,%
colback=\LILLYxColorxBeweis!5!white, colframe=\LILLYxColorxBeweis, #3,%
title={\LILLYxDEFAULTxTYPESETxTITLE{#1}{#2}\ifx\LILLYxBOXxBeweisxLock\true\\\else\\[-0.4\baselineskip]\fi} %
}
\fi


%%% Die Lemma-Box:
\ifx\LILLYxBOXxLemmaxLock\true
\DeclareTColorBox[auto counter]{LILLYxBOXxLemma}{ O{} O{Lemma \thetcbcounter~} O{} }{LillyxBOXxDesignxDefault, %
colback=\LILLYxColorxLemma!5!white, colframe=\LILLYxColorxLemma, #3,%
title={\LILLYxDEFAULTxTYPESETxTITLE{#1}{#2}\ifx\LILLYxBOXxLemmaxLock\true\\\else\\[-0.4\baselineskip]\fi} %
}
\else
\DeclareTColorBox[auto counter, number within=\LILLYxBOXxLemmaxLock]{LILLYxBOXxLemma}{ O{} O{Lemma \thetcbcounter~} O{} }{LillyxBOXxDesignxDefault, %
colback=\LILLYxColorxLemma!5!white, colframe=\LILLYxColorxLemma, #3,%
title={\LILLYxDEFAULTxTYPESETxTITLE{#1}{#2}\ifx\LILLYxBOXxLemmaxLock\true\\\else\\[-0.4\baselineskip]\fi} %
}
\fi

%%% Die Zusammenfassung-Box:
\ifx\LILLYxBOXxZusammenfassungxLock\true
\DeclareTColorBox[auto counter]{LILLYxBOXxZusammenfassung}{ O{} O{Zusammenfassung \thetcbcounter~} O{} }{ LillyxBOXxDesignxDefault, %
colback=\LILLYxColorxZusammenfassung!5!white, colframe=\LILLYxColorxZusammenfassung, #3,%
title={\LILLYxDEFAULTxTYPESETxTITLE{#1}{#2}\ifx\LILLYxBOXxZusammenfassungxLock\true\\\else\\[-0.4\baselineskip]\fi} %
}
\else
\DeclareTColorBox[auto counter, number within=\LILLYxBOXxZusammenfassungxLock]{LILLYxBOXxZusammenfassung}{ O{} O{Zusammenfassung \thetcbcounter~} O{} }{ LillyxBOXxDesignxDefault, %
colback=\LILLYxColorxZusammenfassung!5!white, colframe=\LILLYxColorxZusammenfassung, #3,%
title={\LILLYxDEFAULTxTYPESETxTITLE{#1}{#2}\ifx\LILLYxBOXxZusammenfassungxLock\true\\\else\\[-0.4\baselineskip]\fi} %
}
\fi


%%sloppy - fix => MAKE ALL BOXES ENVIRONMENTS AND PROVIDE OLD COMMAND STYLE AS COMPAT FEATURE TODO:
\DeclareTColorBox{LILLYxBOXxAufgabe}{O{} O{} O{}}{enforce breakable, %enforce breakable für: Mehr Seiten
    enhanced jigsaw, before skip=2mm,after skip=2mm,
    colback=white,colframe=black!50,boxrule=0.2mm,
    attach boxed title to top left={xshift=1cm,yshift*=1mm-\tcboxedtitleheight},
    varwidth boxed title*=-3cm,
    boxed title style={
        frame code={
            \path[fill=white!30!black]
                ([yshift=-1mm,xshift=-1mm]frame.north west)
                arc[start angle=0,end angle=180,radius=1mm]
                ([yshift=-1mm,xshift=1mm]frame.north east)
                arc[start angle=180,end angle=0,radius=1mm];
            \path[left color=white!40!black,right color=white!40!black,
                middle color=white!55!black]
                ([xshift=-2mm]frame.north west) -- ([xshift=2mm]frame.north east)
                [rounded corners=1mm]-- ([xshift=1mm,yshift=-1mm]frame.north east)
                -- (frame.south east) -- (frame.south west)
                -- ([xshift=-1mm,yshift=-1mm]frame.north west)
                [sharp corners]-- cycle;
        },interior engine=empty,
    },
    fonttitle=\bfseries, #3,%
    title={#2 \ifthenelse{\equal{#1}{}}{}{--~}#1}, %Aufgabe
}

\DeclareTColorBox{LILLYxBOXxAufgabexPlain}{O{} O{} O{}}{enforce breakable, enhanced jigsaw, before skip=2mm,after skip=2mm, colback=white,colframe=black!50,boxrule=0.2mm,fonttitle=\bfseries, #3,title={#2 \ifthenelse{\equal{#1}{}}{}{--~}#1}}


