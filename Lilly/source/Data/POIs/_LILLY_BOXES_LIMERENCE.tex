%%% TITEL TITLEPREFIX OPTIONAL-CONTROLLERS
\makeatletter
\tcbset{LillyxBOXxDesignxLimerence/.style={ enforce breakable, lines before break=3, enhanced jigsaw,sharp corners, boxrule=0pt, fonttitle={\larger\bfseries}, coltitle={black}, attach title to upper, opacityfill=0.45, top=3pt, bottom=3pt, right=5pt, frame hidden,colback=LightGray!23!white,grow to right by=3pt}}

%%% small arrows on breaks:
\tcbset{% #1 is the color
    LillyxBOXxDesignxLimerencexArrows/.style={%
    borderline west={3pt}{0pt}{#1},
    extras first and middle={overlay={%
        \begin{scope}[shift={(frame.south west)}]
                \path[fill=#1] (0pt,0) -- ++(3pt,0pt) --  ++(-1.5pt,-3pt) --cycle;
        \end{scope}
    }},%
    extras middle and last={overlay={%
        \begin{scope}[shift={(frame.north west)}]
                \path[fill=#1] (0pt,0) -- ++(3pt,0pt) --  ++(-1.5pt,3pt) --cycle;
        \end{scope}
    }}
}
}

\tcbset{tag/.style={after title=}}

%%Es scheint nicht möglich IfValueTF noch andere Optionen in kombination mithilfe von auto counter zu benutzen - dies ist in jeder hinsicht kacke
%%OLD TITLEC: IfValueTF={#1}{title={$\vcenter{#2\ifthenelse{\equal{#1}{}}{}{--~}\parbox[t]{\linewidth-\widthof{#2}-1.75em}{#1}}$\\}}{title={$\vcenter{#2\parbox{\linewidth-\widthof{#2}-1.75em}{}}$}}
%% DEF: IfValueTF={#1}{title={$\vcenter{#2\ifthenelse{\equal{#1}{}}{}{--~}\parbox[t]{\linewidth-\widthof{#2}-1.75em}{#1}}\hfill\!\vcenter{\parbox[t]{\linewidth}{\textcolor{\LILLYxColorxDefinition}{\large\faThumbTack}}}$\\}}{title={$\vcenter{#2\parbox{\linewidth-\widthof{#2}-1.75em}{}}\hfill\!\vcenter{\parbox[t]{\linewidth}{\textcolor{\LILLYxColorxDefinition}{\large\faThumbTack}}}$}}

%%% Die Definition-Box:
% \ifx\LILLYxBOXxDefinitionxBox\true\tcbset{LILLYxBOXxDefinitionxBoxxLimerence/.style={%
%     if odd page or oneside*={borderline east={3pt}{0pt}{\LILLYxColorxDefinition}}{LillyxBOXxDesignxLimerencexArrows=\LILLYxColorxDefinition}}\else\tcbset{LILLYxBOXxDefinitionxBoxxLimerence/.style={}}\fi
\ifx\@onlypreamble\@notprerr%
    \ifx\LILLYxBOXxDefinitionxBox\true\tcbset{LILLYxBOXxDefinitionxBoxxLimerence/.style={LillyxBOXxDesignxLimerencexArrows=\LILLYxColorxDefinition}}\else\tcbset{LILLYxBOXxDefinitionxBoxxLimerence/.style={opacityfill=0}}\fi
\else%
\AtBeginDocument{
    \ifx\LILLYxBOXxDefinitionxBox\true\tcbset{LILLYxBOXxDefinitionxBoxxLimerence/.style={LillyxBOXxDesignxLimerencexArrows=\LILLYxColorxDefinition}}\else\tcbset{LILLYxBOXxDefinitionxBoxxLimerence/.style={opacityfill=0}}\fi
}\fi

\RenewTColorBox[use counter from=LILLYxBOXxDefinition]{LILLYxBOXxDefinition}{ O{} O{Definition \thetcbcounter~} O{} }{ enlarge left by=-3pt,   %
LillyxBOXxDesignxLimerence, LILLYxBOXxDefinitionxBoxxLimerence, #3, title={$\vcenter{#2\ifthenelse{\equal{#1}{}}{}{--~}\parbox[t]{\linewidth-\widthof{#2}-3.75em}{#1}}$\ifx\LILLYxBOXxDefinitionxBox\true\\[0.1pt]\else\\[-0.4\baselineskip]\fi}, %
}
\RenewTColorBox[use counter from=LILLYxBOXxDefinition]{LILLYxBOXxDefinition*}{ O{} O{Definition \thetcbcounter~} O{} }{ enlarge left by=-3pt,   %
LillyxBOXxDesignxLimerence,LILLYxBOXxDefinitionxBoxxLimerence, #3, title={#2\ifthenelse{\equal{#1}{}}{\parbox[t]{\linewidth-\widthof{#2}-2em}{#1}}{--~\parbox[t]{\linewidth-\widthof{#2}-0.05em-\widthof{--~}}{#1}}\parbox[t]{2em}{\textcolor{\LILLYxColorxDefinition}{\large\faThumbTack}}\ifx\LILLYxBOXxDefinitionxBox\true\\[0.1pt]\else\\[-0.4\baselineskip]\fi},bookmark*={color=\LILLYxColorxDefinition}{#2 -- #1} %
}

%%% Die Beispiel-Box:
\ifx\@onlypreamble\@notprerr%
    \ifx\LILLYxBOXxBeispielxBox\true\tcbset{LILLYxBOXxBeispielxBoxxLimerence/.style={LillyxBOXxDesignxLimerencexArrows=\LILLYxColorxBeispiel}}\else\tcbset{LILLYxBOXxBeispielxBoxxLimerence/.style={opacityfill=0}}\fi
\else%
\AtBeginDocument{
    \ifx\LILLYxBOXxBeispielxBox\true\tcbset{LILLYxBOXxBeispielxBoxxLimerence/.style={LillyxBOXxDesignxLimerencexArrows=\LILLYxColorxBeispiel}}\else\tcbset{LILLYxBOXxBeispielxBoxxLimerence/.style={opacityfill=0}}\fi
}\fi

\RenewTColorBox[use counter from=LILLYxBOXxBeispiel]{LILLYxBOXxBeispiel}{ O{} O{Beispiel \thetcbcounter~} O{} }{ enlarge left by=-3pt, LILLYxBOXxBeispielxBoxxLimerence,   %
LillyxBOXxDesignxLimerence, #3, title={$\vcenter{#2\ifthenelse{\equal{#1}{}}{}{--~}\parbox[t]{\linewidth-\widthof{#2}-1.75em}{#1}}$\ifx\LILLYxBOXxBeispielxBox\true\\\else\\[-0.4\baselineskip]\fi}, %
}


%%% Die Bemerkung-Box:
%% Da Borderline sich nicht einfach löschen lässt - hier über einen extra Marker
\ifx\@onlypreamble\@notprerr%
    \ifx\LILLYxBOXxBemerkungxBox\true\tcbset{LILLYxBOXxBemerkungxBoxxLimerence/.style={LillyxBOXxDesignxLimerencexArrows=\LILLYxColorxBemerkung}}\else\tcbset{LILLYxBOXxBemerkungxBoxxLimerence/.style={opacityfill=0}}\fi
\else%
\AtBeginDocument{
    \ifx\LILLYxBOXxBemerkungxBox\true\tcbset{LILLYxBOXxBemerkungxBoxxLimerence/.style={LillyxBOXxDesignxLimerencexArrows=\LILLYxColorxBemerkung}}\else\tcbset{LILLYxBOXxBemerkungxBoxxLimerence/.style={opacityfill=0}}\fi
}\fi

\RenewTColorBox[use counter from=LILLYxBOXxBemerkung]{LILLYxBOXxBemerkung}{ O{} O{Bemerkung \thetcbcounter~} O{} }{ enlarge left by=-3pt,LILLYxBOXxBemerkungxBoxxLimerence ,  %
LillyxBOXxDesignxLimerence, #3, title={$\vcenter{#2\ifthenelse{\equal{#1}{}}{}{--~}\parbox[t]{\linewidth-\widthof{#2}-1.75em}{#1}}$\ifx\LILLYxBOXxBemerkungxBox\true\\\else\\[-0.4\baselineskip]\fi}, %
}

%%% Die Satz-Box:
\RenewTColorBox[use counter from=LILLYxBOXxSatz]{LILLYxBOXxSatz}{ O{} O{Satz \thetcbcounter~} O{} }{ enlarge left by=-3pt, LillyxBOXxDesignxLimerencexArrows=\LILLYxColorxSatz,   %
LillyxBOXxDesignxLimerence, #3, title={$\vcenter{#2\ifthenelse{\equal{#1}{}}{}{--~}\parbox[t]{\linewidth-\widthof{#2}-1.75em}{#1}}$\\}, %
}


%%% Die Beweis-Box:
\ifx\@onlypreamble\@notprerr%
    \ifx\LILLYxBOXxBeweisxBox\true\typeout{Beweis Box shown}\tcbset{LILLYxBOXxBeweisxBoxxLimerence/.style={LillyxBOXxDesignxLimerencexArrows=\LILLYxColorxBeweis}}\else\typeout{Beweis box hidden}\tcbset{LILLYxBOXxBeweisxBoxxLimerence/.style={opacityfill=0}}\fi
\else%
\AtBeginDocument{
    \ifx\LILLYxBOXxBeweisxBox\true\typeout{Beweis Box shown}\tcbset{LILLYxBOXxBeweisxBoxxLimerence/.style={LillyxBOXxDesignxLimerencexArrows=\LILLYxColorxBeweis}}\else\typeout{Beweis box hidden}\tcbset{LILLYxBOXxBeweisxBoxxLimerence/.style={opacityfill=0}}\fi
}\fi

\RenewTColorBox[use counter from=LILLYxBOXxBeweis]{LILLYxBOXxBeweis}{ O{} O{Beweis \thetcbcounter~} O{} }{ enlarge left by=-3pt,LILLYxBOXxBeweisxBoxxLimerence ,  %
LillyxBOXxDesignxLimerence, #3, title={$\vcenter{#2\ifthenelse{\equal{#1}{}}{}{--~}\parbox[t]{\linewidth-\widthof{#2}-1.75em}{#1}}$\ifx\LILLYxBOXxBeweisxBox\true\\\else\\[-0.4\baselineskip]\fi}, %
}


%%% Die Lemma-Box:

\RenewTColorBox[use counter from=LILLYxBOXxLemma]{LILLYxBOXxLemma}{ O{} O{Lemma \thetcbcounter~} O{} }{ enlarge left by=-3pt, LillyxBOXxDesignxLimerencexArrows=\LILLYxColorxLemma ,  %
LillyxBOXxDesignxLimerence, #3, title={$\vcenter{#2\ifthenelse{\equal{#1}{}}{}{--~}\parbox[t]{\linewidth-\widthof{#2}-0.3em}{#1}}$\\}, %
}


%%% Die Zusammenfassung-Box:
\RenewTColorBox[use counter from=LILLYxBOXxZusammenfassung]{LILLYxBOXxZusammenfassung}{ O{} O{Zusammenfassung \thetcbcounter~} O{} }{ enlarge left by=-3pt, LillyxBOXxDesignxLimerencexArrows=\LILLYxColorxZusammenfassung,   %
LillyxBOXxDesignxLimerence, #3, IfValueTF={#1}{title={#2\ifthenelse{\equal{#1}{}}{\parbox[t]{\linewidth-\widthof{#2}-2em}{#1}}{--~\parbox[t]{\linewidth-\widthof{#2}-0.25em-\widthof{--~}}{#1}}\parbox[t]{2em}{\textcolor{\LILLYxColorxZusammenfassung}{\large\faArchive}}\\}}{title={#2\hbox{}\hfill\parbox[t]{2em}{\textcolor{\LILLYxColorxZusammenfassung}{\large\faArchive}}}}, %
}

%title={\begin{minipage}[t][\baselineskip][l]{\textwidth} \textbf{\textsc{{#2}}} \hfill {\textbf{#1}}\end{minipage}},

\RenewTColorBox{LILLYxBOXxAufgabexPlain}{O{} O{} O{}}{enforce breakable, enhanced jigsaw, before skip=2mm,after skip=2mm, colback=white,colframe=black!50,boxrule=0.0mm,lines before break=4,grow to left by=3pt, fonttitle=\bfseries, #3,title={#2 \ifthenelse{\equal{#1}{}}{}{--~}#1}}

\makeatother