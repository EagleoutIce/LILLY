%%% TITEL TITLEPREFIX OPTIONAL-CONTROLLERS
%% Dies ist die renew umgebung um default boxen zu forcieren auch wenn eine andere option gesetzt wurde
%%% Die Definition-Box:
\RenewTColorBox[use counter from=LILLYxBOXxDefinition]{LILLYxBOXxDefinition}{ O{} O{Definition \thetcbcounter~} O{drop fuzzy shadow} }{ LillyxBOXxDesignxDefault, %
colback=\LILLYxColorxDefinition!5!white, colframe=\LILLYxColorxDefinition, #3, %
title={\LILLYxDEFAULTxTYPESETxTITLE{#1}{#2}\ifx\LILLYxBOXxDefinitionxBox\true\else\\\fi} %
}
\RenewTColorBox[use counter from=LILLYxBOXxDefinition]{LILLYxBOXxDefinition*}{ O{} O{Definition \thetcbcounter~} O{drop fuzzy shadow} }{ LillyxBOXxDesignxDefault, %
colback=\LILLYxColorxDefinition!5!white, colframe=\LILLYxColorxDefinition, #3, %
title={\LILLYxDEFAULTxTYPESETxTITLE{#1}{#2}\ifx\LILLYxBOXxDefinitionxBox\true\else\\\fi},bookmark={#2 -- #1}  %
}
%%% Die Beispiel-Box:
\RenewTColorBox[use counter from=LILLYxBOXxBeispiel]{LILLYxBOXxBeispiel}{ O{} O{Beispiel \thetcbcounter~} O{} }{ LillyxBOXxDesignxDefault, %
colback=\LILLYxColorxBeispiel!5!white, colframe=\LILLYxColorxBeispiel, #3,%
title={\LILLYxDEFAULTxTYPESETxTITLE{#1}{#2}\ifx\LILLYxBOXxBeispielxBox\true\else\\\fi}%
}
%%% Die Bemerkung-Box:
\RenewTColorBox[use counter from=LILLYxBOXxBemerkung]{LILLYxBOXxBemerkung}{ O{} O{Bemerkung \thetcbcounter~} O{} }{ LillyxBOXxDesignxDefault, %
colback=\LILLYxColorxBemerkung!5!white, colframe=\LILLYxColorxBemerkung, #3,%
title={\LILLYxDEFAULTxTYPESETxTITLE{#1}{#2}\ifx\LILLYxBOXxBemerkungxBox\true\else\\\fi} %
}
%%% Die Satz-Box:
\RenewTColorBox[use counter from=LILLYxBOXxSatz]{LILLYxBOXxSatz}{ O{} O{Satz \thetcbcounter~} O{} }{LillyxBOXxDesignxDefault,%
colback=\LILLYxColorxSatz!5!white, colframe=\LILLYxColorxSatz, #3,%
title={\LILLYxDEFAULTxTYPESETxTITLE{#1}{#2}\ifx\LILLYxBOXxSatzxBox\true\else\\\fi} %
}
%%% Die Beweis-Box:
\RenewTColorBox[use counter from=LILLYxBOXxBeweis]{LILLYxBOXxBeweis}{ O{} O{Beweis \thetcbcounter~} O{} }{ LillyxBOXxDesignxDefault,%
colback=\LILLYxColorxBeweis!5!white, colframe=\LILLYxColorxBeweis, #3,%
title={\LILLYxDEFAULTxTYPESETxTITLE{#1}{#2}\ifx\LILLYxBOXxBeweisxBox\true\else\\\fi} %
}
%%% Die Lemma-Box:
\RenewTColorBox[use counter from=LILLYxBOXxLemma]{LILLYxBOXxLemma}{ O{} O{Lemma \thetcbcounter~} O{} }{LillyxBOXxDesignxDefault, %
colback=\LILLYxColorxLemma!5!white, colframe=\LILLYxColorxLemma, #3,%
title={\LILLYxDEFAULTxTYPESETxTITLE{#1}{#2}\ifx\LILLYxBOXxLemmaxBox\true\else\\\fi} %
}
%%% Die Zusammenfassung-Box:
\RenewTColorBox[use counter from=LILLYxBOXxZusammenfassung]{LILLYxBOXxZusammenfassung}{ O{} O{Zusammenfassung \thetcbcounter~} O{} }{ LillyxBOXxDesignxDefault, %
colback=\LILLYxColorxZusammenfassung!5!white, colframe=\LILLYxColorxZusammenfassung, #3,%
title={\LILLYxDEFAULTxTYPESETxTITLE{#1}{#2}\ifx\LILLYxBOXxZusammenfassungxBox\true\else\\\fi} %
}
%%sloppy - fix => MAKE ALL BOXES ENVIRONMENTS AND PROVIDE OLD COMMAND STYLE AS COMPAT FEATURE TODO:
\RenewTColorBox{LILLYxBOXxAufgabe}{O{} O{} O{}}{enforce breakable, %enforce breakable für: Mehr Seiten
enhanced jigsaw, before skip=2mm,after skip=2mm,
colback=white,colframe=black!50,boxrule=0.2mm,
attach boxed title to top left={xshift=1cm,yshift*=1mm-\tcboxedtitleheight},
varwidth boxed title*=-3cm,
boxed title style={
frame code={
\path[fill=white!30!black]
    ([yshift=-1mm,xshift=-1mm]frame.north west)
    arc[start angle=0,end angle=180,radius=1mm]
    ([yshift=-1mm,xshift=1mm]frame.north east)
    arc[start angle=180,end angle=0,radius=1mm];
\path[left color=white!40!black,right color=white!40!black,
    middle color=white!55!black]
    ([xshift=-2mm]frame.north west) -- ([xshift=2mm]frame.north east)
    [rounded corners=1mm]-- ([xshift=1mm,yshift=-1mm]frame.north east)
    -- (frame.south east) -- (frame.south west)
    -- ([xshift=-1mm,yshift=-1mm]frame.north west)
    [sharp corners]-- cycle;
    },interior engine=empty},
fonttitle=\bfseries, #3,%
title={#2 \ifthenelse{\equal{#1}{}}{}{--~}#1}, %Aufgabe
}

\RenewTColorBox{LILLYxBOXxAufgabexPlain}{O{} O{} O{}}{enforce breakable, enhanced jigsaw, before skip=2mm,after skip=2mm, colback=white,colframe=black!50,boxrule=0.2mm,fonttitle=\bfseries, #3,title={#2 \ifthenelse{\equal{#1}{}}{}{--~}#1}}


