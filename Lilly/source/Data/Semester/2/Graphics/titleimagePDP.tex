\documentclass[tikz]{standalone}
\usetikzlibrary{calc,positioning,decorations.text}
\usepackage{pgf}
\pgfmathsetseed{\number\pdfrandomseed}
\pgfmathdeclarerandomlist{color}{{red}{green}{blue}{purple}{yellow}}
\pgfmathrandomitem{\randcol}{color}

\begin{document}

\begin{tikzpicture}[scale=1, line join=bevel]

    % \a and \b are two macros defining characteristic
    % dimensions of the Penrose triangle.       
    \pgfmathsetmacro{\a}{1.8}
    \pgfmathsetmacro{\b}{0.7}
    
    \tikzset{%
      apply style/.code     = {\tikzset{#1}},
      triangle_edges/.style = {thick,draw=black}
    }
    
    \foreach \theta/\facestyle in {%
        0/{triangle_edges, fill = gray!50},
      120/{triangle_edges, fill = \randcol!25},
      240/{triangle_edges, fill = gray!90}%
    }{
      \begin{scope}[rotate=\theta]
        \draw[apply style/.expand once=\facestyle]
          ({-sqrt(3)/2*\a},{-0.5*\a})                     --
          ++  (-\b,0)                                       --
            ({0.5*\b},{\a+3*sqrt(3)/2*\b})                -- % higher point 
            ({sqrt(3)/2*\a+2.5*\b},{-.5*\a-sqrt(3)/2*\b}) -- % rightmost point
          ++({-.5*\b},-{sqrt(3)/2*\b})                    -- % lower point
            ({0.5*\b},{\a+sqrt(3)/2*\b})                  --
          cycle;
    
        \end{scope}
      } 
     \end{tikzpicture}

\end{document}