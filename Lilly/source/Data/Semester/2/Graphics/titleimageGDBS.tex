\documentclass[11pt]{standalone}
 
\usepackage{tikz}
\usepackage{pgfplots} 
\usepackage{xcolor}
\usepackage{tabularx}
\usepackage{listingsutf8}
\pgfplotsset{width=6.05cm,compat=newest}

\definecolor{Awesome}{RGB}{255,32,82}           %%#FF2052
\definecolor{Orchid}{RGB}{180,82,205}           %%#B452CD
\definecolor{AppleGreen}{RGB}{141,182,0}        %%#8DB600
\definecolor{Amber}{RGB}{255,191,0}             %%#FFBF00
\definecolor{AuroMetalSaurus}{RGB}{110,127,128} %%#6E7F80


\lstset{
    showstringspaces=true,  %Damit es keine Verwechslungen gibt
    basicstyle=\ttfamily,   %Damit es auch wirklich ausschaut wie die Schreibmaschine!
    numbers=left,           %Zeilennummern
    escapeinside={!*}{*)}, 
    frame=none,
    language=C,
    morekeywords = [2]{EBOARD_NANO,EBOARD_PWM_SPE,REPT_TASK,TCNT2,F_CPU},
    morekeywords = [3]{ISR, TIMER_2_OVF_vect},
    morekeywords = [4]{rept_task, trig_rept_task},
    morekeywords = [5]{_0pwmValue,_pwmValue,timer_count,timer_ofl},    
    numberstyle=\small\color{gray},
    xleftmargin=15pt,
    alsoletter={\#,\_},
    keywordsprefix={\#},
    numbersep=5pt,
    framexrightmargin=0pt,
    framexleftmargin=0pt,
    keywordstyle=\color{purple}\bfseries,       % keyword style
    keywordstyle = [2]\color{Awesome}, %%KW
    keywordstyle = [3]{\color{Orchid}\bfseries}, %% PARAM
    keywordstyle = [4]{\color{AppleGreen}}, %% FILES
    keywordstyle = [5]{\color{AuroMetalSaurus}}
    commentstyle=\color{gray},    % comment style
    stringstyle=\color{Amber},     % string literal style
    extendedchars=true,
}

\lstdefinelanguage{Cpp}{
    language=C++,
    stringstyle = \color{black},
    keywordstyle=\color{purple}\bfseries,       % keyword style
    morekeywords = [2]{EBOARD_NANO,EBOARD_PWM_SPE,REPT_TASK,TCNT2,F_CPU,PIN_MOTOR,_SPE,EBOARD},
    morekeywords = [3]{ISR, TIMER_2_OVF_vect},
    morekeywords = [4]{rept_task, trig_rept_task,defined, optVAL_t},
    morekeywords = [5]{_OpwmValue,_pwmValue,timer_count,timer_ofl}, 
    otherkeywords = {;,<<,>>,2>,<,|},
    alsoletter={-,|,~,{,},_,>,:},
    comment=[l]{\#},
    basicstyle=\ttfamily,   %Damit es auch wirklich ausschaut wie die Schreibmaschine!
    identifierstyle=\color{black},
    comment=[l]{//},
    sensitive=false,
    morekeywords = {},
    keywordstyle=\color{purple}\bfseries,       % keyword style
    keywordstyle = [2]\color{Awesome}, %%KW
    keywordstyle = [3]{\color{Orchid}\bfseries}, %% PARAM
    keywordstyle = [4]{\color{AppleGreen}}, %% FILES
    keywordstyle = [5]{\color{Amber}}, %% FILES
    morecomment=[s]{/*}{*/},
    commentstyle=\color{gray},
    stringstyle=\color{mint},
    morestring=[b]',
    %literate={@}{{\textcolor{mint}{\$}}}2
    showstringspaces=false,  %Damit es keine Verwechslungen gibt
} 
\usepackage{pgf}


\begin{document}
\begin{tikzpicture}
% \node[opacity = 0.4] at(0,0) {{\begin{lstlisting}[language=Java]
% /**
%  * @brief Konstruktor - Initialisiert alles notwendige
%  */
% public GameCore(){
%     this.state = GameState.PREPARING;
   
%     //Lade Bild - RGB-Farbformat
%     img = new BufferedImage( getConfig().getWidth(),
%                             getConfig().getHeight(), 
%                             BufferedImage.TYPE_INT_RGB); 
   
%     pixels = ((DataBufferInt) img.getRaster().getDataBuffer()).getData(); 
%     //pixels[] zeigt nun auf das Raster des Bildes
    
%     // Setze Fenstergröße
%     setPreferredSize(new Dimension(getConfig().getWidth() * getConfig().getScale(),
%                                    getConfig().getHeight() * getConfig().getScale())); 

%     //initialisiere Fenster
%     JFrame frame = new JFrame(getConfig().getName()); 
%     frame.setDefaultCloseOperation(JFrame.EXIT_ON_CLOSE);
%     frame.setLayout(new BorderLayout());
%     frame.add(this, BorderLayout.CENTER);
%     frame.pack();
%     frame.setResizable(false); frame.setLocationRelativeTo(null); frame.setVisible(true);

%     update_ctrl = new UpdateController(getConfig().getFPS()); //initialisiere update control

%     createBufferStrategy(3); //Buffer Strategie auf triple-Buffering
%     bufferStrategy = getBufferStrategy();
%     g = bufferStrategy.getDrawGraphics(); //Setze Grafik - hier momentan überflüssig

%     this.state = GameState.READY; //Setze Spiel auf bereit - Wird nicht überprüft, kann aber zur Sicherheit verwendet werden
% }
% \end{lstlisting}}};
%% ,draw 
\node[rectangle, rounded corners=14pt, thick,inner sep=25pt] at (0,0) {{\begin{lstlisting}[language=Cpp,numbers=none]
void trig_rept_task() {
  #if EBOARD_NANO == 0x0|
    | defined(DOC) if (_pwm
      Value!=_OpwmValue){ ana
        logWrite(PIN_MOTOR_SPE,
          _pwmValue);_OpwmValue =
            _pwmValue;}#endif #ifde
              f REPT_TASK  rept_task(
                );#endif} int timer_cou
                  nt = 0; bool timer_ofl=
                    false; ISR(TIMER2_OVF_v
                      ect) { timer_count++; i
                        f(timer_count >= EBOARD
                        _PWM_SPE*1000 && EBOARD
                      _PWM_SPE >? 0 && !timer
                    _ofl){ timer_ofl = true
                  ; timer_count -= EBOARD
                _PWM_SPE * 1000; trig_r
              ept_task(); timer_ofl =
            false; } TCNT2 =  256 -
          (int)((float)F_CPU * 0.
        001 / 64); } struct LCD            
      {  #if EBOARD_NANO == 0              d=0x3C);#endif bool changeID(optVAL_t new
    LCD(SoccerBoard &soccer               ID = 0x3C); bool clear(void); void print(co
  Board, optVAL_t id=0x3C                 nst char* data); void print(int data); void 
); #else LCD(optVAL_t i                    print(optVAL_t line, optVAL_t cols, const
\end{lstlisting}}};
%\draw[rounded corners=4pt, fill=white, fill opacity = 0.5] (-3,1) rectangle ++ (3.4,1.5);
%\draw[rounded corners=4pt, fill=white, fill opacity = 0.5] (-7,-2.9) rectangle ++ (3.4,1.5);
%\draw[rounded corners=4pt, fill=white, fill opacity = 0.5] (5.25,-5.2) rectangle ++ (3.4,1.5);
%\draw[rounded corners=14] (-11.07,-6.98) rectangle ++ (22.14,13.96);

\end{tikzpicture}
\end{document} 