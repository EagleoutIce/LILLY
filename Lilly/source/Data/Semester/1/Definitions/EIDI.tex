%%% Diese Datei enthält alle notwendigen Definitionen

\def\PROFESSOR{Prof. Dr. Dipl.-Ing. Thom Früwirth}

\def\UEBUNGSLEITER{}
\def\TUTOR{Andreas Decker}

\def\TITLE{Einführung in die Informatik}
\def\SUBTITLE{\PROFESSOR}


\def\UEBUNGSHEADER{\TITLE\\Übungsblatt \LILLY@n }
\def\FULLTITLE{ \TITLE \\\fontsize{18pt}{16pt}\selectfont{\SUBTITLE} }

\LILLYcommand{\VORLESUNG}{\eidi}

\DeclareRobustCommand{\POLITEINTRO}{
\TOP[disc]{Disclaimer}{Worte des Autors}
Offensichtlich erhebt dieses Dokument keinen Anspruch auf vollständige Richtigkeit. Es wurde auf Basis der zugrundeliegenden Vorlesungsmaterialien erstellt und ist eher als unverbindliche Hilfe zu verstehen. Hinzu kommt, dass die kapitelbasierte Einteilung der Vorlesungsvorlage dort durchbrochen wurde, wo eine andere Gruppierung angenehmer oder passender erschien. Bei Anregungen oder Verbesserungsvorschlägen einfach melden!\\
Alle Grafiken wurden vom Autor mithilfe von \LaTeX{} und \TikZ{} erstellt und basieren zum Teil auf denen des Vorlesungsmaterials.{}\hfill {\tiny Florian Sihler}\\\begin{center}
    \say{Viel Spaß beim Lernen!}
\end{center}
}

%% LAYOUT CONTROL
\providecommand\LILLYxBOXxSatzxLock{TRUE}
\providecommand\LILLYxBOXxDefinitionxLock{TRUE}

\providecommand\LILLYxColorxTITLExOffset{11.6cm}

\providecommand\LILLYxBOXxBemerkungxBox{FALSE}
\providecommand\LILLYxBOXxBeispielxBox{FALSE}
\providecommand\LILLYxBOXxBeweisxBox{FALSE}
\providecommand{\LILLYxUBxSHOWTUTOR}{TRUE}
\input{\LILLYxPATHxDATA/Semester/1/EIDI/HeapNStack}

%%NEW TITLE

\LILLYcommand{\LILLYxFACULTY}{\LILLYxFACULTYxPRAKTISCHEINFORMATIK}
\LILLYcommand{\LILLYxFACULTYxCOLOR}{FacultyPraktischeInformatikxColor}