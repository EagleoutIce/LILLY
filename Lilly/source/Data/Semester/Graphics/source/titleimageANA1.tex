\documentclass[11pt]{standalone}

\usepackage{tikz}
\usepackage{pgfplots}
\usepackage{xcolor}
\usepackage{tabularx}
\pgfplotsset{width=6.05cm,compat=newest}

\usepackage{pgf}
\pgfmathsetseed{\number\pdfrandomseed}
\pgfmathdeclarerandomlist{formula}{{$e^{i\pi}=-1$}{$a^2+b^2=c^2$}{$\sum\limits_{n=1}^k \frac{1}{n}$}{$\sum\limits_{n=1}^k \frac{(-1)^n}{2n+1}$}{$\sum\limits_{n=1}^k = \frac{n(n+1)}{2}$}{$f'(a) = \lim\limits_{h\rightarrow 0} \frac{f(a+h) - f(a)}{h}$}{$d = \sqrt{(x_2 - x_1)^2 + (y_2 + y_1)^2}$}{$\tan(2a) = \frac{2\tan(a)}{1 - \tan^2(a)}$}{$\frac{1}{\sqrt{2}} = \frac{\sqrt{2}}{2}$}{$\frac{\delta x}{\delta y} = \lim\limits_{i \rightarrow 0} \frac{\delta x \cdot i}{\delta y \cdot 7 i}$}{$\lim\limits_{n \rightarrow \infty} \frac{1}{n} = 0$}{$(a^p)^q = a^{pq}$}{$\sqrt[n]{a^n} = a = \left(a^{\frac{1}{n}}\right)^n$}{$|a_n - a| \leq \frac{\epsilon}{2}$}{$\left|\frac{1}{b} - \frac{1}{a}\right| = \frac{|b-a|}{|b||a|}$}{$exp(x) = \sum\limits_{n=0}^\infty \frac{x^n}{n!}$}{$c=5$}{$the-answer=42$}}
\begin{document}
\centering
\begin{tikzpicture}
% mit f(x) = 4e^(-0.5x^2)
% Und [-3.5,3.5,0.5] ergibt sich:
\draw[-latex, opacity = 0.8, very thick, scale=2] (0.75,-0.5) - ++(0,5);
\draw[-latex, opacity = 0.8, very thick, scale=2] (-4.25+0.75,0) -- ++(8,0);
\foreach \ri/\rl/\osi in {13/420/1,0.95/120/2}{
    \pgfmathsetmacro\dumbumbunnyhum{rand};
\foreach \i in {1,...,\rl} {
    \pgfmathsetmacro\rx{abs(rand)}     \pgfmathsetmacro\rs{abs(rand)}
    \pgfmathsetmacro\ry{abs(rand)}     \pgfmathsetmacro\rsign{sign(rand)};
    \pgfmathsetmacro\ro{abs(rand)}     \pgfmathsetmacro\rssign{sign(rand)};
    \pgfmathrandomitem{\form}{formula} \pgfmathsetmacro\offset{abs(rand)};
    \draw[opacity=\ro/\ri,color=black] (12*\rx-5.5 + \rssign\offset/12,6*\ry+ \rsign\offset/12) node[scale=\rs\osi, transform shape] {\textsf{\form}};
}
}

\foreach \y/\c [count=\i, evaluate=\i as \x using {(\i-7)/2}] in {0.01/0.025,0.04/0.03,0.18/0.04,0.54/0.06,1.3/0.10,2.43/0.18,3.53/0.34,4/0.50,3.53/0.34, 2.43/0.18, 1.3/0.10, 0.54/0.06, 0.18/0.04, 0.04/0.03, 0.01/0.025} {
    \draw[scale=2,fill=black, fill opacity = \c] (\x,\y) rectangle ++ (0.5,-\y);
}
\draw[domain=-3.5:3.5, smooth, variable=\x, opacity = 0.8, very thick, scale=2] plot[samples=200] (\x+0.75, {4*e^(-0.5*\x*\x)});
\end{tikzpicture}
\par
\end{document}