\providecommand{\T}[1]{{\text{\LILLYxlstTypeWriter #1}}}%Shortcut Schreibmaschine
\providecommand\narrowitems{\setlength{\itemsep}{0pt}}
\providecommand\closeritems{\setlength{\itemsep}{-4pt}}

\providecommand\tab[1][1cm]{\hspace*{#1}} %find ich logischer so....
\providecommand\lreqn[2]{\noindent\makebox[\textwidth]{\(\displaystyle\)#1\hfill#2}} %für coole Beweisboxen
\providecommand\q[1][0]{\glq#1\grq{}} %''
\providecommand\qq[1][0]{\glqq#1\grqq{}} %""
\providecommand\colvec[1]{\begin{pmatrix}#1\end{pmatrix}} %Vektoren für Mathe
\providecommand\minicolvec[1]{\left(\begin{smallmatrix}#1\end{smallmatrix}\right)} %Vektoren für Mathe

\providecommand{\qedsymbol}{\hbox{}\hfill{}\textcolor{Charcoal}{$\blacksquare$}}  %Beweisbox
\providecommand{\say}[1]{\glqq{#1}\grqq}

\providecommand{\engl}[2][]{#1(engl. \textit{#2})#1}

\providecommand\cd{\ensuremath{\cdot}}

\providecommand{\rom}[1]{\text{\MakeUppercase{\romannumeral #1}}} %Zahlen Römisch (GDRA)

\def\fquad{\hskip1em plus 0.25em minus 0.25em\relax}
\def\fqquad{\hskip2em plus 0.35em minus 0.35em\relax}

%TIKZSYMBOLS
\providecommand\ring[1]{\fill #1 circle [radius=0.7pt];} %%Schortcuts für Schaltkreise, weil es mir so besser gefällt ^^ und ich das Scaling erst nachher angepasst habe ^^
\providecommand\bigRing[1]{\fill #1 circle [radius=1.4pt];}
\providecommand\bigCRing[2][limegreen]{\fill[fill=#1] #2 circle [radius=1.4pt];}

\providecommand\ringC[2][limegreen]{\fill[fill=#1] #2 circle [radius=0.7pt];}



\LILLYcommand{\LILLYcoloredSQ}[1]{\tikz{\draw[fill=#1!75!white,#1!75!white] (-0.05,-0.05) rectangle ++ (0.4,0.4);\draw[fill=#1!50!white,#1!50!white] (0,0) rectangle ++ (0.35,0.35);\draw[fill=#1,#1] (0,0) rectangle ++ (0.3,0.3);}}


\def\firstcircle{(90:0.5cm) circle (0.75cm)} %%Shortcuts für FG, auch danach gerne genutzt
\def\secondcircle{(330:0.5cm) circle (0.75cm)}
\def\thirdcircle{(210:0.5cm) circle (0.75cm)}
\def\bigcircle{(0:0cm) circle (1.5cm)}

\newcommand{\dispnote}[2][]{\parbox[t]{5em}{#1\small#2}}

\newcommand{\note}[2][]{#2\ifthenelse{\equal{#1}{}}{\marginpar[{\dispnote[\raggedleft]{#2}~$\RHD$}]{$\LHD$~\dispnote{#2}}}{\marginpar[{\dispnote[\raggedleft]{#1}~$\RHD$}]{$\LHD$~\dispnote{#1}}}}

%%Smallnote
\newcommand{\snote}[2][1]{\note[\scriptsize #1]{#2}}


\LILLYcommand{\LILLYxCOLORxRainbow}{% who needs foreach
    \LILLYcoloredSQ{\LILLYxColorxDefinition}
    \LILLYcoloredSQ{\LILLYxColorxSatz}
    \LILLYcoloredSQ{\LILLYxColorxBeweis}
    \LILLYcoloredSQ{\LILLYxColorxLemma}
    \LILLYcoloredSQ{\LILLYxColorxBemerkung}
    \LILLYcoloredSQ{\LILLYxColorxZusammenfassung}
    \LILLYcoloredSQ{\LILLYxColorxBeispiel}
    \LILLYcoloredSQ{\LILLYxColorxUebungsaufgabe}
    \LILLYcoloredSQ{\LILLYxColorxZusatzuebung}
    \LILLYcoloredSQ{\LILLYxLINKSxMainColor}
    \LILLYcoloredSQ{\LILLYxLINKSxMainColorDarker}
    \LILLYcoloredSQ{\LILLYxLINKSxCiteColor}
    \LILLYcoloredSQ{\LILLYxLINKSxUrlColor}
}
