%% Das normale Source-Code Paket für die Latex-Mitschriebe
\ifx\LILLYxMODE\LILLYxMODExPRINT
    \providecommand{\LILLYxCODExKEYWORDSTYLE}{\color{DebianRed}\bfseries}
\else
    \providecommand{\LILLYxCODExKEYWORDSTYLE}{\color{DebianRed}}
\fi
%\else
\providecommand{\LILLYxlstTypeWriter}{\fontfamily{AnonymousPro}\selectfont}
%\fi

\lstdefinelanguage{Assembler}{
        keywords={while,if,r,ld,st,sr,sl,beq,bnq,add,sub,and,or,not,xor,dec,inc,jmp,addi,sw,addui,add,sw,lw,slti,j,jal,div,mul,hi,lo, MOV},
        ndkeywords={nop,X,acc},
        ndkeywordstyle=\color{tealblue!80!black}\bfseries,
        comment=[l]{//},
        comment=[l]{\#},
        morecomment=[s]{/*}{*/},
        morestring=[b]',
        sensitive=false,
        %literate={@}{{\textcolor{mint}{\$}}}2
        showstringspaces=true,  %Damit es keine Verwechslungen gibt
}

\lstnewenvironment{assembler}[1][]
  {\lstset{language=Assembler,#1}}
{}



%%%%% Pseudo-Sprachen
    \lstdefinelanguage{lPseudo}{
    keywords={INPUT,REPEAT,ELSE,UNTIL,OR,END,FOR,IF,END,TO,DO,THEN,TRUE,FALSE,END, print,println,goto,system,every,foreach,in},
    keywordstyle=\LILLYxCODExKEYWORDSTYLE,
    basicstyle=\LILLYxlstTypeWriter\color{Charcoal},   %Damit es auch wirklich ausschaut wie die Schreibmaschine!
    %identifierstyle=\color{Azure},
    comment=[l]{//},
    sensitive=false,
    morecomment=[s]{/*}{*/},
    commentstyle=\color{AppleGreen},
    stringstyle=\color{mint},
    morestring=[b]",
    %literate={@}{{\textcolor{mint}{\$}}}2
    showstringspaces=false,  %Damit es keine Verwechslungen gibt
}

\lstnewenvironment{pseudo}[1][]
  {\lstset{language=lPseudo,#1}}
{}



\pushList{RegisteredLanguages}{pseudo/lPseudo}

%%%%% GDRA
    \RequirePackage{caption}

\ExplSyntaxOn
\DeclareCaptionFont{black}{ \color{black} }
\DeclareCaptionFormat{none}{
    \makebox[\textwidth]{#1#2#3}
}
\DeclareCaptionFont{black}{ \color{black} }

\ExplSyntaxOff
\DeclareCaptionFormat{listing}{
    \makebox[\textwidth]{#1#2#3}
}
\newcommand{\gitRAW}{\text{\href{https://www.github.com/EagleoutIce/MIPS_UniUlm_Examples/}{\faGithub}}}
\newcommand{\git}{\hfill \gitRAW}

\def\noprint#1{}

\ExplSyntaxOn
\NewDocumentCommand \mipsspecifier { }
{
    \tl_set:No \l_demo_tl {\the\use:c{lst@token}}
    \regex_replace_all:nnN { .([a-zA-Z]+[0-9]+) } {\$\c{textcolor}\cB\{ limegreen \cE\}\cB{ \1 \cE} } \l_demo_tl
    \tl_use:N \l_demo_tl
    \noprint
}

\ExplSyntaxOff



\lstdefinelanguage{lMips}{
    language = Java,%Dirtymörty
    %keywordstyle = [6]{\toMips},
    otherkeywords = {;,<<,>>},
    keywordsprefix={\$},
    alsoletter={.:},
    comment=[l]{\#},
    extendedchars=true,
    escapeinside={*}{*},
    showstringspaces=true,  %Damit es keine Verwechslungen gibt
    morekeywords=[2]{while,if,r,ld,st,sr,sl,beq,bnq,add,sub,and,or,not,xor,dec,inc,jmp,addi,sw,addui,add,sw,lw,slti,j,jal,div,mul,hi,lo,jr,addiu,
    la, lb, li, bne, mult, mflo},
    morekeywords = [3]{.text,.data,.ascii,.word},
    morekeywords = [4]{nop,limegreen,syscall},
    morekeywords = [5]{ERROR,WARNING},
    morekeywords = [6]{.asciiz, .space},
    literate={HELP}{{\LILLYxPATH}}1 {á}{{\'a}}1 {é}{{\'e}}1 {í}{{\'i}}1 {ó}{{\'o}}1 {ú}{{\'u}}1
    {Á}{{\'A}}1 {É}{{\'E}}1 {Í}{{\'I}}1 {Ó}{{\'O}}1 {Ú}{{\'U}}1
    {à}{{\`a}}1 {è}{{\`e}}1 {ì}{{\`i}}1 {ò}{{\`o}}1 {ù}{{\`u}}1
    {À}{{\`A}}1 {È}{{\'E}}1 {Ì}{{\`I}}1 {Ò}{{\`O}}1 {Ù}{{\`U}}1
    {ä}{{\"a}}1 {ë}{{\"e}}1 {ï}{{\"i}}1 {ö}{{\"o}}1 {ü}{{\"u}}1
    {Ä}{{\"A}}1 {Ë}{{\"E}}1 {Ï}{{\"I}}1 {Ö}{{\"O}}1 {Ü}{{\"U}}1
    {â}{{\^a}}1 {ê}{{\^e}}1 {î}{{\^i}}1 {ô}{{\^o}}1 {û}{{\^u}}1
    {Â}{{\^A}}1 {Ê}{{\^E}}1 {Î}{{\^I}}1 {Ô}{{\^O}}1 {Û}{{\^U}}1
    {œ}{{\oe}}1 {Œ}{{\OE}}1 {æ}{{\ae}}1 {Æ}{{\AE}}1 {ß}{{\ss}}1
    {ű}{{\H{u}}}1 {Ű}{{\H{U}}}1 {ő}{{\H{o}}}1 {Ő}{{\H{O}}}1
    {ç}{{\c c}}1 {Ç}{{\c C}}1 {ø}{{\o}}1 {å}{{\r a}}1 {Å}{{\r A}}1
    {€}{{\euro}}1 {£}{{\pounds}}1 {«}{{\guillemotleft}}1
    {»}{{\guillemotright}}1 {ñ}{{\~n}}1 {Ñ}{{\~N}}1 {¿}{{?`}}1 {li}{{  ~\mipsLI}}1
    %morekeywords = [6]{hallomama},
}

\lstnewenvironment{mips}[1][]
  {\lstset{language=lMips,#1}}
{}



\pushList{RegisteredLanguages}{mips/lMips}


%%%%% Dokumentation - needfull shell
    %% Wird f�r die Dokumentation entsprechend erweitert

\lstdefinelanguage{lBash}{
    language=bash,
    breaklines=true,
    otherkeywords = {<<,>>,2>},
    alsoletter={-,|,~,{,},_,.,>,<},
    comment=[l]{\#},
    keywordsprefix={\$},
    escapeinside={*}{*},
    showstringspaces=true,  %Damit es keine Verwechslungen gibt
    %% numbers=none,           %Zeilennummern
    morekeywords = {bash},
    morekeywords = [2]{mkdir,texhash,make,apt,>,<,cd,java,git},
    morekeywords = [3]{-p, -dir, print, install,-jar,-lilly-path},
    morekeywords = [4]{lilly_compile.sh,sudo,lilly_jake,jake,ghci,ghc,grep,ls,cp,find,man,tail,head},
    morekeywords = [5]{GUI},
}

\lstnewenvironment{bash}[1][]
  {\lstset{language=lBash,#1}}
{}

\DeclareRobustCommand{\cBash}[1]{\LILLYxwriteLst[language=lBash]{#1}}
    %% Wird für die Dokumentation entsprechend erweitert

\lstdefinelanguage{lLatex}{
    language=[LaTeX]TeX,
    keywordstyle = [7]{\color{Azure}}, %%
    keywordstyle = [4]{\color{AppleGreen}},
    morecomment = [l]{\%},
    %morecomment=[s][\color{bondiBlue}]{\$}{\$}, %% see :bmath: and :emath:
    commentstyle={\color{gray}\LILLYxlstTypeWriter},
    alsoletter={\\@_*},
    alsoother={!,"},
    %deletekeywords=[5]{\\},
    columns=[c]flexible,
    escapeinside={!*}{*!},
    morekeywords = [2]{Lilly,Jake,book,lipsum,geometry,LILLYxMATH,LILLYxGRAPHICS,LILLYxCOLOR,LILLYxBOXES,LISTxCompatColors,LILLYxCONTROLLERxBOX,LILLYxLISTINGS,LILLYxLISTINGSxADVANCED,LILLYxLIST,LILLYxFORMATxCONTROL,LISTxColors,\\LISTxCompatColors,\\LISTxColors,\\def,\\gdef,\\@nil,\\\\,\\DeclareDocumentCommand},
    morekeywords = [3]{\\LILLYxTITLE,\\LILLYcommand,\\LILLYxColorxDefinition,\\LILLYxVorlesung,\\LILLYxSemester,\\LILLYxFACULTYxMATHE,\\LILLYxFACULTYxTHEORETISCHEINFORMATIK,\\LILLYxFACULTYxPRAKTISCHEINFORMATIK,\\addcontentsline,\\LILLYxFACULTYxTECHNISCHEINFORMATIK,\\LILLYxPATHxDATA,\\LILLYxPATHxLISTINGS},
    morekeywords = [5]{\\documentclass,documentclass,\\usepackage,\\begin,\\end,\\input,\\caption,\\true,\\FALSE,\\elable,\\examplecube,\\rotateRPY,\\node,\\STATE,\\state,\\jmark,\\hmark,\\listofDEFS,\\clearpage,\\newpage,\\linput,\\include,\\linclude,\\inputUB,\\draw,\\plotline,\\plotseq,\\graphPOI,\\loopTop,\\loopRight,\\loopBot,\\loopLeft,\\tcbset,\\DeclareTColorBox,\\textbf,\\textsc,\\specialrule,\\hfill,\\textwidth,\\VRule,\\tcboxedtitleheight,\\ifthenelse,\\equal,\\bfseries,\\path,\\definecolor,\\toprule,\\hspace,\\midrule,\\bottomrule,\\linewidth,\\arraybackslash,\\centering,\\foreach,\\renewcommand,\\getGraphics,getGraphics,\\framebox,\\vSplitter,\\item,\\xmark,\\ymark,\\ldots,\\psi,\\xi,\\chi,\\listofDEFINITIONS,\\LILLYxMathxMode,\\constructList,\\pushList,\\popList,\\lillyxlist,\\clatex,\\igepard,\\newline,\\containsList,\\left,\\right,\\addplot3,\\PgetXY,\\PgetY,\\PgetX,\\neuronSquare,\\registerColors,\\Acronym,\\PoliteWords,\\ColorfulWords,\\Acronym,\\@Acronym,\\lilly@format@iter,\\LILLYxNOTExLibrary,\\lstdefinelanguage,\\lstset,\\lstnewenvironment,%
    \\LILLY@FORMATTER@CURRENT,\\TextBfFormat,\\newcommand,\\blankcmd,\\textcolor,\\LILLYxlstTypeWriter,\\textbackslash,\\LILLYxlstTypeWriter\\textbackslash,\\LILLYxLIBRARY,\\baselineskip,\\paragraph,\\ilatex,\\ignorespaces,\\NoFormatChar,\\LILLYxTITLExBONUS,\\@Alph,\\def,\\edef,\\xdef,\\gdef,\\typesetList,\\expandafter,\\isLanguageLoaded,\\cpython,\\bpython,\\ccpp,\\bcpp,\\verb,\\section,\\chapter,\\subsection,\\subsubsection,\\smallskip,\\medskip,\\bigskip,\\smallskip\\newline,\\medskip\\newline,\\bigskip\\newline,\\LILLYxListingsxLang,\\LILLYxListingsxPACK,\\cjava,\\bjava,\\pjava,\\getGraphics,\\lstkwone,\\lstkwtwo,\\lstkwthree,\\lstkwfour,\\lstkwfive,\\lstkwsix,\\lststring,\\lstcomment,\\lstnumber,%
    \\getGraphicsPath,\\arabic,\\CTRxDEF,\\isLanguageNameLoaded,\\lstshowcmd,\\getPrerendered,\\LILLYxBOXxMODE,\\LILLYxBOXxDefinitionxEnable,\\pcpp,\\LILLYxBOXxBeweisxBox,\\NewEmblem,\\infoEmblem,\\NewInfoBox,\\dateBox,%
    \\warningEmblem,\\errorEmblem,\\mathEmblem,\\codeEmblem,\\textEmblem,\\btextEmblem,\\DefaultBaseEmblem,\\mathbf,\\lipsum,\\RegisterLanguage,\\hypertarget,\\T,\\pgfkeysvalueof,\\noexpand,\\leavevmode,\\csname,\\endcsname},
    morekeywords = [4]{definition,bemerkung,definition*,beispiel,satz,beweis,lemma,zusammenfassung,aufgabe,uebungsblatt,uebungsblatt*,mrk,pmatrix,tabular,tikzpicture,graph,wgraph,plot,aufgaben,slatex,grid,scope,minipage,Automat,LILLYxBOXxDefinition,LILLYxBOXxAufgabe,enumeratea,document,latex,axis,align*,align,alignat,alignat*,lstlisting,plainjava,plainlatex,sjava,tikzternal,center,lstplain,cpp,java,lstnonum,multicols,ditemize,presentlst,\\wasysymLightning,\\faCode,\\faCalendar,infoBox,warningBox,errorBox,mathBox,codeBox,infoBox*,warningBox*,errorBox*,mathBox*,codeBox*,\\isRuntimeLoaded},
    morekeywords = [6]{Dokumentation,Mitschrieb,Uebungsblatt,Zusammenfassung,samples,opacity,\\storeLillyColorList,\\storeLillyCompatColorList,article,lJava,lLatex,lCpp},
    morekeywords = [7]{\#1,\#2,\#3,\#4,\#5,\#6,\#7,\#8,\#9,Typ,Vorlesung},
}

\RegisterLanguage{latex}{lLatex}

% \lstnewenvironment{latex}[1][]
%   {\lstset{language=lLatex,#1}}
% {}


% %% \DeclareRobustCommand{\clatex}[1]{\LILLYxwriteLst[language=lLatex]{#1}}

% \pushList{RegisteredLanguages}{latex/lLatex}

    %% Wird f�r die Dokumentation entsprechend erweitert

\lstdefinelanguage{lGepard}{
    comment=[l]{\%},
    alsoletter={\#,=,-,[,]},
    morecomment=[s]{!}{!},
    morestring=[b]",
    escapeinside={!*}{*!},
    showstringspaces=false,
    otherkeywords={++\\},
    morekeywords = [1]{@[SELTEXF],@[AUTONUM]},
    morekeywords = [2]{Lilly,Jake},
    morekeywords = [3]{operation,file,debug,lilly-show-boxname,lilly-nameprefix,lilly-modes,lilly-show-boxname,lilly-author,lilly-n,lilly-semester,lilly-vorlesung},
    morekeywords = [4]{true,false,file_compile},
    morekeywords = [5]{=},
    morekeywords = [6]{uebungsblatt}
}

\lstnewenvironment{gepard}[1][]
  {\lstset{language=lGepard,#1}}
{}


\pushList{RegisteredLanguages}{gepard/lGepard}


%% Wird für die Dokumentation entsprechend erweitert

\lstdefinelanguage{lJava}{
    language=java,
    deletekeywords = {<,>,\,},
    otherkeywords = {->},
    alsoletter={@_},
    comment=[l]{//},
    keywordsprefix={@},
    moredelim=**[is][{\lstHLWarning}]{|warn|}{|warn|},
    moredelim=**[is][{\lstHLError}]{|err|}{|err|},
    moredelim=**[is][{\lstHLInfo}]{|info|}{|info|},
    %frameround=tttt, %% mag backgroundcolor nicht
    escapeinside={!*}{*!},
    morecomment=[s]{/*}{*/},
    morekeywords = {System,var,Integer,String}, %% Damit ich nichts mehr machen muss:
    morekeywords = [2]{Stream,List,Map,Queue,Deque,ArrayList,SingleLinkedList,SortedList,SortedSet,Set,HashMap,Tree,HashSet,FileReader,BufferedReader,InputStream,BufferedInputStream,BufferedOutputStream,OutputStream,FileInputStream,FileOutputStream,FileWriter,RandomAccessFile,ObjectOutputStream,ObjectInputStream,File,LocalDate,Node,NodeList,DocumentBuilder,Document,DocumentBuilderFactory,XPathExpression,XPath,XPathFactory,NodeList,Transformer,DOMSource,%
    TransformerFactory,StreamResult,StreamSource,SAXParser,SAXParserFactory,XMLReader,DriverManager,Connection,Statement,ResultSet,PreparedStatement,Text,Font,Application,Pane,Leaf,Component,Composite,Leaf,Runtime,ExecutorService,Math,Executors,LinkedList,Callable,ExecutionException,InterruptedException,TimeUnit,FlowPane,Button,StackPane,Future,Double,Thread,Predicate,Function,Color,Scene,Pattern,Matcher,Event,EventHandler,EventListener,EventType,Observer,Observable,ObservableList,ObservableMap,ObservableArray,Button,TextArea,Pos,%
    Scanner,Calendar,Date,Element,Circle,Rectangle,GridPane,FileChooser,ScrollPane,ExtensionFilter,Worker,MouseEvent,Task,Service,%
    Background,BackgroundFill,CornerRadii,Insets,AtomicInteger,Random,EntityManager,EntityManagerFactory,Query,Persistence,Attributes},
    morekeywords = [3]{contains,toUpperCase,getResource,addAll,getResourceAsStream,setStyle,setPrefSize,relocate,println,push,pop,sort,get,iterator,spliterator,add,matches,Serializable,Cloneable,Iterable,newInstance,parse,newDocumentBuilder,newTransformer,newXPath,compile,evaluate,transform,setTextContent,getTextContent,setAttribute,newSAXParser,getXMLReader,getConnection,setScene,show,createStatement,prepareStatement,executeQuery,execute,executeUpdate,getInt,getString,format,next,close,printStackTrace,isClosed,nextString,nextLine,wasNull,commit,getExensionFilter,getExtensionFilters,showOpenDialog,showSaveDialog,rollback,getMetaData,setContentHandler,item,getChildren,setFont,setX,setY,setText,select,from,where,eq,fetch,equals,availableProcessors,getRuntime,nextDouble,shutdown,invokeAll,awaitTermination,writeObject,setAlignment,readObject,length,group,find,write,getElementsByTagName,getLength,getInstance,getValue,setMaxSize,%
    setOnKeyPressed,setOnMousePressed,setOnSwipeUp,setChanged,intValue,notifyObservers, setOnInputMethodTextChanged,addObserver,setTitle,setPrefColumnCount,add,appendText,setPrefRowCount,setAlignment,setVgap,setHgap,setEditable,matcher,%
    setOnMouseMoved,setOnMouseEntered,setOnMouseExited,getX,getY,setBackground,notify,notifyAll,wait,getTransaction,begin,persist,getResultList,createQuery,size,createEntityManagerFactory,start,newFixedThreadPool,createEntityManager},
    morekeywords = [4]{Files,Paths,Collectors,Collection,Iterable,Iterator,Object,Stage,EOFException,ClassNotFoundException,IOException,XPathConstants,SQLException,Exception,Runnable,IllegalStateException,GenerationType},
    morekeywords = [5]{filter,map,sorted,collect,toList,toSet,toMap,forEach,mapToInt,stream,sum,read,reduce,NODESET,MULTILINE,MAX_VALUE,CENTER_LEFT,CENTER_RIGHT,CENTER,TOP_LEFT,TOP_CENTER,TOP_RIGHT,BLUE,RED,GREEN,YELLOW,EMPTY,SECONDS,MAX_VALUE,IDENTITY},
    morekeywords = [6]{out,in,OBJECT},
}


\lstnewenvironment{java}[1][]
  {\lstset{language=lJava,#1}}
{}



\pushList{RegisteredLanguages}{java/lJava}

%% Wird für die Dokumentation entsprechend erweitert

\lstdefinelanguage{lXml}{
    % language=xml,
    %% identifierstyle=\color{AppleGreen},
    morestring=[b]",
    morestring=[b]',
    %morestring=[s]{>}{<},
    morestring=[s]{"}{"},
    morecomment=[s]{<?}{?>},
    morecomment=[s]{!--}{--},
    moredelim=**[is][{\lstHLWarning}]{|warn|}{|warn|},
    moredelim=**[is][{\lstHLError}]{|err|}{|err|},
    moredelim=**[il][{\lstHLError}]{|err|:},
    moredelim=**[is][{\lstHLInfo}]{|info|}{|info|},
    moredelim=[is][{}]{|plain|}{|plain|},
    alsoletter={:\#!-},
    otherkeywords={<?,?>,<,>,/},
    %frameround=tttt, %% mag background color nicht
    escapeinside={!*}{*!},
    showstringspaces=true,  %Damit es keine Verwechslungen gibt
    morekeywords = {!DOCTYPE,DOCTYPE,!ELEMENT,ELEMENT,!ATTLIST,ATTLIST,/},
    morekeywords = [2]{xml},
    morekeywords = [3]{SYSTEM,PUBLIC,EMPTY,\#PCDATA,\#IMPLIED,\#REQUIRED,\#FIXED,ANY},
    morekeywords = [4]{CDATA,ID,ENTITY,IDREF,IDREFS,NOTATION},
    morekeywords = [5]{xs:element,xs:list,xs:namespace,xs:attribute,xs:simpleType,xs:minInclusive,xs:maxInclusive,xs:restriction,xs:enumeration,xs:union,xs:sequence,xs:complexType,xmlns,xmlns:xs,xmlns:xsi,xsi:schemaLocation,xs:schema,xs:integer,last()},% + xpath
    morekeywords = [6]{name,maxOccurs,minOccurs,version,encoding,targetNamespace,ref,type,memberTypes,itemType,base,value,default,transaction-type,version},
}


\lstnewenvironment{xml}[1][]
  {\lstset{language=lXml,#1}}
{}



\pushList{RegisteredLanguages}{xml/lXml}

%% Wird für die Dokumentation entsprechend erweitert

\lstdefinelanguage{lSql}{
    language=sql,
    comment=[l]{--},
    %alsoletter={\#=-},
    %morecomment=[s]{!}{!},
    escapeinside={!*}{*!},
    moredelim=**[is][{\lstHLWarning}]{|warn|}{|warn|},
    moredelim=**[is][{\lstHLError}]{|err|}{|err|},
    moredelim=**[il][{\lstHLError}]{|err|:},
    moredelim=**[is][{\lstHLInfo}]{|info|}{|info|},
    moredelim=[is][{}]{|plain|}{|plain|},
    showstringspaces=true,  %Damit es keine Verwechslungen gibt
    morekeywords = {WITH,IS,REFERENCES,DEFERRED,OPTION,TO,START,WORK,SHOW,DATABASES,DATABASE,USE,LENGTH,LOCATE,TYPE,UNDER,OF,CURRENT,SEQUENCE,NEXTVAL,PREVVAL,PROCEDURE,FUNCTION,DETERMINISTIC,SQL,CALL,FOR,DO,EACH,STATEMENT,BEFORE,AFTER,INSTEAD OF,ROw,REFERENCING,OLD,NEW,ATOMIC},
    morekeywords = [2]{DOUBLE,REAL,BOOLEAN,USE,LONG},
    morekeywords = [3]{mysql>,ROUND},
    morekeywords = [4]{},
    morekeywords = [5]{},
    morekeywords = [6]{AUTO_INCREMENT},
    morecomment=[s]{/*}{*/},
    %keywordstyle={\color{DebianRed}},
}

\RegisterLanguage{sql}{lSql}

% \lstnewenvironment{sql}[1][]
%   {\lstset{language=lSql,#1}}
% {}

% \pushList{RegisteredLanguages}{sql/lSql}

%% Wird für die Dokumentation entsprechend erweitert

\lstdefinelanguage{lXsl}{
    % language=xslt,
    %% identifierstyle=\color{AppleGreen},
    %keywordsprefix={"},
    morestring=[b]",
    morestring=[b]',
    %morestring=[s]{>}{<},
    morestring=[s]{"}{"},
    morecomment=[s]{<?}{?>},
    morecomment=[s]{!--}{--},
    moredelim=**[is][{\lstHLWarning}]{|warn|}{|warn|},
    moredelim=**[is][{\lstHLError}]{|err|}{|err|},
    moredelim=**[is][{\lstHLInfo}]{|info|}{|info|},
    alsoletter={:/-},
    otherkeywords={/>,<,>},
    morekeywords = {},
    morekeywords = [2]{xml},
    morekeywords = [3]{SYSTEM,PUBLIC,EMPTY,PCDATA,CDATA},
    morekeywords = [4]{},
    morekeywords = [5]{collection,xsl:stylesheet,/xsl:stylesheet,xmlns:xsl,xmlns:xs,xmlns:fn,xsl:output,xsl:template,xsl:copy,/xsl:copy,/xsl:template,xsl:apply-templates,xsl:attribute,/xsl:attribute},% + xpath
    morekeywords = [6]{name,version,encoding,targetNamespace,ref,type,memberTypes,itemType,base,value,default,method,indent,match,select},
}


\lstnewenvironment{xsl}[1][]
  {\lstset{language=lXsl,#1}}
{}



\pushList{RegisteredLanguages}{xsl/lXsl}


\lstdefinelanguage{lChr}{
    alsoletter={-|~{,}_.><=:\*'?\\},
    comment=[l]{\%},
    keywordsprefix={\$},
    morecomment=[s]{/*}{*/},
    escapeinside={!!}{!!},
    morekeywords = {==>,@,<=>,=:=,\\,|,->,<-,>=,=<,builtin,call,:-,is,mod,=,leq}, %% expanded by prolog
    morekeywords = [2]{},
    morekeywords = [3]{},
    morekeywords = [4]{::=,.,chr_constraint,true,false,true.,false.},
    morekeywords = [5]{write,writeq,read},
}
\RegisterLanguage{chr}{lChr}
\lstdefinelanguage{lProlog}{
language=prolog,
alsoletter={-|~_.><=:\*'?},
moredelim=**[is][{\lstHLWarning}]{|warn|}{|warn|},
moredelim=**[is][{\lstHLError}]{|err|}{|err|},
moredelim=**[il][{\lstHLError}]{|err|:},
moredelim=**[is][{\lstHLInfo}]{|info|}{|info|},
moredelim=[is][{}]{|plain|}{|plain|},
comment=[l]{\%},
keywordsprefix={\$},
morecomment=[s]{/*}{*/},
escapeinside={!!}{!!},
deletekeywords= {true,false,consult,write,read,tell,see,told,seen},
otherkeywords = {false,true,creep,!,\\+},
morekeywords = {:-,is,true,false,creep},
morekeywords = [2]{->,=>,<-,>=,=:=,==,=\\=,=<,<,>,=},
morekeywords = [3]{True,False},
morekeywords = [4]{::=,read,get,get0,see,seen,write,writeq,tab,nl,put,tell,hold,fail,fail.,repeat,abort,assert,asserta,assertz,retract,abolish,clause,save,restore,listing,trace,notrace,spy P, nospy P,nospyall,debug,nodebug,atom,atomic,number,integer,float,var,nonvar,halt.,trace.,.,set_prolog_flag,consult,make.,edit.,help,apropos,nodebug.,write,writeln,writeq,read,tell,see,told,seen,use_module,chr_constraint,builtin,call,leq,mod,predicate_property},
morekeywords = [5]{?-,?,swipl,end_of_file,built_in},
morekeywords = [6]{Exit:,Call:,Redo:,Fail:,occurs_check,library},
}
\RegisterLanguage{prolog}{lProlog}
\lstdefinelanguage{lHaskell}{
    language=haskell,
    alsoletter={-|~{,}_.><=:\*'},
    moredelim=**[is][{\lstHLWarning}]{|warn|}{|warn|},
    moredelim=**[is][{\lstHLError}]{|err|}{|err|},
    moredelim=**[is][{\lstHLInfo}]{|info|}{|info|},
    comment=[l]{--},
    deletekeywords={Maybe,Either,Maybe,Either,Nothing,Just,Left,Right,\$},
    %morestring=[b]',
    %keywordsprefix={\$},
    morecomment=[s]{\{-}{-\}},
    escapeinside={!!}{!!},
    morekeywords = {:l,:cd,:q,:r,:t,String,Char,Bool,Show},
    morekeywords = [2]{<,>,->,=>,<-,>=},
    morekeywords = [3]{True,False,Maybe,Either,Nothing,Just,Left,Right},
    morekeywords = [4]{::,\$},
    morekeywords = [5]{Prelude>,\*Main>},
}

\lstnewenvironment{haskell}[1][]
  {\lstset{language=lHaskell,#1}}
{}



\pushList{RegisteredLanguages}{haskell/lHaskell}

\lstdefinelanguage{lCpp}{
    language=C++,
    otherkeywords = {;,<<,>>,2>,<,>,|,std::chrono::,std::},
    alsoletter={-|~{,}_},
    moredelim=**[is][{\lstHLWarning}]{|warn|}{|warn|},
    moredelim=**[is][{\lstHLError}]{|err|}{|err|},
    moredelim=**[il][{\lstHLError}]{|err|:},
    moredelim=**[is][{\lstHLInfo}]{|info|}{|info|},
    moredelim=[is][{}]{|plain|}{|plain|},
    comment=[l]{\#},
    comment=[l]{//},
    sensitive=false,
    morekeywords = {},
    morekeywords = [2]{Pos},
    morekeywords = [3]{cout,cin,endl,flush,ssize_t,size_t,uint8_t,uint16_t,int8_t,int16_t,int_least8_t,int_fast8_t},
    morekeywords = [4]{moveForward, main, printf, sin, cos,strlen,copy_to_user,file_operations,strcmp},
    morekeywords = [5]{iostream, unistd.h},
    morecomment=[s]{/*}{*/},
    morestring=[b]',
}


\lstnewenvironment{cpp}[1][]
  {\lstset{language=lCpp,#1}}
{}



\pushList{RegisteredLanguages}{cpp/lCpp}

\lstdefinelanguage{lPython}{
    language=python,
    otherkeywords = {;,<<,>>,2>,<,|},
    alsoletter={_},
    comment=[l]{\#},
    morekeywords = {as,self},
    morekeywords = [2]{KNeighborsClassifier,KMeans,math},
    morekeywords = [3]{Neuron,Point2D,Point,Image,ImageFilter},
    morekeywords = [4]{fit,predict,__init__,accepts, get_accepting_neuron, get_mean_r,dist,mult_scalar,hypot, is_in,adapt,intersects,accepts,n_clusters,n_samples,centers,cluster_std,random_state,n_neighbors,weights},
    morekeywords = [5]{True,False,None,gp,nc,nr,ir},
}

\lstnewenvironment{python}[1][]
  {\lstset{language=lPython,#1}}
{}



\pushList{RegisteredLanguages}{python/lPython}


\lstdefinelanguage{lJson}{
    %% language=json,
    otherkeywords = {},
    alsoletter={},
    comment=[l]{\#},
    comment=[l]{//},
    sensitive=false,
    %alsoletter={\{\}[]},
    moredelim=**[is][{\lstHLWarning}]{|warn|}{|warn|},
    moredelim=**[is][{\lstHLError}]{|err|}{|err|},
    moredelim=**[is][{\lstHLInfo}]{|info|}{|info|},
    morestring=[s]{"}{"},
    morekeywords = {},
    morekeywords = [2]{true,false,null},
    %morekeywords = [4]{\{,\},[,]},
    %morecomment=[s]{/*}{*/}, there is no String
    morestring=[b]',
    alsoother={\,}
}


\lstnewenvironment{json}[1][]
  {\lstset{language=lJson,#1}}
{}

\pushList{RegisteredLanguages}{json/lJson}


%% Die environments wurden bewusst spezifisch definiert und nicht über ein foreach realisiert


\def\lst@PlaceNumber{\makebox[\dimexpr 1em+\lst@numbersep][l]{\normalfont
  \lst@numberstyle{\thelstnumber}}}%

\lstdefinestyle{MAIN}{
    breaklines=true,
    backgroundcolor=\color{MudWhite},
    rulecolor=\color{MudWhite},
    stringstyle=\color{DarkChromeYellow},
    keywordstyle = \LILLYxCODExKEYWORDSTYLE\bfseries,
    keywordstyle = [2]\color{Azure}, %%KW
    keywordstyle = [3]{\color{AppleGreen}}, %% PARAM
    keywordstyle = [4]{\color{DatmouthGreen}\bfseries}, %% FILES
    keywordstyle = [5]{\color{Orchid}}, %% SHELLPRE
    keywordstyle = [6]{\itshape}, %% SHELLPRE
    otherkeywords = {},
    alsoletter={@,_},
    comment=[l]{//},
    keywordsprefix={@},
    basicstyle=\LILLYxlstTypeWriter\color{black},   %Damit es auch wirklich ausschaut wie die Schreibmaschine!
    commentstyle={\color{gray}\LILLYxlstTypeWriter},
    extendedchars=true,
    numberstyle=\small\color{gray},
    prebreak={\raisebox{0.4\baselineskip}{\rotatebox{270}{\fontsize{4pt}{3pt}\selectfont$\curvearrowright$}}},
    frame=single,
    xleftmargin=15pt,
    xrightmargin=3pt,
    numbers=left,
    numbersep=7pt,
    %frameround=tttt, %% mag backgroundcolor nicht
    framerule=1pt,
}

\lstnewenvironment{lstplain}[1][]
  {\lstset{xleftmargin=0pt,xrightmargin=0pt,%
  numbers=none,numbersep=0pt,%
  rulecolor=\color{white}, backgroundcolor=\color{white}
  ,#1}}
{}

\lstnewenvironment{lstnonum}[1][]
  {\lstset{xleftmargin=0pt,xrightmargin=0pt,%
  numbers=none,numbersep=0pt,%
  ,#1}}
{}

\lstset{
    style=MAIN
}
