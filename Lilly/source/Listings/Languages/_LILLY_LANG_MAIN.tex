%% Das normale Source-Code Paket für die Latex-Mitschriebe
%\ifx\LILLYxMODE\LILLYxMODExPRINT
%\else
\providecommand{\LILLYxlstTypeWriter}{\fontfamily{AnonymousPro}\selectfont}
%\fi

\lstdefinelanguage{assembler}{
        keywords={while,if,r,ld,st,sr,sl,beq,bnq,add,sub,and,or,not,xor,dec,inc,jmp,addi,sw,addui,add,sw,lw,slti,j,jal,div,mul,hi,lo, mov, jne, tas, jne, swp, sei, cli},
        keywordstyle=\color{purple}\bfseries,
        ndkeywords={nop,X,acc},
        basicstyle=\LILLYxlstTypeWriter,   %Damit es auch wirklich ausschaut wie die Schreibmaschine!
        ndkeywordstyle=\color{tealblue!80!black}\bfseries,
        comment=[l]{//},
        comment=[l]{\#},
        morecomment=[s]{/*}{*/},
        commentstyle=\color{gray},
        stringstyle=\color{mint},
        morestring=[b]',
        sensitive=false,
        %literate={@}{{\textcolor{mint}{\$}}}2
        showstringspaces=true,  %Damit es keine Verwechslungen gibt
} 



%%%%% Pseudo-Sprachen
    \lstdefinelanguage{lPseudo}{
    keywords={INPUT,REPEAT,ELSE,UNTIL,OR,END,FOR,IF,END,TO,DO,THEN,TRUE,FALSE,END, print,println,goto,system,every,foreach,in},
    keywordstyle=\LILLYxCODExKEYWORDSTYLE,
    basicstyle=\LILLYxlstTypeWriter\color{Charcoal},   %Damit es auch wirklich ausschaut wie die Schreibmaschine!
    %identifierstyle=\color{Azure},
    comment=[l]{//},
    moredelim=**[is][{\lstHLWarning}]{|warn|}{|warn|},
    moredelim=**[is][{\lstHLError}]{|err|}{|err|},
    moredelim=**[il][{\lstHLError}]{|err|:},
    moredelim=**[is][{\lstHLInfo}]{|info|}{|info|},
    moredelim=[is][{}]{|plain|}{|plain|},
    sensitive=false,
    morecomment=[s]{/*}{*/},
    commentstyle=\color{AppleGreen},
    stringstyle=\color{mint},
    morestring=[b]",
    %literate={@}{{\textcolor{mint}{\$}}}2
    showstringspaces=false,  %Damit es keine Verwechslungen gibt
}

\RegisterLanguage{pseudo}{lPseudo}

% \lstnewenvironment{pseudo}[1][]
%   {\lstset{language=lPseudo,#1}}
% {}

% \pushList{RegisteredLanguages}{pseudo/lPseudo}

%%Compat-Gründe
    \lstdefinelanguage{pseudoNoSpace}{
    keywords={INPUT,REPEAT,ELSE,UNTIL,OR,END,FOR,IF,END,TO,DO,THEN,TRUE,FALSE,END,system,goto},
    keywordstyle=\color{purple}\bfseries,
    basicstyle=\LILLYxlstTypeWriter,   %Damit es auch wirklich ausschaut wie die Schreibmaschine!
    identifierstyle=\color{black},
    comment=[l]{//},
    morecomment=[s]{/*}{*/},
    commentstyle=\color{gray!80!white}\LILLYxlstTypeWriter,
    stringstyle=\color{mint}\LILLYxlstTypeWriter,
    morestring=[b]',
    %literate={@}{{\textcolor{mint}{\$}}}2
    showstringspaces=false,  %Damit es keine Verwechslungen gibt
} 

%%%%% GDRA
    \RequirePackage{caption}

\ExplSyntaxOn
\DeclareCaptionFont{black}{ \color{black} }
\DeclareCaptionFormat{none}{
    \makebox[\textwidth]{#1#2#3}
}
\DeclareCaptionFont{black}{ \color{black} }

\ExplSyntaxOff
\DeclareCaptionFormat{listing}{
    \makebox[\textwidth]{#1#2#3}
}
\newcommand{\gitRAW}{\text{\href{https://www.github.com/EagleoutIce/MIPS_UniUlm_Examples/}{\faGithub}}}
\newcommand{\git}{\hfill \gitRAW}

\def\noprint#1{}

\ExplSyntaxOn
\NewDocumentCommand \mipsspecifier { }
{
    \tl_set:No \l_demo_tl {\the\use:c{lst@token}}
    \regex_replace_all:nnN { .([a-zA-Z]+[0-9]+) } {\$\c{textcolor}\cB\{ limegreen \cE\}\cB{ \1 \cE} } \l_demo_tl
    \tl_use:N \l_demo_tl
    \noprint
}

\ExplSyntaxOff



\lstdefinestyle{MIPS}{
    language = Java,%Dirtymörty
    basicstyle = {\LILLYxlstTypeWriter},
    stringstyle = {\color{candypink}},
    keywordstyle = {\color{tealblue}},
    keywordstyle = [2]{\color{purple}},
    keywordstyle = [3]{\color{dgold}},
    keywordstyle = [4]{\color{limegreen}\bfseries},
    keywordstyle = [5]{\color{thered}},
    keywordstyle = [6]{\color{tealblue!60!black}},
    %keywordstyle = [6]{\toMips},
    otherkeywords = {;,<<,>>},
    keywordsprefix={\$},
    alsoletter={.:},
    comment=[l]{\#},
    commentstyle={\color{gray}\LILLYxlstTypeWriter},
    extendedchars=true,
    numberstyle=\small\color{gray},
    frame=single,
    escapeinside={*}{*}, 
    showstringspaces=true,  %Damit es keine Verwechslungen gibt
    basicstyle=\LILLYxlstTypeWriter,   %Damit es auch wirklich ausschaut wie die Schreibmaschine!
    numbers=left,           %Zeilennummern
    numbersep=5pt,
    xleftmargin=15pt,
    framexleftmargin=0pt,
    prebreak=\raisebox{0ex}[0ex][0ex]{\ensuremath{\hookleftarrow}},
    morekeywords=[2]{while,if,r,ld,st,sr,sl,beq,bnq,add,sub,and,or,not,xor,dec,inc,jmp,addi,sw,addui,add,sw,lw,slti,j,jal,div,mul,hi,lo,jr,addiu,
    la, lb, li, bne, mult, mflo},
    morekeywords = [3]{.text,.data,.ascii,.word},
    morekeywords = [4]{nop,limegreen,syscall},
    morekeywords = [5]{ERROR,WARNING},
    morekeywords = [6]{.asciiz, .space},
    %morekeywords = [6]{hallomama},
} 

\lstdefinestyle{MIPSSNIP}
{
    language = Java,%Dirtymörty
    basicstyle = {\LILLYxlstTypeWriter},
    stringstyle = {\color{candypink}},
    keywordstyle = {\color{tealblue}},
    keywordstyle = [2]{\color{purple}},
    keywordstyle = [3]{\color{dgold}},
    keywordstyle = [4]{\color{limegreen}\bfseries},
    keywordstyle = [5]{\color{red}},
    keywordstyle = [6]{\color{tealblue!60!black}},
    %keywordstyle = [6]{\toMips},
    otherkeywords = {;,<<,>>},
    keywordsprefix={\$},
    alsoletter={.:},
    comment=[l]{\#},
    commentstyle={\color{gray}\LILLYxlstTypeWriter},
    extendedchars=true,
    rulecolor=\color{white},
    escapeinside={!*}{*)}, 
    showstringspaces=true,  %Damit es keine Verwechslungen gibt
    basicstyle=\LILLYxlstTypeWriter,   %Damit es auch wirklich ausschaut wie die Schreibmaschine!
    title={},
    caption={},
    morekeywords=[2]{while,if,r,ld,st,sr,sl,beq,bnq,add,sub,and,or,not,xor,dec,inc,jmp,addi,sw,addui,add,sw,lw,slti,j,jal,div,mul,hi,lo,jr,addiu,
    la, lb, li, bne, mult, mflo},
    morekeywords = [3]{.text,.data,.ascii,.word,fact:,L1:},
    morekeywords = [4]{nop,limegreen,syscall},
    morekeywords = [5]{ERROR,WARNING},
    morekeywords = [6]{.asciiz, .space},
    %morekeywords = [6]{hallomama},
}


%%%%% Dokumentation - needfull shell 
    %% Wird für die Dokumentation entsprechnd erweitert

\lstdefinestyle{bash}{
    language=bash,
    caption={},
    breaklines=true,
    backgroundcolor=\color{black},
    rulecolor=\color{black},
    stringstyle = \color{white},
    keywordstyle = \color{antiVeg}, %%Hey Justin :D
    keywordstyle = [2]\color{Awesome}\bfseries, %%KW
    keywordstyle = [3]{\color{Amber}}, %% PARAM
    keywordstyle = [4]{\color{AppleGreen}\bfseries}, %% FILES
    otherkeywords = {;,<<,>>,>,2>,<,|},
    alsoletter={-,|,~,{,},_,.},
    comment=[l]{\#},
    keywordsprefix={\$},
    commentstyle={\color{gray}\LILLYxlstTypeWriter},
    extendedchars=true,
    numberstyle=\small\color{gray},
    frame=single,
    xleftmargin=3pt,
    xrightmargin=3pt,
    %frameround=tttt, %% mag backgroundcolor nicht
    framerule=1pt,
    escapeinside={*}{*}, 
    showstringspaces=true,  %Damit es keine Verwechslungen gibt
    basicstyle=\LILLYxlstTypeWriter\color{white},   %Damit es auch wirklich ausschaut wie die Schreibmaschine!
    numbers=none,           %Zeilennummern
    morekeywords = [2]{mkdir,texhash,make,apt},
    morekeywords = [3]{-p, -dir, print, install},
    morekeywords = [4]{lilly_compile.sh,sudo},
}


    %% Wird für die Dokumentation entsprechnd erweitert

\lstdefinestyle{haskell}{
    language=haskell,
    caption={},
    breaklines=true,
    backgroundcolor=\color{MudWhite},
    rulecolor=\color{MudWhite},
    stringstyle = \color{ChromeYellow},
    keywordstyle = \color{DebianRed},
    keywordstyle = [2]\color{Azure}\bfseries, %%KW
    keywordstyle = [3]{\color{Amber}}, %% PARAM
    keywordstyle = [4]{\color{AppleGreen}\bfseries}, %% FILES
    keywordstyle = [5]{\color{Orchid}\bfseries}, %% SHELLPRE
    otherkeywords = {;,<<,>>,2>,|},
    alsoletter={-,|,~,{,},_,.,>,<,:,\*},
    comment=[l]{--},
    keywordsprefix={\$},
    commentstyle={\color{gray}\LILLYxlstTypeWriter},
    extendedchars=true,
    numberstyle=\small\color{gray},
    frame=single,
    xleftmargin=3pt,
    morecomment=[s]{\{-}{-\}},
    xrightmargin=3pt,
    %frameround=tttt, %% mag backgroundcolor nicht
    framerule=1pt,
    escapeinside={!*!}{!*!}, 
    showstringspaces=true,  %Damit es keine Verwechslungen gibt
    basicstyle=\LILLYxlstTypeWriter\color{black},   %Damit es auch wirklich ausschaut wie die Schreibmaschine!
    numbers=none,           %Zeilennummern
    morekeywords = {:l,:cd,:q,:r,String,Char,Bool},
    morekeywords = [2]{<,>,->,=>,<-},
    morekeywords = [3]{True,False},
    morekeywords = [4]{::},
    morekeywords = [5]{Prelude>,\*Main>},
}


    %% Wird für die Dokumentation entsprechend erweitert

\lstdefinestyle{latex}{
    stringstyle = \color{Amber},
    keywordstyle = \color{Azure}\bfseries,
    keywordstyle = [2]\color{Awesome}\bfseries, %%KW
    keywordstyle = [3]{\color{ChromeYellow}}, %% PARAM
    keywordstyle = [4]{\bfseries}, %% FILES
    keywordstyle = [5]{\color{Ao}}, %% PARAM
    keywordstyle = [6]{\color{Azure}}, %% PARAM
    comment=[l]{\%},
    commentstyle={\color{gray}\LILLYxlstTypeWriter},
    alsoletter={\#,\\,@,\$,\_,-},
    alsoother={!,\"},
    extendedchars=true,
    breaklines=true,
    frame=single,
    columns=flexible,
    xleftmargin=3pt,
    xrightmargin=3pt,
    %frameround=tttt, %% mag backgroundcolor nicht
    framerule=0.5pt,
    escapeinside={!*}{*!}, 
    showstringspaces=true,  %Damit es keine Verwechslungen gibt
    basicstyle=\LILLYxlstTypeWriter,   %Damit es auch wirklich ausschaut wie die Schreibmaschine!
    numbers=none,           %Zeilennummern
    morekeywords = [2]{Lilly,Jake},
    morekeywords = [3]{Typ,sqrt,\\LILLYxTITLE,\\LILLYcommand,\\LILLYxColorxDefinition,\\LILLYxVorlesung,\\LILLYxSemester,\\LILLYxFACULTYxMATHE,\\LILLYxFACULTYxTHEORETISCHEINFORMATIK,\\LILLYxFACULTYxPRAKTISCHEINFORMATIK,\\LILLYxFACULTYxTECHNISCHEINFORMATIK},
    morekeywords = [4]{document,\\documentclass,\\begin,\\end,\\input,\\caption,\\TRUE,\\FALSE,\\elable,\\examplecube,\\rotateRPY,\\node,\\STATE,\\state,\\jmark,\\hmark,\\listofDEFS,\\clearpage,\\newpage,\\linput,\\include,\\linclude,\\inputUB,\\draw,\\plotline,\\graphPOI,\\loopTop,\\loopRight,\\loopBot,\\loopLeft,\\tcbset,\\DeclareTColorBox,\\textbf,\\textsc,\\specialrule,\\hfill,\\textwidth,\\tcboxedtitleheight,\\ifthenelse,\\equal,\\bfseries,\\path,\\definecolor,\\toprule,\\hspace,\\midrule,\\bottomrule,\\linewidth,\\arraybackslash,\\centering,\\foreach,\\renewcommand,\\getGraphics,\\framebox},
    morekeywords = [5]{definition,bemerkung,definition*,beispiel,satz,beweis,lemma,zusammenfassung,aufgabe,uebungsblatt,mrk,pmatrix,tabular,tikzpicture,graph,wgraph,plot,grid,scope,minipage,Automat,LILLYxBOXxDefinition,LILLYxBOXxAufgabe},
    morekeywords = [6]{\#1,\#2,\#3,\#4,\#5,\#6,\#7,\#8,\#9}
}


    %% Wird für die Dokumentation entsprechend erweitert

\lstdefinestyle{gepard}{
    stringstyle = \color{Amber},
    keywordstyle = \bfseries, 
    keywordstyle = [2]\color{Awesome}\bfseries, %%KW
    keywordstyle = [3]{\color{ChromeYellow}}, %% PARAM
    keywordstyle = [4]{\bfseries}, %% FILES
    keywordstyle = [5]{\color{Ao}}, %% PARAM
    keywordstyle = [6]{\color{Azure}}, %% PARAM
    comment=[l]{\%},
    commentstyle={\color{gray}\LILLYxlstTypeWriter},
    alsoletter={\#,=,-},
    morecomment=[s]{!}{!},
    extendedchars=true,
    breaklines=true,
    frame=single,
    xleftmargin=3pt,
    xrightmargin=3pt,
    %frameround=tttt, %% mag backgroundcolor nicht
    framerule=0.5pt,
    escapeinside={!*}{*!}, 
    showstringspaces=true,  %Damit es keine Verwechslungen gibt
    basicstyle=\LILLYxlstTypeWriter,   %Damit es auch wirklich ausschaut wie die Schreibmaschine!
    numbers=none,           %Zeilennummern
    morekeywords = [2]{Lilly,Jake},
    morekeywords = [3]{operation,file,debug},
    morekeywords = [4]{true,false,file_compile},
    morekeywords = [5]{=},
    morekeywords = [6]{lilly-show-boxname,lilly-nameprefix}
}



\lstdefinelanguage{Cpp}{
    language=C++,
    stringstyle = \color{black},
    keywordstyle=\LILLYxCODExKEYWORDSTYLE,       % keyword style
    keywordstyle = [2]\color{Awesome}, %%KW
    keywordstyle = [3]{\color{Orchid}\bfseries}, %% PARAM
    keywordstyle = [4]{\color{AppleGreen}}, %% FILES
    otherkeywords = {;,<<,>>,2>,<,|},
    alsoletter={-,|,~,{,},_,>,:},
    comment=[l]{\#},
    basicstyle=\LILLYxlstTypeWriter,   %Damit es auch wirklich ausschaut wie die Schreibmaschine!
    identifierstyle=\color{black},
    comment=[l]{//},
    sensitive=false,
    breaklines=true,
    backgroundcolor=\color{MudWhite},
    rulecolor=\color{MudWhite},
    numberstyle=\small\color{gray},
    frame=single,
    xleftmargin=15pt,
    xrightmargin=3pt,
    numbers=left,
    numbersep=7pt,
    framerule=1pt,
    morekeywords = {},
    morekeywords = [2]{Pos},
    morekeywords = [4]{moveForward, main, printf, sin, cos},
    morecomment=[s]{/*}{*/},
    commentstyle=\color{gray},
    stringstyle=\color{mint},
    morestring=[b]',
    %literate={@}{{\textcolor{mint}{\$}}}2
    showstringspaces=false,  %Damit es keine Verwechslungen gibt
}

\lstdefinelanguage{Py}{
    language=python,
    stringstyle = \color{black},
    keywordstyle=\LILLYxCODExKEYWORDSTYLE,       % keyword style
    keywordstyle = [2]\color{Azure}, %%KW
    keywordstyle = [3]{\color{Orchid}\bfseries}, %% PARAM
    keywordstyle = [4]{\color{AppleGreen}}, %% FILES
    keywordstyle = [5]{\color{ChromeYellow}}, %% FILES
    otherkeywords = {;,<<,>>,2>,<,|},
    alsoletter={_},
    comment=[l]{\#},
    breaklines=true,
    frame=none,
    xleftmargin=15pt,
    prebreak={\raisebox{0.4\baselineskip}{\rotatebox{270}{\fontsize{4pt}{3pt}\selectfont$\curvearrowright$}}},
    basicstyle=\small\LILLYxlstTypeWriter,   %Damit es auch wirklich ausschaut wie die Schreibmaschine!
    identifierstyle=\color{black},
    sensitive=false,
    numberstyle=\footnotesize\color{gray},
    morekeywords = {as,self},
    morekeywords = [2]{KNeighborsClassifier,KMeans,math},
    morekeywords = [3]{Neuron,Point2D,Point,Image},
    morekeywords = [4]{fit,predict,__init__,accepts, get_accepting_neuron, get_mean_r,dist,mult_scalar,hypot, is_in,adapt,intersects,accepts,n_clusters,n_samples,centers,cluster_std,random_state,n_neighbors,weights,resize},
    morekeywords = [5]{True,False,None,gp,nc,nr,ir},
    commentstyle=\color{gray},
    stringstyle=\color{mint},
    morestring=[b]',
    showstringspaces=true
}


\def\lst@PlaceNumber{\makebox[\dimexpr 1em+\lst@numbersep][l]{\normalfont
  \lst@numberstyle{\thelstnumber}}}%
    
\lstset{
    showstringspaces=true,  %Damit es keine Verwechslungen gibt
    basicstyle=\LILLYxlstTypeWriter,   %Damit es auch wirklich ausschaut wie die Schreibmaschine!
    numbers=left,           %Zeilennummern
    escapeinside={!*}{*)}, 
    frame=single,
    language=assembler,
    numberstyle=\small\color{gray},
    xleftmargin=15pt,
    numbersep=5pt,
    framexrightmargin=0pt,
    framexleftmargin=0pt,
    keywordstyle=\color{purple}\bfseries,       % keyword style
    commentstyle=\color{gray},    % comment style
    stringstyle=\color{mint},     % string literal style
    extendedchars=true,
}
