%% Das normale Source-Code Paket für die Latex-Mitschriebe
\ifx\LILLYxMODE\LILLYxMODExPRINT
    \providecommand{\LILLYxCODExKEYWORDSTYLE}{\color{DebianRed}\bfseries}
\else
    \providecommand{\LILLYxCODExKEYWORDSTYLE}{\color{DebianRed}}
\fi
%\else

%\fi

%% Die explizite Angabe erfolgt gewollt :D

\lstdefinelanguage{assembler}{
        keywords={while,if,r,ld,st,sr,sl,beq,bnq,add,sub,and,or,not,xor,dec,inc,jmp,addi,sw,addui,add,sw,lw,slti,j,jal,div,mul,hi,lo, mov, jne, tas, jne, swp, sei, cli},
        keywordstyle=\color{purple}\bfseries,
        ndkeywords={nop,X,acc},
        basicstyle=\LILLYxlstTypeWriter,   %Damit es auch wirklich ausschaut wie die Schreibmaschine!
        ndkeywordstyle=\color{tealblue!80!black}\bfseries,
        comment=[l]{//},
        comment=[l]{\#},
        morecomment=[s]{/*}{*/},
        commentstyle=\color{gray},
        stringstyle=\color{mint},
        morestring=[b]',
        sensitive=false,
        %literate={@}{{\textcolor{mint}{\$}}}2
        showstringspaces=true,  %Damit es keine Verwechslungen gibt
} 


%%%%% Pseudo-Sprachen
\lstdefinelanguage{lPseudo}{
    keywords={INPUT,REPEAT,ELSE,UNTIL,OR,END,FOR,IF,END,TO,DO,THEN,TRUE,FALSE,END, print,println,goto,system,every,foreach,in},
    keywordstyle=\LILLYxCODExKEYWORDSTYLE,
    basicstyle=\LILLYxlstTypeWriter\color{Charcoal},   %Damit es auch wirklich ausschaut wie die Schreibmaschine!
    %identifierstyle=\color{Azure},
    comment=[l]{//},
    moredelim=**[is][{\lstHLWarning}]{|warn|}{|warn|},
    moredelim=**[is][{\lstHLError}]{|err|}{|err|},
    moredelim=**[il][{\lstHLError}]{|err|:},
    moredelim=**[is][{\lstHLInfo}]{|info|}{|info|},
    moredelim=[is][{}]{|plain|}{|plain|},
    sensitive=false,
    morecomment=[s]{/*}{*/},
    commentstyle=\color{AppleGreen},
    stringstyle=\color{mint},
    morestring=[b]",
    %literate={@}{{\textcolor{mint}{\$}}}2
    showstringspaces=false,  %Damit es keine Verwechslungen gibt
}

\RegisterLanguage{pseudo}{lPseudo}

% \lstnewenvironment{pseudo}[1][]
%   {\lstset{language=lPseudo,#1}}
% {}

% \pushList{RegisteredLanguages}{pseudo/lPseudo}

%%%%% GDRA
% \ExplSyntaxOn
% \DeclareCaptionFont{black}{ \color{black} }
% \DeclareCaptionFormat{none}{
%     \makebox[\textwidth]{#1#2#3}
% }
% \DeclareCaptionFont{black}{ \color{black} }

% \ExplSyntaxOff
% \DeclareCaptionFormat{listing}{
%     \makebox[\textwidth]{#1#2#3}
% }
\newcommand{\mipsgitRAW}{\text{\href{https://www.github.com/EagleoutIce/MIPS_UniUlm_Examples/}{\faGithub}}}
\newcommand{\mipsgit}{\hbox{}\hfill \mipsgitRAW}

\def\noprint#1{}

\ExplSyntaxOn
\NewDocumentCommand \mipsspecifier { }
{
    \tl_set:No \l_demo_tl {\the\use:c{lst@token}}
    \regex_replace_all:nnN { .([a-zA-Z]+[0-9]+) } {\$\c{textcolor}\cB\{ limegreen \cE\}\cB{ \1 \cE} } \l_demo_tl
    \tl_use:N \l_demo_tl
    \noprint
}

\ExplSyntaxOff



\lstdefinelanguage{lMips}{%
    language=TeX,
    %keywordsprefix={\$},
    alsoletter={.:\$1234567890},
    comment=[l]{\#},
    moredelim=**[is][{\lstHLWarning}]{|warn|}{|warn|},
    moredelim=**[is][{\lstHLError}]{|err|}{|err|},
    moredelim=**[il][{\lstHLError}]{|err|:},
    moredelim=**[is][{\lstHLInfo}]{|info|}{|info|},
    moredelim=[is][{}]{|plain|}{|plain|},
    keywordstyle = \color{Azure}, %%KW
    keywordstyle = [2]\LILLYxCODExKEYWORDSTYLE\bfseries, %%KW
    escapeinside={*}{*},
    morestring=[b]',
    morestring=[b]",
    otherkeywords={;,<<,>>},
    morekeywords={\$0,\$1,\$2,\$3,\$4,\$5,\$6,\$7,\$8,\$9,\$10,\$11,\$12,\$13,\$14,\$15,\$16,\$17,\$18,\$19,\$20,\$21,\$22,\$23,\$24,\$25,\$26,\$27,\$28,\$29,\$30,\$zero,\$r0,\$at,\$v0,\$v1,\$a0,\$a1,\$a2,\$a3,\$t0,\$t1,\$t2,\$t3,\$t4,\$t5,\$t6,\$t7,\$s0,\$s1,\$s2,\$s3,\$s4,\$s5,\$s6,\$s7,\$t8,\$t9,\$k0,\$k1,\$gp,\$sp,\$fp,\$ra},
    morekeywords=[2]{while,if,r,ld,st,sr,sl,beq,bnq,add,sub,and,or,not,xor,dec,inc,jmp,addi,sw,addui,add,sw,lw,slti,j,jal,div,mul,hi,lo,jr,addiu,%
    la, lb, li, bne, mult, mflo},
    morekeywords = [3]{.text,.data,.ascii,.word},
    morekeywords = [4]{nop,limegreen,syscall},
    morekeywords = [5]{ERROR,WARNING},
    morekeywords = [6]{.asciiz, .space},
    morekeywords = [6]{hallomama},
    literate={HELP}{{\LILLYxPATH}}1 {á}{{\'a}}1 {é}{{\'e}}1 {í}{{\'i}}1 {ó}{{\'o}}1 {ú}{{\'u}}1
            {Á}{{\'A}}1 {É}{{\'E}}1 {Í}{{\'I}}1 {Ó}{{\'O}}1 {Ú}{{\'U}}1
            {à}{{\`a}}1 {è}{{\`e}}1 {ì}{{\`i}}1 {ò}{{\`o}}1 {ù}{{\`u}}1
            {À}{{\`A}}1 {È}{{\'E}}1 {Ì}{{\`I}}1 {Ò}{{\`O}}1 {Ù}{{\`U}}1
            {ä}{{\"a}}1 {ë}{{\"e}}1 {ï}{{\"i}}1 {ö}{{\"o}}1 {ü}{{\"u}}1
            {Ä}{{\"A}}1 {Ë}{{\"E}}1 {Ï}{{\"I}}1 {Ö}{{\"O}}1 {Ü}{{\"U}}1
            {â}{{\^a}}1 {ê}{{\^e}}1 {î}{{\^i}}1 {ô}{{\^o}}1 {û}{{\^u}}1
            {Â}{{\^A}}1 {Ê}{{\^E}}1 {Î}{{\^I}}1 {Ô}{{\^O}}1 {Û}{{\^U}}1
            {œ}{{\oe}}1 {Œ}{{\OE}}1 {æ}{{\ae}}1 {Æ}{{\AE}}1 {ß}{{\ss}}1
            {ű}{{\H{u}}}1 {Ű}{{\H{U}}}1 {ő}{{\H{o}}}1 {Ő}{{\H{O}}}1
            {ç}{{\c c}}1 {Ç}{{\c C}}1 {ø}{{\o}}1 {å}{{\r a}}1 {Å}{{\r A}}1
            {€}{{\euro}}1 {£}{{\pounds}}1 {«}{{\guillemotleft}}1
            {»}{{\guillemotright}}1 {ñ}{{\~n}}1 {Ñ}{{\~N}}1 {¿}{{?`}}1 %{li}{{  ~\mipsLI}}2
            {:cdot:}{{$\cdot$}}1
}


\RegisterLanguage{mips}{lMips}

% \lstnewenvironment{mips}[1][]
%   {\lstset{language=lMips,#1}}
% {}

% \pushList{RegisteredLanguages}{mips/lMips}


%%%%% Dokumentation - needfull shell
%% Wird für die Dokumentation entsprechend erweitert

\lstdefinelanguage{lBash}{
    language=bash,
    otherkeywords = {<<,>>,2>},
    alsoletter={|~_.><:-},
    moredelim=**[is][{\lstHLWarning}]{|warn|}{|warn|},
    moredelim=**[is][{\lstHLError}]{|err|}{|err|},
    moredelim=**[il][{\lstHLError}]{|err|:},
    moredelim=**[is][{\lstHLInfo}]{|info|}{|info|},
    moredelim=[is][{}]{|plain|}{|plain|},
    moredelim=**[il][{\color{Charcoal}\bfseries}]{>imp>},
    comment=[l]{\#},
    keywordsprefix={\$},
    escapeinside={*}{*},
    morekeywords = {bash},
    morekeywords = [2]{mkdir,texhash,make,apt,>,<,cd,git,pdflatex,mvn},
    morekeywords = [3]{print,install,update,get,dump,config,tokentest,clean,help,buildfile_compile,generate},
    morekeywords = [4]{lilly_compile.sh,sudo,lilly_jake,jake,java,javac,ghci,ghc,grep,ls,cp,sort,find,man,tail,head,setfacl,getfacl,groups,ls,pdflatex,apt-get,texdoc,traceroute,docker,netcat},
    morekeywords = [5]{GUI,REI,DEI},
    morekeywords = [6]{javafx:compile,javafx:run},
    morestring=[b]'
}

\RegisterLanguage{bash}{lBash}


% \lstnewenvironment{bash}[1][]
%   {\lstset{language=lBash,#1}}
% {}


% \pushList{RegisteredLanguages}{bash/lBash}

%% Wird für die Dokumentation entsprechend erweitert

 %

\lstdefinelanguage{lLatex}{
    %language=[LaTeX]TeX,
    keywordstyle = [7]{\color{Azure}}, %%
    keywordstyle = [4]{\color{AppleGreen}},
    morecomment = [l]{\%},
    %morecomment=[s][\color{bondiBlue}]{\$}{\$}, %% see :bmath: and :emath:
    commentstyle={\color{gray}\LILLYxlstTypeWriter},
    add to literate={\\$}{{\$}}1 {\\$}{{\$}}1 {\\\%}{{\%}}1 {\\(}{{{\textcolor{bondiBlue}{\textbackslash(}}}}2 {\\)}{{{\textcolor{bondiBlue}{\textbackslash)}}}}2 {\\[}{{{\textcolor{bondiBlue}{\textbackslash[}}}}2 {\\]}{{{\textcolor{bondiBlue}{\textbackslash]}}}}2 {\\;}{{{\textcolor{Purple}{\textbackslash;}}}}2,
    alsoletter={\\\#@*},
    alsoother={!,"_},
    % moredelim=**[s][{\color{bondiBlue}}]{$$}{$$},
    % moredelim=**[s][{\color{bondiBlue}}]{$}{$},
    % moredelim=**[s][{\color{bondiBlue}}]{\\(}{\\)},
    % moredelim=**[s][{\color{bondiBlue}}]{\\[}{\\]},
    %deletekeywords=[5]{\\},
    columns=[c]flexible,
    escapeinside={!*}{*!},
    morekeywords = [2]{Lilly,Jake,book,lipsum,geometry,babel,inputenc,fontenc,xcolor,LILLYxMATH,LILLYxGRAPHICS,LILLYxCOLOR,LILLYxBOXES,LISTxCompatColors,LILLYxCONTROLLERxBOX,LILLYxLISTINGS,LILLYxLISTINGSxADVANCED,lillyKIZ,,beamer,LILLYxTIMETABLESxUNIVERSITY,LILLYxUTIL,LILLYxCORE,LILLYxLIST,LILLYxFORMATxCONTROL,LISTxColors,LILLYxPRESENTER,\\LISTxCompatColors,\\LISTxColors,\\def,\\gdef,\\edef,\\xdef,\\gdef,\\@nil,\\DeclareDocumentCommand,\\if,\\ifx,\\else,\\fi,blindtext,\\LILLY@n},
    morekeywords = [3]{\\LILLYxTITLE,\\LILLYcommand,\\LILLYxVorlesung,\\LILLYxFACULTYxMATHE,\\LILLYxFACULTYxTHEORETISCHEINFORMATIK,\\LILLYxFACULTYxPRAKTISCHEINFORMATIK,\\addcontentsline,\\LILLYxFACULTYxTECHNISCHEINFORMATIK,\\LILLYxPATHxDATA,\\LILLYxPATHxLISTINGS},
    morekeywords = [5]{\\documentclass,documentclass,\\usepackage,\\setLinkColor,\\lpage,\\LILLYxColorxDefinition,\\begin,\\ifthenelse,\\silentHmark,\\end,\\input,\\NewTimeTable,\\PresentTimeTable,\\caption,\\true,\\FALSE,\\elable,\\examplecube,\\rotateRPY,\\node,\\LoadLillyBoxMode,\\RenewTColorBox,\\STATE,\\state,\\jmark,\\hmark,\\listofDEFS,\\clearpage,\\codierscheibe,\\Psi,\\Chi,\\Omega,\\psi,\\omega,\\Pi,\\Rho,\\rho,\\lambda,\\Lambda,\\cA,\\cAname,\\cAadj,%
    \\newpage,\\linput,\\include,\\linclude,\\inputUB,\\draw,\\plotline,\\plotseq,\\graphPOI,\\loopTop,\\loopRight,\\loopBot,\\loopLeft,\\tcbset,\\DeclareTColorBox,\\textbf,\\textsc,\\specialrule,\\hfill,\\textwidth,\\VRule,\\tcboxedtitleheight,\\ifthenelse,\\equal,\\bfseries,\\path,\\definecolor,\\toprule,\\hspace,\\midrule,\\bottomrule,\\linewidth,\\arraybackslash,\\centering,\\foreach,\\renewcommand,\\getGraphics,getGraphics,\\framebox,\\vSplitter,\\item,\\xmark,\\ymark,\\ldots,\\psi,\\xi,\\chi,\\listofDEFINITIONS,\\NewLectureSeries,\\LILLYxMathxMode,\\ProgressBar,%
    \\constructList,\\pushList,\\popList,\\lillyxlist,\\clatex,\\igepard,\\newline,\\containsList,\\left,\\right,\\addplot3,\\PgetXY,\\PgetY,\\PgetX,\\neuronSquare,\\registerColors,\\Acronym,\\PoliteWords,\\ColorfulWords,\\Acronym,\\@Acronym,\\lilly@format@iter,\\LILLYxNOTExLibrary,\\lstdefinelanguage,\\lstset,\\lstnewenvironment,\\thetcbcounter,\\fontsize,\\abstractMarginBox,\\sum,\\infty,\\pm,\\prod,\\sqrt,\\frac,\\lim,\\to,\\StartCode,\\KwData,\\KwResult,\\StartCode,\\;,\\While,\\If,\\Comment,\\For,\\eIf,\\dist,%
    \\LILLY@FORMATTER@CURRENT,\\TextBfFormat,\\newcommand,\\blankcmd,\\textcolor,\\LILLYxlstTypeWriter,\\textbackslash,\\LILLYxlstTypeWriter\\textbackslash,\\LILLYxLIBRARY,\\baselineskip,\\paragraph,\\ilatex,\\ignorespaces,\\NoFormatChar,\\LILLYxTITLExBONUS,\\@Alph,\\typesetList,\\expandafter,\\isLanguageLoaded,\\cpython,\\bpython,\\ccpp,\\bcpp,\\verb,\\section,\\section*,\\subsection*,\\chapter*,\\chapter,\\subsection,\\subsubsection,\\smallskip,\\medskip,\\bigskip,\\smallskip\\newline,\\medskip\\newline,\\bigskip\\newline,\\LILLYxListingsxLang,\\LILLYxListingsxPACK,\\cjava,\\bjava,\\pjava,\\getGraphics,\\lstkwone,\\lstkwtwo,\\lstkwthree,\\lstkwfour,\\lstkwfive,\\lstkwsix,\\lststring,\\lstcomment,\\lstnumber,\\clearrow,\\headerrow,\\headerrow*,\\forall,\\in,\\exists,\\gets,%
    \\getGraphicsPath,\\arabic,\\CTRxDEF,\\isLanguageNameLoaded,\\lstshowcmd,\\getPrerendered,\\LILLYxBOXxMODE,\\LILLYxBOXxDefinitionxEnable,\\pcpp,\\LILLYxBOXxBeweisxBox,\\NewEmblem,\\infoEmblem,\\NewInfoBox,\\dateBox,\\pi,\\userput,\\newcounter,\\setcounter,\\doublealph,\\mathbb,\\mathcal,%
    \\warningEmblem,\\errorEmblem,\\mathEmblem,\\codeEmblem,\\textEmblem,\\btextEmblem,\\DefaultBaseEmblem,\\mathbf,\\lipsum,\\RegisterLanguage,\\hypertarget,\\T,\\pgfkeysvalueof,\\noexpand,\\leavevmode,\\csname,\\endcsname,\\lillyPathData,\\textit,\\PickRandom,\\colvec,\\minicolvec,\\say,\\firstcircle,\\secondcircle,\\thirdcircle,\\bigcircle,\\setrow,\\itshape,\\makeatletter,\\makeatother,\\usetheme,\\narrowitems,\\setname,\\colorprofile,\\setemail,\\setwebsite,\\setimage,\\setphone,\\setlocation,\\addskill,\\addskilltext,\\StartApplication,\\href,%
    \\makerenewglobal,\\lreqn,\\makerenewlocal,\\newenvironment,\\tiny,\\LILLYcoloredSQ,\\makeenvglobal,\\RandomInt,\\engl,\\rom,\\dispnote,\\note,\\snote,\\case,\\default,\\LILLYxDemandPackage,\\LILLYxLoadPackage,\\OrnamentsBoxTitle,\\OrnamentsUpper,\\OrnamentsLower,\\LILLYxPATHxFALLBACKS,\\cheaderrow,\\normalrow,\\smallrow,\\footnotesizerow,\\scriptsizerow,\\tinyrow,\\tabreset,\\tabprint,\\etabadd,\\tabadd,\\tabforeach,\\obeylines,\\subduelines,\\singlequote,\\NewTimeTableEvent,\\smallNumber,\\PersonAlias,\\PersonName,\\PersonFullName,\\CreateNewPerson,\\ShowPerson,\\ShowPersonTag,%
    \\imp,\\<,\\>,\\mto,\\reg,\\customex,\\kw,\\sw,\\sr,\\TOP,\\startAppendix,\\LillyNewPaletteColor,\\TOPskip,\\showcase,\\emph,\\blinddocument,\\TableOfContents,\\SetPartFlavour,\\printMiniToc,\\Hcolor,\\part,\\LILLYxGENxFACULTY,\\LILLYxFACULTY,\\LILLYxFACULTYxCOLOR,\\providecommand,\\LILLYxColorxTITLExSETTINGSxGENERAL,\\textsl,\\LILLYxColorxTITLExSETTINGSxVORLESUNG,\\resizebox,\\LILLYxPHILOSOPHERxBORDERBLOCK,\\rectat,\\crectat,\\paperwidth,\\tikz,\\MonthToName,\\@Session,\\@Session@End,\\CreateNewFileType,\\CreateNewFolderType,\\setprice,\\setquote,\\seteffects,\\closeritems,\\scriptsize,\\tableofcontents,\\includegraphics,%
    \\PROFESSOR,\\UEBUNGSLEITER,\\TUTOR,\\TITLE,\\SUBTITLE,\\UEBUNGSHEADER,\\applicationset,\\FULLTITLE,\\VORLESUNG,\\anaI,\\DeclareRobustCommand,\\POLITEINTRO,\\LaTeX,\\TeX,\\TikZ,\\providedef,\\setlength,\\itemsep,\\selectfont,\\entity,\\relation,\\attribute,\\kattribute,\\filldraw,\\par,\\SetFolderFileSameIndent,\\sign,\\progressbar,\\CreateCardGame,\\CardFan,\\CardBoard,\\text,\\cdot,%
    \\settitle,\\setheading,\\setsubheading,\\setauthor,\\setsubtitle,\\setdate,\\setresourcepath,\\setbrief,\\settitleimage,\\settitlewidth,\\setlogoimage,\\setsignature,\\setsignatureDarker},
    morekeywords = [4]{definition,bemerkung,bemerkung*,definition*,beispiel,beispiel*,satz,satz*,beweis,beweis*,lemma,lemma*,zusammenfassung,zusammenfassung*,aufgabe,aufgabe*,uebungsblatt,uebungsblatt*,mrk,pmatrix,tabular,tabular*,session,telegram,tikzpicture,graph,mtable,mltable,cases,itemize,enumerate,mtabular,mltabular,wgraph,plot,aufgaben,algorithm,slatex,grid,scope,minipage,Automat,LILLYxBOXxDefinition,poem,poem*,LILLYxBOXxAufgabe,enumeratea,document,latex,axis,align*,align,alignat,alignat*,lstlisting,quote,figure,quotes,quotes*,plainjava,plainlatex,sjava,tikzternal,center,application,block,timeline,event,bulletpoints,text,lstplain,cpp,java,lstnonum,multicols,ditemize,presentlst,switch,NewFantasyCard,\\wasysymLightning,\\faCode,\\faMagic,\\faGraduationCap,\\faTasks,\\faThumbsOUp,\\faApple,\\faLinux,\\faCalendar,infoBox,warningBox,errorBox,mathBox,codeBox,infoBox*,java*,warningBox*,errorBox*,mathBox*,codeBox*,\\isRuntimeLoaded,defaultlst,ditemize,smalldite,smalldesc,directory,slide,fancydir},
    morekeywords = [4]{\\\\},
    morekeywords = [6]{Dokumentation,Mitschrieb,Uebungsblatt,Zusammenfassung,samples,opacity,\\storeLillyColorList,\\storeLillyCompatColorList,article,lJava,lLatex,lCpp},
    morekeywords = [7]{\#1,\#2,\#3,\#4,\#5,\#6,\#7,\#8,\#9},
}

\RegisterLanguage{latex}{lLatex}
%% Wird f�r die Dokumentation entsprechend erweitert

\lstdefinelanguage{lGepard}{
    comment=[l]{\%},
    alsoletter={\#,=,-,[,]},
    morecomment=[s]{!}{!},
    morestring=[b]",
    escapeinside={!*}{*!},
    showstringspaces=false,
    otherkeywords={++\\},
    morekeywords = [1]{@[SELTEXF],@[AUTONUM]},
    morekeywords = [2]{Lilly,Jake},
    morekeywords = [3]{operation,file,debug,lilly-show-boxname,lilly-nameprefix,lilly-modes,lilly-show-boxname,lilly-author,lilly-n,lilly-semester,lilly-vorlesung},
    morekeywords = [4]{true,false,file_compile},
    morekeywords = [5]{=},
    morekeywords = [6]{uebungsblatt}
}

\lstnewenvironment{gepard}[1][]
  {\lstset{language=lGepard,#1}}
{}
\gdef\cgepard#1{\LILLYxwriteLst[language=lGepard]{#1}}
\gdef\igepard#1{\lstinputlisting[language=lGepard]{#1}}
\pushList{RegisteredLanguages}{gepard/lGepard}


%% Wird für die Dokumentation entsprechend erweitert

\lstdefinelanguage{lJava}{
    language=java,
    otherkeywords = {->,<,>},
    alsoletter={@,_},
    comment=[l]{//},
    keywordsprefix={@},
    %frameround=tttt, %% mag backgroundcolor nicht
    escapeinside={!*}{*!},
    morecomment=[s]{/*}{*/},
    morekeywords = {String,System,Integer},
    morekeywords = [2]{Stream,List,Queue,ArrayList,SingleLinkedList,SortedList,SortedSet,Set,HashMap,Tree,HashSet,FileReader,BufferedReader,InputStream,BufferedInputStream,BufferedOutputStream,OutputStream,FileInputStream,FileOutputStream,FileWriter,RandomAccessFile,ObjectOutputStream,ObjectInputStream,File,LocalDate,Node,NodeList,DocumentBuilder,Document,DocumentBuilderFactory,XPathExpression,XPath,XPathFactory,NodeList,Transformer,DOMSource,TransformerFactory,StreamResult,StreamSource,SAXParser,SAXParserFactory,XMLReader,DriverManager,Connection,Statement,ResultSet,PreparedStatement,Text,Font,Application,Pane,Leaf,Component,Composite,Leaf,},
    morekeywords = [3]{contains,toUpperCase,println,push,pop,sort,get,iterator,spliterator,add,matches,Serializable,Cloneable,Iterable,newInstance,parse,newDocumentBuilder,newTransformer,newXPath,compile,evaluate,transform,setTextContent,getTextContent,setAttribute,newSAXParser,getXMLReader,getConnection,createStatement,prepareStatement,executeQuery,execute,executeUpdate,getInt,getString,format,next,close,printStackTrace,isClosed,nextString,nextLine,wasNull,commit,rollback,getMetaData,setContentHandler,item,getChildren,setFont,setX,setY,setText,select,from,where,eq,fetch,equals},
    morekeywords = [4]{Files,Paths,Collectors,Collection,Iterable,Iterator,Object,EOFException,ClassNotFoundException,IOException,XPathConstants,SQLException},
    morekeywords = [5]{filter,map,sorted,collect,toList,forEach,mapToInt,stream,sum,read,reduce,NODESET},
}


\lstnewenvironment{java}[1][]
  {\lstset{language=lJava,#1}}
{}

\DeclareRobustCommand{\cjava}[1]{\LILLYxwriteLst[language=lJava,morekeywords={System}]{#1}}

%% Wird für die Dokumentation entsprechend erweitert

\lstdefinelanguage{lXml}{
    % language=xml,
    %% identifierstyle=\color{AppleGreen},
    alsoletter={?},
    morestring=[b]",
    morestring=[b]',
    %morestring=[s]{>}{<},
    morestring=[s]{"}{"},
    morecomment=[s]{<?}{?>},
    morecomment=[s]{!--}{--},
    alsoletter={:,/},
    otherkeywords={/>,<,>},
    %frameround=tttt, %% mag background color nicht
    escapeinside={!*}{*!},
    showstringspaces=true,  %Damit es keine Verwechslungen gibt
    morekeywords = {DOCTYPE,!DOCTYPE,ELEMENT,!ELEMENT,ATTLIST,!ATTLIST,/},
    morekeywords = [2]{xml},
    morekeywords = [3]{SYSTEM,PUBLIC,EMPTY,PCDATA,CDATA},
    morekeywords = [4]{},
    morekeywords = [5]{collection,xs:element,xs:list,xs:namespace,xs:attribute,xs:simpleType,xs:minInclusive,xs:maxInclusive,xs:restriction,xs:enumeration,xs:union,xs:sequence,xs:complexType,xmlns,xmlns:xs,xmlns:xsi,xsi:schemaLocation,xs:schema,/xs:element,/xs:namespace,/xs:attribute,/xs:simpleType,/xs:minInclusive,/xs:maxInclusive,/xs:restriction,/xs:enumeration,/xs:union,/xs:sequence,/xs:complexType,/xs:list,/xs:schema,xs:integer,/collection,last()},% + xpath
    morekeywords = [6]{name,maxOccurs,minOccurs,version,encoding,targetNamespace,ref,type,memberTypes,itemType,base,value,default},
}


\lstnewenvironment{xml}[1][]
  {\lstset{language=lXml,#1}}
{}

\DeclareRobustCommand{\cxml}[1]{\LILLYxwriteLst[language=lXml]{#1}}

% Wird für die Dokumentation entsprechend erweitert
\lstdefinelanguage{lSql}{
language=sql,
comment=[l]{--},
escapeinside={!*}{*!},
moredelim=**[is][{\lstHLWarning}]{|warn|}{|warn|},
moredelim=**[is][{\lstHLError}]{|err|}{|err|},
moredelim=**[il][{\lstHLError}]{|err|:},
moredelim=**[is][{\lstHLInfo}]{|info|}{|info|},
moredelim=[is][{}]{|plain|}{|plain|},
moredelim=**[s][{\itshape}]{`}{`},
showstringspaces=true,  %Damit es keine Verwechslungen gibt
morekeywords = {WITH,IS,REFERENCES,DEFERRED,OPTION,TO,START,WORK,SHOW,DATABASES,DATABASE,USE,LENGTH,LOCATE,TYPE,UNDER,OF,CURRENT,SEQUENCE,NEXTVAL,PREVVAL,PROCEDURE,FUNCTION,DETERMINISTIC,SQL,CALL,FOR,DO,EACH,STATEMENT,BEFORE,AFTER,INSTEAD OF,ROw,REFERENCING,OLD,NEW,ATOMIC,CONCAT_WS,CONCAT,DELIMITER},
morekeywords = [2]{DOUBLE,REAL,BOOLEAN,USE,LONG,UNSIGNED},
morekeywords = [3]{mysql>,ROUND},
morekeywords = [4]{},
morekeywords = [5]{},
morekeywords = [6]{AUTO_INCREMENT},
morecomment=[s]{/*}{*/},
}
\RegisterLanguage{sql}{lSql}
%% Wird für die Dokumentation entsprechend erweitert

\lstdefinelanguage{lXsl}{
    % language=xslt,
    %% identifierstyle=\color{AppleGreen},
    %keywordsprefix={"},
    morestring=[b]",
    morestring=[b]',
    %morestring=[s]{>}{<},
    morestring=[s]{"}{"},
    morecomment=[s]{<?}{?>},
    morecomment=[s]{!--}{--},
    moredelim=**[is][{\lstHLWarning}]{|warn|}{|warn|},
    moredelim=**[is][{\lstHLError}]{|err|}{|err|},
    moredelim=**[il][{\lstHLError}]{|err|:},
    moredelim=**[is][{\lstHLInfo}]{|info|}{|info|},
    moredelim=[is][{}]{|plain|}{|plain|},
    alsoletter={:/-},
    otherkeywords={/>,<,>},
    morekeywords = {},
    morekeywords = [2]{xml},
    morekeywords = [3]{SYSTEM,PUBLIC,EMPTY,PCDATA,CDATA},
    morekeywords = [4]{},
    morekeywords = [5]{collection,xsl:stylesheet,/xsl:stylesheet,xmlns:xsl,xmlns:xs,xmlns:fn,xsl:output,xsl:template,xsl:copy,/xsl:copy,/xsl:template,xsl:apply-templates,xsl:attribute,/xsl:attribute},% + xpath
    morekeywords = [6]{name,version,encoding,targetNamespace,ref,type,memberTypes,itemType,base,value,default,method,indent,match,select},
}


\RegisterLanguage{xsl}{lXsl}

% \lstnewenvironment{xsl}[1][]
%   {\lstset{language=lXsl,#1}}
% {}

% \pushList{RegisteredLanguages}{xsl/lXsl}


\lstdefinelanguage{lChr}{
    % language=prolog,
    alsoletter={-,|,~,{,},_,.,>,<,=,:,\*,',?,\\},
    comment=[l]{\%},
    keywordsprefix={\$},
    % morecomment=[s]{\{-}{-\}},
    escapeinside={!!}{!!},
    morekeywords = {==>,@,<=>,=:=,\\,|,->,<-,>=,=<},
    morekeywords = [2]{},
    morekeywords = [3]{true,false},
    morekeywords = [4]{::=},
    morekeywords = [5]{},
}

\lstnewenvironment{chr}[1][]
  {\lstset{language=lChr,#1}}
{}
\gdef\cchr#1{\LILLYxwriteLst[language=lChr]{#1}}
\gdef\ichr#1{\lstinputlisting[language=lChr]{#1}}

\pushList{RegisteredLanguages}{chr/lChr}

\lstdefinelanguage{lProlog}{
    language=prolog,
    alsoletter={-,|,~,{,},_,.,>,<,=,:,\*,',?},
    comment=[l]{\%},
    keywordsprefix={\$},
    % morecomment=[s]{\{-}{-\}},
    escapeinside={!!}{!!},
    morekeywords = {:-},
    morekeywords = [2]{->,=>,<-,>=,=:=, =\=,=<,<,>},
    morekeywords = [3]{True,False},
    morekeywords = [4]{::=},
    morekeywords = [5]{?-,?},
}

\lstnewenvironment{prolog}[1][]
  {\lstset{language=lProlog,#1}}
{}

\DeclareRobustCommand{\cprolog}[1]{\LILLYxwriteLst[language=lProlog]{#1}}
\lstdefinelanguage{lHaskell}{
    language=haskell,
    alsoletter={-|~_.><=:\*'},
    moredelim=**[is][{\lstHLWarning}]{|warn|}{|warn|},
    moredelim=**[is][{\lstHLError}]{|err|}{|err|},
    moredelim=**[il][{\lstHLError}]{|err|:},
    moredelim=**[is][{\lstHLInfo}]{|info|}{|info|},
    moredelim=[is][{}]{|plain|}{|plain|},
    comment=[l]{--},
    deletekeywords={Maybe,Either,Maybe,Either,Nothing,Just,Left,Right,\$},
    morecomment=[s]{\{-}{-\}},
    escapeinside={!*}{*!}, % !! would collide with the '!!' List select operator
    morekeywords = {:l,:cd,:q,:r,:t,String,Char,Bool,Show},
    morekeywords = [2]{<,>,->,=>,<-,>=},
    morekeywords = [3]{True,False,Maybe,Either,Nothing,Just,Left,Right},
    morekeywords = [4]{::,\$},
    morekeywords = [5]{Prelude>,\*Main>},
}
\RegisterLanguage{haskell}{lHaskell}
\lstdefinelanguage{lCpp}{
    language=C++,
    otherkeywords = {;,<<,>>,2>,<,>,|,std::chrono::,std::},
    alsoletter={-,|,~,{,},_},
    comment=[l]{\#},
    comment=[l]{//},
    sensitive=false,
    morekeywords = {},
    morekeywords = [2]{Pos},
    morekeywords = [3]{cout,cin,endl,flush,ssize_t,size_t,uint8_t,uint16_t,int8_t,int16_t,int_least8_t,int_fast8_t},
    morekeywords = [4]{moveForward, main, printf, sin, cos,strlen,copy_to_user,file_operations},
    morekeywords = [5] {iostream, unistd.h},
    morecomment=[s]{/*}{*/},
    morestring=[b]',
}


\lstnewenvironment{cpp}[1][]
  {\lstset{language=lCpp,#1}}
{}
\gdef\ccpp#1{\LILLYxwriteLst[language=lCpp]{#1}}
\gdef\icpp#1{\lstinputlisting[language=lCpp]{#1}}

\pushList{RegisteredLanguages}{cpp/lCpp}

\lstdefinelanguage{lPython}{
    language=python,
    otherkeywords = {;,<<,>>,2>,<,|},
    alsoletter={_},
    comment=[l]{\#},
    breaklines=true,
    frame=none,
    prebreak={\raisebox{0.4\baselineskip}{\rotatebox{270}{\fontsize{4pt}{3pt}\selectfont$\curvearrowright$}}},
    basicstyle=\small\LILLYxlstTypeWriter,
    identifierstyle=\color{black},
    sensitive=false,
    numberstyle=\footnotesize\color{gray},
    morekeywords = {as,self},
    morekeywords = [2]{KNeighborsClassifier,KMeans,math},
    morekeywords = [3]{Neuron,Point2D,Point,Image,ImageFilter},
    morekeywords = [4]{fit,predict,__init__,accepts, get_accepting_neuron, get_mean_r,dist,mult_scalar,hypot, is_in,adapt,intersects,accepts,n_clusters,n_samples,centers,cluster_std,random_state,n_neighbors,weights},
    morekeywords = [5]{True,False,None,gp,nc,nr,ir},
    morestring=[b]',
    showstringspaces=true
}

\lstnewenvironment{python}[1][]
  {\lstset{language=lPython,#1}}
{}

\DeclareRobustCommand{\cpython}[1]{\LILLYxwriteLst[language=lPython]{#1}}

%% Wird für die Dokumentation entsprechend erweitert

\lstdefinelanguage{lDocker}{
    alsoletter={|~_.><-},
    moredelim=**[is][{\lstHLWarning}]{|warn|}{|warn|},
    moredelim=**[is][{\lstHLError}]{|err|}{|err|},
    moredelim=**[il][{\lstHLError}]{|err|:},
    moredelim=**[is][{\lstHLInfo}]{|info|}{|info|},
    moredelim=[is][{}]{|plain|}{|plain|},
    moredelim=**[il][{\color{Charcoal}\bfseries}]{>imp>},
    comment=[l]{\#},
    morestring=[s]"",
    escapeinside={!*}{*!},
    morekeywords = {FROM,RUN,LABEL,VOLUME,ENTRYPOINT,EXPOSE,ADD},
    %morekeywords = [2]{mkdir,texhash,make,apt,>,<,cd,git,pdflatex,mvn},
    morekeywords = [3]{pull,run,build,image,inspect,rmi},
    %morekeywords = [4]{lilly_compile.sh,sudo,lilly_jake,jake,java,javac,ghci,ghc,grep,ls,cp,sort,find,man,tail,head,setfacl,getfacl,groups,ls,pdflatex,apt-get,texdoc,traceroute,docker},
    morekeywords = [5]{docker},
    morekeywords = [6]{ls},
    morestring=[b]'
}

\RegisterLanguage{docker}{lDocker}


% \lstnewenvironment{bash}[1][]
%   {\lstset{language=lBash,#1}}
% {}


% \pushList{RegisteredLanguages}{bash/lBash}

%% SonarCüüüüübööööööö
\lstdefinelanguage{lSonarCube}{
    alsoletter={|~_><-},
    moredelim=**[is][{\lstHLWarning}]{|warn|}{|warn|},
    moredelim=**[is][{\lstHLError}]{|err|}{|err|},
    moredelim=**[il][{\lstHLError}]{|err|:},
    moredelim=**[is][{\lstHLInfo}]{|info|}{|info|},
    moredelim=[is][{}]{|plain|}{|plain|},
    moredelim=**[il][{\color{Charcoal}\bfseries}]{>imp>},
    comment=[l]{\#},
    morestring=[s]"",
    escapeinside={!*}{*!},
    morekeywords = {},
    % morekeywords = [2]{},
    morekeywords = [3]{projectKey,projectName,projectVersion,sources,sourceEncoding,scm,java},
    %morekeywords = [4]{lilly_compile.sh,sudo,lilly_jake,jake,java,javac,ghci,ghc,grep,ls,cp,sort,find,man,tail,head,setfacl,getfacl,groups,ls,pdflatex,apt-get,texdoc,traceroute,docker},
    morekeywords = [5]{sonar},
    morekeywords = [6]{provider,binaries},
    morestring=[b]'
}

\RegisterLanguage{sonarcube}{lSonarCube}
\lstdefinelanguage{lGit}{
    alsoletter={|~_><-},
    moredelim=**[is][{\lstHLWarning}]{|warn|}{|warn|},
    moredelim=**[is][{\lstHLError}]{|err|}{|err|},
    moredelim=**[il][{\lstHLError}]{|err|:},
    moredelim=**[is][{\lstHLInfo}]{|info|}{|info|},
    moredelim=[is][\itshape\bfseries]{|ihl|}{|ihl|},
    moredelim=[is][{}]{|plain|}{|plain|},
    moredelim=**[il][{\color{Charcoal}\bfseries}]{>imp>},
    comment=[l]{\#},
    morestring=[s]"",
    escapeinside={!*}{*!},
    morekeywords = {remote},
    % morekeywords = [2]{},
    morekeywords = [3]{commit,push,add,init,status,rm,pull,clone,checkout,branch,merge,config,fetch,stash,diff,reset,tag},
    %morekeywords = [4]{lilly_compile.sh,sudo,lilly_jake,jake,java,javac,ghci,ghc,grep,ls,cp,sort,find,man,tail,head,setfacl,getfacl,groups,ls,pdflatex,apt-get,texdoc,traceroute,docker},
    morekeywords = [5]{git},
    morekeywords = [6]{origin,master,user-input,--global,epic-feature},
    morestring=[b]'
}

\RegisterLanguage{git}{lGit}


\lstdefinelanguage{lJson}{
    %% language=json,
    otherkeywords = {},
    alsoletter={},
    comment=[l]{\#},
    comment=[l]{//},
    sensitive=false,
    %alsoletter={\{\}[]},
    moredelim=**[is][{\lstHLWarning}]{|warn|}{|warn|},
    moredelim=**[is][{\lstHLError}]{|err|}{|err|},
    moredelim=**[il][{\lstHLError}]{|err|:},
    moredelim=**[is][{\lstHLInfo}]{|info|}{|info|},
    moredelim=[is][{}]{|plain|}{|plain|},
    morestring=[s]{"}{"},
    morekeywords = {},
    morekeywords = [2]{true,false,null},
    %morekeywords = [4]{\{,\},[,]},
    %morecomment=[s]{/*}{*/}, there is no String
    morestring=[b]',
    alsoother={\,}
}


\lstnewenvironment{json}[1][]
  {\lstset{language=lJson,#1}}
{}

\pushList{RegisteredLanguages}{json/lJson}


% Main erlaubt das konfigurieren der Definitionen wie Farben etc. via pgfkeys :D

\def\lstPackkeys#1{%
  \pgfkeys{%
    /lillyxLISTINGS/packages/MAIN/.cd,#1%
  }
}
\def\lstPackget#1{\pgfkeysvalueof{/lillyxLISTINGS/packages/MAIN/#1}}

\pgfkeys{/lillyxLISTINGS/packages/MAIN/.is family,%
         /lillyxLISTINGS/packages/MAIN,%
        % Frame
        backgroundcolor/.initial={\color{lstmainbackcol}},%
        rulecolor/.initial={\color{lstmainbordercol}},%
        % frame/.initial={single},%
        framerule/.initial={1pt},%
        xleftmargin/.initial={15pt},%
        xrightmargin/.initial={3pt},%
        %Numbers
        %numbers/.initial={left},%
        numbersep/.initial={7pt},%
        numberstyle/.initial=\color{gray}\LILLYxLISTINGSxNUMxFONTSIZE\sffamily\@lstnumconsumer,%
        % Generic Configurations
        % breaklines/.initial={true},% Currently won't expand correctly
        % extendedchars/.initial={true},%
        prebreak/.initial={\raisebox{0.35\baselineskip}{\rotatebox{270}{\normalcolor\fontsize{6pt}{6pt}\selectfont$\curvearrowright$}}},%
        % core styles
        basicstyle/.initial={\LILLYxlstTypeWriter\color{black}\LILLYxLISTINGSxFONTSIZE},%
        stringstyle/.initial={\color{DarkChromeYellow}},%
        commentstyle/.initial={\color{gray}\LILLYxlstTypeWriter},%
        % Additional Definitions
        % otherkeywords/.initial={},%
        % alsoletter/.initial={@_},%
        % keywordsprefix/.initial={},%
        % Keywordstyles
        keywordstyle/.initial={\LILLYxCODExKEYWORDSTYLE},%
        keywordstyle 2/.initial={\color{Azure}},%
        keywordstyle 3/.initial={\color{AppleGreen}},%
        keywordstyle 4/.initial={\color{DatmouthGreen}\ifx\LILLYxMODE\LILLYxMODExPRINT\bfseries\fi},%
        keywordstyle 5/.initial={\color{Orchid}},%
        keywordstyle 6/.initial={\itshape}
}

%% Typesets values like the listings comment:D
\def\lstcomment#1{\bgroup\lstPackget{basicstyle}\lstPackget{commentstyle}#1\egroup}
\foreach \kw/\sig in {{keywordstyle}/one,{keywordstyle 2}/two,{keywordstyle 3}/three,{keywordstyle 4}/four,{keywordstyle 5}/five,{keywordstyle 6}/six}{
  \expandafter\xdef\csname lstkw\sig\endcsname##1{\bgroup\noexpand\lstPackget{basicstyle}\noexpand\lstPackget{\kw}##1\egroup}
}
\def\lststring#1{\bgroup\lstPackget{basicstyle}\lstPackget{stringstyle}#1\egroup}
\def\lstnumber#1{\bgroup\lstPackget{basicstyle}\digitstyle#1\egroup}


\lstdefinestyle{MAIN}{
    breaklines      = true,
    backgroundcolor = \lstPackget{backgroundcolor},
    rulecolor       = \lstPackget{rulecolor},
    stringstyle     = \lstPackget{stringstyle},
    keywordstyle    = \lstPackget{keywordstyle},
    keywordstyle    = [2]\lstPackget{keywordstyle 2}, %%KW
    keywordstyle    = [3]\lstPackget{keywordstyle 3}, %% PARAM
    keywordstyle    = [4]\lstPackget{keywordstyle 4}, %% FILES
    keywordstyle    = [5]\lstPackget{keywordstyle 5}, %% SHELLPRE
    keywordstyle    = [6]\lstPackget{keywordstyle 6}, %% SHELLPRE
    otherkeywords   = {},
    alsoletter      = {@},
    keywordsprefix  = {},
    basicstyle      = \lstPackget{basicstyle},   %Damit es auch wirklich ausschaut wie die Schreibmaschine!
    commentstyle    = \lstPackget{commentstyle},
    extendedchars   = true,
    numberstyle     = \lstPackget{numberstyle},
    prebreak        = \lstPackget{prebreak},
    frame           = single,
    xleftmargin     = \lstPackget{xleftmargin},
    xrightmargin    = \lstPackget{xrightmargin},
    numbers         = left,
    numbersep       = \lstPackget{numbersep},
    framerule       = \lstPackget{framerule},
    escapeinside={!*}{*!},
}


\ifx\LILLYxMODE\LILLYxMODExPRINT
  \lstPackkeys{backgroundcolor=\color{black!0}}
\fi

\lstset{%
    style=MAIN
}
