%%% Die Verwaltung der Listings verläuft auf verschiedene Highlighting-Module welche auch über die Kommandozeile deifniert werden können

\RequirePackage{listingsutf8}                                   %UTF8-Formatiertes Code-Highlighting



\lstset{literate=
  {á}{{\'a}}1 {é}{{\'e}}1 {í}{{\'i}}1 {ó}{{\'o}}1 {ú}{{\'u}}1
  {Á}{{\'A}}1 {É}{{\'E}}1 {Í}{{\'I}}1 {Ó}{{\'O}}1 {Ú}{{\'U}}1
  {à}{{\`a}}1 {è}{{\`e}}1 {ì}{{\`i}}1 {ò}{{\`o}}1 {ù}{{\`u}}1
  {À}{{\`A}}1 {È}{{\'E}}1 {Ì}{{\`I}}1 {Ò}{{\`O}}1 {Ù}{{\`U}}1
  {ä}{{\"a}}1 {ë}{{\"e}}1 {ï}{{\"i}}1 {ö}{{\"o}}1 {ü}{{\"u}}1
  {Ä}{{\"A}}1 {Ë}{{\"E}}1 {Ï}{{\"I}}1 {Ö}{{\"O}}1 {Ü}{{\"U}}1
  {â}{{\^a}}1 {ê}{{\^e}}1 {î}{{\^i}}1 {ô}{{\^o}}1 {û}{{\^u}}1
  {Â}{{\^A}}1 {Ê}{{\^E}}1 {Î}{{\^I}}1 {Ô}{{\^O}}1 {Û}{{\^U}}1
  {œ}{{\oe}}1 {Œ}{{\OE}}1 {æ}{{\ae}}1 {Æ}{{\AE}}1 {ß}{{\ss}}1
  {ű}{{\H{u}}}1 {Ű}{{\H{U}}}1 {ő}{{\H{o}}}1 {Ő}{{\H{O}}}1
  {ç}{{\c c}}1 {Ç}{{\c C}}1 {ø}{{\o}}1 {å}{{\r a}}1 {Å}{{\r A}}1
  {€}{{\euro}}1 {£}{{\pounds}}1 {«}{{\guillemotleft}}1
  {»}{{\guillemotright}}1 {ñ}{{\~n}}1 {Ñ}{{\~N}}1 {¿}{{?`}}1
} %Damit die Umlaute in lstlistings auch gescheit formatiert werden. Entnommen der Dokumentation: https://en.wikibooks.org/wiki/LaTeX/Source_Code_Listings


\providecommand{\LILLYxListingsxLang}{MAIN}
\DeclareRobustCommand{\LILLYxwriteLst}[2][style=bash]{\tikz[baseline=-0.6ex]{\node[rectangle, minimum height=0.95\baselineskip, inner sep=1pt,draw, rounded corners=4pt,MudWhite, fill=MudWhite, centered]{\,\vphantom{I}\lstinline[#1]!#2!\,};}}

\input{\LILLYxPATHxLISTINGS/Languages/_LILLY_LANG_\LILLYxListingsxLang}



\DeclareRobustCommand{\java}[1]{\LILLYxwriteLst[style=Java,morekeywords={System}]{#1}}
\DeclareRobustCommand{\pseudo}[1]{\LILLYxwriteLst[style=bash,language=pseudo]{#1}}

