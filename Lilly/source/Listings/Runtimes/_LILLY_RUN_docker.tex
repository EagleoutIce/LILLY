
% Diese Datei liefert einen Befehl mit folgenden Eigenschaften:
% Laden einer Shell-Datei
% Ausführen dieser Shell-Datei mit einem angebenen Interpreter
% Evaluation zu Ergebnis der Ausführung
\lstdefinestyle{DockerOutput}{
  basicstyle=\LILLYxLISTINGSxFONTSIZE\LILLYxlstTypeWriter,
  frame=tblr,numbers=none,frame=single,
  xleftmargin=3pt,xrightmargin=3pt,
  backgroundcolor=\color{black!0},
  rulecolor=\color{lstmainbordercol}
}
\gdef\rdocker{\@ifnextchar[{\rdocker@opt}{\rdocker@noopt}}
\gdef\rdocker@opt[#1]#2{
  \cdocker{#2}~#1% TODO: configure layout with #1
  \getBashFileExecutionOutput{#2}
}
\gdef\rdocker@noopt#1{\cdocker{#1}\,:\getBashFileExecutionOutput{#1}}

\def\lstrdockerlistopts{}
% oldies are goldies :D
\ifx\LILLYxCMD\undefined
% {file} (coming soon: [input])
\gdef\getDockerFileExecutionOutput#1{%
\immediate\write18{ #1 > \jobname-docker.txt 2> \jobname-docker.txt} % safest way :D
\lstinputlisting[style=DockerOutput,language=lDocker]{\jobname-docker.txt}
\immediate\write18{ rm \jobname-docker.txt }
}
\else
\gdef\getDockerFileExecutionOutput#1{Direkte Befehlsevaluation deaktiviert!}
\fi

\gdef\previewDocker#1{%
\begingroup
\colorbox{MudWhite!93!black}{\lstinline[language=lDocker]{docker #1}}
\getDockerFileExecutionOutput{docker #1}
\endgroup}
\NewDocumentCommand{\previewDockerFile}{ m } {
\begingroup
\lstinputlisting[language=lDocker, caption={Inhalt der Datei #1}]{#1}
\endgroup
\previewBash{docker #1}}