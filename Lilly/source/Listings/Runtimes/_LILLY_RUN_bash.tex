
%% Diese Datei liefert einen Befehl mit folgenden Eigenschaften:
% Laden einer Shell-Datei
% Ausführen dieser Shell-Datei mit einem angebenen Interpreter
% Evaluation zu Ergebnis der Ausführung

\lstdefinestyle{BashOutput}{
  basicstyle=\LILLYxLISTINGSxFONTSIZE\LILLYxlstTypeWriter,
  frame=tblr,
  numbers=none,
  frame=single,
  xleftmargin=3pt,
  xrightmargin=3pt,
  backgroundcolor=\color{black!0},
  rulecolor=\color{lstmainbordercol}
}

\gdef\rbash{\@ifnextchar[{\rbash@opt}{\rbash@noopt}}
\gdef\rbash@opt[#1]#2{
  \cbash{#2}~#1%% TODO: configure layout with #1
  \getBashFileExecutionOutput{#2}
}
\gdef\rbash@noopt#1{\cbash{#1}\,:\getBashFileExecutionOutput{#1}}

\def\lstrbashlistopts{}
%% oldies are goldies :D
\ifx\LILLYxCMD\undefined
%% {file} (coming soon: [input])
\gdef\getBashFileExecutionOutput#1{%
\immediate\write18{ #1 > \jobname.txt 2> \jobname.txt} % safest way :D
\lstinputlisting[style=BashOutput,language=lBash,keywords={[4]{}}]{\jobname.txt}
\immediate\write18{ rm \jobname.txt }
}
\else
\gdef\getBashFileExecutionOutput#1{Direkte Befehlsevaluation deaktiviert!}
\fi

\gdef\previewBash#1{%
\begingroup
\colorbox{MudWhite!93!black}{\lstinline[language=lBash]{$ #1}}
\getBashFileExecutionOutput{#1}
\endgroup
}



\NewDocumentCommand{\previewBashFile}{ m } {
\begingroup
\lstinputlisting[language=lBash, caption={Inhalt der Datei #1}]{#1}
\endgroup
\previewBash{bash #1}
}