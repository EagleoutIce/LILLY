
%% Diese Datei liefert einen Befehl mit folgenden Eigenschaften:
% Laden einer Shell-Datei
% Ausführen dieser Shell-Datei mit einem angebenen Interpreter
% Evaluation zu Ergebnis der Ausführung

\lstdefinestyle{HaskellOutput}{
  basicstyle=\small\LILLYxlstTypeWriter,
  frame=tblr,
  numbers=none,
  frame=single,
  xleftmargin=3pt,
  xrightmargin=3pt,
  backgroundcolor=\color{MudWhite},
  rulecolor=\color{MudWhite}
}

\gdef\rhaskell{\@ifnextchar[{\rhaskell@opt}{\rhaskell@noopt}}
\gdef\rhaskell@opt[#1]#2{
  \chaskell{print (#2)}#1\newline%% TODO: configure layout with #1
  \getHaskellFileExecutionOutput{echo 'main = print (#2)' > \jobname.hs && runhaskell \jobname.hs}
}
\gdef\rhaskell@noopt#1{\chaskell{print (#1)}:\getHaskellFileExecutionOutput{echo 'main = print (#1)' > \jobname.hs && runhaskell \jobname.hs}}


%% oldies are goldies :D
\ifx\LILLYxCMD\undefined
%% {file} (coming soon: [input])
\gdef\getHaskellFileExecutionOutput#1{%
\immediate\write18{ #1 > \jobname.txt }
\lstinputlisting[style=HaskellOutput,language=lHaskell]{\jobname.txt}
\immediate\write18{ rm \jobname.txt }
%\immediate\write18{ rm \jobname.hs }
}
\else
\gdef\getHaskellFileExecutionOutput#1{Direkte Befehlsevaluation deaktiviert!}
\fi

\gdef\previewHaskell#1{%
\begingroup
\colorbox{MudWhite!93!black}{\lstinline[language=lHaskell]{runhaskell #1}}
\getHaskellFileExecutionOutput{#1}
\endgroup
}



\NewDocumentCommand{\previewHaskellFile}{ m } {
\begingroup
\lstinputlisting[language=HaskellOutput, caption={Inhalt der Datei #1}]{#1}
\endgroup
\previewHaskell{runhaskell #1}
}