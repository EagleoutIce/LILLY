\chapter[Farben \LILLYxBOXxVersion{\small 1.0.4}]{Farben}
\TitleSUB{Viele viele bunte Farben \hfill \LILLYxBOXxVersion{\small 1.0.4}}
{\centering \framebox{Alle folgenden Pfade sind relativ zu \T{Data/\ldots}}\vspace*{0.5\baselineskip}\par}
Im folgenden wird beschrieben wie grundlegend die Einbettung eines neuen Farbprofils ab \LILLYxBOXxVersion{\small 1.0.4} funktioniert. Bitte beachte, dass vor dieser Version ein Farbprofil noch alle Farben überschreiben und liefern musste, während seit dieser Version mit dem Überschreiben der Standard-Farben gearbeitet wird. Wichtig ist:\begin{center}
    \small\bfseries Jedes Farbprofil kann eigene Farben hinzufügen - hiervon wird aber stark abgeraten, da somit nichtmehr die Design-Unabhängigkeit von LILLY garantiert ist!
\end{center}
\reversemarginpar
\section{Die normalen Farbprofile}
\subsection{Das Standardfarbprofil}
{\centering \framebox{Diese Definitionen befinden sich in der Datei: \T{Colors/\_LILLY\_DEFAULT\_COLORPROFILE}}\vspace*{0.5\baselineskip}\par}
Dieses Farbprofil wird nur geladen, wenn die Variable \CMDpreview{LILLYxColorxInject} nicht definiert ist. Es selbst bindet die Pakete \T{xcolor} und \T{pgf} (?) mit ein und definiert seinerseits eine Menge Farben, die zum einen einfach gut aussehen \smiley{} und zum anderen von Einzelpersonen gewünscht wurden. Alle Farben, welche LILLY darüber zur Verfügung gestellt bekommt, werden im folgenden aufgelistet:  %% define colordefault preview:
\newcommand{\csXshow}[2][]{\tikz{\draw[fill=#2,#1] (0,0) circle (0.15);}}
\newcommand{\csXcolor}[4]{#1\({}^{~(r:~#2,~g:~#3,~b:~#4)}\)}
\begin{multicols}{2}
    \begin{itemize}[label=$\diamond$]\narrowitems
        \foreach \c/\r/\g/\b in {Aureolin/253/238/0,
                                 Amber/255/191/0,
                                 ChromeYellow/255/167/0,
                                 Coquelicot/255/56/0,
                                 Cinnabar/227/66/52,
                                 BrightMaroon/195/33/72,
                                 Cherry/222/49/99,
                                 AlizarinCrimson/227/28/54,
                                 Amaranth/229/43/80,
                                 AmericanRose/255/3/62,
                                 Awesome/255/32/82,
                                 BrightPink/255/0/127,
                                 DebianRed/215/10/83,
                                 Crimson/220/20/60,
                                 DarkMidnightBlue/0/51/102,
                                 Azure/0/127/255,
                                 bondiBlue/0/149/182,
                                 antiVeg/190/238/239,
                                 DarkOrchid/104/34/139,
                                 Veronica/180/82/205,
                                 Amethyst/153/102/204,
                                 AntiqueFuchsia/145/92/131,
                                 BritishRacingGreen/0/66/37,
                                 DatmouthGreen/0/105/62,
                                 Ao/0/128/0,
                                 AppleGreen/141/82/0, 
                                 BrightGreen/102/255/0,
                                 LightGray/224/224/224,
                                 AuroMetalSaurus/110/127/128,
                                 Charcoal/54/69/79} {
            \item[\csXshow{\c}] \csXcolor{\c}{\r}{\g}{\b}
        }
    \end{itemize}
\end{multicols}
\begin{bemerkung}[Kompatibilität]
Weiter gibt es die folgenden Farben, welche aus Kompatibilitätsgründen aus dem \T{eagleStudiPackage} übernommen wurden:

\begin{multicols}{2}
    \begin{itemize}[label=$\diamond$]\narrowitems
        \foreach \c/\r/\g/\b in {dpurple/86/60/92,
                                 ddpurple/128/0/128,
                                 beauty/104/55/107,
                                 candypink/227/112/122,
                                 thered/255/47/47,
                                 dorange/255/102/0,
                                 mint/255/128/0,
                                 gold/255/215/50,
                                 dgold/235/198/13,
                                 limegreen/51/204/51,
                                 skyblue/60/179/113,
                                 tealblue/51/153/255,
                                 superlightgray/240/240/240
                                 } {
            \item[\csXshow{\c}] \csXcolor{\c}{\r}{\g}{\b}
        }
    \end{itemize}
\end{multicols}

Sie sollten nicht mehr verwendet werden!
\end{bemerkung}
\newcommand{\csXcslave}[3][Color]{\T{#1x#2}${}^{~(#3)}$}

Weiter definiert dieses Farbprofil die Farben, welche LILLY für Links, Boxen usw. verwenden soll. Alle diese Befehle sollten auch bei eigenen Implementationen und Erweiterungen angewendet werden, darum folgt hier eine Auflistung für die Boxen (alle Befehle beginnen mit \T{\textbackslash LILLYx}):
\begin{multicols}{2}
    \begin{itemize}[label=$\diamond$]\narrowitems
        \foreach \c/\l/\r in {Definition/DebianRed/Color,
                           Satz/Ao/Color,
                           Beweis/DarkMidnightBlue/Color,
                           Lemma/DarkMidnightBlue/Color,
                           Bemerkung/Charcoal/Color,
                           Zusammenfassung/ChromeYellow/Color,
                           Beispiel/Aureolin/Color,
                           Uebungsaufgabe/Veronica/Color,
                           Zusatzuebung/Veronica/Color,
                           MainColor/DebianRed!85!black/LINKS,
                           CiteColor/DarkMidnightBlue/LINKS,
                           UrlColor/DarkMidnightBlue/LINKS,
                           COLOR/DebianRed/TITLE%
                           } {
            \item[\csXshow{\l}] \csXcslave[\r]{\c}{\l}
        }
    \end{itemize}
\end{multicols}
Weiter gibt es noch die Farbe: \CMDshow{LILLYxLINKSxMainColorDarker} (\csXshow{\LILLYxLINKSxMainColorDarker}). Sie wird gemäß: \T{\textbackslash LILLYxLINKSxMainColor!90!black} generiert. \newline
Beispielhaft lässt sich die Definitionsfarbe mit: \CMDshow{LILLYxColorxDefinition} abfragen (\csXshow{\LILLYxColorxDefinition})

\subsection{Das Druckprofil}
{\centering \framebox{Diese Definitionen befinden sich in der Datei: \T{Colors/\_LILLY\_PRINT\_COLORPROFILE}}\vspace*{0.5\baselineskip}\par}
Auch dieses Profil definiert seine Farben nur, wenn \T{\textbackslash LILLYxColorxInject} nicht definiert ist!
Aus Kompatibilitätsgründen zu Versionen vor \LILLYxBOXxVersion{\small 1.0.4} definiert dieses Profil grundlegend die gleichen Farben wie das Standardfarbprofil. Hier wird nur auf die alternativen LILLY-Farben eingegangen, da sich die Profile nur hierin unterscheiden:
\begin{multicols}{2}
    \begin{itemize}[label=$\diamond$]\narrowitems
        \foreach \c/\l/\r in {Definition/DebianRed/Color,
                           Satz/Charcoal/Color,
                           Beweis/Charcoal/Color,
                           Lemma/Charcoal/Color,
                           Bemerkung/Charcoal/Color,
                           Zusammenfassung/ChromeYellow/Color,
                           Beispiel/Charcoal/Color,
                           Uebungsaufgabe/Charcoal/Color,
                           Zusatzuebung/Charcoal/Color,
                           MainColor/Charcoal/LINKS,
                           CiteColor/Charcoal/LINKS,
                           UrlColor/Charcoal/LINKS,
                           COLOR/DebianRed/TITLE%
                           } {
            \item[\csXshow{\l}] \csXcslave[\r]{\c}{\l}
        }
    \end{itemize}
\end{multicols}
Die Farbe \CMDshow{LILLYxLINKSxMainColorDarker} (\csXshow{\LILLYxLINKSxMainColorDarker}) wird hier mithilfe von: \T{\textbackslash LILLYxLINKSxMainColor!95!black} generiert.

\section{Weitere Planungen}
\begin{itemize}[label=$\diamond$]\narrowitems
    \item Elysium \LILLYxNOTExWarning{Ausstehend}
    \item Besseres Druckprofil \LILLYxNOTExWarning{Ausstehend}
    \item Weitere Farben \LILLYxNOTExWarning{Ausstehend} - Generische Farben wie \say{Rot} auch als Befehl - zudem Lösung für Druckversion, sodass nirgendwo steht - der \say{Rote Kreis} - wenn er dann eigentlich schwarz ist.
\end{itemize}
