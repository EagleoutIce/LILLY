\chapter[Mathe\lilib{LILLYxMATH}{1.0.0}]{Mathe}
\TitleSUB{Einzelne Variationen und eine Menge Abkürzungen \hfill \LILLYxBOXxVersion{\small 2.0.0}}
\bigskip\newline
\renewcommand{\arraystretch}{1.5}
\elable{chp:MATH}\hypertarget{LILLYxMATH}An sich ändert LILLY nicht viel an der normalen Implementation der Mathewelt. Dieses Paket liegt hier: \begin{center}
    \blankcmd{LILLYxPATHxMATHS} = \T{\LILLYxPATHxMATHS}
\end{center}

%
%
%

\presentCommand[1.0.3]{LILLYxMathxMode} %% Overbar
Der verwendete Mathemodus lässt sich mithilfe des Befehls \clatex[morekeywords={[5]{LILLYxMathxMode}}]{:bs:LILLYxMathxMode} frei einstellen. Standardmäßig wird dieser Wert auf \emph{normal} gesetzt.
\begin{bemerkung}[Standalone-Math]
    Mit \LILLYxBOXxVersion{2.0.0} wurde die Mathe-Integration als eigenes Paket \LILLYxNOTExLibrary{LILLYxMATH} etabliert, welches sich eigenständig über \begin{latex*}
\usepackage{LILLYxMATH}
        \end{latex*}
        auch ohne das Verwenden der restlichen LILLY-Welt benutzen lässt.
\end{bemerkung}

\section{Weitere Befehle}

\subsection[Operatoren \LILLYxBOXxVersion{\small 1.0.3}]{Operatoren}
Diese Definitionen befinden sich in der Datei: \T{Maths/\_LILLY\_MATHS\_OPERATORS}. Sie werden mit \LILLYxBOXxVersion{2.0.0} automatisch mit dem Einbinden von \LILLYxNOTExLibrary{LILLYxMATH} geladen.

%
%
%

\presentCommand[1.0.3]{overbar}[\manArg{text}] %% Overbar
Lilly liefert den Befehl auf Basis von \T{mkern} so, dass er direkt Abstände zwischen den Overlines definiert, sodass kein manueller Abstand eingefügt werden muss. So ergibt sich:
\begin{center}
    \begin{tabular}{!{\VRule[1pt]}@{\hspace{1em}}l@{\hspace{1em}}|@{\hspace{1em}}c@{\hspace{1em}}!{\VRule[1pt]}}
        \specialrule{1pt}{0pt}{0pt}
         {\blatex[morekeywords={[5]{overbar}}]{:bs:overbar\{a\_1\} :bs:overbar\{a\_2\}}} & \(\overbar{a_1}\overbar{a_2}\)\\\hline
        {\blatex[morekeywords={[5]{overline}}]{:bs:overline\{a\_1\} :bs:overline\{a\_2\}}} & \(\overline{a_1}\overline{a_2}\)\\
        \specialrule{1pt}{0pt}{0pt}
    \end{tabular}
\end{center}

%
%
%

\presentCommand[1.0.3]{das}[\cmdlist \anothercmd[1.0.3]{sad}\cmdlist \anothercmd[1.0.3]{daseq}\cmdlist \anothercmd[1.0.3]{qesad}\cmdlist \anothercmd[1.0.6]{shouldeq}]
Für Definitionen gibt es die Befehle \blankcmd{das} ($\das$), \blankcmd{sad} ($\sad$), \blankcmd{daseq} ($\daseq$), \blankcmd{qesad} ($\qesad$) sowie \blankcmd{shouldeq} ($\shouldeq$). All diese Befehle funktionieren sowohl in einer Matheumgebung, das auch im normalen text, sie werden mit \verb|\ensuremath| abgesichert!\par
Bis auf den letzten werden zudem alle Befehle mithilfe von \verb|\vcentcolon| realisiert.

%
%
%

\presentCommand[1.0.3]{sqrt}[\optArg{n}\manArg{math-Ausdruck}]
Weiter wurde das Aussehen der Wurzel verändert und die Möglichkeit hinzugefügt, über das optionale Argument \say{\T{n}} höhere Wurzeln zu Formulieren, wir erhalten folgendes:
\begin{center}
    \begin{tabular}{!{\VRule[1pt]}@{\hspace{1em}}l@{\hspace{1em}}|@{\hspace{1em}}c@{\hspace{1em}}!{\VRule[1pt]}}
        \specialrule{1pt}{0pt}{0pt}
        {\blatex[morekeywords={[5]{sqrt}}]{:bs:sqrt[3]\{42\}}} & \(\sqrt[3]{42}\)\\\hline
        {\blatex[morekeywords={[5]{oldsqrt}}]{:bs:oldsqrt[3]\{42\}}} & \(\oldsqrt[3]{42}\)\\
        \specialrule{1pt}{0pt}{0pt}
    \end{tabular}\smallskip
\end{center}

%% todo error patterns:
%% error, missing $, improper alphabetic constant, ! Too many }'s (in general, lines starting with '!')
%
%
%

\presentCommand[1.0.3]{det}[\cmdlist \anothercmd[1.0.3]{adj}\cmdlist \anothercmd[1.0.3]{LH}\cmdlist \anothercmd[1.0.3]{eig}\cmdlist \anothercmd[1.0.3]{Dim}\cmdlist \anothercmd[1.0.3]{sel}\cmdlist \anothercmd[1.0.3]{sign}\cmdlist \anothercmd[1.0.3]{diag}\cmdlist \anothercmd[1.0.3]{LK}\cmdlist \anothercmd[1.0.3]{rg}\cmdlist \anothercmd[1.0.3]{KER}\cmdlist \anothercmd[1.0.3]{Eig}]
Diese vereinfachenden Operatoren solles es ermöglichen Schneller verschiedene mathematische Operatoren zu setzen
\displayCommandList[3]{det,adj,LH,eig,Dim,sel,sign,diag,LK,rg,KER,Eig}

%
%
%

\presentCommand[1.0.2]{Im}[\cmdlist \anothercmd[1.0.2]{mod}\cmdlist \anothercmd[1.0.2]{Re}\cmdlist \anothercmd[1.0.2]{emptyset}]
Auch wurde das Aussehen von \verb|\mod|, \verb|\Im|, \verb|\Re| und \verb|\emptyset| modifiziert:
\displayCommandList{mod,Im,Re,emptyset}

%
%
%

\presentCommand[1.0.6]{inf}[\cmdlist \anothercmd[1.0.6]{sup}\cmdlist \anothercmd[1.0.6]{min}\cmdlist \anothercmd[1.0.6]{max}]
Auch hierbei handelt es sich wieder um stupide Abbildungen im Operator-Style:
\displayCommandList{inf,sup,min,max}

%
%
%

\presentCommand[1.0.9]{abs}[\manArg{math-Ausdruck}]
Dieser Befehl vereinfacht das Schreiben von Betragsstrichen. Diese passen sich zudem automatisch an die vertikalen Dimensionen des Ausdrucks an:
\begin{center}
    \begin{tabular}{!{\VRule[1pt]}@{\hspace{1em}}C{7cm}@{\hspace{1em}}|@{\hspace{1em}}C{2cm}@{\hspace{1em}}!{\VRule[1pt]}}
        \specialrule{1pt}{0pt}{0pt}
        {\blatex[morekeywords={[5]{abs,frac,log,pi}}]{:bmath::bs:abs\{:bs:frac\{:bs:pi-x^2\}\{:bs:log 3x\}\}:emath:}} & \(\displaystyle\abs{\frac{\pi - x^2}{\log 3x}}\)\\\hline
        {\blatex[morekeywords={[5]{abs,frac,log,pi}}]{:bmath:|:bs:frac\{:bs:pi-x^2\}\{:bs:log 3x\}|:emath:}} & \(\displaystyle|\frac{\pi - x^2}{\log 3x}|\)\\
        \specialrule{1pt}{0pt}{0pt}
    \end{tabular}\smallskip
\end{center}

%
%
%

\presentEnvironment[1.0.2]{matrix}[\optArg{Spaltendefinition}\cmdlist \anotherenv[1.0.0]{pmatrix}\optArg{Spaltendefinition}]
Des Weiteren wurde noch die Matrixumgebung (\verb|\env@matrix|) so erweitert, dass sie als optionales Argument eine gültige Array-Spaltendefinition entgegennimmt:
\begin{center}\renewcommand{\arraystretch}{0.75}
    \begin{tabular}{!{\VRule[1pt]}@{\hspace{1em}}C{5.5cm}@{\hspace{1em}}|@{\hspace{1em}}C{2cm}@{\hspace{1em}}!{\VRule[1pt]}}
        \specialrule{1pt}{0pt}{0pt}
{\begin{latex}
:bmath:\begin{pmatrix}[cc|c]
    1 & 2 & 3 \\
    4 & 5 & 6
\end{pmatrix}:emath:
\end{latex}}
&{$\begin{pmatrix}[cc|c]
            1 & 2 & 3 \\
            4 & 5 & 6
        \end{pmatrix}$}\\
        \specialrule{1pt}{0pt}{0pt}
    \end{tabular}
\end{center}

%
%
%

\presentCommand[1.0.8]{val}[\cmdlist \anothercmd[1.0.8]{sch}\cmdlist \anothercmd[1.0.8]{dom}\cmdlist \anothercmd[1.0.8]{grad}]
Auch hier handelt es sich um weitere Mathe-Operatoren, die selbstredend implementiert werden:
\displayCommandList{val,sch,dom,grad}

%
%
%

\presentCommand[1.0.8]{arccot}
Da der ach so wichtige Arkuskotangens erstaunlicherweise nicht standardmäßig dabei ist, hier: \blankcmd{arccot} ($\arccot$).

%
%
%

\presentCommand[2.0.0]{dif}[\cmdlist \anothercmd[2.0.0]{dint}\optArg[x]{Variable}]
Auch hierbei handelt es sich wieder um stupide Abbildungen im Operator-Style für Integration und Differenzierung
\displayCommandList{dif,dint}








\subsection[Symbole \LILLYxBOXxVersion{\small 1.0.3}]{Symbole}
Diese Definitionen befinden sich in der Datei: \T{Maths/\_LILLY\_MATHS\_SYMBOLS}. Sie werden mit \LILLYxBOXxVersion{2.0.0} automatisch mit dem Einbinden von \LILLYxNOTExLibrary{LILLYxMATH} geladen.

%Weitere (nicht nennenswert und vermutlich\cmdold) sind die Befehle: \CMDpreview{x} (\(\x\)), \CMDpreview{y} (\(\y\)) und \CMDpreview{z} (\(\z\))
%\normalmarginpar

\presentCommand[1.0.0]{N}[\cmdlist \anothercmd[1.0.0]{Z}\cmdlist \anothercmd[1.0.0]{Q}\cmdlist \anothercmd[1.0.0]{R}\cmdlist \anothercmd[1.0.0]{C}]
Für die einzelnen Zahlenräume werden einige Befehle zur Verfügung gestellt, die alle über \verb|\ensuremath| abgesichert sind:
\blankcmd{N} ($\N$), \blankcmd{Z} ($\Z$), \blankcmd{Q} ($\Q$), \blankcmd{R} ($\R$), \blankcmd{C} ($\C$). Sie werden mithilfe von \verb|\mathbb| generiert.

%
%
%

\presentCommand[1.0.1]{i}
Die komplexe Einheit \i{} wird mit \blankcmd{i} zur Verfügung gestellt.

%
%
%

\presentCommand[1.0.3]{epsilon}[\cmdlist \anothercmd[1.0.3]{phi}]
Weiter wurden die griechischen Buchstaben Epsilon und Phi modifiziert:
\begin{center}
    \begin{tabular}{*{2}{!{\VRule[1pt]}@{\hspace{1em}}l@{\hspace{1em}}|@{\hspace{1em}}c@{\hspace{1em}}}!{\VRule[1pt]}}
        \specialrule{1pt}{0pt}{0pt}
        \T{\CMDshow{oldepsilon}} & \(\oldepsilon\) & \T{\CMDshow{epsilon}} & \(\epsilon\) \\\hline
        \T{\CMDshow{oldphi}} & \(\oldphi\)& \T{\CMDshow{phi}} & \(\phi\) \\
        \specialrule{1pt}{0pt}{0pt}
    \end{tabular}
\end{center}

%
%
%

\presentCommand[1.0.3]{B}[\cmdlist \anothercmd[1.0.3]{X}\cmdlist \anothercmd[1.0.3]{K}\cmdlist \anothercmd[1.0.3]{P}\cmdlist \anothercmd[1.0.3]{F}\cmdlist \anothercmd[1.0.3]{O}]
Zudem wird zum Beispiel die Menge der Binärzahlen über \blankcmd{B} ($\B$), die chromatische Zahl über \blankcmd{X} ($\X$) und der generelle Körper mit \blankcmd{K} ($\K$) zur Verfügung gestellt. Für die Potenzmenge liefert LILLY \blankcmd{P} ($\P$), für die Menge der Funktionen \blankcmd{F} ($\F$) und für die Groß-O-Notation \blankcmd{O} ($\O$).

%
%
%

\presentCommand[2.0.0]{join}[\cmdlist \anothercmd[2.0.0]{leftouterjoin}\cmdlist \anothercmd[2.0.0]{rightouterjoin}\cmdlist \anothercmd[2.0.0]{fullouterjoin}]
Da auch die Relationenalgebra Teil der Mathematik ist, hier die entsprechenden Symbole für die Joins:
\displayCommandList{join,leftouterjoin,rightouterjoin,fullouterjoin}
%
%
%
\begin{bemerkung}[Weitere Symbole]
    Weiter bindet LILLY das \T{\href{http://ctan.math.utah.edu/ctan/tex-archive/macros/latex/required/psnfss/psnfss2e.pdf}{pifont}} Paket ein und liefert so zum Beispiel \verb|\ding{51}| (\ding{51}) und \verb|\ding{55}| (\ding{55}).
\end{bemerkung}









\subsection[Kompatibilität \LILLYxBOXxVersion{\small 1.0.3}]{Kompatibilität}
Diese Definitionen befinden sich in der Datei: \T{Maths/\_LILLY\_MATHS\_COMPATIBILITIES}. Sie werden mit \LILLYxBOXxVersion{2.0.0} automatisch mit dem Einbinden von \LILLYxNOTExLibrary{LILLYxMATH} geladen.\medskip

Hier werden einige Befehle eingerichtet, die entweder noch nicht zugeordnet wurden\({}^{\tiny \LILLYxBOXxVersion{\tiny 2.0.0}}\) oder während der Vorlesung (im Überlebenskampf :P) ins \T{eagleStudiPackage} eingebaut worden sind.

%
%
%

\presentCommand[1.0.0]{enum}[\manArg{items}\cmdlist \anothercmd[1.0.0]{liste}\manArg{items}]
Hier befinden sich die für \la kreierten: \T{\blankcmd{enum}\manArg{items}} (\verb|enumerate| mit \newline\blankcmd{narrowitems}) und \T{\blankcmd{liste}\manArg{items}} (\verb|enumerate| mit römischen Zahlen und \newline\blankcmd{narrowitems}).

%
%
%

\presentCommand[1.0.1]{xa}[\cmdlist \anothercmd[1.0.1]{xb}\cmdlist \anothercmd[1.0.1]{xc}]
Weiter existieren die Befehle \blankcmd{xa} ($\xa$), \blankcmd{xb} ($\xb$), \blankcmd{xc} ($\xc$), welche einen etwas größeren Abstand für eine bessere Lesbarkeit einfügen.

%
%
%


\presentCommand[1.0.1]{crossAT}[\manArg{(PosX,PosY)}\cmdlist \anothercmd[1.0.1]{circAT}\manArg{(PosX,PosY)}\cmdlist \anothercmd[1.0.1]{bblock}\manArg{(PosX,PosY)}\manArg{text}]
Für Ti\textit{k}Z gibt es noch die Befehle \T{\blankcmd{crossAT}}\manArg{(PosX,PosY)} (\raisebox{-0.35\baselineskip}{\tikz{\crossAT{(0,0)};}}\footnote{\T{\blankcmd{tikz}\{\blankcmd{crossAT}\{(0,0)\};\}} -- Zum Erhalt der Textzeile vertikal um \T{-0.35\textbackslash baselineskip} verschoben.}) und analog\newline \T{\blankcmd{circAT}\manArg{(PosX,PosY)}} (\tikz{\circAT{(0,0)};} \footnote{\T{\blankcmd{tikz}\{\blankcmd{circAT}\{(0,0)\};\}}}), sowie \T{\blankcmd{bblock}\manArg{(PosX,PosY)}\manArg{text}} (\raisebox{-0.2\baselineskip}{\tikz{\bblock{(0,0)}{42};}} \footnote{\T{\blankcmd{tikz}\{\blankcmd{bblock}\{(0,0)\}\{42\};\}} -- Wieder vertikal um \T{-0.2\textbackslash baselineskip} verschoben.}). Hier fragt man sich nun vielleicht, warum diese nicht in einem entsprechenden Ti\textit{k}Z-Paket sind. Im Rahmen der mit \LILLYxBOXxVersion{2.0.0} eingeführten Modularisierung hat sich diese Verteilung als günstig erwiesen. %% vielleicht doch verschieben?

%
%
%

\presentEnvironment[1.0.2]{nstabbing}[\cmdlist \anotherenv[1.0.2]{centered}\cmdlist \anotherenv[1.0.2]{sqcases} \cmdold]
Weiter werden drei (mittlerweile obsolete) Umgebungen definiert: \begin{ditemize}\narrowitems
    \item \blankenv{nstabbing}:  \verb|tabbing|-Umgebung, ohne Abstände
    \item \blankenv{centered}: \verb|center|-Umgebung, ohne Abstände
    \item \blankenv{sqcases}: Ähnelt \verb|cases| - nur mit '\T{]}'.
\end{ditemize}

%
%
%

\presentCommand[1.0.4]{VRule}[\manArg{width}]
Zudem definiert sich noch für Tabellen der Befehl \T{\blankcmd{VRule}\manArg{width}}, welcher eine Spalte variabler Größe für Tabellen zur Verfügung stellt. Eine exemplarische Verwendung findet sich hier:
\begin{center}
    \begin{tabular}{!{\VRule[1pt]}@{\hspace{1em}}C{8cm}@{\hspace{1em}}|@{\hspace{1em}}c@{\hspace{1em}}!{\VRule[1pt]}}
        \specialrule{1pt}{0pt}{0pt}
        {\begin{latex}
\begin{tabular}{c!{\VRule[6pt]}c}
    \specialrule{2pt}{0pt}{0pt}
    You!*'*!re my & Wonder Wall\\
    \specialrule{2pt}{0pt}{0pt}
\end{tabular}
        \end{latex}
}&  {
            \begin{tabular}{c!{\VRule[6pt]}c}
            \specialrule{2pt}{0pt}{0pt}
            You're my & Wonder Wall\\
            \specialrule{2pt}{0pt}{0pt}\end{tabular}
            }\\
        \specialrule{1pt}{0pt}{0pt}
    \end{tabular}
\end{center}
%
%
%
\presentCommand[1.0.0]{trenner}[\cmdold]
Fügt einen großen senkrechten Strich ein: \blankcmd{trenner} ($\trenner$).

%% MOVED TO LILLYxTABLES ! TODO
% \begin{bemerkung}[Tabellenspalten]
%     Weiter gibt es noch einige verschiedene Tabellen-Spalten, deren Kurzbezeichner den Anschein erwecken wild zusammengewürfelt zu sein: \begin{multicols}{2}
%         \begin{ditemize}\narrowitems
%             \item \verb|b|: Fettgedruckt zentriert
%             \item \verb|u|: Mathematisch zentriert
%             \item \verb|g|: Fußnotengröße linksbündig
%             \item \verb|w|: Fußnotengröße linksbündig (X-Spalte)
%             \item \verb|L|\manArg{width}: Linksbündig mit Breite \emph{width}
%             \item \verb|C|\manArg{width}: Zentriert mit Breite \emph{width}
%             \item \verb|R|\manArg{width}: Rechtsbündig mit Breite \emph{width}
%         \end{ditemize}
%     \end{multicols}
% \end{bemerkung}







\subsection[Shortcuts \LILLYxBOXxVersion{\small 1.0.8}]{Shortcuts}
Diese Definitionen befinden sich in der Datei: \T{Maths/\_LILLY\_MATHS\_SHORTCUTS}. Sie werden mit \LILLYxBOXxVersion{2.0.0} automatisch mit dem Einbinden von \LILLYxNOTExLibrary{LILLYxMATH} geladen.\medskip\newline
Hier befinden sich einige abkürzende Befehle, die primär das Schreiben beschleunigen sollen. Sie werden auf Bedarf stetig erweitert.

%
%
%

\presentCommand[1.0.7]{folge}[\optArg[a]{Folgenglied}]
Setzt eine Folge, welche mit dem Index $n$ arbeitet: \blankcmd{folge} (\folge).

%
%
%

\presentCommand[1.0.7]{reihe}[\optArg[a\_k]{Folgenglied}\optArg[0]{Start}]
Setzt eine Reihe über die Glieder \emph{Folgenglied} an \emph{Start}: \blankcmd{reihe} (\reihe)

%
%
%

\presentCommand[1.0.8]{obda}[\cmdlist \anothercmd[1.0.8]{Obda}]
Schreibt entsprechend \T{o.B.d.A} (\blankcmd{obda}) und \T{O.B.d.A.} (\blankcmd{Obda}) und beschleunigt damit das Tippen von Beweisen \Smiley.

%
%
%

\presentCommand[1.0.7]{gdw}[\cmdlist \anothercmd[1.0.7]{limn}\cmdlist \anothercmd[1.0.7]{sumn}\cmdlist \anothercmd[1.0.7]{limk}\cmdlist \anothercmd[1.0.7]{sumk}]
Setzt verschiedene mathematische Ausdrücke:
\displayCommandList[3]{gdw,limn,sumn,limk,sumk}

%
%
%

\presentCommand[1.0.2]{x}[\optArg[$\sim$]{spacing}\cmdlist \anothercmd[1.0.2]{y}\optArg[$\sim$]{spacing}\cmdlist \anothercmd[1.0.2]{z}\optArg[$\sim$]{spacing} \cmdold]
Setzt entsprechend: \x[], \y und \z[].

%
%
%

\presentCommand[2.0.0]{ceil}[\optArg{math-Ausdruck}\cmdlist \anothercmd[2.0.0]{floor}\optArg{math-Ausdruck}]
Verkürzt das Schreiben von: \clatex[morekeywords={[5]{left,right,lfloor,rfloor}}]{:bs:left:bs:lfloor\<Ausdruck\>:bs:right:bs:rfloor} beziehungsweise \clatex[morekeywords={[5]{lceil}}]{:bs:lceil} \& \clatex[morekeywords={[5]{rceil}}]{:bs:rceil} entsprechend:
\begin{multicols}{2}%
    \begin{ditemize}
        \item \blankcmd{ceil} (\ceil{\frac{a}{b}})%
        \item \blankcmd{floor} (\floor{\frac{a}{b}})%
    \end{ditemize}
\end{multicols}


\section{Plots \tiny\LILLYxBOXxVersion{1.0.8}}
\begin{center}
    Für die Spezifikationen siehe hier: \jmark[klick mich]{jmp:PLOTSSPEC}!
\end{center}
Diese Definitionen befinden sich in der Datei: \T{Maths/\_LILLY\_MATHS\_PLOTS}. Sie werden mit \LILLYxBOXxVersion{2.0.0} automatisch mit dem Einbinden von \LILLYxNOTExLibrary{LILLYxMATH} geladen.

%
%
%

\presentCommand[1.0.8]{plotline}[\optArg[Ao]{Farbe}\optArg[\textbackslash x]{Variable}\manArg{Term}\optArg[0]{offset}]
Zeichnet in eine \jmark[Graph-Umgebung]{env:graph} eine Funktion (siehe Umgebung für Beispiel). Existiert auch außerhalb von \blankenv{graph}, ist aber hier nur eingeschränkt nutzbar. Mit \emph{offset}${}^{\T{v2.0.0}}$ lässt sich die Funktion entsprechend verschieben.

%
%
%

\presentCommand[1.0.8]{plotseq}[\optArg[Ao]{Farbe}\optArg[\textbackslash x]{Variable}\manArg{Term}\optArg[maxX]{Obergrenze}\secline\optArg[1]{Untergrenze}\optArg[1pt]{Dicke}]
Zeichnet in eine \jmark[Graph-Umgebung]{env:graph} eine Folge zwischen \emph{Unter-} und \emph{Obergrenze} mit Punkten der Größe \emph{Dicke} (siehe Umgebung für Beispiel). Existiert auch außerhalb von \blankenv{graph}, ist aber hier nur eingeschränkt nutzbar.

%
%
%

\presentCommand[2.0.0]{xmark}[\optArg[x]{text}\manArg{PosX}\optArg[0.15]{linelength}]
Setzt einen Marker auf der $x$-Achse bei \emph{PosX} mit dem text \emph{text}. Für ein Beispiel, siehe \jmark[Graph-Umgebung]{env:graph}.

%
%
%

\presentCommand[2.0.0]{ymark}[\optArg[xy]{text}\manArg{PosY}\optArg[0.15]{linelength}]
Setzt einen Marker auf der $y$-Achse bei \emph{PosY} mit dem text \emph{text}. Für ein Beispiel, siehe \jmark[Graph-Umgebung]{env:graph}.

%
%
%
%
%
%

\subsection{graph-Environment}\elable{env:graph}
%
%
%
\presentEnvironment[1.0.8]{graph}[\optArg{Konfigurationen}\optArg{Tikz-Argumente}]
Es existiert die folgende Implementation der Graph-Umgebung:
\begin{center}
    \begin{tabular}{!{\VRule[1pt]}@{\hspace{0.5em}}C{0.5\textwidth}@{\hspace{0.5em}}|@{\hspace{0.5em}}C{0.4\textwidth}@{\hspace{0.5em}}!{\VRule[1pt]}}
        \specialrule{1pt}{0pt}{0pt}
{\begin{latex}[breaklines=true]
\begin{graph}[scale=1.15,maxY=3,numbers]
    \plotline[purple]{sqrt(\x+2.5)};
    \plotline[Ao][\y]{(\y*\y)/1.5};
    \plotseq[Azure]
        {sin(deg(\x))^2};
    \xmark[\xi]{1.5}; \ymark[\psi]{2.5};
\end{graph}
\end{latex}}
&
        \begin{graph}[scale=1.15,maxY=3,numbers]
            \plotline[purple]{sqrt(\x+2.5)};
            \plotline[Ao][\y]{(\y*\y)/1.5};
            \plotseq[Azure]
                {sin(deg(\x))^2};
            \xmark[$\xi$]{1.5}; \ymark[$\psi$]{2.5};
        \end{graph}\\
        \specialrule{1pt}{0pt}{0pt}
    \end{tabular}
\end{center}
Für die \emph{Konfiguration} gibt es die folgenden Parameter:
\begin{center}% \begin{tabularx}{0.75\linewidth}
    \begin{tabularx}{0.7\linewidth}{^t>{\em}^l^c^l+}
        \toprule
            \headerrow Bezeichner & \normalfont\bfseries Typ & Standard & Beschreibung\\
        \midrule
        scale & Zahl & 1 & Skalierungsfaktor \\
        xscale & Zahl & 1 & $x$-Skalierungsfaktor${}^{\typesetVersion{2.0.0}}$\\
        yscale & Zahl & 1 & $y$-Skalierungsfaktor${}^{\typesetVersion{2.0.0}}$\\
        minX & Zahl & -2 & X-Achse Start \\
        maxX & Zahl & 2 & X-Achse Ende \\
        minY & Zahl & 0 & Y-Achse Start \\
        maxY & Zahl & 4 & Y-Achse Ende \\
        offset & Zahl & 0.4 & Zusatzlänge Achsen \\
        loffset & Zahl & 0.1 & Unbeachteter Zusatz Achsen\\
        labelX & String & \$x\$ & Bezeichner X-Achse \\
        labelY & String & \$y\$ & Bezeichner Y-Achse \\
        samples & Zahl & 250 & Anzahl an Kalkulationen \\
        numbers & <> & false & Zeigt Zahlen an \\
        numXMin & Zahl & 0 & Nummernstart $x$ \\
        numYMin & Zahl & 0 & Nummernstart $y$ \\
        numbersize & Zahl & 5 & Schriftgröße Nummerierung \\
        labelsize & Zahl & 10 & Schriftgröße Texte \\
        \bottomrule
    \end{tabularx}
\end{center}

\presentEnvironment[2.0.0]{egraph}[\optArg{Konfigurationen}]
Funktioniert analog zu \blankenv{egraph}, erlaubt allerdings keine weiteren \emph{Tikz-Argumente}, sondern macht von \blankenv{tikzternal} gebrauch, kann also ausgelagert werden. (TODO: LINK)

%
%
%

\presentEnvironment[1.0.8]{wgraph}[\manArg{Ausrichtung}\optArg{Konfigurationen}\optArg{Tikz-Argumente}\secline\optArg{wrapfig-Zusatz}\optArg[0pt]{width}]
Um die Graph-Umgebung noch vielfälter zu Gestalten wurde \blankenv{wgraph} geschaffen.
Nach reichlicher Überlegung wurde ein neuer Befehl etabliert anstelle es in das
normale \jmark[graph]{env:graph}-Environment einzubetten. Er funktioniert mit der Syntax:
\begin{latex}
\begin{wgraph}{l}[][][\caption{Wichtiger Graph}][400pt]
    \plotline{\x*\x}
\end{wgraph}
\end{latex}
\clearpage
%
%
%
%
%
%
%
%
%
\section{3D-Plots} % \LILLYxNOTExWarning{Ausstehend}
Bisher sind noch keine Definitionen für 3-Dimensionale Plots integriert. Deswegen hier die exemplarische Definition eines 3D-Plots:
\begin{center}
    \begin{tabular}{!{\VRule[1pt]}@{\hspace{0.5em}}C{0.7\textwidth}@{\hspace{0.5em}}|@{\hspace{0.5em}}C{0.3\textwidth}@{\hspace{0.5em}}!{\VRule[1pt]}}
        \specialrule{1pt}{0pt}{0pt}
        {\begin{latex}[breaklines=true]
\begin{tikzternal}[scale=0.5]
\begin{axis}[3d box=complete, colormap/bluered,
             grid=major,view={60}{40},
             z buffer=sort, data cs=polar]
    :bcmd:addplot3:ecmd:[data cs=cart,surf,domain=-3:3,samples=20, opacity=0.5] {x+y};
\end{axis}
\end{tikzternal}
        \end{latex}
        } &  \begin{tikzternal}[scale=0.6]
            \begin{axis}[3d box=complete, colormap/bluered,grid=major,view={60}{40},z buffer=sort, data cs=polar]
              \addplot3[data cs=cart,surf,domain=-3:3,samples=20, opacity=0.5] {x+y};
            \end{axis}
          \end{tikzternal}\\
        \hline
{\begin{latex}[breaklines=true]
\begin{tikzternal}[scale=0.6]
    \begin{axis}[3d box=complete, axis equal image, colormap/bluered,grid=major,view={60}{40},z buffer=sort,enlargelimits=0.2,scale=2.3]
    :bcmd:addplot3:ecmd:[%
        opacity = 0.5, surf,
        samples = 21, variable = \u,
        variable y = \v, domain = 0:180,
        y domain = 0:360,
    ]
    ({cos(u)*sin(v)}, {sin(u)*sin(v)},
     {cos(v)});
    \end{axis}
\end{tikzternal}
\end{latex}
} & \begin{tikzternal}[scale=0.6]
    \begin{axis}[3d box=complete, axis equal image, colormap/bluered,grid=major,view={60}{40},z buffer=sort,enlargelimits=0.2,scale=2.3]
    \addplot3[%
        opacity = 0.5,
        surf,
        samples = 21,
        variable = \u,
        variable y = \v,
        domain = 0:180,
        y domain = 0:360,
    ]
    ({cos(u)*sin(v)}, {sin(u)*sin(v)}, {cos(v)});
    \end{axis}
\end{tikzternal}\\
\specialrule{1pt}{0pt}{0pt}
    \end{tabular}
\end{center}

\renewcommand{\arraystretch}{1}
