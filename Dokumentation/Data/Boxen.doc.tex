\renewcommand{\arraystretch}{1.5}
\chapter[Boxen \LILLYxBOXxVersion{\small 1.0.0}]{Boxen}
\TitleSUB{Boxes in Boxes in Boxes in Boxes\ldots \hfill \LILLYxBOXxVersion{\small 1.0.0}}
\section{Grundlegendes}
\subsection{Eine kleine Einführung}

Die 3 Standard-Designs, welche mit LILLY ausgeliefert werden lauten wie folgt:
\begin{tabularx}{\linewidth}{!{\VRule[1pt]}@{\hspace{1em}}c@{\hspace{1em}}|@{\hspace{1em}}c@{\hspace{1em}}|@{\hspace{1em}}c@{\hspace{1em}}!{\VRule[1pt]}}
    \specialrule{1pt}{0pt}{0pt}
    \bfseries DEFAULT & \bfseries ALTERNATE & \bfseries LIMERENCE \\\hline %% Leider ist es nicht möglich zu mischen ^^
{%%% TITEL TITLEPREFIX OPTIONAL-CONTROLLERS 

%% Dies ist die renew umgebung um default boxen zu forcieren auch wenn eine andere option gesetzt wurde

%%% Die Definition-Box: 

\RenewTColorBox[use counter from=LILLYxBOXxDefinition]{LILLYxBOXxDefinition}{ O{} O{Definition \thetcbcounter~} O{drop fuzzy shadow} }{ LillyxBOXxDesignxDefault, %
colback=\LILLYxColorxDefinition!5!white, colframe=\LILLYxColorxDefinition, #3, %
title={\begin{minipage}[t][\baselineskip][l]{\textwidth} \textbf{\textsc{{#2}}} \hfill {\textbf{#1}}\end{minipage}} %
}
\RenewTColorBox[use counter from=LILLYxBOXxDefinition]{LILLYxBOXxDefinition*}{ O{} O{Definition \thetcbcounter~} O{drop fuzzy shadow} }{ LillyxBOXxDesignxDefault, %
colback=\LILLYxColorxDefinition!5!white, colframe=\LILLYxColorxDefinition, #3, %
title={\begin{minipage}[t][\baselineskip][l]{\textwidth} \textbf{\textsc{{#2}}} \hfill {\textbf{#1}}\end{minipage}} %
}


%%% Die Beispiel-Box: 
\RenewTColorBox[use counter from=LILLYxBOXxBeispiel]{LILLYxBOXxBeispiel}{ O{} O{Beispiel \thetcbcounter~} O{} }{ LillyxBOXxDesignxDefault, %
colback=\LILLYxColorxBeispiel!5!white, colframe=\LILLYxColorxBeispiel, #3,%
title={\begin{minipage}[t][\baselineskip][l]{\textwidth} \textbf{\textsc{{#2}}} \hfill {\textbf{#1}}\end{minipage}}%
}


%%% Die Bemerkung-Box: 
\RenewTColorBox[use counter from=LILLYxBOXxBemerkung]{LILLYxBOXxBemerkung}{ O{} O{Bemerkung \thetcbcounter~} O{} }{ LillyxBOXxDesignxDefault, %
colback=\LILLYxColorxBemerkung!5!white, colframe=\LILLYxColorxBemerkung, #3,%
title={\begin{minipage}[t][\baselineskip][l]{\textwidth} \textbf{\textsc{{#2}}} \hfill {\textbf{#1}}\end{minipage}} %
}



%%% Die Satz-Box: 
\RenewTColorBox[use counter from=LILLYxBOXxSatz]{LILLYxBOXxSatz}{ O{} O{Satz \thetcbcounter~} O{} }{LillyxBOXxDesignxDefault,%
colback=\LILLYxColorxSatz!5!white, colframe=\LILLYxColorxSatz, #3,%
title={\begin{minipage}[t][\baselineskip][l]{\textwidth} \textbf{\textsc{{#2}}} \hfill {\textbf{#1}}\end{minipage}} %
}


%%% Die Beweis-Box: 
\RenewTColorBox[use counter from=LILLYxBOXxBeweis]{LILLYxBOXxBeweis}{ O{} O{Beweis \thetcbcounter~} O{} }{ LillyxBOXxDesignxDefault,%
colback=\LILLYxColorxBeweis!5!white, colframe=\LILLYxColorxBeweis, #3,%
title={\begin{minipage}[t][\baselineskip][l]{\textwidth} \textbf{\textsc{{#2}}} \hfill {\textbf{#1}}\end{minipage}} %
}


%%% Die Lemma-Box:
\RenewTColorBox[use counter from=LILLYxBOXxLemma]{LILLYxBOXxLemma}{ O{} O{Lemma \thetcbcounter~} O{} }{LillyxBOXxDesignxDefault, %
colback=\LILLYxColorxLemma!5!white, colframe=\LILLYxColorxLemma, #3,%
title={\begin{minipage}[t][\baselineskip][l]{\textwidth} \textbf{\textsc{{#2}}} \hfill {\textbf{#1}}\end{minipage}} %
}

%%% Die Zusammenfassung-Box:
\RenewTColorBox[use counter from=LILLYxBOXxZusammenfassung]{LILLYxBOXxZusammenfassung}{ O{} O{Zusammenfassung \thetcbcounter~} O{} }{ LillyxBOXxDesignxDefault, %
colback=\LILLYxColorxZusammenfassung!5!white, colframe=\LILLYxColorxZusammenfassung, #3,%
title={\begin{minipage}[t][\baselineskip][l]{\textwidth} \textbf{\textsc{{#2}}} \hfill {\textbf{#1}}\end{minipage}} %
}



%%sloppy - fix => MAKE ALL BOXES ENVIRONMENTS AND PROVIDE OLD COMMAND STYLE AS COMPAT FEATURE TODO:
\RenewTColorBox{LILLYxBOXxAufgabe}{O{} O{} O{}}{enforce breakable, %enforce breakable für: Mehr Seiten
enhanced jigsaw, before skip=2mm,after skip=2mm,
colback=white,colframe=black!50,boxrule=0.2mm,
attach boxed title to top left={xshift=1cm,yshift*=1mm-\tcboxedtitleheight},
varwidth boxed title*=-3cm,
boxed title style={
    frame code={
        \path[fill=white!30!black]
            ([yshift=-1mm,xshift=-1mm]frame.north west)
            arc[start angle=0,end angle=180,radius=1mm]
            ([yshift=-1mm,xshift=1mm]frame.north east)
            arc[start angle=180,end angle=0,radius=1mm];
        \path[left color=white!40!black,right color=white!40!black,
            middle color=white!55!black]
            ([xshift=-2mm]frame.north west) -- ([xshift=2mm]frame.north east)
            [rounded corners=1mm]-- ([xshift=1mm,yshift=-1mm]frame.north east)
            -- (frame.south east) -- (frame.south west)
            -- ([xshift=-1mm,yshift=-1mm]frame.north west)
            [sharp corners]-- cycle;
    },interior engine=empty,
},
fonttitle=\bfseries, #3,%
title={#2 \ifthenelse{\equal{#1}{}}{}{--~}#1}, %Aufgabe
}

\RenewTColorBox{LILLYxBOXxAufgabexPlain}{O{} O{} O{}}{enforce breakable, enhanced jigsaw, before skip=2mm,after skip=2mm, colback=white,colframe=black!50,boxrule=0.2mm,fonttitle=\bfseries, #3,title={#2 \ifthenelse{\equal{#1}{}}{}{--~}#1}}


\hspace{-3.25em}\begin{minipage}{0.32\linewidth}
    \begin{satz}[Nice]
        Superwichtig
    \end{satz}
\end{minipage}} & {
%%% TITEL TITLEPREFIX OPTIONAL-CONTROLLERS 


\tcbset{LillyxBOXxDesignxAlternate/.style={enforce breakable,leftrule=5mm,lines before break=4,left=2mm,frame code={\path[tcb fill frame] (frame.south west)--(frame.north west)
--([xshift=-5mm]frame.north east)--([yshift=-5mm]frame.north east)
--([yshift=5mm]frame.south east)--([xshift=-5mm]frame.south east)--cycle; },
interior code={\path[tcb fill interior] (interior.south west)--(interior.north west)
--([xshift=-4.8mm]interior.north east)--([yshift=-4.8mm]interior.north east)
--([yshift=4.8mm]interior.south east)--([xshift=-4.8mm]interior.south east)
--cycle; },
% code for the first part of a break sequence:
skin first is subskin of={emptyfirst}{%
frame code={\path[tcb fill frame] (frame.south west)--(frame.north west)
--([xshift=-5mm]frame.north east)--([yshift=-5mm]frame.north east)
--(frame.south east)--cycle;
\path[fill=white] ([xshift=2.5mm,yshift=1mm]frame.south west) -- +(120:2mm)
-- +(60:2mm)-- cycle; },
interior code={\path[tcb fill interior] (interior.south west|-frame.south)
--(interior.north west)--([xshift=-4.8mm]interior.north east)
--([yshift=-4.8mm]interior.north east)--(interior.south east|-frame.south)
--cycle; },
},%
% code for the middle part of a break sequence:
skin middle is subskin of={emptymiddle}{%
frame code={\path[tcb fill frame] (frame.south west)--(frame.north west)
--(frame.north east)--(frame.south east)--cycle;
\path[fill=white] ([xshift=2.5mm,yshift=-1mm]frame.north west) -- +(240:2mm)
-- +(300:2mm) -- cycle;
\path[fill=white] ([xshift=2.5mm,yshift=1mm]frame.south west) -- +(120:2mm)
-- +(60:2mm) -- cycle;
},
interior code={\path[tcb fill interior] (interior.south west|-frame.south)
--(interior.north west|-frame.north)--(interior.north east|-frame.north)
--(interior.south east|-frame.south)--cycle; },
},
% code for the last part of a break sequence:
skin last is subskin of={emptylast}{%
frame code={\path[tcb fill frame] (frame.south west)--(frame.north west)
--(frame.north east)--([yshift=5mm]frame.south east)
--([xshift=-5mm]frame.south east)--cycle;
\path[fill=white] ([xshift=2.5mm,yshift=-1mm]frame.north west) -- +(240:2mm)
-- +(300:2mm) -- cycle;
},
interior code={\path[tcb fill interior] (interior.south west)
--(interior.north west|-frame.north)--(interior.north east|-frame.north)
--([yshift=4.8mm]interior.south east)--([xshift=-4.8mm]interior.south east)
--cycle; },
}}}

%%Es scheint nicht möglich IfValueTF noch andere Optionen in kombination mithilfe von auto counter zu benutzen - dies ist in jeder hinsicht kacke

%%% Die Definition-Box:
\RenewTColorBox[use counter from=LILLYxBOXxDefinition]{LILLYxBOXxDefinition}{ O{} O{Definition \thetcbcounter~} O{} }{ empty, %
frame style={fill,top color=\LILLYxColorxDefinition,bottom color=\LILLYxColorxDefinition,middle color=\LILLYxColorxDefinition},colback=\LILLYxColorxDefinition!15!white,  %
LillyxBOXxDesignxAlternate, #3, IfValueTF={#1}{before upper=\textbf{\textcolor{\LILLYxColorxDefinition}{#2-- #1}}\newline}{before upper=\textbf{\textcolor{\LILLYxColorxDefinition}{#2}~}}, %
}
\RenewTColorBox[use counter from=LILLYxBOXxDefinition]{LILLYxBOXxDefinition*}{ O{} O{Definition \thetcbcounter~} O{} }{ empty, %
frame style={fill,top color=\LILLYxColorxDefinition,bottom color=\LILLYxColorxDefinition,middle color=\LILLYxColorxDefinition},colback=\LILLYxColorxDefinition!15!white,  %
LillyxBOXxDesignxAlternate, #3, IfValueTF={#1}{before upper=\textbf{\textcolor{\LILLYxColorxDefinition}{#2-- #1}}\newline}{before upper=\textbf{\textcolor{\LILLYxColorxDefinition}{#2}~}}, %
}

%%% Die Beispiel-Box:
\RenewTColorBox[use counter from=LILLYxBOXxBeispiel]{LILLYxBOXxBeispiel}{ O{} O{Beispiel \thetcbcounter~} O{} }{ empty, %
frame style={fill,top color=\LILLYxColorxBeispiel,bottom color=\LILLYxColorxBeispiel,middle color=\LILLYxColorxBeispiel},colback=\LILLYxColorxBeispiel!15!white,  %
LillyxBOXxDesignxAlternate, #3, IfValueTF={#1}{before upper=\textbf{\textcolor{\LILLYxColorxBeispiel}{#2-- #1}}\newline}{before upper=\textbf{\textcolor{\LILLYxColorxBeispiel}{#2}~}}, %
}


%%% Die Bemerkung-Box:
\RenewTColorBox[use counter from=LILLYxBOXxBemerkung]{LILLYxBOXxBemerkung}{ O{} O{Bemerkung \thetcbcounter~} O{} }{ empty, %
frame style={fill,top color=\LILLYxColorxBemerkung,bottom color=\LILLYxColorxBemerkung,middle color=\LILLYxColorxBemerkung},colback=\LILLYxColorxBemerkung!15!white,  %
LillyxBOXxDesignxAlternate, #3, IfValueTF={#1}{before upper=\textbf{\textcolor{\LILLYxColorxBemerkung}{#2-- #1}}\newline}{before upper=\textbf{\textcolor{\LILLYxColorxBemerkung}{#2}~}}, %
}


%%% Die Satz-Box:
\RenewTColorBox[use counter from=LILLYxBOXxSatz]{LILLYxBOXxSatz}{ O{} O{Satz \thetcbcounter~} O{} }{ empty, %
frame style={fill,top color=\LILLYxColorxSatz,bottom color=\LILLYxColorxSatz,middle color=\LILLYxColorxSatz},colback=\LILLYxColorxSatz!15!white,  %
LillyxBOXxDesignxAlternate, #3, IfValueTF={#1}{before upper=\textbf{\textcolor{\LILLYxColorxSatz}{#2-- #1}}\newline}{before upper=\textbf{\textcolor{\LILLYxColorxSatz}{#2}~}}, %
}


%%% Die Beweis-Box:
\RenewTColorBox[use counter from=LILLYxBOXxBeweis]{LILLYxBOXxBeweis}{ O{} O{Beweis \thetcbcounter~} O{} }{ empty, %
frame style={fill,top color=\LILLYxColorxBeweis,bottom color=\LILLYxColorxBeweis,middle color=\LILLYxColorxBeweis},colback=\LILLYxColorxBeweis!15!white,  %
LillyxBOXxDesignxAlternate, #3, IfValueTF={#1}{before upper=\textbf{\textcolor{\LILLYxColorxBeweis}{#2-- #1}}\newline}{before upper=\textbf{\textcolor{\LILLYxColorxBeweis}{#2}~}}, %
}


%%% Die Lemma-Box:
\RenewTColorBox[use counter from=LILLYxBOXxLemma]{LILLYxBOXxLemma}{ O{} O{Lemma \thetcbcounter~} O{} }{ empty, %
frame style={fill,top color=\LILLYxColorxLemma,bottom color=\LILLYxColorxLemma,middle color=\LILLYxColorxLemma},colback=\LILLYxColorxLemma!15!white,  %
LillyxBOXxDesignxAlternate, #3, IfValueTF={#1}{before upper=\textbf{\textcolor{\LILLYxColorxLemma}{#2-- #1}}\newline}{before upper=\textbf{\textcolor{\LILLYxColorxLemma}{#2}~}}, %
}


%%% Die Zusammenfassung-Box:
\RenewTColorBox[use counter from=LILLYxBOXxZusammenfassung]{LILLYxBOXxZusammenfassung}{ O{} O{Zusammenfassung \thetcbcounter~} O{} }{ empty, %
frame style={fill,top color=\LILLYxColorxZusammenfassung,bottom color=\LILLYxColorxZusammenfassung,middle color=\LILLYxColorxZusammenfassung},colback=\LILLYxColorxZusammenfassung!15!white,  %
LillyxBOXxDesignxAlternate, #3, IfValueTF={#1}{before upper=\textbf{\textcolor{\LILLYxColorxZusammenfassung}{#2-- #1}}\newline}{before upper=\textbf{\textcolor{\LILLYxColorxZusammenfassung}{#2}~}}, %
}


%title={\begin{minipage}[t][\baselineskip][l]{\textwidth} \textbf{\textsc{{#2}}} \hfill {\textbf{#1}}\end{minipage}},

\RenewTColorBox{LILLYxBOXxAufgabe}{ O{} O{} O{} }{enforce breakable, enhanced,skin=enhancedlast jigsaw, attach boxed title to top left={xshift=-4mm,yshift=-0.5mm},fonttitle=\bfseries\sffamily,varwidth boxed title=0.7\linewidth,colbacktitle=Charcoal!45!white,colframe=Charcoal!50!black,interior style={top color=gray!2!white,bottom color=gray!2!white},boxed title style={empty,arc=0pt,outer arc=0pt,boxrule=0pt},underlay boxed title={\fill[\Hcolor] (title.north west) -- (title.north east)-- +(\tcboxedtitleheight-1mm,-\tcboxedtitleheight+1mm)-- ([xshift=4mm,yshift=0.5mm]frame.north east) -- +(0mm,-1mm)-- (title.south west) -- cycle;\fill[Charcoal!45!white!50!black] ([yshift=-0.5mm]frame.north west)-- +(-0.4,0) -- +(0,-0.3) -- cycle;\fill[Charcoal!45!white!50!black] ([yshift=-0.5mm]frame.north east)-- +(0,-0.3) -- +(0.4,0) -- cycle;  },  #3,  title={#2 \ifthenelse{\equal{#1}{}}{}{--~}#1}, %Aufgabe
}



\providecommand{\LILLYxBOXxIPOI}[3]{
    ~\\[0.2cm]
    \def\slctmp{#1}\ifx\slctmp\empty
        \noindent{\textbf{#2}}
    \else
        \noindent{\textbf{#2 - #1}}\\
    \fi
        #3
    \vspace{0.45cm}
} 


%%TODO: BREAKS ON DIFFERENT AMOUNT OF LINES FOR NORMAL AND ALTERNATE - FIXpo


%%TODO: BREAKS ON DIFFERENT AMOUNT OF LINES FOR NORMAL AND ALTERNATE - FIXpo
\hspace{-3.25em}\begin{minipage}{0.32\linewidth}
    \begin{satz}[Nice]
        Superwichtig
    \end{satz}
\end{minipage}} & {% TITEL TITLEPREFIX OPTIONAL-CONTROLLERS
\def\LILLYxBOXxLIMERENCExBORDERxWIDTH{3.145pt}
\makeatletter
\tcbset{LillyxBOXxDesignxLimerence/.style={breakable, lines before break=3, enhanced jigsaw,sharp corners, boxrule=0pt, fonttitle={\bfseries}, coltitle={black}, attach title to upper, opacityfill=0.45, top=3pt, bottom=3pt, right=5pt, frame hidden,colback=LightGray!26!white,grow to right by=\LILLYxBOXxLIMERENCExBORDERxWIDTH,enlargepage flexible=\baselineskip,after title app={\hbox{}~\vspace*{-.85\baselineskip}\newline\unskip\widowpenalties=3 10000 10000 150}}}

% small arrows on breaks:
\tcbset{% #1 is the color
LillyxBOXxDesignxLimerencexArrows/.style={%
borderline west={\LILLYxBOXxLIMERENCExBORDERxWIDTH}{0pt}{#1},
extras first and middle={overlay={%
    \begin{scope}[shift={(frame.south west)}]
            \path[fill=#1] (0pt,0) -- ++(\LILLYxBOXxLIMERENCExBORDERxWIDTH,0pt) --  ++(-1.75pt,-\LILLYxBOXxLIMERENCExBORDERxWIDTH) --cycle;
    \end{scope}
}},%
extras middle and last={overlay={%
    \begin{scope}[shift={(frame.north west)}]
            \path[fill=#1] (0pt,0) -- ++(\LILLYxBOXxLIMERENCExBORDERxWIDTH,0pt) --  ++(-1.75pt,\LILLYxBOXxLIMERENCExBORDERxWIDTH) --cycle;
    \end{scope}
}}}}
\tcbset{tag/.style={after title=}}

\ifx\@onlypreamble\@notprerr
    \ifx\LILLYxBOXxDefinitionxBox\true\tcbset{LILLYxBOXxDefinitionxBoxxLimerence/.style={LillyxBOXxDesignxLimerencexArrows=\LILLYxColorxDefinition}}\else\tcbset{LILLYxBOXxDefinitionxBoxxLimerence/.style={opacityfill=0}}\fi
\else
\AtBeginDocument{
    \ifx\LILLYxBOXxDefinitionxBox\true\tcbset{LILLYxBOXxDefinitionxBoxxLimerence/.style={LillyxBOXxDesignxLimerencexArrows=\LILLYxColorxDefinition}}\else\tcbset{LILLYxBOXxDefinitionxBoxxLimerence/.style={opacityfill=0}}\fi
}\fi

\RenewTColorBox[use counter from=LILLYxBOXxDefinition]{LILLYxBOXxDefinition}{ O{} O{Definition \thetcbcounter~} O{} }{ enlarge left by=-\LILLYxBOXxLIMERENCExBORDERxWIDTH,   %
LillyxBOXxDesignxLimerence, LILLYxBOXxDefinitionxBoxxLimerence, #3, title={$\vcenter{#2\ifthenelse{\equal{#1}{}}{}{--~}\parbox[t]{\linewidth-\widthof{#2}-3.75em}{#1}}$\ifx\LILLYxBOXxDefinitionxBox\true\\[0.1pt]\else\\[-0.4\baselineskip]\fi}, %
}
\RenewTColorBox[use counter from=LILLYxBOXxDefinition]{LILLYxBOXxDefinition*}{ O{} O{Definition \thetcbcounter~} O{} }{ enlarge left by=-\LILLYxBOXxLIMERENCExBORDERxWIDTH,   %
LillyxBOXxDesignxLimerence,LILLYxBOXxDefinitionxBoxxLimerence, #3, title={#2\ifthenelse{\equal{#1}{}}{\parbox[t]{\linewidth-\widthof{#2}-2em}{#1}}{--~\parbox[t]{\linewidth-\widthof{#2}-.05em-\widthof{--~}}{#1}}\parbox[t]{2em}{\textcolor{\LILLYxColorxDefinition}{\large\faThumbTack}}\ifx\LILLYxBOXxDefinitionxBox\true\\[.1pt]\else\\[-.4\baselineskip]\fi},bookmark*={color=\LILLYxColorxDefinition}{#2 -- #1} %
}

% Die Beispiel-Box:
\ifx\@onlypreamble\@notprerr
    \ifx\LILLYxBOXxBeispielxBox\true\tcbset{LILLYxBOXxBeispielxBoxxLimerence/.style={LillyxBOXxDesignxLimerencexArrows=\LILLYxColorxBeispiel}}\else\tcbset{LILLYxBOXxBeispielxBoxxLimerence/.style={opacityfill=0}}\fi
\else
\AtBeginDocument{
    \ifx\LILLYxBOXxBeispielxBox\true\tcbset{LILLYxBOXxBeispielxBoxxLimerence/.style={LillyxBOXxDesignxLimerencexArrows=\LILLYxColorxBeispiel}}\else\tcbset{LILLYxBOXxBeispielxBoxxLimerence/.style={opacityfill=0}}\fi
}\fi

\RenewTColorBox[use counter from=LILLYxBOXxBeispiel]{LILLYxBOXxBeispiel}{ O{} O{Beispiel \thetcbcounter~} O{} }{ enlarge left by=-\LILLYxBOXxLIMERENCExBORDERxWIDTH, LILLYxBOXxBeispielxBoxxLimerence,   %
LillyxBOXxDesignxLimerence, #3, title={$\vcenter{#2\ifthenelse{\equal{#1}{}}{}{--~}\parbox[t]{\linewidth-\widthof{#2}-1.75em}{#1}}$\ifx\LILLYxBOXxBeispielxBox\true\\\else\\[-.4\baselineskip]\fi}, %
}


% Die Bemerkung-Box:
% Da Borderline sich nicht einfach löschen lässt - hier über einen extra Marker
\ifx\@onlypreamble\@notprerr
    \ifx\LILLYxBOXxBemerkungxBox\true\tcbset{LILLYxBOXxBemerkungxBoxxLimerence/.style={LillyxBOXxDesignxLimerencexArrows=\LILLYxColorxBemerkung}}\else\tcbset{LILLYxBOXxBemerkungxBoxxLimerence/.style={opacityfill=0}}\fi
\else
\AtBeginDocument{
    \ifx\LILLYxBOXxBemerkungxBox\true\tcbset{LILLYxBOXxBemerkungxBoxxLimerence/.style={LillyxBOXxDesignxLimerencexArrows=\LILLYxColorxBemerkung}}\else\tcbset{LILLYxBOXxBemerkungxBoxxLimerence/.style={opacityfill=0}}\fi
}\fi

\RenewTColorBox[use counter from=LILLYxBOXxBemerkung]{LILLYxBOXxBemerkung}{ O{} O{Bemerkung \thetcbcounter~} O{} }{ enlarge left by=-\LILLYxBOXxLIMERENCExBORDERxWIDTH,LILLYxBOXxBemerkungxBoxxLimerence, %
LillyxBOXxDesignxLimerence, #3, title={$\vcenter{#2\ifthenelse{\equal{#1}{}}{}{--~}\parbox[t]{\linewidth-\widthof{#2}-1.75em}{#1}}$\ifx\LILLYxBOXxBemerkungxBox\true\\\else\\[-.4\baselineskip]\fi}, %
}
% Die Satz-Box:
\RenewTColorBox[use counter from=LILLYxBOXxSatz]{LILLYxBOXxSatz}{ O{} O{Satz \thetcbcounter~} O{} }{ enlarge left by=-\LILLYxBOXxLIMERENCExBORDERxWIDTH, LillyxBOXxDesignxLimerencexArrows=\LILLYxColorxSatz, %
LillyxBOXxDesignxLimerence, #3, title={$\vcenter{#2\ifthenelse{\equal{#1}{}}{}{--~}\parbox[t]{\linewidth-\widthof{#2}-1.75em}{#1}}$\\}}
% Die Beweis-Box:
\ifx\@onlypreamble\@notprerr%
    \ifx\LILLYxBOXxBeweisxBox\true\typeout{Beweis Box shown}\tcbset{LILLYxBOXxBeweisxBoxxLimerence/.style={LillyxBOXxDesignxLimerencexArrows=\LILLYxColorxBeweis}}\else\typeout{Beweis box hidden}\tcbset{LILLYxBOXxBeweisxBoxxLimerence/.style={opacityfill=0}}\fi
\else%
\AtBeginDocument{
    \ifx\LILLYxBOXxBeweisxBox\true\typeout{Beweis Box shown}\tcbset{LILLYxBOXxBeweisxBoxxLimerence/.style={LillyxBOXxDesignxLimerencexArrows=\LILLYxColorxBeweis}}\else\typeout{Beweis box hidden}\tcbset{LILLYxBOXxBeweisxBoxxLimerence/.style={opacityfill=0}}\fi
}\fi

\RenewTColorBox[use counter from=LILLYxBOXxBeweis]{LILLYxBOXxBeweis}{ O{} O{Beweis \thetcbcounter~} O{} }{ enlarge left by=-\LILLYxBOXxLIMERENCExBORDERxWIDTH,LILLYxBOXxBeweisxBoxxLimerence,%
LillyxBOXxDesignxLimerence, #3, title={$\vcenter{#2\ifthenelse{\equal{#1}{}}{}{--~}\parbox[t]{\linewidth-\widthof{#2}-1.75em}{#1}}$\ifx\LILLYxBOXxBeweisxBox\true\\\else\\[-.4\baselineskip]\fi}}

% Die Lemma-Box:
\ifx\@onlypreamble\@notprerr%
    \ifx\LILLYxBOXxLemmaxBox\true\typeout{Lemma Box shown}\tcbset{LILLYxBOXxLemmaxBoxxLimerence/.style={LillyxBOXxDesignxLimerencexArrows=\LILLYxColorxLemma}}\else\typeout{Lemma box hidden}\tcbset{LILLYxBOXxLemmaxBoxxLimerence/.style={opacityfill=0}}\fi
\else%
\AtBeginDocument{
    \ifx\LILLYxBOXxLemmaxBox\true\typeout{Lemma Box shown}\tcbset{LILLYxBOXxLemmaxBoxxLimerence/.style={LillyxBOXxDesignxLimerencexArrows=\LILLYxColorxLemma}}\else\typeout{Lemma box hidden}\tcbset{LILLYxBOXxLemmaxBoxxLimerence/.style={opacityfill=0}}\fi
}\fi

\RenewTColorBox[use counter from=LILLYxBOXxLemma]{LILLYxBOXxLemma}{ O{} O{Lemma \thetcbcounter~} O{} }{ enlarge left by=-\LILLYxBOXxLIMERENCExBORDERxWIDTH, LILLYxBOXxLemmaxBoxxLimerence ,  %
LillyxBOXxDesignxLimerence, #3, title={$\vcenter{#2\ifthenelse{\equal{#1}{}}{}{--~}\parbox[t]{\linewidth-\widthof{#2}-.3em}{#1}}$\\}}
% Die Zusammenfassung-Box:
\RenewTColorBox[use counter from=LILLYxBOXxZusammenfassung]{LILLYxBOXxZusammenfassung}{ O{} O{Zusammenfassung \thetcbcounter~} O{} }{ enlarge left by=-\LILLYxBOXxLIMERENCExBORDERxWIDTH, LillyxBOXxDesignxLimerencexArrows=\LILLYxColorxZusammenfassung,   %
LillyxBOXxDesignxLimerence, #3, IfValueTF={#1}{title={#2\ifthenelse{\equal{#1}{}}{\parbox[t]{\linewidth-\widthof{#2}-2em}{#1}}{--~\parbox[t]{\linewidth-\widthof{#2}-0.75em-\widthof{--~}}{#1}}\parbox[t]{2em}{\textcolor{\LILLYxColorxZusammenfassung}{\large\faArchive}}\\}}{title={#2\hbox{}\hfill\parbox[t]{2em}{\textcolor{\LILLYxColorxZusammenfassung}{\large\faArchive}}}}, %
}

\RenewTColorBox{LILLYxBOXxAufgabexPlain}{O{} O{} O{}}{enforce breakable, enhanced jigsaw, before skip=2mm,after skip=2mm, colback=white,colframe=black!50,boxrule=0.0mm,lines before break=4,grow to left by=3pt, fonttitle=\bfseries, #3,title={#2 \ifthenelse{\equal{#1}{}}{}{--~}#1}}\hspace{-3.25em}\begin{minipage}{0.32\linewidth}
    \begin{satz}[Nice]
        Superwichtig
    \end{satz}
\end{minipage}}
\\
    \specialrule{1pt}{0pt}{0pt}
    \end{tabularx}
Auch wenn sie hier explizit forciert wurden ist es grundlegend möglich (und auch so gedacht) sie mithilfe des Makefiles konfiguerieren. Die allgemeine Syntax hierfür lautet:
\begin{lstlisting}[language=lBash]
make "BOXMODE=<Name>"
\end{lstlisting}
\marginpar{\tiny LILLY lädt übrigens nicht DEFAULT sondern immer DEFAULT(init)!}wobei \T{<Name>} mit einem der oben stehenden Bezeichner ersetzt wird. Die Bezeichner werden vom weiter unten näher beschriebenem Box-Controller wie folgt aufgelöst:
\begin{lstlisting}[language=lLatex,frame=none]
\input{\LILLYxPATHxDATA/POIs/_LILLY_BOXES_\LILLYxBOXxMODE}
\end{lstlisting}
\marginpar{\tiny Natürlich wäre es schöner auch andere Verzeichnisse zuzulasen und hierbei dann den gesamten Pfad anzugeben - dies ist aber bisher auch TODO:}Über genau dieses Verfahren lassen sich auch beliebig die Box-Designs erweitern.
\subsection{Der Box-Controller}
{\centering \framebox{Diese Definitionen befinden sich in der Datei: \T{Controllers/\_LILLY\_BOX\_CONTROLLER}}\vspace*{0.5\baselineskip}\par}\reversemarginpar
Alle Box-Desings werden über den Box-Controller geladen, der über \CMDpreview{LILLYxBOXxMODE} die Möglichkeit zur Verfügung stellt die jeweilige \T{POI}-Datei zu laden (TODO: LINK). Er definiert ein gigantisches Paket an Befehlen (TODO: pgf foreach) die allerdings für jeden Boxtyp identisch sind. allgemein werden: \CMDpreview{LILLYxBOXx*xLock} und \CMDpreview{LILLYxBOXx*xEnable} für alle Boxen definiert. So kann man zum Beispiel durch das Setzen von \CMDshow{LILLYxBOXxDefinitionxEnable} auf \T{FALSE} das Anzeigen von Definitionsboxen deaktivieren (Information: Sie werden einfach entfernt, es wird kein adäquater Platzhalter als Ersatz eingefügt) und durch das Setzen von \CMDshow{LILLYxBOXxBeispielxLock} auf\T{section} das Nummerieren der Box auf die Sektionen festlegen (\T{TRUE} für ungebunden).
Weiter definiert es die folgenden Box-Environments:
%%PYSCRIPT:
%
% b = ("definition", "definition*", "bemerkung", "beispiel", "satz", "beweis", "lemma", "zusammenfassung", "aufgabe")
% for a in b:
%   print("\\begingroup\\begin{%s}[Titel]\nmoin" % (a))
%   print("\\begin{lstlisting}[language=lLatex]")
%   print("\\begin{%s}[Titel]\n\tmoin" % (a))
%   print("\\end{%s}\end{lstlisting}" % (a))
%   print("\\end{%s}\\endgroup" % (a))
%   print()

{\begin{multicols}{2}
\begingroup\begin{definition}[Titel]
moin
\begin{lstlisting}[language=lLatex]
\begin{definition}[Titel]
        moin
\end{definition}\end{lstlisting}
\end{definition}\endgroup

\begingroup\begin{definition*}[Titel]
moin
\begin{lstlisting}[language=lLatex]
\begin{definition*}[Titel]
        moin
\end{definition*}\end{lstlisting}
\end{definition*}\endgroup

\begingroup\begin{bemerkung}[Titel]
moin
\begin{lstlisting}[language=lLatex]
\begin{bemerkung}[Titel]
        moin
\end{bemerkung}\end{lstlisting}
\end{bemerkung}\endgroup

\begingroup\begin{beispiel}[Titel]
moin
\begin{lstlisting}[language=lLatex]
\begin{beispiel}[Titel]
        moin
\end{beispiel}\end{lstlisting}
\end{beispiel}\endgroup

\begingroup\begin{satz}[Titel]
moin
\begin{lstlisting}[language=lLatex]
\begin{satz}[Titel]
        moin
\end{satz}\end{lstlisting}
\end{satz}\endgroup

\columnbreak

\begingroup\begin{beweis}[Titel]
moin
\begin{lstlisting}[language=lLatex]
\begin{beweis}[Titel]
        moin
\end{beweis}\end{lstlisting}
\end{beweis}\endgroup

\begingroup\begin{lemma}[Titel]
moin
\begin{lstlisting}[language=lLatex]
\begin{lemma}[Titel]
        moin
\end{lemma}\end{lstlisting}
\end{lemma}\endgroup

\begingroup\begin{zusammenfassung}[Titel]
moin
\begin{lstlisting}[language=lLatex]
\begin{zusammenfassung}[Titel]
        moin
\end{zusammenfassung}\end{lstlisting}
\end{zusammenfassung}\endgroup

\begingroup\begin{aufgabe}[Titel][3]
moin
\begin{lstlisting}[language=lLatex]
\begin{aufgabe}[Titel][3]
        moin
\end{aufgabe}\end{lstlisting}
\end{aufgabe}\endgroup

Nicht richtig darstellbar aber weiter existiert:
\begin{lstlisting}[language=lLatex]
\begin{uebungsblatt}[Titel][2]
        moin
\end{uebungsblatt}\end{lstlisting}

\end{multicols}}
Für bisher leider noch nicht alle Boxen wird zudem der Befehl: \CMDpreview{LILLYxBOXx*xBox} definiert. Bisher unterstützt werden:
\begin{multicols}{3}
    \begin{itemize}[label=$\diamond$]\narrowitems
        \item Bemerkung
        \item Beispiel
        \item Beweis
        \item Aufgabe
        \item Uebungsblatt
    \end{itemize}
\end{multicols}
Setzt man den Wert auf \T{FALSE} so wird das sogenannte \T{plain}-Design angewendet, welches jedes Design wieder selbst definieren kann! (TODO: custom Box counters).\newline
Zudem existieren aus Kompatibilitätsgründen auch noch die alten Befehle aus dem \T{eagleStudiPackage}: \CMDpreview[(2)]{DEF}, \CMDpreview[(2)]{BEM}, \ENVpar{task}\T{task},\ldots\medskip\newline
Mit \LILLYxBOXxVersion{\small 1.0.3} wurden in LILLY zudem Kurzbefehle für das Einbinden von Übungsblättern integriert:
\CMDpreview[(3)]{inputUB} (mit Signatur: \verb|{Name}{Nummer}{Pfad}|) und \CMDpreview[(3)]{inputUBS} (analog für \T{uebungsblatt*})\normalfont

\section{Die Boxmodi}
\subsection{Default-Design}
Mit \LILLYxBOXxVersion{\small 1.0.0} stellt dieses Design den Urvater dar. Im Folgenden wird auf die genaue Implementation eingegangen:\newline
Auf Basis des Pakets \T{tcolorbox} definiert LILLY das Design \T{LillyxBOXxDesignxDefault} - auf das Großschreiben von Lilly wurde hier bewusst verzichtet - mit folgender Implementation:
{\begin{lstlisting}[language=lLatex]
\tcbset{LillyxBOXxDesignxDefault/.style={enhanced jigsaw, pad before break*=2mm %
    pad after break=2mm, lines before break=4, before skip=0pt, boxrule = 0mm, toprule=0.5mm,%
    bottomtitle=0.5mm,bottomrule=1.2mm, after skip=0pt, enlarge top by=\baselineskip,%
    enlarge bottom by=\baselineskip, sharp corners=south, enforce breakable}%
}
\end{lstlisting}}
Bisher definiert LILLY die Counter über die Einstellung \verb|auto counter| - dies soll aber bald auf das vom eagleStudiPackage Package verwendete \T{counter}-Verfahren umgestellt werden. Bis dato sieht eine exemplarische Definition einer Box wie folgt aus:
{\begin{lstlisting}[language=lLatex]
\DeclareTColorBox[auto counter]%
    {LILLYxBOXxDefinition}%
    { O{} O{Definition \thetcbcounter~} O{drop fuzzy shadow} }%
    {LillyxBOXxDesignxDefault, colback=\LILLYxColorxDefinition!5!white,%
        colframe=\LILLYxColorxDefinition, #3,%
        title={%
            \begin{minipage}[t][\baselineskip][l]{\textwidth}%
                \textbf{\textsc{{#2}}} \hfill {\textbf{#1}}%
            \end{minipage}%
        }%
    }
\end{lstlisting}}
Hiervon weichen nur 2 Definitionen ab. Die der Aufgaben-Box:
\begingroup{\begin{lstlisting}[language=lLatex]
\DeclareTColorBox{LILLYxBOXxAufgabe}{O{} O{} O{}}{enforce breakable,%
    colback=white,colframe=black!50,boxrule=0.2mm,%
    attach boxed title to top left={xshift=1cm,yshift*=1mm-\tcboxedtitleheight},%
    varwidth boxed title*=-3cm,%
    boxed title style={
        frame code={
            \path[fill=white!30!black]%
                ([yshift=-1mm,xshift=-1mm]frame.north west)%
                    arc[start angle=0,!*\T{end}*! angle=180,radius=1mm]%
                ([yshift=-1mm,xshift=1mm]frame.north east)%
                    arc[start angle=180,!*\T{end}*! angle=0,radius=1mm];
            \path[left color=white!40!black,right color=white!40!black,
                    middle color=white!55!black]
                ([xshift=-2mm]frame.north west) -- ([xshift=2mm]frame.north east)%
                [rounded corners=1mm]-- ([xshift=1mm,yshift=-1mm]frame.north east)%
                -- (frame.south east) -- (frame.south west)%
                -- ([xshift=-1mm,yshift=-1mm]frame.north west)%
                [sharp corners]-- cycle;%
        },interior engine=empty,%
    },
    enhanced jigsaw, before skip=2mm,after skip=2mm,%
    fonttitle=\bfseries, #3,%
    title={#2 \ifthenelse{\equal{#1}{}}{}{--~}#1}, %Aufgabe
}
\end{lstlisting}}
Und die der Plain-Box:
{\begin{lstlisting}[language=lLatex]
\DeclareTColorBox{LILLYxBOXxAufgabexPlain}{O{} O{} O{}} {%
        enforce breakable, enhanced jigsaw, before skip=2mm,after skip=2mm,%
        colback=white,colframe=black!50,boxrule=0.2mm,fonttitle=\bfseries,%
        #3,title={#2 \ifthenelse{\equal{#1}{}}{}{--~}#1}%
    }
\end{lstlisting}}
