\chapter{Einleitung}
\TitleSUB{Integrieren von LILLY -- Die Grundlagen von A-Z}
\section{Installieren von Lilly}
\LILLYxNOTExVersion{1.0.10}Aktuell kommt die Dokumentklasse ohne \T{.ins} oder \T{.dtx} Datei, dafür allerdings mit einem Installer für alle Debian (Linux) basierten Betriebsysteme, an einer Variante für MacOS und Windows wird momentan entwickelt.

\begin{bemerkung}[Mithilfe]
    Wenn du dich mit \TeX{} oder \LaTeX{} auskennst, schreibe an folgende Email-Adresse \T{\AUTHORMAIL}.\smallskip\newline
    Mittlerweile gibt es auch ein offizielles Github-Repository (\url{https://github.com/EagleoutIce/LILLY} \href{https://github.com/EagleoutIce/LILLY}{\faGithub})
    über das die gesamte Entwicklung abläuft. Hier werden noch Helfer für folgende Aufgaben gesucht:
    \begin{multicols}{2}
        \begin{itemize}[label=$\diamond$]
            \item Java - Entwicklung
            \item Bash, Konsolen - Entwicklung
            \item Kommentieren in Markdown
            \item Maintaining (\TeX, \LaTeX)
            \item Kommentieren in Doxygen
            \item Layout Gestaltung
            \item \TeX, \LaTeX{} -Entwicklung
            \item Tester (\faLinux, \faApple, \faWindows)
        \end{itemize}
    \end{multicols}
\end{bemerkung}

\subsection{Linux}
\begin{center}
    Für Versionen < 1.0.10 klicke hier: \jmark[klick mich]{jmp:oldInstallLinux1009}!
\end{center}
Mit der Portierung von \Jake in die Programmiersprache Java hat sich die Installation von LILLY, immens vereinfacht.
Da man hierfür allerdings \Jake benötigt, der sich dann um alles weitere kümmert, sei hier einmal nur kurz erklärt, wie man die \T{stable}-Version von Jake installiert, für mehr Infos siehe: \jmark[Jake Installieren]{mrk:InstallJake}.\newline
Mit dem Bezug dieser Dokumentation sollte eine \T{jake.jar} Datei einhergegangen sein, die es nun gilt auszuführen. Natürlich wird hierfür Java benötigt, auf einem apt-Basierten Betriebsystem installiert man Java wie folgt:
\begin{bash}
sudo apt install default-jdk
\end{bash}
Für alle anderen Derivate gilt es sich auf \url{https://www.oracle.com/de/java/} entsprechend zu informieren.
Einmal installiert, genügt ein Ausführen der \T{.jar} Datei mithilfe von \cbash{java -jar jake.jar} oder durch
einen Doppelklick, sofern die entsprechende \T{.jar} als Ausführbar markiert ist. Zieht man bunte Fenster der Kommandozeile vor, so ist man in der Lage mit \newline\cbash{java -jar jake.jar GUI} eine grafische Unterstützung zu erhalten, die allerdings momentan noch in Arbeit und noch lange davon entfernt ist, dieselbe Mächtigkeit wie die Kommandozeile zu erreichen.
\paragraph{Lilly mit Jake installieren}
Nun genügt ein Ausführen von \cbash{jake install}, wobei mithilfe der Option \cbash{-lilly-path} der Pfad angegeben werden kann, an dem sich die LILLY.cls befindet:
\begin{bash}
jake install -lilly-path: '/absoluter/Pfad/zum/Lilly/Ordner'
\end{bash}
Anschließend sollte es möglich sein Dokumente mit LILLY zu kompilieren. Gemeinsam mit LILLY werden eine Vielzahl an Beispieldokumenten ausgeliefert, die die Verwendung anschaulich machen sollen und somit auch als Test für eine erfolgreiche Installation verwendet werden können.
Exemplarisch sei \T{test \& bonus/map\_tests/test.conf} genannt, welches auch die \T{getGraphics}-Schnittstelle etabliert.

\subsection{Windows  \LILLYxNOTExWarning{Ausstehend}}
\subsection{MacOS  \LILLYxNOTExWarning{Ausstehend}}

\subsection{Keine Installation}
\begin{bemerkung}
    Von dieser Methode wird abgeraten
\end{bemerkung}
Natürlich lässt sich Lilly auch so nutzen, hierfür muss einfach nur die zu kompilierende Latex-Datei im selben Verzeichnis wie die Datei \T{Lilly.cls} liegen (also: \T{Lilly}). Natürlich kann dies bei mehreren Dateien, die auf Lilly zugreifen, unübersichtlich werden.
\clearpage
\section[Erstellen eines Dokuments mit Lilly]{Erstellen eines Dokuments mit Lilly \tiny\LILLYxBOXxVersion{1.0.10}}
\subsection{Das Gerüst}
Es ist recht einfach ein Dokument mit Lilly zu erstellen. Da es sich ja um eine Dokumentklasse handelt, wird sie wie folgt eingebunden:
\begin{latex}
\documentclass[Typ=Dokumentation]{Lilly}
\end{latex}
\LILLYxNOTExVersion{1.0.7}Für den Typ gibt es hierfür 4 Optionen:
\begin{multicols}{2}
    \begin{itemize}[label=$\diamond$]\narrowitems
        \item Dokumentation
        \item Mitschrieb
        \item Uebungsblatt
        \item Zusammenfassung
    \end{itemize}
\end{multicols}
Mit \LILLYxBOXxVersion{1.0.10} ist es möglich auch nur \clatex{Dokumentation} anstelle von \clatex{Typ=Dokumentation} zu schreiben. Die Definition für dieses Dokument lautet zum Beispiel:
\begin{latex}
\documentclass[Dokumentation]{Lilly}
\end{latex}
In Kombination mit \Jake ist es zudem noch möglich die Option \clatex{Jake} anzugeben, die es Jake gestattet die Dokumentspezifischen Parameter zu bestimmen.\newline %% TODO: NOTE 'ub' and 'zsfg', more shortcuts?
Zu beachten ist, dass die anderen Optionen weitere Parameter \emph{fordern}. \newline
So benötigt \T{Mitschrieb} noch den Parameter \T{Vorlesung}, der zusammen mit dem Parameter \T{Semester}\marginpar{\tiny  Semester ist Standardmäßig 1} gemäß:
\begin{latex}
\input{\LILLYxPATHxDATA/Semester/\LILLYxSemester/Definitions/
       \LILLYxVorlesung}
\end{latex}
die für die jeweilige Vorlesung definierten Daten lädt.
Erklärungen für die geladenen Daten befinden sich in den jeweiligen README-Dateien:
\begin{description}
    \item[1. Semester] \url{../Lilly/\LILLYxPATHxDATA/Semester/1/Readme.md}
    \item[2. Semester] \url{../Lilly/\LILLYxPATHxDATA/Semester/2/Readme.md}
\end{description}


Weiter nutzt \emph{Uebungsblatt} ebenfalls \T{Vorlesung}\&\T{Semester} sowie noch die optionale Option (tihihi) \T{n} die angibt, um das wievielte Übungsblatt es sich handelt. Darüber müssen wir uns aber in der Regel keine Gedanken machen. Trägt unser Übungsblatt einen Namen wie \T{uebungablatt-gdbs-42.tex}, so kann \Jake über sogenannte \jmark[NameMaps]{mrk:nmaps} entsprechend alles konfigurieren, in diesem Fall benötigt dein Übungsblatt auch kein \T{documentclass} mehr, es genügt das direkte Schreiben von Latex-Code, der Rest wird von Jake übernommen.\\
Entsprechend des Dokumenttyps werden gegebenenfalls auch bereits etliche Seiten generiert, dies gilt es zu beachten, wenn man vielleicht nur etwas testen möchte. In diesem Fall gibt es (wie später auch noch weiter aufgeführt) den sogenannten \emph{Bonustyp} \T{PLAIN}, welcher ein leeres Dokument erstellt! %% TODO: LINK
\clearpage
\subsection[Wie funktionieren Boxen]{Die Böxli}
Jede Box besteht als Environment und lässt sich wie folgt nutzen:
\begin{multicols}{2}
\begin{definition*}[Titel]
    Hallo Welt
\begin{lstlisting}[language=lLatex]
\begin{definition*}[Titel]
    Hallo welt
\end{definition*}\end{lstlisting}
\end{definition*}

\begin{definition}
    Hallo Welt
\begin{lstlisting}[language=lLatex]
\begin{definition}
    Hallo welt
\end{definition}\end{lstlisting}
\end{definition}

\begin{satz}[Titel]
    Hallo Welt
\begin{lstlisting}[language=lLatex]
\begin{satz}[Titel]
    Hallo welt
\end{satz}\end{lstlisting}
\end{satz}
\LILLYcommand{\LILLYxBOXxAufgabexBox}{FALSE}

\begin{aufgabe}{Titel}{3}
    Hallo Welt
\begin{lstlisting}[language=lLatex]
\begin{aufgabe}{Titel}{3}
    Hallo welt
\end{aufgabe}\end{lstlisting}
\end{aufgabe}
\end{multicols}
Letztere ändert sich zum Beispiel mit dem Dokumenttyp, so wird die Aufgabenbox in einem Übungsblatt immernoch wie folgt veranschaulicht:
\LILLYcommand{\LILLYxBOXxAufgabexBox}{TRUE}
\begin{aufgabe}{Titel}{3}
    Hallo Welt
\begin{lstlisting}[language=lLatex]
\begin{aufgabe}{Titel}{3}
    Hallo welt
\end{aufgabe}\end{lstlisting}
\end{aufgabe}
Hier eine Liste aller Boxen:
\begin{multicols}{3}
    \begin{itemize}[label=$\diamond$]\narrowitems
        \item definition
        \item bemerkung
        \item beispiel
        \item satz
        \item beweis
        \item lemma
        \item zusammenfassung
        \item aufgabe
        \item uebungsblatt
    \end{itemize}
\end{multicols}
Sie können alle mithilfe von:
\begin{latex}
%% Allgemein
% \LILLYcommand{\LILLYxBOXx<FirstLetterUp-Name>xEnable}{FALSE}
\LILLYcommand{\LILLYxBOXxDefinitionxEnable}{FALSE}
\end{latex}
jeweils deaktiviert und damit aus dem Dokument entfernt werden (auch nur abschnittsweise, das Reaktivieren funktioniert analog mit \T{TRUE}). \newline
Eine Auflistung ihrer lässt sich mit dem \T{\textbackslash listof} Befehl erzeugen. \marginpar{\tiny  Die Bezeichnung der Listen sind bisher noch inkonsistent :/}Beispielhaft:
\begin{lstlisting}[language=lLatex]
\listofDEFINITIONS
\end{lstlisting}
\marginpar{\tiny  Natürlich sind die Linien nur zur Trennung eingfügt}erzeugt hierbei:\clearpage
\tocloftpagestyle{scrheadings}
\rule{\linewidth}{1.2pt}\vspace{-0.75\baselineskip}
\rule{\linewidth}{0.6pt}\vspace*{-1.5cm}
\listofDEFINITIONS
\rule{\linewidth}{0.6pt}\vspace{-0.7\baselineskip}
\rule{\linewidth}{1.2pt}

\newcommand{\printmark}[2][Linkname]{\ensuremath{\text{#1}^{\rightarrow~\text{\pageref{#2}}}}}





\subsection{Hyperlinks}
\LILLYxNOTExVersion{1.0.0}\elable{mrk:Hey}Eine Sprungmarke innerhalb eines Dokuments lässt sich mit:
\begin{lstlisting}[language=lLatex]
\elable{mrk:Hey} %% \elable{<Sprungmarke>}
\end{lstlisting}
erstellen. Referenziert werden kann sie mithilfe des Befehls \blankcmd{jmark}:
\begin{lstlisting}[language=lLatex]
\jmark[Klick mich]{mrk:Hey} %% \jmark[Text]{Sprungmarke}
\end{lstlisting}
der erzeugte Link: \jmark[Klick mich]{mrk:Hey}, passt sich zudem der Akzentfarbe der aktuellen Boxumgebung und dem Druckmodus an:\smallskip
\begin{zusammenfassung}[Testzusammenfassung]
Siehe hier: \jmark[Klick mich]{mrk:Hey} (Wenn Druck: \printmark[Klick mich]{mrk:Hey})
\end{zusammenfassung}~\smallskip\newline
\marginpar{\tiny \textbf{\T{jmark*}}, welcher die Akzentfarbe ignoriert, ist bereits geplant, wurde allerdings bisher nicht zwangsläufig benötigt }Der alternative Vertreter für \blankcmd{jmark} ist \blankcmd{hmark}, er ignoriert sämtliche Farbattribute:
\begin{lstlisting}[language=lLatex]
\hmark[Klick mich]{mrk:Hey} %% \hmark[Text]{Sprungmarke}
\end{lstlisting}
und erzeugt damit: \hmark[Klick mich]{mrk:Hey}.





\section{Einbinden von weiteren Dokumenten}
\subsection{Aufgliedern eines Dokuments}
\LILLYxNOTExVersion{1.0.4}Um Dokumente portabel kompilierbar zu machen, setzt das Makefile gemäß der Konfiguration \verb|\LILLYxPATH| (hier: \say{\T{\LILLYxPATH}}). Nun lässt sich mithilfe des Befehls \T{\textbf{\textbackslash linput}\{<Pfad>\}} eine Datei relativ zur Quelldatei angeben (beachte, dass absolute Pfade bei \T{\textbf{\textbackslash linput}} keinen Sinn machen. Hierfür solltest du weiterhin \verb|\input| verwenden).\newline
Zudem lässt sich damit über \T{\textbf{\textbackslash LILLYxDOC\-UM\-ENTxSUB\-NAME}} der Name der zuletzt eingebundenen Datei (\T{\LILLYxDOCUMENTxSUBNAME}) abfragen.\newline
\LILLYxNOTExVersion{1.0.7}Weiter gilt zu beachten, dass es \emph{nicht} möglich ist, das klassische \verb|\include| zu verwenden! Dieser Befehl wird aber von LILLY deswegen direkt entsprechend erneuert (hierzu wird das klassische Latex \verb|\input| im Zusammenspiel mit \verb|\clearpage| verwendet, nicht LILLYs \T{\textbf{\textbackslash linput}}!). Es ist also im Endeffekt doch möglich Dokumente mit \verb|\include|
zu verwenden.




\clearpage
\subsection{Übungsblätter}
\marginpar{\tiny Aktuell wird daran gearbeitet eine \T{make}-Regel für Übungsblätter zu integrieren}Da es von Bedeutung ist Übungsblätter so zu erstellen, dass die Abgaben direkt in die Mitschrift eingebunden werden können,\marginpar{\tiny \vspace*{2em}\tiny Die Verwendung von \T{\textbackslash tcblower} ist noch in arbeit} gibt es hierfür eine einfache Möglichkeit:\marginpar{\vspace*{1em} \tiny Es gibt auch eine Umgebung mit * und gleichermaßen \T{\textbf{\textbackslash inputUBS}}. Diese setzen den Zähler für die Aufgaben \emph{nicht} zurück!}
\begin{lstlisting}[language=lLatex]
%% \inputUB{<Name>}{<Nummer>}{<Pfad - linput>}
\inputUB{Mengen}{1}{Aufgaben_Data/Uebungsblatt_1.tex}

%% Wird zu:
\clearpage
\begin{uebungsblatt}[Mengen][1]
    \linput{Aufgaben_Data/Uebungsblatt_1.tex}
\end{uebungsblatt}
\newpage
\end{lstlisting}
Übungsblätter sind nur in \textbf{complete}-Varianten verfügbar, werden also sonst nicht eingebunden! \paragraph{Ein Übungsblatt erstellen}
Doch wie erstellt man nun ein Fachgerechtes Übungsblatt. Nun, da es sich hier um die Schnelleinführung handelt, ein paar Vorgaben. Benenne dein Übungsblatt nach dem Schema: \begin{center}
    \T{uebungsblatt-<VORLESUNG>-<BLATTNUMMER>.tex}
\end{center}
Die Reihenfolge spielt keine Rolle, ein beispielhafter Name könnte sein: \newline\T{gdbs-uebungsblatt-13.tex}. Nun erstelle dir eine \T{jake.conf}-Datei, wobei egal ist wie sie heißt, solange sei auf \T{.conf} endet (fürs Autocomplete \Smiley). In sie trägst du folgendes ein:
\begin{gepard}
file        = @[SELTEXF]
operation   = file_compile

lilly-modes = uebungsblatt

lilly-show-boxname = false

lilly-nameprefix = FlorianS-Partner-
lilly-author = Florian Sihler, Mein Partner

lilly-n = @[AUTONUM]
\end{gepard}
Natürlich kannst du die Namen entsprechend ändern. Das sieht jetzt aus wie viel, aber das musst du nur einmal machen, sofern du die Konfigurationsdatei immer in das Verzeichnis mitkopierst, indem sich die Übungsblatt-\T{.tex} und \textbf{nur} diese \T{.tex}-Datei befindet. Wir werden uns später %% TODO: LINK
mit besseren Konfigurationen beschäftigen, die keinerlei Nachaufwand benötigen und galanter sind.
In das Übungsblatt können wir nun unsere Aufgaben stecken. Hier ist der \emph{gesamte} Inhalt der oben genanten \TeX-Datei:
\begin{latex}
\begin{aufgabe}{Tolle Aufgabe}{400} % 400 Punkte
    Die Aufgabenbeschreibung, blah, blah, blah, \ldots
    \begin{enumeratea}
        \item Teilaufgabe a)
        \item Teilaufgabe b)
        \item \ldots
    \end{enumeratea}
\vSplitter
    \begin{enumeratea}
        \item Antwort zu Teilaufgabe a)
        \item Antwort zu Teilaufgabe b)
        \item \ldots
    \end{enumeratea}
\end{aufgabe}

%% Weitere Aufgaben, wenn gewünscht
\end{latex}

Kompilieren kann man den Spaß nun mit: \cbash{jake jake.conf}. Und das wars, Boom \Smiley.
\clearpage
