\chapter{Einleitung}
\TitleSUB{Integrieren von LILLY -- Die Grundlagen von A-Z}
\section{Installieren von Lilly}
\LILLYxNOTExVersion{1.0.7}Aktuell kommt die Dokumentklasse ohne \T{.ins} oder \T{.dtx} Datei und besitzt keinen Installer - Jaaay \Smiley. Deswegen gibt es im folgenden Erklärungen für die manuelle Installation!
\begin{bemerkung}[Mithilfe]
    Wenn du dich mit \TeX{} oder \LaTeX{} auskennst, schreibe an folgende Email-Adresse \T{\AUTHORMAIL} (Ein Repository wird aktuell eingerichtet).
\end{bemerkung}

\subsection{Linux}
\LILLYxNOTExVersion{1.0.0}Da LILLY komplett auf einem Linux-Betriebsystem entwickelt wurde, gestaltet sich die Implementierung relativ einfach. 
Zuerst gilt es einen neuen Ordner zu erstellen\marginpar{\tiny  Es wird \emph{nicht} auf die Semantik einzelner Befehle eingegangen! Copy\&Paste ist doof, tippen! ;)}:
\begin{lstlisting}[style=bash]
mkdir -p "${HOME}/texmf/tex/latex/"
\end{lstlisting}
In diesen Ordner (wenn nicht sogar bereits existent) kann nun der gesamte \T{Lilly}-Ordner verschoben werden (oder mithilfe eines symbolischen Links verknüpft). Als letztes muss man nun noch \TeX{} über das neue Verzeichnis informieren:\marginpar{\tiny  Dies sichert uns die Persistenz des Pakets im Falle einer Neuinstallation oder eines Updates von \LaTeX}
\begin{lstlisting}[style=bash]
texhash "${HOME}/texmf"
\end{lstlisting}

Nun gilt es sich den anderen mitgelieferten Dateien zu widmen! Von besonderer Relevanz ist hierbei \T{lilly\_compile.sh}, welches hier ausführlicher beschrieben wird(TODO: LINK). Grundlegend generiert es ein Makefile, das dann zum Kompilieren des Dokuments gedacht ist.\newline
Mithilfe von folgendem Befehl wurde das Makefile für diese Dokumentation generiert:
\begin{lstlisting}[style=bash]
*\textcolor{white}{\bfseries \T{./}}*lilly_compile.sh "Lilly-Dokumentation.doc.tex" \
                 -dir="Dokumentation/"
\end{lstlisting}
\LILLYxNOTExVersion{1.0.2}Hierbei wird das Makefile gemäß folgenden Regeln erzeugt:
\begin{itemize}[label=$\diamond$]\narrowitems
    \item Es soll die tex-Datei \say{Lilly-Dokumentation.doc.tex} kompiliert werden.\marginpar{\tiny  Dies muss nicht sein siehe hierfür (TODO) idir und odir}
    \item Das ganze soll (relativ zu \T{lilly\_compile.sh}) im Verzeichnis \T{Dokumentation} stattfinden - hier wird ebenfalls das Makefile generiert. \marginpar{\vspace*{2em}\tiny  Es wird mit den Regeln \T{default, all} und \T{clean} generiert, selbstredend lässt sich dies erweitern}\begin{bemerkung}[make]
        Logischerweise muss damit auch \T{make} auf dem System vorhanden sein:
\begin{lstlisting}[style=bash]
sudo apt install "make"
\end{lstlisting}
    \end{bemerkung}
\end{itemize}
Mit diesem Makefile kann man nun das Dokument generieren lassen. Zu beachten sei hierbei, dass \T{make} - im Falle der Regel \T{all} - Regeln parallel ausführen wird! \newline
\marginpar{\tiny  Die Anführungszeichen dienen hier und in anderen Codebeispielen lediglich zur Übersicht!}Diese Dokumentation wurde mit folgendem Befehl erstellt:
\begin{lstlisting}[style=bash]
make "BOXMODE=LIMERENCE"
\end{lstlisting}
\marginpar{\tiny Dies ändert sich mit Version 1.0.8 und der Übernahme von Jake TODO}Hierbei lässt sich ebenfalls erkennen, wie sich noch mit dem Makefile einzelne Komponenten (wie das verwendete Boxdesign) ändern lassen!
\subsection{Windows \LILLYxNOTExWarning{Ausstehend}}
\subsection{MacOSX \LILLYxNOTExWarning{Ausstehend}}

\subsection{Keine Installation}
\begin{bemerkung}
    Von dieser Methode wird abgeraten
\end{bemerkung}
Natürlich lässt sich Lilly auch so nutzen. Hierfür muss einfach nur die zu kompilierende Latex-Datei im selben Verzeichnis wie die Datei \T{Lilly.cls} liegen (also: \T{Lilly}). Dementsprechend kann dies bei mehreren Dateien, die auf Lilly zugreifen, unübersichtlich werden.
\clearpage
\section[Erstellen eines Dokuments mit Lilly]{Erstellen eines Dokuments mit Lilly \tiny\LILLYxBOXxVersion{1.0.5}}
\subsection{Das Gerüst}
Es ist recht einfach ein Dokument mit Lilly zu erstellen. Da es sich ja um eine Dokumentklasse handelt, wird sie wie folgt eingebunden:
\begin{lstlisting}[style=latex]
\documentclass[Typ=Dokumentation]{Lilly} 
\end{lstlisting}
\LILLYxNOTExVersion{1.0.7}Für den Typ gibt es hierfür 3 Optionen:
\begin{multicols}{3}
    \begin{itemize}[label=$\diamond$]\narrowitems
        \item Dokumentation
        \item Mitschrieb
        \item Uebungsblatt
    \end{itemize}
\end{multicols}
\marginpar{\tiny \bfseries TODO}Zu beachten ist, dass die anderen Optionen weitere Parameter \emph{fordern}. \newline
So benötigt \emph{Mitschrieb} noch den Parameter \T{Vorlesung}, der zusammen mit dem Parameter \T{Semester}\marginpar{\tiny  Semester ist standartmäßig 1} gemäß:
\begin{lstlisting}[style=latex,frame=none]
\input{Lilly_Files/Data/Semster/\LILLY@Semster/Definitions/
       \LILLY@Vorlesung}
\end{lstlisting}
die für die jeweilige Vorlesung definierten Daten lädt (TODO: SIEHE DEFINITIONS).
Weiter nutzt \emph{Uebungsblatt} ebenfalls \T{Vorlesung}\&\T{Semster} sowie noch die optionale Option (tihihi) \T{n} die angibt, um das wievielte Übungsblatt es sich handelt. \\
Im Folgenden wird angenommen der Typ Mitschrieb ist gewählt - für ausführliche Informationen siehe (LINK \LILLYxNOTExWarning{Kommt}).
Nun stehen alle grundlegenden Befehle zur Verfügung.\marginpar{\tiny  Es wird versucht die Befehle für jeden Typ identisch zu deklarieren!} Entsprechend des Dokumenttyps werden gegebenenfalls auch bereits etliche Seiten generiert!
\subsection[Wie funktionieren Boxen]{Die Böxli}
Jede Box besteht als Environment und bringt (leider noch) ihren eigenen Counter mit (einzige Ausnahme: Übungsblätter mit Aufgaben).
Sie lassen sich wie folgt benutzen:
\begin{multicols}{2}
\begin{definition*}[Titel]
    Hallo Welt
\begin{lstlisting}[style=latex]
\begin{definition*}[Titel]
    Hallo welt
\end{definition*}\end{lstlisting}
\end{definition*}

\begin{definition}
    Hallo Welt
\begin{lstlisting}[style=latex]
\begin{definition}
    Hallo welt
\end{definition}\end{lstlisting}
\end{definition}

\begin{satz}[Titel]
    Hallo Welt
\begin{lstlisting}[style=latex]
\begin{satz}[Titel]
    Hallo welt
\end{satz}\end{lstlisting}
\end{satz}
\LILLYcommand{\LILLYxBOXxAufgabexBox}{FALSE}

\begin{aufgabe}[Titel][3]
    Hallo Welt
\begin{lstlisting}[style=latex]
\begin{aufgabe}[Titel][3]
    Hallo welt
\end{aufgabe}\end{lstlisting}
\end{aufgabe}
\end{multicols}
Letztere ändert sich zum Beispiel mit dem Dokumenttyp, so wird die Aufgabenbox in einem Übungsblatt immernoch wie folgt veranschaulicht:
\LILLYcommand{\LILLYxBOXxAufgabexBox}{TRUE}
\begin{aufgabe}[Titel][3]
    Hallo Welt
\begin{lstlisting}[style=latex]
\begin{aufgabe}[Titel][3]
    Hallo welt
\end{aufgabe}\end{lstlisting}
\end{aufgabe}
Hier eine Liste aller Boxen:
\begin{multicols}{3}
    \begin{itemize}[label=$\diamond$]\narrowitems
        \item definition
        \item bemerkung
        \item beispiel
        \item satz
        \item beweis
        \item lemma
        \item zusammenfassung
        \item aufgabe
        \item uebungsblatt
    \end{itemize}
\end{multicols}
Sie können alle mithilfe von:
\begin{lstlisting}[style=latex]
%% Allgemein
% \LILLYcommand{\LILLYxBOXx<FirstLetterUp-Name>xEnable}{FALSE}
\LILLYcommand{\LILLYxBOXxDefinitionxEnable}{FALSE}
\end{lstlisting}
jeweils deaktiviert und damit aus dem Dokument entfernt werden (auch nur abschnittsweise, das Reaktivieren funktioniert analog mit \T{TRUE}). \newline
Eine Auflistung ihrer lässt sich mit dem \T{\textbackslash listof} Befehl erzeugen. \marginpar{\tiny  Die Bezeichnung der Listen sind bisher noch inkonsitent :/}Beispielhaft:
\begin{lstlisting}[style=latex]
\listofDEFINITIONS
\end{lstlisting}
\marginpar{\tiny  Natürlich sind die Linien nur zur Trennung eingfügt}erzeugt hierbei:\\
\tocloftpagestyle{scrheadings}
\rule{\linewidth}{1.2pt}\vspace{-0.75\baselineskip}
\rule{\linewidth}{0.6pt}\vspace*{-1.5cm}
\listofDEFINITIONS    
\rule{\linewidth}{0.6pt}\vspace{-0.7\baselineskip}
\rule{\linewidth}{1.2pt}

\clearpage
\newcommand{\printmark}[2][Linkname]{\ensuremath{\text{#1}^{\rightarrow~\text{\pageref{#2}}}}}

\subsection{Hyperlinks}
\LILLYxNOTExVersion{1.0.0}\elable{mrk:Hey}Eine Sprungmarke innerhalb eines Dokuments lässt sich mit:
\begin{lstlisting}[style=latex]
\elable{mrk:Hey} %% \elable{<Sprungmarke>}
\end{lstlisting}
erstellen. Referenziert werden kann sie mithilfe des Befehls \textbf{\T{jmark}}:
\begin{lstlisting}[style=latex]
\jmark[Klick mich]{mrk:Hey} %% \jmark[Text]{Sprungmarke}
\end{lstlisting}
der erzeugte Link: \jmark[Klick mich]{mrk:Hey}, passt sich zudem der Akzentfarbe der aktuellen Boxumgebung und dem Druckmodus an:\smallskip
\begin{zusammenfassung}[Testzusammenfassung]
Siehe hier: \jmark[Klick mich]{mrk:Hey} (Wenn Druck: \printmark[Klick mich]{mrk:Hey})
\end{zusammenfassung}~\smallskip\newline
\marginpar{\tiny \textbf{\T{jmark*}}, welcher die Akzentfarbe ignoriert, ist bereits geplant, wurde allerdings bisher nicht zwangsläufig benötigt }Der alternative Vertreter für \textbf{\T{jmark}} ist \textbf{\T{hmark}}, er ignoriert sämtliche Farbattribute:
\begin{lstlisting}[style=latex]
\hmark[Klick mich]{mrk:Hey} %% \hmark[Text]{Sprungmarke}
\end{lstlisting}
und erzeugt damit: \hmark[Klick mich]{mrk:Hey}. 

\section{Einbinden von weiteren Dokumenten}
\subsection{Aufgliedern eines Dokuments}
\LILLYxNOTExVersion{1.0.4}Um Dokumente portabel kompilierbar zu machen, setzt das Makefile gemäß der Konfiguration \verb|\LILLYxPATH| (hier: \T{\LILLYxPATH}). Nun lässt sich mithilfe des Befehls \T{\textbf{\textbackslash linput}\{<Pfad>\}} eine Datei relativ zur Quelldatei angeben (beachte, dass absolute Pfade bei \T{\textbf{\textbackslash linput}} keinen Sinn machen. Siehe hierzu: TODO LINK).
Zudem lässt sich damit über \T{\textbf{\textbackslash LILLYxDOC\-UM\-ENTxSUB\-NAME}} der Name der zuletzt eingebundenen Datei (\T{\LILLYxDOCUMENTxSUBNAME}) abfragen.\newline
\LILLYxNOTExVersion{1.0.7}Weiter gilt zu beachten, dass es \emph{nicht} möglich ist, das klassische \verb|\include| zu verwenden! Dieser Befehl wird aber von LILLY deswegen direkt entsprechend erneuert (hierzu wird das klassische Latex \verb|\input| im Zusammenspiel mit \verb|\clearpage| verwendet, nicht LILLYs \T{\textbf{\textbackslash linput}}!).
\subsection{Übungsblätter}
\marginpar{\tiny Aktuell wird daran gearbeitet eine \T{make}-Regel für Übungsblätter zu integrieren}Da es von Bedeutung war Übungsblätter so zu erstellen, dass die Abgaben direkt (passend formatiert) in die Mitschrift eingebunden werden können,\marginpar{\tiny \vspace*{2em}\tiny Die Verwendung von \T{\textbackslash tcblower} ist noch in arbeit} gibt es hierfür eine einfache Möglichkeit:\marginpar{\vspace*{1em} \tiny Es gibt auch eine Umgebung mit * und gleichermaßen \T{\textbf{\textbackslash inputUBS}}. Diese setzen den Zähler für die Aufgaben \emph{nicht} zurück!}
\begin{lstlisting}[style=latex]
%% \inputUB{<Name>}{<Nummer>}{<Pfad - linput>}
\inputUB{Mengen}{1}{Aufgaben_Data/Uebungsblatt_1.tex} 

%% Wird zu: 
\clearpage
\begin{uebungsblatt}[Mengen][1]
    \linput{Aufgaben_Data/Uebungsblatt_1.tex}
\end{uebungsblatt}
\newpage
\end{lstlisting}
Übungsblätter können gleich, wie alle anderen Boxen deaktiviert werden\marginpar{\tiny Hierfür gibt es dann noch TODO die Option noextra}.


\clearpage
