\chapter{Einleitung}
\TitleSUB{Integrieren von LILLY -- Die Grundlagen von A-Z}
\section{Installieren von Lilly}
\LILLYxNOTExVersion{1.0.8}Aktuell kommt die Dokumentklasse ohne \T{.ins} oder \T{.dtx} Datei, dafür allerdings mit einem Installer für alle Debian (Linux) basierten Betriebsysteme und für MacOSX wobei letzerer bisher nur in einer
\emph{light}-Variante zur Verfügung steht.

\begin{bemerkung}[Mithilfe]
    Wenn du dich mit \TeX{} oder \LaTeX{} auskennst, schreibe an folgende Email-Adresse \T{\AUTHORMAIL}.\smallskip\newline
    Mittlerweile gibt es auch ein offizielles Github-Repository (\url{https://github.com/EagleoutIce/LILLY} \href{https://github.com/EagleoutIce/LILLY}{\faGithub})
    über das die gesamte Entwicklung abläuft. Hier werden noch Helfer für folgende Aufgaben gesucht:
    \begin{multicols}{2}
        \begin{itemize}[label=$\diamond$]
            \item C++11/C++14 - Entwicklung
            \item Makefile/Bash - Entwicklung
            \item Kommentieren in Markdown
            \item Maintainig (\TeX, \LaTeX)
            \item Kommentieren in Doxygen
            \item Layout Gestaltung
            \item \TeX, \LaTeX{} -Entwicklung
            \item Tester (\faLinux, \faApple, \faWindows)
        \end{itemize}
    \end{multicols}
\end{bemerkung}

\subsection{Linux}
\begin{center}
    Für Versionen < 1.0.8 klicke hier: \jmark[klick mich]{jmp:oldInstallLinux}!
\end{center}
\LILLYxNOTExVersion{1.0.8}Da LILLY komplett auf einem Linux-Betriebsystem entwickelt wurde, gestaltet sich die Implementierung relativ einfach. 
Hierzu nutzen wir das Hilfsprogramm \Jake welches selbst in C++ geschrieben wurde. 
Im Folgendenen sind die Schritte kurz erklärt. 
\paragraph{Installation von \Jake:}Eine ausführliche Erklärung von \Jake selbst findest sich weiter hinten (\jmark[hier]{jmp:iJake}) in dieser Dokumentation:\smallskip
\marginpar{\tiny Für ausführliche Informationen zur Installation konsultiere bitte die README-Datei in: 
\url{../Lilly/Jake/jake_source/README.md}. \newline
Für Informationen zur Nutzung konsultiere: \url{../Lilly/Jake/README.md}}
\begin{enumerate}\setlength{\itemsep}{0.25\baselineskip}
    \item Navigiere mit dem Terminal in das Verzeichnis: \verb|Lilly/Jake/jake_source|
    \item Führe nun \verb|make| aus um \Jake zu kompilieren. 
          Es wird vermutlich kurz dauern, aber danach wird dir das Programm \LJake zur Verfügung stehen.
    \item Nun kannst du dein Terminal neu starten und von überall her \verb|lilly_jake install| aufrufen.
          Dies sollte den Installationsprozess in Gang setzen.\smallskip
\end{enumerate}
Sollte das Ganze fehlerfrei verlaufen sein, dann: Glückwunsch, du hast Lilly erfolgreich installiert!
Betrachte im Falle eines Fehlers bitte erst die Readme-Dateien und die bereits beantworteten Fehler
auf Github (\href{https://github.com/EagleoutIce/LILLY/issues}{\faGithub}) bevor du einen neuen Fehler
eröffnest oder mir eine Nachricht schreibst \Smiley.
\paragraph{Erstellen eines Makefiles:}
Nun möchtest du natürlich auch ausprobieren ob die Installation funktioniert hat. 
Hierzu kannst du in das Testverzeichnis navigieren (\verb|Lilly/Jake/tests|). 
Hier befinden sich eine Menge Dateien die in dieser Dokumentation auch als Beispiele benutzt werden.
Du gibst nun folgendes in die Konsole ein: 
\begin{lstlisting}[style=bash]
lilly_jake test.tex
\end{lstlisting}
\Jake erstellt nun ein entsprechendes Makefile für dich, welches du nun ausführen kannst:
\begin{lstlisting}[style=bash]
make 
\end{lstlisting}
Im standartmäßig konfigurierten Ausgabe-Ordner \verb|test-OUT| befindet sich nun eine entsprechende PDF
Datei \Smiley.\smallskip
\begin{bemerkung}[make]
    Logischerweise muss damit auch \T{make} auf dem System vorhanden sein:
\begin{lstlisting}[style=bash]
sudo apt install "make"
\end{lstlisting}
\end{bemerkung}

\subsection{Windows \LILLYxNOTExWarning{Ausstehend}}
\subsection{MacOS}
\begin{center}
    Entspricht, dank \Jake, der Linux-Installation.
\end{center}
Hierzu nutzen wir das Hilfsprogramm \Jake welches selbst in C++ geschrieben wurde. 
Im Folgendenen sind die Schritte kurz erklärt. 
\paragraph{Installation von \Jake:}Eine ausführliche Erklärung von \Jake selbst findest sich weiter hinten (TODO: LINK) in dieser Dokumentation:\smallskip
\marginpar{\tiny Für ausführliche Informationen zur Installation konsultiere bitte die README-Datei in: 
\url{../Lilly/Jake/jake_source/README.md}. \newline
Für Informationen zur Nutzung konsultiere: \url{../Lilly/Jake/README.md}}
\begin{enumerate}\setlength{\itemsep}{0.25\baselineskip}
    \item Navigiere mit dem Terminal in das Verzeichnis: \verb|Lilly/Jake/jake_source|
    \item Führe nun \verb|make| aus um \Jake zu kompilieren. 
          Es wird vermutlich kurz dauern, aber danach wird dir das Programm \LJake zur Verfügung stehen.
    \item Nun kannst du dein Terminal neu starten und von überall her \verb|lilly_jake install| aufrufen.
          Dies sollte den Installationsprozess in Gang setzen.\smallskip
\end{enumerate}
Sollte das Ganze fehlerfrei verlaufen sein, dann: Glückwunsch, du hast Lilly erfolgreich installiert!
Betrachte im Falle eines Fehlers bitte erst die Readme-Dateien und die bereits beantworteten Fehler
auf Github (\href{https://github.com/EagleoutIce/LILLY/issues}{\faGithub}) bevor du einen neuen Fehler
eröffnest oder mir eine Nachricht schreibst \Smiley.
\paragraph{Erstellen eines Makefiles:}
Nun möchtest du natürlich auch ausprobieren ob die Installation funktioniert hat. 
Hierzu kannst du in das Testverzeichnis navigieren (\verb|Lilly/Jake/tests|). 
Hier befinden sich eine Menge Dateien die in dieser Dokumentation auch als Beispiele benutzt werden.
Du gibst nun folgendes in die Konsole ein: 
\begin{lstlisting}[style=bash]
lilly_jake test.tex
\end{lstlisting}
\Jake erstellt nun ein entsprechendes Makefile für dich, welches du nun ausführen kannst:
\begin{lstlisting}[style=bash]
make 
\end{lstlisting}
Im standartmäßig konfigurierten Ausgabe-Ordner \verb|test-OUT| befindet sich nun eine entsprechende PDF
Datei \Smiley.\smallskip
\begin{bemerkung}[make]
    Logischerweise muss damit auch \T{make} auf dem System vorhanden sein:
\begin{lstlisting}[style=bash]
sudo apt install "make"
\end{lstlisting}
\end{bemerkung}

\subsection{Keine Installation}
\begin{bemerkung}
    Von dieser Methode wird abgeraten
\end{bemerkung}
Natürlich lässt sich Lilly auch so nutzen, hierfür muss einfach nur die zu kompilierende Latex-Datei im selben Verzeichnis wie die Datei \T{Lilly.cls} liegen (also: \T{Lilly}). Natürlich kann dies bei mehreren Dateien, die auf Lilly zugreifen, unübersichtlich werden.
\clearpage
\section[Erstellen eines Dokuments mit Lilly]{Erstellen eines Dokuments mit Lilly \tiny\LILLYxBOXxVersion{1.0.5}}
\subsection{Das Gerüst}
Es ist recht einfach ein Dokument mit Lilly zu erstellen. Da es sich ja um eine Dokumentklasse handelt, wird sie wie folgt eingebunden:
\begin{lstlisting}[style=latex]
\documentclass[Typ=Dokumentation]{Lilly} 
\end{lstlisting}
\LILLYxNOTExVersion{1.0.7}Für den Typ gibt es hierfür 4 Optionen:
\begin{multicols}{2}
    \begin{itemize}[label=$\diamond$]\narrowitems
        \item Dokumentation
        \item Mitschrieb
        \item Uebungsblatt
        \item Zusammenfassung
    \end{itemize}
\end{multicols}
Zu beachten ist, dass die anderen Optionen weitere Parameter \emph{fordern}. \newline
So benötigt \verb|Mitschrieb| noch den Parameter \T{Vorlesung}, der zusammen mit dem Parameter \T{Semester}\marginpar{\tiny  Semester ist standartmäßig 1} gemäß:
\begin{lstlisting}[style=latex,frame=none]
\input{\LILLYxPATHxDATA/Semester/\LILLY@Semester/Definitions/
       \LILLY@Vorlesung}
\end{lstlisting}
die für die jeweilige Vorlesung definierten Daten lädt.
Erklärungen für die geladenen Daten befinden sich in den jeweiligen README-Dateien:
\begin{description}
    \item[1. Semester] \url{../Lilly/\LILLYxPATHxDATA/Semester/1/Readme.md}
    \item[2. Semester] \url{../Lilly/\LILLYxPATHxDATA/Semester/2/Readme.md}
\end{description}


Weiter nutzt \emph{Uebungsblatt} ebenfalls \T{Vorlesung}\&\T{Semester} sowie noch die optionale Option (tihihi) \T{n} die angibt, um das wievielte Übungsblatt es sich handelt. \\

Entsprechend des Dokumenttyps werden gegebenenfalls auch bereits etliche Seiten generiert!

\subsection[Wie funktionieren Boxen]{Die Böxli}
Jede Box besteht als Environment und lässt sich wie folgt nutzen: 
\begin{multicols}{2}
\begin{definition*}[Titel]
    Hallo Welt
\begin{lstlisting}[style=latex]
\begin{definition*}[Titel]
    Hallo welt
\end{definition*}\end{lstlisting}
\end{definition*}

\begin{definition}
    Hallo Welt
\begin{lstlisting}[style=latex]
\begin{definition}
    Hallo welt
\end{definition}\end{lstlisting}
\end{definition}

\begin{satz}[Titel]
    Hallo Welt
\begin{lstlisting}[style=latex]
\begin{satz}[Titel]
    Hallo welt
\end{satz}\end{lstlisting}
\end{satz}
\LILLYcommand{\LILLYxBOXxAufgabexBox}{FALSE}

\begin{aufgabe}[Titel][3]
    Hallo Welt
\begin{lstlisting}[style=latex]
\begin{aufgabe}[Titel][3]
    Hallo welt
\end{aufgabe}\end{lstlisting}
\end{aufgabe}
\end{multicols}
Letztere ändert sich zum Beispiel mit dem Dokumenttyp, so wird die Aufgabenbox in einem Übungsblatt immernoch wie folgt veranschaulicht:
\LILLYcommand{\LILLYxBOXxAufgabexBox}{TRUE}
\begin{aufgabe}[Titel][3]
    Hallo Welt
\begin{lstlisting}[style=latex]
\begin{aufgabe}[Titel][3]
    Hallo welt
\end{aufgabe}\end{lstlisting}
\end{aufgabe}
Hier eine Liste aller Boxen:
\begin{multicols}{3}
    \begin{itemize}[label=$\diamond$]\narrowitems
        \item definition
        \item bemerkung
        \item beispiel
        \item satz
        \item beweis
        \item lemma
        \item zusammenfassung
        \item aufgabe
        \item uebungsblatt
    \end{itemize}
\end{multicols}
Sie können alle mithilfe von:
\begin{lstlisting}[style=latex]
%% Allgemein
% \LILLYcommand{\LILLYxBOXx<FirstLetterUp-Name>xEnable}{FALSE}
\LILLYcommand{\LILLYxBOXxDefinitionxEnable}{FALSE}
\end{lstlisting}
jeweils deaktiviert und damit aus dem Dokument entfernt werden (auch nur abschnittsweise, das Reaktivieren funktioniert analog mit \T{TRUE}). \newline
Eine Auflistung ihrer lässt sich mit dem \T{\textbackslash listof} Befehl erzeugen. \marginpar{\tiny  Die Bezeichnung der Listen sind bisher noch inkonsitent :/}Beispielhaft:
\begin{lstlisting}[style=latex]
\listofDEFINITIONS
\end{lstlisting}
\marginpar{\tiny  Natürlich sind die Linien nur zur Trennung eingfügt}erzeugt hierbei:\\
\tocloftpagestyle{scrheadings}
\rule{\linewidth}{1.2pt}\vspace{-0.75\baselineskip}
\rule{\linewidth}{0.6pt}\vspace*{-1.5cm}
\listofDEFINITIONS    
\rule{\linewidth}{0.6pt}\vspace{-0.7\baselineskip}
\rule{\linewidth}{1.2pt}

\clearpage
\newcommand{\printmark}[2][Linkname]{\ensuremath{\text{#1}^{\rightarrow~\text{\pageref{#2}}}}}





\subsection{Hyperlinks}
\LILLYxNOTExVersion{1.0.0}\elable{mrk:Hey}Eine Sprungmarke innerhalb eines Dokuments lässt sich mit:
\begin{lstlisting}[style=latex]
\elable{mrk:Hey} %% \elable{<Sprungmarke>}
\end{lstlisting}
erstellen. Referenziert werden kann sie mithilfe des Befehls \textbf{\T{jmark}}:
\begin{lstlisting}[style=latex]
\jmark[Klick mich]{mrk:Hey} %% \jmark[Text]{Sprungmarke}
\end{lstlisting}
der erzeugte Link: \jmark[Klick mich]{mrk:Hey}, passt sich zudem der Akzentfarbe der aktuellen Boxumgebung und dem Druckmodus an:\smallskip
\begin{zusammenfassung}[Testzusammenfassung]
Siehe hier: \jmark[Klick mich]{mrk:Hey} (Wenn Druck: \printmark[Klick mich]{mrk:Hey})
\end{zusammenfassung}~\smallskip\newline
\marginpar{\tiny \textbf{\T{jmark*}}, welcher die Akzentfarbe ignoriert, ist bereits geplant, wurde allerdings bisher nicht zwangsläufig benötigt }Der alternative Vertreter für \textbf{\T{jmark}} ist \textbf{\T{hmark}}, er ignoriert sämtliche Farbattribute:
\begin{lstlisting}[style=latex]
\hmark[Klick mich]{mrk:Hey} %% \hmark[Text]{Sprungmarke}
\end{lstlisting}
und erzeugt damit: \hmark[Klick mich]{mrk:Hey}. 





\section{Einbinden von weiteren Dokumenten}
\subsection{Aufgliedern eines Dokuments}
\LILLYxNOTExVersion{1.0.4}Um Dokumente portabel kompilierbar zu machen, setzt das Makefile gemäß der Konfiguration \verb|\LILLYxPATH| (hier: \say{\T{\LILLYxPATH}}). Nun lässt sich mithilfe des Befehls \T{\textbf{\textbackslash linput}\{<Pfad>\}} eine Datei relativ zur Quelldatei angeben (beachte, dass absolute Pfade bei \T{\textbf{\textbackslash linput}} keinen Sinn machen. Hierfür solltest du weiterhin \verb|\input| verwenden).\newline
Zudem lässt sich damit über \T{\textbf{\textbackslash LILLYxDOC\-UM\-ENTxSUB\-NAME}} der Name der zuletzt eingebundenen Datei (\T{\LILLYxDOCUMENTxSUBNAME}) abfragen.\newline
\LILLYxNOTExVersion{1.0.7}Weiter gilt zu beachten, dass es \emph{nicht} möglich ist, das klassische \verb|\include| zu verwenden! Dieser Befehl wird aber von LILLY deswegen direkt entsprechend erneuert (hierzu wird das klassische Latex \verb|\input| im Zusammenspiel mit \verb|\clearpage| verwendet, nicht LILLYs \T{\textbf{\textbackslash linput}}!). Es ist also im Endeffekt doch möglich Dokumente mit \verb|\include| 
zu verwenden.





\subsection{Übungsblätter}
\marginpar{\tiny Aktuell wird daran gearbeitet eine \T{make}-Regel für Übungsblätter zu integrieren}Da es von Bedeutung war Übungsblätter so zu erstellen, dass die Abgaben direkt in die Mitschrift eingebunden werden können,\marginpar{\tiny \vspace*{2em}\tiny Die Verwendung von \T{\textbackslash tcblower} ist noch in arbeit} gibt es hierfür eine einfache Möglichkeit:\marginpar{\vspace*{1em} \tiny Es gibt auch eine Umgebung mit * und gleichermaßen \T{\textbf{\textbackslash inputUBS}}. Diese setzen den Zähler für die Aufgaben \emph{nicht} zurück!}
\begin{lstlisting}[style=latex]
%% \inputUB{<Name>}{<Nummer>}{<Pfad - linput>}
\inputUB{Mengen}{1}{Aufgaben_Data/Uebungsblatt_1.tex} 

%% Wird zu: 
\clearpage
\begin{uebungsblatt}[Mengen][1]
    \linput{Aufgaben_Data/Uebungsblatt_1.tex}
\end{uebungsblatt}
\newpage
\end{lstlisting}
Übungsblätter sind nur in \textbf{complete}-Varianten Verfügbar! TODO LINK


\clearpage
