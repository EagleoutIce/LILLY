\normalmarginpar

\renewcommand{\arraystretch}{1.5}
\chapter[Grafiken \LILLYxBOXxVersion{\small 1.0.0}]{Grafiken}
\TitleSUB{Etliche Vereinfachungen und andere Freuden :D\hfill \LILLYxBOXxVersion{\small 1.0.2}}
\bigskip\newline
\elable{chp:GRAPHICS}Dieses Paket liegt hier: \begin{center}
    \blankcmd{LILLYxPATHxGRAPHICS} = \T{\LILLYxPATHxGRAPHICS}
\end{center}
\begin{bemerkung}[Standalone-Graphics]
    Mit \LILLYxBOXxVersion{2.0.0} wurde die Grafik-Integration als eigenes Paket (\LILLYxNOTExLibrary{LILLYxGRAPHICS}) etabliert, welches sich auch eigenständig über \begin{latex}
\usepackage{LILLYxGRAPHICS}
        \end{latex}
        auch ohne das Verwenden der restlichen LILLY-Welt benutzen.
\end{bemerkung}
%
%
%
\section{Grundlegende Symbole}
Diese Definitionen befinden sich in der Datei: {\ltt\LILLYxPATHxGRAPHICS/Tikz-Core/\_LILLY\_TIKZ\_SYMBOLS}. Sie werden mit \LILLYxBOXxVersion{2.0.0} automatisch mit dem Einbinden von\newline \LILLYxNOTExLibrary{LILLYxGRAPHICS} geladen.\medskip\newline
Dieses Paket liefert grundlegende, mal mehr und mal weniger, nützliche Tikz-Grafiken, welche zum Großteil aus denen in der Vorlesung verwendeten Grafiken entstanden sind. Alle diese Grafiken benötigen Ti\textit{k}Z (\url{https://www.ctan.org/pkg/pgf}).
\subsection{Die Ampeln}
Diese Definitionen befinden sich in der Datei: {\ltt../Tikz-Core/\_LILLY\_TIKZ\_AMPELN}
An sich handelt es sich hierbei um ein kleines Shortcut-Sammelsurium für Ampeln:\medskip
%
%
%
\presentCommand[1.0.2]{ampelG}[\cmdlist \textbackslash ampelY\cmdlist \textbackslash ampelR\cmdlist \textbackslash ampelH]
Explizit verwendet werden diese Befehle in zum Beispiel in den Erklärungen zum Moore-\&Mealy-Automaten auf Basis der Ampelschaltung (\!\ampelG\ampelY\ampelH):
\displayCommandList{ampelG,ampelY,ampelR,ampelH}

\subsection{Emoticons  \LILLYxNOTExWarning{Ausstehend}}
Dieses Paket soll weitere lustige Begleiter im Textgeschehen zur Verfügung stellen:
\displayCommandList[3]{Ninja,Smiley,Sadey,Xey,Innocey,Walley,dSadey,Fire,Autumntree}

\subsection{Utility  \LILLYxNOTExWarning{Ausstehend}}
Dieses Paket soll die bisher von FontAwesome verwendeten Symbolen ersetzen und durch eigens erstellte Grafiken ersetzen.
%
%
%
%
%
\section{Diagramme \& Graphen}
\subsection{Graphen}
Diese Definitionen befinden sich in der Datei: {\ltt\LILLYxPATHxGRAPHICS/Tikz-Core/\_LILLY\_TIKZ\_GRAPHEN}. Sie werden mit \LILLYxBOXxVersion{2.0.0} automatisch mit dem Einbinden von\newline \LILLYxNOTExLibrary{LILLYxGRAPHICS} geladen.\newline
\begin{bemerkung}[Motivation]
Dieses Paket liefert grundlegende, mal mehr und mal weniger, nützliche Tikz-Grafiken, welche zum Großteil aus denen in der Vorlesung verwendeten Grafiken entstanden sind. Alle diese Grafiken benötigen Ti\textit{k}Z (\url{https://www.ctan.org/pkg/pgf}).
\end{bemerkung}
%
%
%
\presentCommand[1.0.2]{POLYRAD}[~\tiny$\langle$length$\rangle$]
Grundlegend wird für den Radius aller Polygone empfohlen \blankcmd{POLYRAD} zu verwenden (Standardmäßig: \T{1.61cm}).\medskip\newline
Weiter definiert diese Bibliothek etliche sogenannte \T{graphdot}s, welche alle nur in einer tikzpicture-Umgebung funktionieren, allen vorran die Ur-Funktion:\medskip
%
%
%
\presentCommand[1.0.2]{graphdot}[\manArg{fill-color}\manArg{(PosX,PosY)}\manArg{node-name}\manArg{border-color}\cmdlist\newline\hbox{}~~\textbackslash tgraphdot\manArg{fill-color}\manArg{(PosX,PosY)}\manArg{node-name}\manArg{border-color}]
Die Befehle unterscheiden sich darin, dass der \blankcmd{tgraphdot} das Farbargument ignoriert und entsprechend transparent (\T{fill opacity = 0})als Füllfarbe verwendet:
\begin{center}\renewcommand{\arraystretch}{1.75}
    \begin{tabular}{!{\VRule[1pt]}@{\hspace{1em}}C{0.6\linewidth}@{\hspace{1em}}|@{\hspace{1em}}C{1cm}@{\hspace{1em}}!{\VRule[1pt]}}
        \specialrule{1pt}{0pt}{0pt}
        \clatex{:bs:graphdot\{DebianRed\}\{(0,0)\}\{42\}\{a\}\{Azure\}} &\raisebox{-0.375\baselineskip}{\tikz{\graphdot{DebianRed}{(0,0)}{42}{a}{Azure}}}\\\hline
        \clatex{:bs:tgraphdot\{DebianRed\}\{(0,0)\}\{42\}\{a\}\{Azure\}} &\raisebox{-0.375\baselineskip}{\tikz{\tgraphdot{DebianRed}{(0,0)}{42}{a}{Azure}}}\\\hline
        \specialrule{1pt}{0pt}{0pt}
    \end{tabular}
\end{center}
%
%
%
\presentCommand[1.0.2]{oragraphdot}[\cmdlist \textbackslash blugraphdot\cmdlist \textbackslash gregraphdot\cmdlist\textbackslash purgraphdot\cmdlist\textbackslash golgraphdot\cmdlist\secline\textbackslash blagraphdot\cmdlist\textbackslash norgraphdot\cmdlist\textbackslash margraphdot]
Alle weiteren graphdots sind nun nichts weiteres als Shortcuts für die eben genannten Befehle und besitzen die Signatur: \blankcmd{oragraphdot\manArg{(PosX,PosY)}\manArg{Text}\manArg{node-name}}:\vspace{-1em}
\begin{multicols}{3}%
    \begin{ditemize}\narrowitems%
        \foreach \cmd in {ora,blu,gre,pur,gol,bla,nor,mar}{%
    \item \blankcmd{\cmd graphdot} (\,\raisebox{-0.375\baselineskip}{\tikz{\csname\cmd graphdot\endcsname{(0,0)}{42}{a};}}\,)%
        }%
    \end{ditemize}%
\end{multicols}%
Zur Information, alle diese Befehle wurden wie folgt präsentiert:
\begin{latex}
:bs:tikz{:bs::lan:graphdot:ran:{(0,0)}{42}{a}};
\end{latex}
wobei \clatex{:lan:graphdot:ran:} entsprechend ersetzt wurde, weiter wurde für den Textfluss noch die Boxposition angepasst, dies spielt allerdings für den Graphen keine Rolle. Mit \LILLYxBOXxVersion{2.0.0} wurden die Farben der Dots der neuen Palette %%TODO: link
entsprechend portiert.\medskip
%
%
%
\presentCommand[1.0.4]{graphPOI}[\manArg{(PosX,PosY)}\manArg{accent-color}\manArg{year}\manArg{obj-name}\manArg{brief}\secline\manArg{img-path}\manArg{img-link}\manArg{extra}]
Präsentiert ein Timeline Point-of-interest, der schnell einen einheitlichen look für Timelines garantiert. Im Folgenden eine repräsentation, die den Wirrwarr an Optionen etwas übersichtlicher macht. Es gilt zu beachten, dass \clatex{:lan:extra:ran:} hier die Rolle des entsprechendes Landes einnimmt:\vspace{-1.15\baselineskip}
%
%%%% COMPAT CHEAT
\LILLYcommand{\graphPOICOMPAT}[8]{
\filldraw [thick,color=#2] #1 circle (1.45pt) node[align=left] {}  -- ++(0.5,0) node [right,color=#2] {\begin{minipage}{0.49\textwidth}\small{\textbf{#3}}\hfill #8\end{minipage}} ++(-0.3,-0.2) node[below right] {\begin{minipage}{0.5\textwidth}
    \textbf{\textcolor{#2!80}{#4:}} \textcolor{#2!65}{#5}
\end{minipage} } ++(9cm,0.3) node [below right] {\begin{minipage}[c]{0.3\textwidth}
    \href{#7}{\fcolorbox{#2}{white}{\includegraphics[height=2cm]{\LILLYxPATH#6}}}
\end{minipage}} ;}%%%%%%%%%%
\begin{center}%\renewcommand{\arraystretch}{0.25}
\begin{tabular}{!{\VRule[1pt]}@{\hspace{1em}}C{0.45\textwidth}@{\hspace{1em}}|@{\hspace{1em}}C{0.55\textwidth}@{\hspace{1em}}!{\VRule[1pt]}}
    \specialrule{1pt}{0pt}{0pt}
    {\tiny\begin{latex}[numberstyle=\tiny\color{gray}]
\begin{tikzternal}[scale=0.75,
        every node/.style={transform shape}]
    \graphPOI{(0,0)}{purple}{1999 n.Chr.}
      {Florian Sihler}
      {Florian Sihler ist der Autor dieses Dokuments.}
      {Data/2003.jpg}
      {https://github.com/EagleoutIce/Quickblit}
      {Deutschland};
\end{tikzternal}
    \end{latex}}&{%
    \begin{tikzternal}[scale=0.75,
        every node/.style={transform shape}]
            \graphPOICOMPAT{(0,0)}{purple}{1999 n.Chr.}{Florian Sihler}{Florian Sihler ist der Autor dieses Dokuments.}{Data/2003.jpg}{https://github.com/EagleoutIce/Quickblit}{Deutschland};
    \end{tikzternal}}\\
    \specialrule{1pt}{0pt}{0pt}
    \end{tabular}
\end{center}
Hier wurde aus Platzgründen die Größe angepasst. Es gibt auch einen weiteren Befehl der es ermöglicht den \blankcmd{graphPOI}-Befehl einzuschränken:\medskip
%
%
%
\presentCommand[1.0.4]{LILLYxMODExEXTRA}
Wir dieser Befehl auf \verb|\true| (\true) gesetzt, so wird \blankcmd{graphPOI} so konfiguriert, dass die zugehörige Grafik angezeigt wird. Ist dies nicht der Fall (in anderen Worten: \blankcmd{LILLYxMODExEXTRA}=\verb|\false|), so wird kein Bild angezeigt (auch der Link existiert dann nicht). Diese Version wurde erstellt um Urheberrechtsverletzungen zu vermeiden.\medskip
%
%
%
\presentCommand[2.0.0]{PgetXY}[\manArg{Point}\manArg{out:x-cord}\manArg{out:y-cord}\cmdlist\secline\textbackslash PgetX\manArg{out:x-cord}\cmdlist \textbackslash PgetY\manArg{out:y-cord}]
Da es oft notwendig ist die Koordinate eines Punktes weiter zu benutzen und da das Kreuzen von Koordinaten nervig ist, gibt es verschiedene Befehle die es erlauben, die entsprechenden Koordinaten zu speichern, wobei die letzteren beiden nur lesbarere Alternativen für die erste sind, sofern die entsprechend andere Koordinate nicht benötigt wird:\vspace{-1.15\baselineskip}
\begin{center}%\renewcommand{\arraystretch}{0.25}
    \begin{tabular}{!{\VRule[1pt]}@{\hspace{1em}}C{0.6\textwidth}@{\hspace{1em}}|@{\hspace{1em}}C{0.4\textwidth}@{\hspace{1em}}!{\VRule[1pt]}}
        \specialrule{1pt}{0pt}{0pt}
{\begin{latex}
\begin{tikzternal}
  \node (A) at (1,2) {A};
  \PgetXY{(A)}{\myX}{\myY};
  % Befehle werden gebunden
  \node (B) at (\myX,0) {B};
  \PgetY{(B)}{\anotherY};
  \node (C) at (1.5*\myY,\anotherY) {C};
\end{tikzternal}
\end{latex}} &
\begin{tikzternal}
    \draw[thin,xshift=0.5cm,yshift=0.5cm] (-1,-2) grid (3,2); %step=1.0
    \node (A) at (1,2) {A};
    \PgetXY{(A)}{\myX}{\myY};
    \node (B) at (\myX,0) {B};
    \PgetY{(B)}{\anotherY};
    \node (C) at (1.5*\myY,\anotherY) {C};
\end{tikzternal} \\
    \specialrule{1pt}{0pt}{0pt}
    \end{tabular}
\end{center}
Was hierbei auch interessant ist: die Skalierung von $X$- und $Y$-Koordinaten wird unabhängig voneinander getroffen, das heißt die $Y$-Koordinate eines Punktes als die $X$-Koordinate eines anderen zu verwenden funktioniert (meist) nicht ohne mathematische Operationen. Das Gitter wurde natürlich nachträglich hinzugefügt:%\newline
\begin{latex}
\draw[thin,xshift=0.5cm,yshift=0.5cm] (-1,-2) grid (3,2);
\end{latex}
%
%
%
%
%
%
\subsection{Rotation}
Diese Definitionen befinden sich in der Datei: {\ltt\LILLYxPATHxGRAPHICS/Tikz-Core/\_LILLY\_TIKZ\_ROTATION}. Sie werden mit \LILLYxBOXxVersion{2.0.0} automatisch mit dem Einbinden von\newline \LILLYxNOTExLibrary{LILLYxGRAPHICS} geladen.\medskip\newline
%
%
%
\presentCommand[1.0.4]{rotateRPY}[\optArg[0/0/0]{transform-point}\manArg{roll}\manArg{pitch}\manArg{yaw}]
Dieser Befehl wird verwendet um erstellte Ti\textit{k}Z Grafiken zu drehen und dementsprechend anzupassen. Dieser Code entstammt der Feder von David Carlisle und Tom Bombadil\footnote{\url{https://tex.stackexchange.com/questions/67573/tikz-shift-and-rotate-in-3d}} und wird hier beispielhaft illustriert:
\newcommand{\examplecube}%
{   \coordinate (a) at (-2,-2,-2);
    \coordinate (b) at (-2,-2,2);
    \coordinate (c) at (-2,2,-2);
    \coordinate (d) at (-2,2,2);
    \coordinate (e) at (2,-2,-2);
    \coordinate (f) at (2,-2,2);
    \coordinate (g) at (2,2,-2);
    \coordinate (h) at (2,2,2);
    \draw (a)--(b) (a)--(c) (a)--(e) (b)--(d) (b)--(f) (c)--(d) (c)--(g) (d)--(h) (e)--(f) (e)--(g) (f)--(h) (g)--(h);
    \fill[Ao] (a) circle (0.1cm);
    \fill[tealblue] (d) ++(0.1cm,0.1cm) rectangle ++(-0.2cm,-0.2cm);
}
\begin{center}
\begin{tabular}{!{\VRule[1pt]}@{\hspace{0.5em}}C{0.55\textwidth}@{\hspace{0.5em}}|@{\hspace{0.5em}}C{0.5\textwidth}@{\hspace{0.5em}}!{\VRule[1pt]}}
    \specialrule{1pt}{0pt}{0pt}
    {\tiny\begin{lstplain}[language=lLatex,numberstyle=\tiny\color{gray}]
\begin{tikzternal}
    \examplecube
    \rotateRPY[-2/2/2]{13}{171}{55} %% Rotate set
    \begin{scope}[draw=purple, text=purple,
                    fill=purple, densely dashed, RPY]
        \examplecube
    \end{scope}
    \draw[tealblue,ultra thick] (-2,2,2) -- (\savedx,2,2) -- (\savedx,\savedy,2) -- (\savedx,\savedy,\savedz) circle (0.25);
    \rotateRPY[-2/-2/-2]{13}{171}{55}
    \draw[Ao,ultra thick] (-2,-2,-2) -- (\savedx,-2,-2) -- (\savedx,\savedy,-2) -- (\savedx,\savedy,\savedz) circle (0.25);
    \node at (2.5,-3.5){RPY: 13,171,55};
\end{tikzternal}
    \end{lstplain}} &  \begin{tikzternal}
        \examplecube
        \rotateRPY[-2/2/2]{13}{171}{55}
        \begin{scope}[draw=purple, text=purple,fill=purple,densely dashed,RPY]
            \examplecube
        \end{scope}
        \draw[tealblue,ultra thick] (-2,2,2) -- (\savedx,2,2) -- (\savedx,\savedy,2) -- (\savedx,\savedy,\savedz) circle (0.25);
        \rotateRPY[-2/-2/-2]{13}{171}{55}
        \draw[Ao,ultra thick] (-2,-2,-2) -- (\savedx,-2,-2) -- (\savedx,\savedy,-2) -- (\savedx,\savedy,\savedz) circle (0.25);
        \node at (2.5,-3.5) {RPY: 13,171,55};
    \end{tikzternal} \\
    \specialrule{1pt}{0pt}{0pt}
    \end{tabular}
\end{center}
%
%
%
%
%
\subsection{Automaten \LILLYxNOTExWarning{Work in Progress}}
Diese Definitionen befinden sich in der Datei: {\ltt\LILLYxPATHxGRAPHICS/Tikz-Core/\_LILLY\_TIKZ\_AUTOMATEN}. Sie werden mit \LILLYxBOXxVersion{2.0.0} automatisch mit dem Einbinden von\newline \LILLYxNOTExLibrary{LILLYxGRAPHICS} geladen.\medskip\newline
Obwohl bereits Ti\textit{k}Z eine Bibliothek für das Generieren von Automaten zur Verfügung stellt, wurde dieses (Work in Progress) Paket erstellt um darauf aufbauend schnell Automaten erstellen zu können. Der Grundbefehl lautet:\medskip
%
%
%
\presentCommand[1.0.3]{loopTo}[\optArg[1]{looseness}\manArg{arc}\manArg{node-name}\manArg{Text}\manArg{Orientierung}]
Dieser Befehl setzt grundlegend einen Pfeil, der von einem Knoten aus wieder zu sich selbst führt. Im folgenden sind $4$ verschiedene Shortcuts, die für die klassischen Himmelsrichtungen die Pfeile vordefinieren:\medskip
%
%
%
\presentCommand[1.0.3]{loopTop}[\optArg[1]{looseness}\manArg{node-name}\manArg{Text}\cmdlist \textbackslash loopRight\optArg[1]{lsns}\secline\manArg{node-name}\manArg{Text}\cmdlist\textbackslash loopLeft\optArg[1]{lsns}\manArg{node-name}\manArg{Text}\cmdlist\secline\textbackslash loopBot\optArg[1]{lsns}\manArg{node-name}\manArg{Text}]
Im folgenden sei eine beispielhafte Verwendung gezeigt (der Automat muss keinen Sinn ergeben es soll lediglich die Nutzung verdeutlicht werden):
\begin{center}
\begin{tabular}{!{\VRule[1pt]}@{\hspace{0.5em}}C{0.65\textwidth}@{\hspace{0.5em}}|@{\hspace{0.5em}}C{0.4\textwidth}@{\hspace{0.5em}}!{\VRule[1pt]}}
    \specialrule{1pt}{0pt}{0pt}
    {\scriptsize\begin{latex}
\begin{tikzternal}[scale=1,
    every node/.style={minimum size=12pt,transform shape},
    state/.style={circle, !*\T{draw}*!, minimum size=20pt},
    every path/.style={!*\T{draw}*!, -latex},
    every initial by arrow/.style={-latex, initial text=}]

    \node[initial,accepting,state] (1) at (180:1){\T{1}};
    \node[state] (2) at (  0:1){\T{2}};

    \draw (1) to node[pos=0.5,above,sloped]{\T{0}} (2);
    \loopTop[4]{1}{\T{4}};
    \loopRight[4]{2}{\T{2}};
\end{tikzternal}
    \end{latex}} &      \begin{tikzternal}[scale=1,
        every node/.style={minimum size=12pt,transform shape},
        state/.style={circle, draw, minimum size=20pt},
        every path/.style={draw, -latex},
        every initial by arrow/.style={-latex, initial text=},
        ]
        \node[initial,accepting,state] (1) at (180:1){\T{1}};
        \node[state]                   (2) at (  0:1){\T{2}};

        \draw (1) to node[pos=0.5,above,sloped]{\T{0}} (2);

        \loopTop[4]{1}{\T{4}};
        \loopRight[4]{2}{\T{2}};

        \end{tikzternal} \\
        \specialrule{1pt}{0pt}{0pt}
        \end{tabular}
\end{center}

Natürlich soll dieses Erstellen noch weiter stark vereinfacht werden. Des Weiteren wird darüber nachgedacht, einen akzeptierten Endzustand klarer zu markieren (Linien dicker, mehr abstand etc). Der Traum wäre, dass das Erstellen eines Automaten wie folgt funktioniert:
\begin{lstlisting}[language=lLatex]
\begin{Automat}
    \STATE[1]{180:1}{1};
    \state[2]{0:1}{2};

    \draw (1) to node[midway,above]{0} (2);

    \loopTop[4]{1}{\T{4}};
    \loopRight[4]{2}{\T{2}};
\end{Automat}
\end{lstlisting}
Die Befehle \blankcmd{state} und \blankcmd{STATE} sollen hierbei automatisch hochzählen können - pro Automat - aber über das optionale Argument lesbar einer Zahl zugewiesen werden. Die Umgebung Automat soll hierbei zusätzlich auch handhaben, dass automatisch alle Nodes mithilfe von \T{\textbackslash T} geschrieben werden. Der entstehende Automat soll optisch identisch zum obigen sein, dies wird allerdings erst auf das Bedürfnis hin übernommen.
\subsection{Schaltkreise \LILLYxNOTExWarning{Ausstehend}}

\subsection{Neuronen \LILLYxNOTExWarning{Work in Progress}}
Diese Definitionen befinden sich in der Datei: {\ltt\LILLYxPATHxGRAPHICS/Tikz-Core/\_LILLY\_TIKZ\_NEURONS}. Sie werden mit \LILLYxBOXxVersion{2.0.0} automatisch mit dem Einbinden von\newline \LILLYxNOTExLibrary{LILLYxGRAPHICS} geladen.\medskip\newline
Da vor allem mit \fg der Wunsch danach aufkam, neuronale Netze schnell zu Texen, wurde dieses Paket entwickelt um das Paket mit den Schaltkreisen so zu erweitern, dass es erlaubt Perzeptronen darin einzubauen, das  Paket an sich befindet sich ebenfalls im Work in Progress-Status. \textit{Das Schaltkreise-Paket ist ebenfalls noch nicht in LILLY integriert. Es befindet sich ebenfalls in einem Anfangsstadium und deswegen wird auch hierbei um Mithilfe bei der Weiterentwicklung gebeten.}\medskip
%
%
%
\presentCommand[1.0.5]{neuronSquare}[\manArg{pos:00}\manArg{pos:01}\manArg{pos:10}\manArg{pos:11}]
Es wurde bisher auch nur durch das Bereitstellen eines einzelnen Befehls implementiert: \blankcmd{neuronSquare}. Dieser funktioniert seinerseits lediglich in einer \emph{tikzpicture}/\emph{tikzternal} Umgebung und zeichnet nichtmal ein Neuron, sondern lediglich die 2-D Repräsentation eines booleschen Raums, der wiedergibt unter welchen Eingabevektoren das Perzeptron welchen Wert zurückliefert. Die 4 Parameter, die hierzu \blankcmd{neuronSquare} benötigt, entsprechen der jeweiligen Binärdarstellung der Eingabevektoren. Eine beispielhafte Anwendung ist hier zu finden:
\[\begin{tabular}{!{\VRule[1pt]}@{\hspace{0.5em}}C{0.35\textwidth}@{\hspace{0.5em}}|@{\hspace{0.5em}}C{0.3\textwidth}@{\hspace{0.5em}}!{\VRule[1pt]}}
    \specialrule{1pt}{0pt}{0pt}
    {\scriptsize\begin{lstlisting}[language=lLatex,frame=none,breaklines=true]
\begin{tikzternal}
    \neuronSquare{T}{F}{T}{T};
\end{tikzternal}
    \end{lstlisting}} & \begin{tikzternal}
        \node at(0,2.25) {};
        \neuronSquare{T}{F}{T}{T};
    \end{tikzternal} \\
        \specialrule{1pt}{0pt}{0pt}
        \end{tabular}\]
Hierbei steht ein \T{T} (true) natürlich für einen akzeptierten, ein \T{F} (false) entsprechend für einen nicht akzeptierten Befehl. Aktuell ist geplant, dass der Befehl auch für 1-, 3- und 4-dimensionale Räume eine Option anbietet (siehe für 4D: Titelgrafik \gdra[]), die dann über einen einfacheren Namen abgegriffen werden kann. Weiter sollen dann \fg und \gdra (boolesche Räume) diese Befehle nutzen anstelle der dafür eigens implementierten Grafiken. Weiter soll es möglich sein über ein optionales Argument die Position (relativ) zu bestimmen!
%
%
%
\section{Mitgelieferte Grafiken}
Diese Definitionen befinden sich in der Datei: {\ltt\LILLYxPATHxGRAPHICS/LILLYxGRAPHICSxPROVIDER.sty}. Sie werden mit \LILLYxBOXxVersion{2.0.0} automatisch mit dem Einbinden von\newline \LILLYxNOTExLibrary{LILLYxGRAPHICS} geladen.\medskip\newline
Dieser Teil existiert weiter auch als eigenes Paket mit: \LILLYxNOTExLibrary{LILLYxGRAPHICSxPROVIDER} und hängt vom Mutterpaket ab. 

\presentCommand[2.0.0]{getGraphics}[\optArg[\textbackslash linewidth]{width}\manArg{path}\optArg{height}]
Erlaubt den Zugriff auf zahlreiche Grafiken, die im Rahmen der Arbeit entstanden sind. Bei einer Angabe von Breite und Höhe gewinnt die Breite, da stets nur eine Dimension skaliert wird! Die bisher enthaltenen Grafiken können durch \cbash{jake get} abgerufen werden: \begin{center}
    \begin{tabular}{C{0.45\linewidth}_C{0.45\linewidth}}
        \getGraphics{Rechner/PLAAmpel} & \getGraphics{Software/ThreadStates} \\
        \tiny \blatex{:bs:getGraphics\{Rechner/PLAAmpel\}} & \tiny\blatex{:bs:getGraphics\{Software/ThreadState\}}
    \end{tabular}
\end{center}
Die Größe skaliert sich in der Regel automatisch, allerdings existieren auch Grafiken, die automatisch nicht skaliert werden, da sie Code oder andere nicht skalierfähige Elemente enthalten: \begin{center}
    \getGraphics{Software/XMLUebersicht}
    \blatex{:bs:getGraphics\{Software/XMLUebersicht\}}
\end{center}

\presentCommand[2.0.0]{getGraphicsPath}[\manArg{path}]
Liefert den absoluten Pfad zu einer Grafik. Beispiel: \T{\blankcmd{getGraphicsPath}\{Software/XMLUebersicht\}} liefert:\newline \T{\getGraphicsPath{Software/XMLUebersicht}}. 

\presentCommand[2.0.0]{getPrerendered}[\optArg[\textbackslash linewidth]{width}\manArg{path}\optArg{height}]
Erlaubt eine automatisch an die Seitenbreite skalierte Implementation von bereits vorberechneten Grafiken. Bei einer Angabe von Breite und Höhe gewinnt die Breite, da stehts nur eine Dimension skaliert wird! Sie werden in der Grafiksammlung durch den Tag \T{pdf} gekennzeichnet (die Breite wurde im Beispiel angepasst, oder lassen sich durch das Anfügen eines \say{\T{-pdf}}-Suffix:
\begin{center}
    \begin{tabular}{C{0.45\linewidth}_C{0.45\linewidth}}
        \getPrerendered{Eigene/Proseminar/Cluster/rolf-special} & \getPrerendered{Software/ThreadStates-pdf} \\
        \tiny \blatex{:bs:getPrerendered\{Eigene/Proseminar/Cluster/rolf-special\}} & \tiny\blatex{:bs:getPrerendered\{Software/ThreadState-pdf\}}
    \end{tabular}
\end{center}
%Weiter existiert der Befehl \blankcmd{lillyGraphicsUsePrerenderedtrue}, mit dem automatisch jeder \blankcmd{getGraphics}-Aufruf in einen entsprechenden \blankcmd{getPrerendered} Aufruf umgewandelt wird. 
\emph{Es gilt zu beachten, dass die bereits vorgenerierten Grafiken von den manuell generierten abweichen können!}