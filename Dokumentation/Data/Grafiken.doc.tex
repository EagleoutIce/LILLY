\renewcommand{\arraystretch}{1.5}
\chapter[Grafiken\lilib{LILLYxGRAPHICS}{1.0.0}]{Grafiken}
\TitleSUB{Etliche Vereinfachungen und andere Freuden :D\hfill \LILLYxBOXxVersion{\small 1.0.2}}
\bigskip\newline
\elable{chp:GRAPHICS}\hypertarget{LILLYxGRAPHICS}Dieses Paket liegt hier: \begin{center}
    \blankcmd{LILLYxPATHxGRAPHICS} = \T{\LILLYxPATHxGRAPHICS}
\end{center}
\begin{bemerkung}[Standalone-Graphics]
    Mit \LILLYxBOXxVersion{2.0.0} wurde die Grafik-Integration als eigenes Paket \LILLYxNOTExLibrary{LILLYxGRAPHICS} etabliert, welches sich eigenständig über \begin{latex*}
\usepackage{LILLYxGRAPHICS}
        \end{latex*}
        auch ohne das Verwenden der restlichen LILLY-Welt benutzen lässt.
\end{bemerkung}
\hypertarget{LILLYxTIKZxCORE}Dieses Paket basiert übrigens auf \LILLYxNOTExLibrary{LILLYxTIKZxCORE}, welche selbst nicht direkt sondern nur durch die hier verfügbaren Befehle dokumentiert ist und auch nicht frei verwendet werden sollte (ausgenommen natürlich, man weiß, wie \Smiley).\medskip\\
Das Laden des Pakets mit LILLY kann durch die Option \T{graphics} aktiviert werden (Standard) und durch \T{nographics} entsprechend dekativiert. So sorgt das Deklarieren von:
\begin{latex*}
\documentclass[nographics]{Lilly}
\end{latex*}
Für ein Lilly-Dokument ohne \LILLYxNOTExLibrary{LILLYxGRAPHICS}.

%
%
%

\section{Grundlegende Symbole}
Diese Definitionen befinden sich in der Datei: {\ltt\LILLYxPATHxGRAPHICS/Tikz-Core/\_LILLY\_TIKZ\_SYMBOLS}. Sie werden mit \LILLYxBOXxVersion{2.0.0} automatisch mit dem Einbinden von\newline \LILLYxNOTExLibrary{LILLYxGRAPHICS} geladen.\medskip\newline
Dieses Paket liefert grundlegende, mal mehr und mal weniger, nützliche Tikz-Grafiken, welche zum Großteil aus denen in der Vorlesung verwendeten Grafiken entstanden sind. Alle diese Grafiken benötigen Ti\textit{k}Z (\url{https://www.ctan.org/pkg/pgf}).

%
%
%

\presentCommand[1.0.0]{rectat}[\manArg{point}\cmdlist\anothercmd[1.0.0]{crectat}\manArg{point}\manArg{color}]
Ersterer Befehl setzt in einem \blankenv{tikzpicture} ein (rot-)farbiges Rechteck, der zweite erlaubt die Auswahl der jeweiligen Farbe:
\begin{defaultlst}[][listing side text,righthand width=2cm]{lLatex}
\begin{tikzpicture}
    \rectat{(0,1)};
    \crectat{(0,0)}{AppleGreen};
\end{tikzpicture}
\end{defaultlst}



\subsection{Die Ampeln}
Diese Definitionen befinden sich in der Datei: {\ltt../Tikz-Core/\_LILLY\_TIKZ\_AMPELN}.
An sich handelt es sich hierbei um ein kleines Shortcut-Sammelsurium für Ampeln:\medskip

%
%
%

\presentCommand[1.0.2]{ampelG}[\cmdlist \anothercmd[1.0.2]{ampelY}\cmdlist \anothercmd[1.0.2]{ampelR}\cmdlist \anothercmd[1.0.2]{ampelH}]
Explizit verwendet werden diese Befehle in zum Beispiel in den Erklärungen zum Moore-\&Mealy-Automaten auf Basis der Ampelschaltung (\!\ampelG\ampelY\ampelH):
\displayCommandList{ampelG,ampelY,ampelR,ampelH}

\subsection{Emoticons  \LILLYxNOTExWarning{Ausstehend}}
Dieses Paket soll weitere lustige Begleiter im Textgeschehen zur Verfügung stellen:
\displayCommandList[3]{Ninja,Smiley,Sadey,Xey,Innocey,Walley,dSadey,Fire,Autumntree}

\subsection{Utility  \LILLYxNOTExWarning{Ausstehend}}
Dieses Paket soll die bisher von FontAwesome verwendeten Symbolen ersetzen und durch eigens erstellte Grafiken ersetzen.

%
%
%
%
%

\section{Diagramme \& Graphen}
\subsection{Graphen}
Diese Definitionen befinden sich in der Datei: {\ltt\LILLYxPATHxGRAPHICS/Tikz-Core/\_LILLY\_TIKZ\_GRAPHEN}. Sie werden mit \LILLYxBOXxVersion{2.0.0} automatisch mit dem Einbinden von\newline \LILLYxNOTExLibrary{LILLYxGRAPHICS} geladen.\newline
\begin{bemerkung}[Motivation]
Dieses Paket liefert grundlegende, mal mehr und mal weniger, nützliche Tikz-Grafiken, welche zum Großteil aus denen in der Vorlesung verwendeten Grafiken entstanden sind. Alle diese Grafiken benötigen Ti\textit{k}Z (\url{https://www.ctan.org/pkg/pgf}).
\end{bemerkung}

%
%
%

\presentCommand[1.0.2]{POLYRAD}[~\tiny$\langle$length$\rangle$]
Grundlegend wird für den Radius aller Polygone empfohlen \blankcmd{POLYRAD} zu verwenden (Standardmäßig: \T{1.61cm}).\medskip\newline
Weiter definiert diese Bibliothek etliche sogenannte \T{graphdot}s, welche alle nur in einer tikzpicture-Umgebung funktionieren, allen vorran die Ur-Funktion:

%
%
%

\presentCommand[1.0.2]{graphdot}[\manArg{fill-color}\manArg{(PosX,PosY)}\manArg{node-name}\manArg{border-color}\cmdlist\newline\hbox{}~~\anothercmd[1.0.2]{tgraphdot}\manArg{fill-color}\manArg{(PosX,PosY)}\manArg{node-name}\manArg{border-color}]
Die Befehle unterscheiden sich darin, dass der \blankcmd{tgraphdot} das Farbargument ignoriert und entsprechend transparent (\T{fill opacity = 0}) als Füllfarbe verwendet:
\begin{center}\renewcommand{\arraystretch}{1.75}
    \begin{tabular}{!{\VRule[1pt]}@{\hspace{1em}}C{0.6\linewidth}@{\hspace{1em}}|@{\hspace{1em}}C{1cm}@{\hspace{1em}}!{\VRule[1pt]}}
        \specialrule{1pt}{0pt}{0pt}
        {\blatex[morekeywords={[5]{graphdot}}]{:bs:graphdot\{DebianRed\}\{(0,0)\}\{42\}\{a\}\{Azure\}}} &\raisebox{-0.375\baselineskip}{\tikz{\graphdot{DebianRed}{(0,0)}{42}{a}{Azure}}}\\\hline
        {\blatex[morekeywords={[5]{tgraphdot}}]{:bs:tgraphdot\{DebianRed\}\{(0,0)\}\{42\}\{a\}\{Azure\}}} &\raisebox{-0.375\baselineskip}{\tikz{\tgraphdot{DebianRed}{(0,0)}{42}{a}{Azure}}}\\\hline
        \specialrule{1pt}{0pt}{0pt}
    \end{tabular}
\end{center}

%
%
%

\presentCommand[1.0.2]{oragraphdot}[\cmdlist \anothercmd[1.0.2]{bluegraphdot}\cmdlist \anothercmd[1.0.2]{gregraphdot}\cmdlist\anothercmd[1.0.2]{purgraphdot}\cmdlist\secline\anothercmd[1.0.2]{golgraphdot}\cmdlist\anothercmd[1.0.2]{blagraphdot}\cmdlist\anothercmd[1.0.2]{norgraphdot}\cmdlist\anothercmd[1.0.2]{margraphdot}]
Alle weiteren graphdots sind nun nichts weiteres als Shortcuts für die eben genannten Befehle und besitzen die Signatur: \T{\blankcmd{oragraphdot}\manArg{(PosX,PosY)}\manArg{Text}\manArg{node-name}}:
\begin{multicols}{3}%
    \begin{ditemize}\narrowitems%
        \foreach \cmd in {ora,blu,gre,pur,gol,bla,nor,mar}{%
    \item \blankcmd{\cmd graphdot} (\,\raisebox{-0.375\baselineskip}{\tikz{\csname\cmd graphdot\endcsname{(0,0)}{42}{a};}}\,)%
        }%
    \end{ditemize}%
\end{multicols}%
Zur Information, alle diese Befehle wurden wie folgt präsentiert:
\begin{latex}[morekeywords={[5]{tikz}}]
:bs:tikz!**!\:lan:graphdot:ran:{(0,0)}{42}{a}};
\end{latex}
wobei \clatex{:lan:graphdot:ran:} entsprechend ersetzt wurde, weiter wurde für den Textfluss noch die Boxposition angepasst, dies spielt allerdings für den Graphen keine Rolle. Mit \LILLYxBOXxVersion{2.0.0} wurden die Farben der Dots der neuen Palette entsprechend portiert.

%
%
%

\presentCommand[1.0.4]{graphPOI}[\manArg{(PosX,PosY)}\manArg{accent-color}\manArg{year}\manArg{obj-name}\manArg{brief}\secline\manArg{img-path}\manArg{img-link}\manArg{extra}]
Präsentiert ein Timeline Point-of-interest, der schnell einen einheitlichen Look für Timelines garantiert.
Im Folgenden eine Repräsentation, die den Wirrwarr an Optionen etwas übersichtlicher macht. Es gilt zu beachten, dass \clatex{:lan:extra:ran:} hier die Rolle des entsprechendes Landes einnimmt.\iflillycompact\else
\vspace{-1.15\baselineskip}
%
% COMPAT CHEAT
\LILLYcommand\graphPOICOMPAT[8]{
\filldraw [thick,color=#2] #1 circle (1.45pt) node[align=left] {}  -- ++(0.5,0) node [right,color=#2] {\begin{minipage}{0.49\textwidth}\small{\textbf{#3}}\hfill #8\end{minipage}} ++(-0.3,-0.2) node[below right] {\begin{minipage}{0.5\textwidth}
    \textbf{\textcolor{#2!80}{#4:}} \textcolor{#2!65}{#5}
\end{minipage} } ++(9cm,0.3) node [below right] {\begin{minipage}[c]{0.3\textwidth}
    \href{#7}{\fcolorbox{#2}{white}{\includegraphics[height=2cm]{\LILLYxPATH#6}}}
\end{minipage}} ;}%
\begin{center}%\renewcommand{\arraystretch}{0.25}
\begin{tabular}{!{\VRule[1pt]}@{\hspace{1em}}C{0.45\textwidth}@{\hspace{1em}}|@{\hspace{1em}}C{0.55\textwidth}@{\hspace{1em}}!{\VRule[1pt]}}
    \specialrule{1pt}{0pt}{0pt}
    {\tiny\begin{plainlatex}
\begin{tikzternal}[scale=0.75,
        every node/.style={transform shape}]
    \graphPOI{(0,0)}{purple}{1999 n.Chr.}
        {Florian Sihler}
        {Florian Sihler ist der Autor dieses Dokuments.}
        {Data/2003.jpg}
        {https://github.com/EagleoutIce/Quickblit}
        {Deutschland};
\end{tikzternal}
    \end{plainlatex}}&{%
    \begin{tikzternal}[scale=0.75,
        every node/.style={transform shape}]
            \graphPOICOMPAT{(0,0)}{purple}{1999 n.Chr.}{Florian Sihler}{Florian Sihler ist der Autor dieses Dokuments.}{Data/2003.jpg}{https://github.com/EagleoutIce/Quickblit}{Deutschland};
    \end{tikzternal}}\\
    \specialrule{1pt}{0pt}{0pt}
    \end{tabular}
\end{center}
Hier wurde aus Platzgründen die Größe angepasst.\fi
Es gibt auch noch \blankcmd{LILLYxMODExEXTRA} der es ermöglicht den \blankcmd{graphPOI}-Befehl einzuschränken.
Wir dieser Befehl auf \verb|\true| (\true) gesetzt, so wird \blankcmd{graphPOI} so konfiguriert, dass die zugehörige Grafik angezeigt wird. Ist dies nicht der Fall (in anderen Worten: \blankcmd{LILLYxMODExEXTRA}=\verb|\false|), so wird kein Bild angezeigt (auch der Link existiert dann nicht). Diese Version wurde erstellt um Urheberrechtsverletzungen zu vermeiden.

%
%
%

\presentCommand[2.0.0]{PgetXY}[\manArg{Point}\manArg{out:x-cord}\manArg{out:y-cord}\cmdlist\secline\anothercmd[2.0.0]{PgetX}\manArg{out:x-cord}\cmdlist \anothercmd[2.0.0]{PgetY}\manArg{out:y-cord}]
Da es oft notwendig ist die Koordinate eines Punktes weiter zu benutzen und da das Kreuzen von Koordinaten nervig ist, gibt es verschiedene Befehle die es erlauben, die entsprechenden Koordinaten zu speichern, wobei die letzteren beiden nur lesbarere Alternativen für die erste sind, sofern die entsprechend andere Koordinate nicht benötigt wird:\vspace{-1.15\baselineskip}
\begin{center}%\renewcommand{\arraystretch}{0.25}
    \begin{tabular}{!{\VRule[1pt]}@{\hspace{1em}}C{0.6\textwidth}@{\hspace{1em}}|@{\hspace{1em}}C{0.4\textwidth}@{\hspace{1em}}!{\VRule[1pt]}}
        \specialrule{1pt}{0pt}{0pt}
{\begin{plainlatex}
\begin{tikzternal}
  \node (A) at (1,2) {A};
  \PgetXY{(A)}{\myX}{\myY};
  % Befehle werden gebunden
  \node (B) at (\myX,0) {B};
  \PgetY{(B)}{\anotherY};
  \node (C) at (1.5*\myY,\anotherY) {C};
\end{tikzternal}
\end{plainlatex}
} &
\begin{tikzternal}
    \draw[thin,xshift=0.5cm,yshift=0.5cm] (-1,-2) grid (3,2); %step=1.0
    \node (A) at (1,2) {A};
    \PgetXY{(A)}{\myX}{\myY};
    \node (B) at (\myX,0) {B};
    \PgetY{(B)}{\anotherY};
    \node (C) at (1.5*\myY,\anotherY) {C};
\end{tikzternal} \\
    \specialrule{1pt}{0pt}{0pt}
    \end{tabular}
\end{center}
Was hierbei auch interessant ist: die Skalierung von $X$- und $Y$-Koordinaten wird unabhängig voneinander getroffen, das heißt die $Y$-Koordinate eines Punktes als die $X$-Koordinate eines anderen zu verwenden funktioniert (meist) nicht ohne mathematische Operationen. Das Gitter wurde natürlich nachträglich hinzugefügt:%\newline
\begin{latex*}
\draw[thin,xshift=0.5cm,yshift=0.5cm] (-1,-2) grid (3,2);
\end{latex*}

%
%
%
%
%
%

\subsection{Binäre Arrays}
Diese Definitionen befinden sich in der Datei: {\ltt\LILLYxPATHxGRAPHICS/Tikz-Core/\_LILLY\_TIKZ\_BINARY\_ARRAY}. Sie werden mit \LILLYxBOXxVersion{2.2.0} automatisch mit dem Einbinden von\newline \LILLYxNOTExLibrary{LILLYxGRAPHICS} geladen.

\presentCommand[2.2.0]{BinaryArray}[\manArg{BinData}\manArg{NumOfElements}\manArg{WantedWidth}]
Setzt ein Binäres Array in einer schönen Repräsentation. Die Farbe für \say{aktive} Felder wird mittels \blankcmd{BinaryArray@activecol} definiert.
\begin{latex*}
\begin{tikzpicture}
    \BinaryArray{1100110010101010}{16}{9cm}
\end{tikzpicture}
\end{latex*}
Ergibt: \begin{center}
    \begin{tikzpicture}
        \BinaryArray{1100110010101010}{16}{9cm}
    \end{tikzpicture}
\end{center}

%
%
%
%
%
%

\subsection{Rotation}
Diese Definitionen befinden sich in der Datei: {\ltt\LILLYxPATHxGRAPHICS/Tikz-Core/\_LILLY\_TIKZ\_ROTATION}. Sie werden mit \LILLYxBOXxVersion{2.0.0} automatisch mit dem Einbinden von\newline \LILLYxNOTExLibrary{LILLYxGRAPHICS} geladen.

%
%
%

\presentCommand[1.0.4]{rotateRPY}[\optArg[0/0/0]{transform-point}\manArg{roll}\manArg{pitch}\manArg{yaw}]
Dieser Befehl wird verwendet um erstellte Ti\textit{k}Z Grafiken zu drehen und dementsprechend anzupassen. Dieser Code entstammt der Feder von David Carlisle und Tom Bombadil\footnote{\url{https://tex.stackexchange.com/questions/67573/tikz-shift-and-rotate-in-3d}} und wird hier beispielhaft illustriert:
\newcommand{\examplecube}%
{   \coordinate (a) at (-2,-2,-2);
    \coordinate (b) at (-2,-2,2);
    \coordinate (c) at (-2,2,-2);
    \coordinate (d) at (-2,2,2);
    \coordinate (e) at (2,-2,-2);
    \coordinate (f) at (2,-2,2);
    \coordinate (g) at (2,2,-2);
    \coordinate (h) at (2,2,2);
    \draw (a)--(b) (a)--(c) (a)--(e) (b)--(d) (b)--(f) (c)--(d) (c)--(g) (d)--(h) (e)--(f) (e)--(g) (f)--(h) (g)--(h);
    \fill[Ao] (a) circle (0.1cm);
    \fill[tealblue] (d) ++(0.1cm,0.1cm) rectangle ++(-0.2cm,-0.2cm);
}
\begin{center}
\begin{tabular}{!{\VRule[1pt]}@{\hspace{0.5em}}C{0.55\textwidth}@{\hspace{0.5em}}|@{\hspace{0.5em}}C{0.5\textwidth}@{\hspace{0.5em}}!{\VRule[1pt]}}
    \specialrule{1pt}{0pt}{0pt}
    {\tiny\begin{lstplain}[language=lLatex,numberstyle=\tiny\color{gray}]
\begin{tikzternal}
    \examplecube
    \rotateRPY[-2/2/2]{13}{171}{55} % Rotate set
    \begin{scope}[draw=purple, text=purple,
                    fill=purple, densely dashed, RPY]
        \examplecube
    \end{scope}
    \draw[tealblue,ultra thick] (-2,2,2) -- (\savedx,2,2) -- (\savedx,\savedy,2) -- (\savedx,\savedy,\savedz) circle (0.25);
    \rotateRPY[-2/-2/-2]{13}{171}{55}
    \draw[Ao,ultra thick] (-2,-2,-2) -- (\savedx,-2,-2) -- (\savedx,\savedy,-2) -- (\savedx,\savedy,\savedz) circle (0.25);
    \node at (2.5,-3.5){RPY: 13,171,55};
\end{tikzternal}
    \end{lstplain}
    } &  \begin{tikzternal}
        \examplecube
        \rotateRPY[-2/2/2]{13}{171}{55}
        \begin{scope}[draw=purple, text=purple,fill=purple,densely dashed,RPY]
            \examplecube
        \end{scope}
        \draw[tealblue,ultra thick] (-2,2,2) -- (\savedx,2,2) -- (\savedx,\savedy,2) -- (\savedx,\savedy,\savedz) circle (0.25);
        \rotateRPY[-2/-2/-2]{13}{171}{55}
        \draw[Ao,ultra thick] (-2,-2,-2) -- (\savedx,-2,-2) -- (\savedx,\savedy,-2) -- (\savedx,\savedy,\savedz) circle (0.25);
        \node at (2.5,-3.5) {RPY: 13,171,55};
    \end{tikzternal} \\
    \specialrule{1pt}{0pt}{0pt}
    \end{tabular}
\end{center}

%
%
%
%
%

\subsection{Automaten \LILLYxNOTExWarning{Work in Progress}}
Diese Definitionen befinden sich in der Datei: {\ltt\LILLYxPATHxGRAPHICS/Tikz-Core/\_LILLY\_TIKZ\_AUTOMATEN}. Sie werden mit \LILLYxBOXxVersion{2.0.0} automatisch mit dem Einbinden von\newline \LILLYxNOTExLibrary{LILLYxGRAPHICS} geladen.\medskip\newline
Obwohl bereits Ti\textit{k}Z eine Bibliothek für das Generieren von Automaten zur Verfügung stellt, wurde dieses (Work in Progress) Paket erstellt um darauf aufbauend schnell Automaten erstellen zu können. Der Grundbefehl lautet:

%
%
%

\presentCommand[1.0.3]{loopTo}[\optArg[1]{looseness}\manArg{arc}\manArg{node-name}\manArg{Text}\manArg{Orientierung}]
Dieser Befehl setzt grundlegend einen Pfeil, der von einem Knoten aus wieder zu sich selbst führt. Im folgenden sind $4$ verschiedene Shortcuts, die für die klassischen Himmelsrichtungen die Pfeile vordefinieren:

%
%
%

\presentCommand[1.0.3]{loopTop}[\optArg[1]{looseness}\manArg{node-name}\manArg{Text}\cmdlist \anothercmd[1.0.3]{loopRight}\optArg[1]{lsns}\secline\manArg{node-name}\manArg{Text}\cmdlist\anothercmd[1.0.3]{loopLeft}\optArg[1]{lsns}\manArg{node-name}\manArg{Text}\cmdlist\secline\anothercmd[1.0.3]{loopBot}\optArg[1]{lsns}\manArg{node-name}\manArg{Text}]
Im folgenden sei eine beispielhafte Verwendung gezeigt (der Automat muss keinen Sinn ergeben es soll lediglich die Nutzung verdeutlicht werden):
\begin{center}
\begin{tabular}{!{\VRule[1pt]}@{\hspace{0.5em}}C{0.65\textwidth}@{\hspace{0.5em}}|@{\hspace{0.5em}}C{0.4\textwidth}@{\hspace{0.5em}}!{\VRule[1pt]}}
    \specialrule{1pt}{0pt}{0pt}
    {\scriptsize\begin{plainlatex}
\begin{tikzternal}[scale=1,
    every node/.style={minimum size=12pt,transform shape},
    state/.style={circle, !*\T{draw}*!, minimum size=20pt},
    every path/.style={!*\T{draw}*!, -latex},
    every initial by arrow/.style={-latex, initial text=}]

    \node[initial,accepting,state] (1) at (180:1){\T{1}};
    \node[state] (2) at (  0:1){\T{2}};

    \draw (1) to node[pos=0.5,above,sloped]{\T{0}} (2);
    \loopTop[4]{1}{\T{4}};
    \loopRight[4]{2}{\T{2}};
\end{tikzternal}
    \end{plainlatex}
    } &      \begin{tikzternal}[scale=1,
        every node/.style={minimum size=12pt,transform shape},
        state/.style={circle, draw, minimum size=20pt},
        every path/.style={draw, -latex},
        every initial by arrow/.style={-latex, initial text=},
        ]
        \node[initial,accepting,state] (1) at (180:1){\T{1}};
        \node[state]                   (2) at (  0:1){\T{2}};

        \draw (1) to node[pos=0.5,above,sloped]{\T{0}} (2);

        \loopTop[4]{1}{\T{4}};
        \loopRight[4]{2}{\T{2}};

        \end{tikzternal} \\
        \specialrule{1pt}{0pt}{0pt}
        \end{tabular}
\end{center}

Natürlich soll dieses Erstellen noch weiter stark vereinfacht werden. Des Weiteren wird darüber nachgedacht, einen akzeptierten Endzustand klarer zu markieren (Linien dicker, mehr abstand etc). Der Traum wäre, dass das Erstellen eines Automaten wie folgt funktioniert:
\begin{lstlisting}[language=lLatex]
\begin{Automat}
    \STATE[1]{180:1}{1};
    \state[2]{0:1}{2};

    \draw (1) to node[midway,above]{0} (2);

    \loopTop[4]{1}{\T{4}};
    \loopRight[4]{2}{\T{2}};
\end{Automat}
\end{lstlisting}
Die Befehle \blankcmd{state} und \blankcmd{STATE} sollen hierbei automatisch hochzählen können - pro Automat - aber über das optionale Argument lesbar einer Zahl zugewiesen werden. Die Umgebung Automat soll hierbei zusätzlich auch handhaben, dass automatisch alle Nodes mithilfe von \T{\textbackslash T} geschrieben werden. Der entstehende Automat soll optisch identisch zum obigen sein, dies wird allerdings erst auf das Bedürfnis hin übernommen.
\subsection{Schaltkreise \LILLYxNOTExWarning{Ausstehend}}

\subsection{Neuronen \LILLYxNOTExWarning{Work in Progress}}
Diese Definitionen befinden sich in der Datei: {\ltt\LILLYxPATHxGRAPHICS/Tikz-Core/\_LILLY\_TIKZ\_NEURONS}. Sie werden mit \LILLYxBOXxVersion{2.0.0} automatisch mit dem Einbinden von\newline \LILLYxNOTExLibrary{LILLYxGRAPHICS} geladen.\medskip\newline
Da vor allem mit \fg der Wunsch danach aufkam, neuronale Netze schnell zu Texen, wurde dieses Paket entwickelt um das Paket mit den Schaltkreisen so zu erweitern, dass es erlaubt Perzeptronen darin einzubauen, das  Paket an sich befindet sich ebenfalls im Work in Progress-Status. \textit{Das Schaltkreise-Paket ist ebenfalls noch nicht in LILLY integriert. Es befindet sich ebenfalls in einem Anfangsstadium und deswegen wird auch hierbei um Mithilfe bei der Weiterentwicklung gebeten.}

%
%
%

\presentCommand[1.0.5]{neuronSquare}[\manArg{pos:00}\manArg{pos:01}\manArg{pos:10}\manArg{pos:11}]
Es wurde bisher auch nur durch das Bereitstellen eines einzelnen Befehls implementiert: \blankcmd{neuronSquare}. Dieser funktioniert seinerseits lediglich in einer \emph{tikzpicture}/\emph{tikzternal} Umgebung und zeichnet nichtmal ein Neuron, sondern lediglich die 2-D Repräsentation eines booleschen Raums, der wiedergibt unter welchen Eingabevektoren das Perzeptron welchen Wert zurückliefert. Die 4 Parameter, die hierzu \blankcmd{neuronSquare} benötigt, entsprechen der jeweiligen Binärdarstellung der Eingabevektoren. Eine beispielhafte Anwendung ist hier zu finden:
\[\begin{tabular}{!{\VRule[1pt]}@{\hspace{0.5em}}C{0.35\textwidth}@{\hspace{0.5em}}|@{\hspace{0.5em}}C{0.3\textwidth}@{\hspace{0.5em}}!{\VRule[1pt]}}
    \specialrule{1pt}{0pt}{0pt}
    {\scriptsize\begin{plainlatex}
\begin{tikzternal}
    \neuronSquare{T}{F}{T}{T};
\end{tikzternal}
    \end{plainlatex}
    } & \begin{tikzternal}
        \node at(0,2.25) {};
        \neuronSquare{T}{F}{T}{T};
    \end{tikzternal} \\
        \specialrule{1pt}{0pt}{0pt}
        \end{tabular}\]
Hierbei steht ein \T{T} (true) natürlich für einen akzeptierten, ein \T{F} (false) entsprechend für einen nicht akzeptierten Befehl. Aktuell ist geplant, dass der Befehl auch für 1-, 3- und 4-dimensionale Räume eine Option anbietet (siehe für 4D: Titelgrafik \gdra[]), die dann über einen einfacheren Namen abgegriffen werden kann. Weiter sollen dann \fg und \gdra (boolesche Räume) diese Befehle nutzen anstelle der dafür eigens implementierten Grafiken. Weiter soll es möglich sein über ein optionales Argument die Position (relativ) zu bestimmen!

%
%
%

\subsection{ER-Diagramme}
Diese Definitionen liegen in der Datei \blatex[morekeywords={[5]{LILLYxPATHxGRAPHICS}}]{:bs:LILLYxPATHxGRAPHICS/Tikz-Core/\_LILLY\_TIKZ\_ER.tex}, sie werden mit \LILLYxBOXxVersion{2.0.0} automatisch mit dem Einbinden von \LILLYxNOTExLibrary{LILLYxGRAPHICS} geladen.\medskip\newline
Ersteinmal ein Beispiel:
\begin{center}
    \begin{tikzternal}[scale=0.7,every node/.style={transform shape}]
        \newcommand{\ttw}{3};\newcommand{\tth}{2} % tth heightsteps ttw widthsteps
        % could be for each ^^ sorry
        \entity{(2*\ttw,4*\tth)}{Teile};
        \entity{(4*\ttw,4*\tth)}{Lieferanten}; \entity{(4*\ttw,2*\tth)}{Bestellungen};

        \relation{(4*\ttw,3*\tth)}{erhielt};   \relation{(3*\ttw,4*\tth)}{liefert};

        % Teile
        \attribute{(1.5*\ttw,3.5*\tth)}{Farbe}; \attribute{(1.5*\ttw,3*\tth)}{KKosten};
        \attribute{(1.5*\ttw,2.5*\tth)}{Bestand}; \attribute{(1.5*\ttw,2*\tth)}{MinBest};
        %\draw(TeileNr)--(Teiletypen) -- (TeileName);
        \draw (Farbe)++(-0.9\ttw,0) node[left]  {\scriptsize $\{0,1,2\}\ni$};
        \draw (Farbe) -| (Teile) (KKosten) -| (Teile) (Bestand) -| (Teile) (MinBest) -| (Teile);

        % liefert
        \attribute{(3*\ttw, 3.25*\tth)}{Preis};
        \draw(liefert) -- (Preis);

        % Lieferanten
        \kattribute{(3.5*\ttw,4.75*\tth)}{LiefNr}; \attribute[LiefName]{(4.5*\ttw,4.75*\tth)}{\tiny LiefName};
        \attribute{(5*\ttw,4*\tth)}{LiefStadt}; \attribute{(5*\ttw,3.5*\tth)}{Bewert};
        \draw(LiefNr)--(Lieferanten) -- (LiefName) (LiefStadt) -- (Lieferanten) -- (Bewert);

        \draw (Bewert)++(0.35*\ttw,-0.1*\tth) node[right] (beweinsch) {\scriptsize $\in [-2,+2]$};

        % Bestellungen
        \kattribute{(5*\ttw,2.25*\tth)}{BestNr}; \attribute{(5*\ttw,1.75*\tth)}{BDatum};
        \draw(BestNr)--(Bestellungen) -- (BDatum);


        %kunden to Bestellpos
        \draw (Teile) +(1.25,0) node[above right] {\small $0..*$} -- (liefert) -- (Lieferanten) +(-1.25,0) node[above left] {\small $0..*$} %
        +(0,-0.377) node[below right] {\small $1$} -- (erhielt) -- (Bestellungen) +(0,0.377) node[above left] (bpkard)  {\small $0..*$};

        \providecommand{\descNode}[4][above]{%
            \draw #2 node[#1]  {\parbox{3cm}{\textbf{#3}:\\\itshape #4}};
        }

        \begin{scope}[every path/.style={latex-}]
            \descNode{(Teile) -- ++(-0.75,0.75)}{Entität}{Identifizierbares Objekt};

            \descNode{(liefert) -- ++(-0.75,1.75)}{Beziehung}{Verhältnis zwischen Entitäten};

            \descNode{(LiefName) -- ++(0,0.75)}{Attribut}{Eigenschaften einer Entität};

            \descNode{(beweinsch) -- ++(1.25,0.75)}{Einschränkung}{Schränkt mgl. Werte ein};

            \descNode[right]{(BestNr) -- ++(1.5,0)}{Primärattribut}{Eindeutiger Attributwert};

            \descNode[left]{(bpkard) -- ++(-1.25,0)}{Kardinalität}{Ein Liefr. erhält bel. viele Bestellungen};
        \end{scope}
    \end{tikzternal}
\end{center}
Erzeugt wurde dieses durch die folgenden Befehle:

%
%
%

\presentCommand[1.0.9]{entity}[\optArg{Node Name}\manArg{Point}\manArg{Text}]
Setzt eine Entität im Kontext eines Entity-Relationship-Diagramms. Wird kein expliziter \T{Node Name} angegeben, so wird er gleich dem \T{Text} gesetzt. Da dies natürlich nicht immer geht, bietet die Option hierfür einen Ausweg:
\begin{defaultlst}[][listing side text,righthand width=3cm]{lLatex}
\begin{tikzpicture}
    \entity{(0,0)}{Dieter};
\end{tikzpicture}
\end{defaultlst}

%
%
%

\presentCommand[1.0.9]{relation}[\optArg{Node Name}\manArg{Point}\manArg{Text}\optArg[1]{Width}\optArg[0.5]{Height}]
Setzt eine Relation im Kontext eines Entity-Relationship-Diagramms. Wird kein expliziter \T{Node Name} angegeben, so wird er gleich dem \T{Text} gesetzt. Da dies natürlich nicht immer geht, bietet die Option hierfür einen Ausweg:
\begin{defaultlst}[][listing side text,righthand width=3cm]{lLatex}
\begin{tikzpicture}
    \relation{(0,0)}{Dieter};
\end{tikzpicture}
\end{defaultlst}

%
%
%

\presentCommand[1.0.9]{attribute}[\optArg{Node Name}\manArg{Point}\manArg{Text}\cmdlist\secline\anothercmd[1.0.0]{kattribute}\optArg{Node Name}\manArg{Point}\manArg{Text}]
Setzt ein Attribut beziehungsweise ein Schlüsselattribut im Kontext eines Entity-Relationship-Diagramms. Wird kein expliziter \T{Node Name} angegeben, so wird er gleich dem \T{Text} gesetzt. Da dies natürlich nicht immer geht, bietet die Option hierfür einen Ausweg:
\begin{defaultlst}[][listing side text,righthand width=3cm]{lLatex}
\begin{tikzpicture}
    \attribute{(0,1)}{Dieter};
    \kattribute{(0,0)}{Dieter};
\end{tikzpicture}
\end{defaultlst}


%
%
%

\subsection{Verzeichnis-Bäume}
Diese Definitionen liegen in der Datei \blatex[morekeywords={[5]{LILLYxPATHxGRAPHICS}}]{:bs:LILLYxPATHxGRAPHICS/Tikz-Core/\_LILLY\_TIKZ\_DIRECTORY.tex}, sie werden mit \LILLYxBOXxVersion{2.0.0} automatisch mit dem Einbinden von \LILLYxNOTExLibrary{LILLYxGRAPHICS} geladen.

%
%
%

\presentEnvironment[2.0.0]{directory}[\optArg{tikz args}]
Setzt ein Verzeichnis. Allgemein liefert diese Umgebung eine Möglichkeit relativ einfach etwas \emph{Verzeichnisähnliches} zu setzen. Ein Beispiel:
\begin{defaultlst}[][listing side text,righthand width=4.5cm]{lLatex}
\begin{directory}[scale=0.5]
\node{Toplevel}
    child { node {Level 1}
        child { node {Level 2}
            child { node {Level 3} }
            child { node {\ldots} }
        }
        child { node {Level 2}
            child { node {Level 3} }
            child { node {\ldots} }
        }
        child { node {\bfseries Readme.md} }
    }
    child { node {\bfseries Level 1}
        child { node {Level 2}
                child {node {test.txt} } }
    };
\end{directory}
\end{defaultlst}

%
%
%

\presentEnvironment[2.1.0]{fancydir}
Setzt eine Verzeichnisstruktur, \emph{ähnlich} zu \blankenv{directory}. Ein Beispiel:
\begin{defaultlst}[][listing side text,righthand width=4.5cm]{lLatex}
\begin{fancydir}
[Toplevel
    [Level 1, dir=AppleGreen
        [Level 2
            [Level 3]
            [\ldots, cfile={ChromeYellow}{X}]
        ]
        [Level 2
            [Level 3, ldir={\faApple}]
            [\ldots]
        ]
        [Readme.md, ifile]
    ]
    [Level 1, idir
        [Level 2]
        [test.tex, file]
    ]
]
\end{fancydir}
\end{defaultlst}
Standardmäßig ist jedes Element ein Verzeichnis. Es gibt allerdings einige Möglichkeiten das Aussehen zu verändern: \begin{ditemize}\narrowitems
        \item Für Ordner: \begin{ditemize}\narrowitems
        \item \T{dir=\manArg{Color}} Setzt ein Verzeichnis in der übergebenen Farbe. Wird keine Farbe  übergeben, so wird die Standardfarbe (\T{folderbg}) verwendet.
        \item \T{idir=\manArg{Color}} Setzt ein Verzeichnis in der übergebenen Farbe. Wird keine Farbe übergeben, so wird die Standardfarbe (\T{ifolderbg}) verwendet um einen wichtigen Ordner zu markieren.
        \item \T{ldir=\manArg{Logo}} Setzt ein Verzeichnis in \T{folderbg} mit dem übergebenen Logo als Inhalt. Wird ein leeres Logo übergeben, so wird auch keins gesetzt.
        \item \T{cdir=\manArg{Color}\manArg{Logo}} Setzt ein Verzeichnis in \T{Color} mit dem übergebenen \T{Logo} als Inhalt. Wird ein leeres Logo übergeben, so wird auch keins gesetzt.
    \end{ditemize}
    \item Für Dateien: \begin{ditemize}\narrowitems % TODO:
        \item \T{file=\manArg{Color}} Setzt eine Datei in der übergebenen Farbe. Wird keine Farbe  übergeben, so wird die Standardfarbe (\T{filebg}) verwendet.
        \item \T{ifile=\manArg{Color}} Setzt eine Datei in der übergebenen Farbe. Wird keine Farbe übergeben, so wird die Standardfarbe (\T{ifilebg}) verwendet um einen wichtigen Ordner zu markieren.
        \item \T{lfile=\manArg{Logo}} Setzt eine Datei in \T{filebg} mit dem übergebenen Logo als Inhalt. Wird ein leeres Logo übergeben, so werden die Textstreifen gesetzt.
        \item \T{cfile=\manArg{Color}\manArg{Logo}} Setzt eine Datei in \T{Color} mit dem übergebenen \T{Logo} als Inhalt. Wird ein leeres Logo übergeben, so werden die Textstreifen gesetzt.
    \end{ditemize}
\end{ditemize}
Mit den folgenden Befehlen lassen sich zudem eigene Dateitypen kreieren:

%
%
%

\presentCommand[2.1.0]{CreateNewFolderType}[\optArg{Color}\manArg{Name}\optArg{Logo}\cmdlist\secline\anothercmd[2.1.0]{CreateNewFileType}\optArg{Color}\manArg{Name}\optArg{Logo}]
Erzeugt einen neuen Ordner- beziehungsweise Dateityp. Ein Beispiel:
\begin{defaultlst}[][listing side text,righthand width=2.75cm]{lLatex}
% C-Files
\CreateNewFileType[DarkMidnightBlue]{.c}[\textbf{C}]
\CreateNewFolderType[Veronica]{templates}
\begin{fancydir}
[test
    [super, templates
        [UH.TXT, file=ChromeYellow]
        [my.c, .c]
    ]
]
\end{fancydir}
\end{defaultlst}

%
%
%

\presentCommand[2.1.0]{SetFolderFileSameIndent}
Standardmäßig werden die Texte für eine Datei näher an die Datei gesetzt. Dieser Befehl sorgt dafür, dass die Texte auf einem Level gleich weit eingerückt werden:
{
\begin{defaultlst}[][listing side text,righthand width=2.75cm]{lLatex}
\begin{fancydir}
[test
    [super,
        [Hallo, file]
        [Hallo]
    ]
    [duper, file]
]
\end{fancydir}
\SetFolderFileSameIndent
\begin{fancydir}
[test
    [super,
        [Hallo, file]
        [Hallo]
    ]
    [duper, file]
]
\end{fancydir}
\end{defaultlst}
}



% Note commands \createnew etc.

%
%
%
%
%

\subsection{Bilder und Bildlinks}
Diese Definitionen liegen in der Datei \blatex[morekeywords={[5]{LILLYxPATHxGRAPHICS}}]{:bs:LILLYxPATHxGRAPHICS/Tikz-Core/\_LILLY\_TIKZ\_GRAPHICS.tex}, sie werden mit \LILLYxBOXxVersion{2.0.0} automatisch mit dem Einbinden von \LILLYxNOTExLibrary{LILLYxGRAPHICS} geladen.\medskip\newline
Teile dieser Integration wurden durch \url{https://tex.stackexchange.com/questions/120861/making-tikz-path-into-a-hyperlink} inspiriert. Dieses Paket liefert keinen neuen Befehl, allerdings $2$ neue \TikZ-Argumente, die im Folgenden an einem Beispiel präsentiert werden:

\begin{defaultlst}[][listing side text,righthand width=2cm]{lLatex}
\begin{tikzpicture}
    % clip image to a path {#1: path to the image}
    %   {#2 scaled-height of the image}:
    \filldraw[path image={\LILLYxPATHxDATA/Images/me.jpg}{2cm},AppleGreen] (0,2) circle (1cm);
    \filldraw[path image={\LILLYxPATHxDATA/Images/me.jpg}{2cm},AppleGreen] (0,0) ellipse (1cm and 0.5cm);
\end{tikzpicture}
\end{defaultlst}

\begin{defaultlst}[][listing side text,righthand width=2cm]{lLatex}
\hypertarget{marker}{Hallo Welt} [\ldots] \par
%
\tikz{\filldraw[hyperlink={marker},AppleGreen] (0,0) rectangle ++(2,1);}
\end{defaultlst}


%
%
%

\section{Mitgelieferte Grafiken}
\hypertarget{LILLYxGRAPHICSxPROVIDER}Diese Definitionen befinden sich in der Datei: {\ltt\LILLYxPATHxGRAPHICS/LILLYxGRAPHICSxPROVIDER.sty}. Sie werden mit \LILLYxBOXxVersion{2.0.0} automatisch mit dem Einbinden von\newline \LILLYxNOTExLibrary{LILLYxGRAPHICS} geladen.\medskip\newline
Dieser Teil existiert weiter auch als eigenes Paket mit: \LILLYxNOTExLibrary{LILLYxGRAPHICSxPROVIDER} und hängt vom Mutterpaket ab.

\presentCommand[2.0.0]{getGraphics}[\optArg[\textbackslash linewidth]{width}\manArg{path}\optArg{height}]
Erlaubt den Zugriff auf zahlreiche Grafiken, die im Rahmen der Arbeit entstanden sind. Bei einer Angabe von Breite und Höhe gewinnt die Breite, da stets nur eine Dimension skaliert wird! Die bisher enthaltenen Grafiken können durch \cbash{jake get} abgerufen werden.
\iflillycompact\else\begin{center}
    \begin{tabular}{C{0.45\linewidth}_C{0.45\linewidth}}
        \getGraphics{Rechner/PLAAmpel} & \getGraphics{Software/ThreadStates} \\
        {\tiny \blatex{:bs:getGraphics\{Rechner/PLAAmpel\}}} & {\tiny\blatex{:bs:getGraphics\{Software/ThreadState\}}}
    \end{tabular}
\end{center}\fi
Die Größe skaliert sich in der Regel automatisch, allerdings existieren auch Grafiken, die automatisch nicht skaliert werden, da sie Code oder andere nicht skalierfähige Elemente enthalten.
\iflillycompact\else\begin{center}
    \getGraphics{Software/XML/XMLUebersicht}
    \blatex{:bs:getGraphics\{Software/XML/XMLUebersicht\}}
\end{center}\fi

\presentCommand[2.0.0]{getGraphicsPath}[\manArg{path}]
Liefert den absoluten Pfad zu einer Grafik. Beispiel:
\begin{latex*}
\getGraphicsPath{Software/XML/XMLUebersicht}
\end{latex*}
Liefert: \T{\getGraphicsPath{Software/XML/XMLUebersicht}}.

%
%
%

\presentCommand[2.0.0]{getPrerendered}[\optArg[\textbackslash linewidth]{width}\manArg{path}\optArg{height}]
Erlaubt eine automatisch an die Seitenbreite skalierte Implementation von bereits vorberechneten Grafiken. Bei einer Angabe von Breite und Höhe gewinnt die Breite, da stehts nur eine Dimension skaliert wird! Sie werden in der Grafiksammlung durch den Tag \T{pdf} gekennzeichnet (die Breite wurde im Beispiel angepasst), oder lassen sich durch das Anfügen eines \say{\T{-pdf}}-Suffix erhalten.
\iflillycompact\else\begin{center}
    \begin{tabular}{C{0.45\linewidth}_C{0.45\linewidth}}
        \getPrerendered{Rechner/PLAAmpel-pdf} & \getPrerendered{Software/ThreadStates-pdf} \\
        {\tiny \blatex[morekeywords={[5]{getPrerendered}}]{:bs:getPrerendered\{Rechner/PLAAmpel-pdf\}}} & {\tiny\blatex[morekeywords={[5]{getPrerendered}}]{:bs:getPrerendered\{Software/ThreadState-pdf\}}}
    \end{tabular}
\end{center}\fi
%Weiter existiert der Befehl \blankcmd{lillyGraphicsUsePrerenderedtrue}, mit dem automatisch jeder \blankcmd{getGraphics}-Aufruf in einen entsprechenden \blankcmd{getPrerendered} Aufruf umgewandelt wird.
\emph{Es gilt zu beachten, dass die bereits vorgenerierten Grafiken von den manuell generierten abweichen können!}
Mit Version \LILLYxBOXxVersion{2.1.0} wird \Jake (um die Größe zu reduzieren) ohne die vorkompilierten Grafiken ausgeliefert. Diese können allerdings durch \bbash{jake get} generiert werden (\Jake fragt nach, wenn die Dateien nicht gefunden werden können!). Dieser Prozess kann einige Minuten in Anspruch nehmen und erübrigt sich bei einer Installation über die Entwicklungs-Version. Durch \T{:force} kann das Generieren dieser Grafiken forciert werden, wobei jeweils nur Änderungen neu kompiliert werden.

%
%
%
%
%

\section{Zusätzliche Optionen}

%
%
%

\subsection{Externalisierung}
Diese Definitionen liegen in der Datei \blatex[morekeywords={[5]{LILLYxPATHxGRAPHICS}}]{:bs:LILLYxPATHxGRAPHICS/Tikz-Core/\_LILLY\_TIKZ\_GRAPHEN.tex}, sie werden mit \LILLYxBOXxVersion{2.0.0} automatisch mit dem Einbinden von \LILLYxNOTExLibrary{LILLYxGRAPHICS} geladen.\medskip\newline
Auf Basis von \T{environ} und der \TikZ-Bibliothek \T{external} bietet diese Datei eine Möglichkeit Grafiken zu externalisieren. Wirklich aktiviert wird es allerdings nur auf Basis der \Jake[-]\jmark[Einstellung]{mrk:jakesettings} \T{lilly-external} (genau genommen läuft die kontrolle über \blankcmd{LILLYxEXTERNALIZE}), lagert dann allerdings alle Grafiken die durch folgende Umgebung definiert werden aus:

%
%
%

\presentEnvironment[1.0.7]{tikzternal}[\optArg{tikz args}]
Wenn die Externalisierung (\Jake[-]\jmark[Einstellung]{mrk:jakesettings} \T{lilly-external}) aktiviert ist, wird die hierin enthaltene Grafik automatisch externalisiert. Sonst fungiert die Umgebung als normales \blankenv{tikzpicture}.

%
%
%

\subsection{Platzhalter}
Diese Definitionen liegen in der Datei \blatex[morekeywords={[5]{LILLYxPATHxGRAPHICS}}]{:bs:LILLYxPATHxGRAPHICS/Tikz-Core/\_LILLY\_TIKZ\_PLATZHALTER.tex}, sie werden mit \LILLYxBOXxVersion{2.0.0} automatisch mit dem Einbinden von \LILLYxNOTExLibrary{LILLYxGRAPHICS} geladen.

%
%
%

\presentCommand[1.0.1]{imgplaceholder}
Ein Platzhalter, der da eingebaut werden kann, wo ein Bild hingehört, allerdings noch keins ist \Smiley. So liefert \blankcmd{imgplaceholder}:
\imgplaceholder

%
%
%
%
%

\section{Weiterführende Symbole}

\subsection{Embleme}
\hypertarget{LILLYxEMBLEMS}Diese Definitionen befinden sich in der Datei: {\ltt\LILLYxPATHxGRAPHICS/LILLYxEMBLEMS.sty}. Sie werden mit \LILLYxBOXxVersion{2.0.0} automatisch mit dem Einbinden von \LILLYxNOTExLibrary{LILLYxGRAPHICS} geladen.\medskip\newline
Dieser Teil existiert weiter auch als eigenes Paket mit: \LILLYxNOTExLibrary{LILLYxEMBLEMS} und hängt vom Mutterpaket ab.

%
%
%

\presentCommand[2.0.0]{infoEmblem}[\cmdlist\anothercmd[2.0.0]{warningEmblem}\cmdlist\anothercmd[2.0.0]{errorEmblem}\cmdlist\anothercmd[2.0.0]{mathEmblem}\cmdlist\secline\anothercmd[2.0.0]{codeEmblem}]
Hierbei handelt es sich um Shortcuts um einige Embleme direkt zu Setzen:\medskip
\begin{center}
    \begin{tikzpicture}
        \foreach \emblem [count=\ang] in {\infoEmblem,\warningEmblem,\errorEmblem,\mathEmblem,\codeEmblem} {
            \node at(72*\ang+18:1cm) {\emblem};
        }
    \end{tikzpicture}
\end{center}
Hierbei bedienen sich die Befehle der Emblem-Definition \blankcmd{DefaultBaseEmblem}.

%
%
%

\presentCommand[2.0.0]{NewEmblem}[\optArg{Emblem-Keys}\optArg{Tikz-Args}\manArg{name}]
Definiert ein neues Emblem, wobei folgende Emblem-Keys zur Verfügung stehen, diese werden persistiert:
\begin{center}
    \begin{tabularx}{0.96\linewidth}{^t>{\em}^l^c^l+}
        \toprule
            \headerrow Bezeichner & \normalfont\bfseries Typ & Standard & Beschreibung\\
        \midrule
        radius & Length & $0.369cm$ & Radius des Symbols\\
        shape & Enum\footnote{Allowed: none, quadratic, round, hexagon, octagon} & \T{shape/hexagon} & Form des Hintergrunds \\
        bgcolor & Farbe & DebianRed & Hintegrundsfarbe \\
        bordercolor & Farbe & DebianRed & Rahmenfarbe \\
        fgcolor & Farbe & MudWhite & Textfarbe \\
        font    & Code & \blankcmd{bfseries}\blankcmd{large}\blankcmd{sffamily} & Schrift\\
        \bottomrule
    \end{tabularx}\nskip
\end{center}
So lassen sich relativ einfach GrundEmbleme definieren:
\begin{latex}
\NewEmblem[shape/none]{NoneEmblem}
\NewEmblem[shape/quadratic]{QuadraticEmblem}
\NewEmblem[shape/round]{RoundEmblem}
\NewEmblem[shape/hexagon]{HexagonEmblem}
\NewEmblem[shape/octagon]{OctagonEmblem}
\end{latex}

So ist es zum Beispiel möglich durch die jeweilige Form das aussehne der mitgelieferten Embleme zu modifizieren:
\begin{center}
        \hspace*{-1cm}\begin{tikzpicture}[scale=0.75,transform shape]
            \foreach \defTar[count=\x] in {{shape/none},{shape/quadratic},{shape/round},{shape/hexagon},{shape/octagon}} {%
                \node at(3.5*\x,0){\NewEmblem[\defTar][rotate=90]{DefaultBaseEmblem}\begin{tikzpicture}
                    \foreach \emblem [count=\ang] in {\infoEmblem,\warningEmblem,\errorEmblem,\mathEmblem,\codeEmblem} {
                        \node at(72*\ang+18:1cm) {\emblem};
                    }
                \end{tikzpicture}
                };
                \node at(3.5*\x+0.75,-2) [below] {\T{\defTar}};
            }
        \end{tikzpicture}%
\end{center}
Das Erzeugen eines neuen Emblems mithilfe von \blankcmd{NewEmblem} erzeugt einen neuen Befehl, entsprechend des Namens des Emblems. Der Befehl besitzt jeweils die folgende Signatur:

%
%
%

\presentCommand[2.0.0]{<name>}[\optArg{Tikz-Keys}\manArg{text}]
So liefert zum Beispiel: \T{\blankcmd{OctagonEmblem}\{Hu\}}: \OctagonEmblem{Hu}.\newline
Oder: \T{\blankcmd{OctagonEmblem}\{\blankcmd{tiny} stop\}} \OctagonEmblem{\tiny Stop}.\medskip\newline
Die Shortcuts von oben, wurden hierbei wie folgt definiert:
\begin{latex}
\gdef!**!\infoEmblem{\,\DefaultBaseEmblem[draw=Leaf,fill=Leaf!75]{i}\,}
\gdef!**!\warningEmblem{\,\DefaultBaseEmblem[draw=ChromeYellow,fill=ChromeYellow!75]{!}\,}
\gdef!**!\errorEmblem{\,\DefaultBaseEmblem[draw=DebianRed,fill=DebianRed!75]{\wasysymLightning}\,}
\gdef!**!\mathEmblem{\,\DefaultBaseEmblem[draw=DarkMidnightBlue,fill=DarkMidnightBlue!75]{:bmath:\mathbf{\pi}:emath:}\,}
\gdef!**!\codeEmblem{\,\DefaultBaseEmblem[draw=DarkOrchid,fill=DarkOrchid!75]{\faCode}\,}
\end{latex}

%
%
%

\presentCommand[2.0.0]{textEmblem}[\manArg{Emblem}]
Setzt ein Emblem für den Fließtext: \textEmblem\codeEmblem{} anstelle von \codeEmblem. Die Argumentklammern können Vernachlässigt werden, das Bedeutet Es genügt das Schreiben von \T{\blankcmd{textEmblem}\blankcmd{codeEmblem}}.

%
%
%

\presentCommand[2.0.0]{btextEmblem}[\manArg{Emblem}]
Funktioniert identisch, setzt allerdings ein Emblem, welches die komplette Zeilenhöhe ausfüllt: \btextEmblem\infoEmblem{} anstelle von \infoEmblem.