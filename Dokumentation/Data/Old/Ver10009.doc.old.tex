\section{Version \small\LILLYxBOXxVersion{1.0.9}}
\elable{jmp:oldInstallLinux1009}
\section{Installieren von Lilly mit Cpp-Jake}
\subsection{Linux}
\begin{center}
    Für Versionen < 1.0.8 klicke hier: \jmark[klick mich]{jmp:oldInstallLinux}!
\end{center}
\LILLYxNOTExVersion{1.0.8}Da LILLY komplett auf einem Linux-Betriebsystem entwickelt wurde, gestaltet sich die Implementierung relativ einfach.
Hierzu nutzen wir das Hilfsprogramm \Jake welches selbst in C++ geschrieben wurde.
Im Folgendenen sind die Schritte kurz erklärt.
\paragraph{Installation von \Jake:}Eine ausführliche Erklärung von \Jake selbst findest sich weiter hinten (\jmark[hier]{jmp:iJake}) in dieser Dokumentation:\smallskip
\marginpar{\tiny Für ausführliche Informationen zur Installation konsultiere bitte die README-Datei in:
\url{../Lilly/Jake/jake_source/README.md}. \newline
Für Informationen zur Nutzung konsultiere: \url{../Lilly/Jake/README.md}}
\begin{enumerate}\setlength{\itemsep}{0.25\baselineskip}
    \item Navigiere mit dem Terminal in das Verzeichnis: \verb|Lilly/Jake/jake_source|
    \item Führe nun \verb|make| aus um \Jake zu kompilieren.
          Es wird vermutlich kurz dauern, aber danach wird dir das Programm \LJake zur Verfügung stehen.
    \item Nun kannst du dein Terminal neu starten und von überall her \verb|lilly_jake install| aufrufen.
          Dies sollte den Installationsprozess in Gang setzen.\smallskip
\end{enumerate}
Sollte das Ganze fehlerfrei verlaufen sein, dann: Glückwunsch, du hast Lilly erfolgreich installiert!
Betrachte im Falle eines Fehlers bitte erst die Readme-Dateien und die bereits beantworteten Fehler
auf Github (\href{https://github.com/EagleoutIce/LILLY/issues}{\faGithub}) bevor du einen neuen Fehler
eröffnest oder mir eine Nachricht schreibst \Smiley.
\paragraph{Erstellen eines Makefiles:}
Nun möchtest du natürlich auch ausprobieren ob die Installation funktioniert hat.
Hierzu kannst du in das Testverzeichnis navigieren (\verb|Lilly/Jake/tests|).
Hier befinden sich eine Menge Dateien die in dieser Dokumentation auch als Beispiele benutzt werden.
Du gibst nun folgendes in die Konsole ein:
\begin{lstlisting}[language=lBash]
lilly_jake test.tex
\end{lstlisting}
\Jake erstellt nun ein entsprechendes Makefile für dich, welches du nun ausführen kannst:
\begin{lstlisting}[language=lBash]
make
\end{lstlisting}
Im Standardmäßig konfigurierten Ausgabe-Ordner \verb|test-OUT| befindet sich nun eine entsprechende PDF
Datei \Smiley.\smallskip
\begin{bemerkung}[make]
    Logischerweise muss damit auch \T{make} auf dem System vorhanden sein:
\begin{lstlisting}[language=lBash]
sudo apt install "make"
\end{lstlisting}
\end{bemerkung}

\subsection{Windows \LILLYxNOTExWarning{Ausstehend}}
\subsection{MacOS}
\begin{center}
    Entspricht, dank \Jake, der Linux-Installation.
\end{center}
Hierzu nutzen wir das Hilfsprogramm \Jake welches selbst in C++ geschrieben wurde.
Im Folgendenen sind die Schritte kurz erklärt.
\paragraph{Installation von \Jake:}Eine ausführliche Erklärung von \Jake selbst findest sich weiter hinten (TODO: LINK) in dieser Dokumentation:\smallskip
\marginpar{\tiny Für ausführliche Informationen zur Installation konsultiere bitte die README-Datei in:
\url{../Lilly/Jake/jake_source/README.md}. \newline
Für Informationen zur Nutzung konsultiere: \url{../Lilly/Jake/README.md}}
\begin{enumerate}\setlength{\itemsep}{0.25\baselineskip}
    \item Navigiere mit dem Terminal in das Verzeichnis: \verb|Lilly/Jake/jake_source|
    \item Führe nun \verb|make| aus um \Jake zu kompilieren.
          Es wird vermutlich kurz dauern, aber danach wird dir das Programm \LJake zur Verfügung stehen.
    \item Nun kannst du dein Terminal neu starten und von überall her \verb|lilly_jake install| aufrufen.
          Dies sollte den Installationsprozess in Gang setzen.\smallskip
\end{enumerate}
Sollte das Ganze fehlerfrei verlaufen sein, dann: Glückwunsch, du hast Lilly erfolgreich installiert!
Betrachte im Falle eines Fehlers bitte erst die Readme-Dateien und die bereits beantworteten Fehler
auf Github (\href{https://github.com/EagleoutIce/LILLY/issues}{\faGithub}) bevor du einen neuen Fehler
eröffnest oder mir eine Nachricht schreibst \Smiley.
\paragraph{Erstellen eines Makefiles:}
Nun möchtest du natürlich auch ausprobieren ob die Installation funktioniert hat.
Hierzu kannst du in das Testverzeichnis navigieren (\verb|Lilly/Jake/tests|).
Hier befinden sich eine Menge Dateien die in dieser Dokumentation auch als Beispiele benutzt werden.
Du gibst nun folgendes in die Konsole ein:
\begin{lstlisting}[language=lBash]
lilly_jake test.tex
\end{lstlisting}
\Jake erstellt nun ein entsprechendes Makefile für dich, welches du nun ausführen kannst:
\begin{lstlisting}[language=lBash]
make
\end{lstlisting}
Im Standardmäßig konfigurierten Ausgabe-Ordner \verb|test-OUT| befindet sich nun eine entsprechende PDF
Datei \Smiley.\smallskip
\begin{bemerkung}[make]
    Logischerweise muss damit auch \T{make} auf dem System vorhanden sein:
\begin{lstlisting}[language=lBash]
sudo apt install "make"
\end{lstlisting}
\end{bemerkung}
