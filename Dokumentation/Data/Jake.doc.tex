\renewcommand{\arraystretch}{1.5}
\chapter[Jake \LILLYxBOXxVersion{\small 1.0.8}]{Jake}
\TitleSUB{\Jake[]! Would you get me the cake please?\ldots \hfill \LILLYxBOXxVersion{\small 1.0.8}}
\section{Grundlegendes}\elable{jmp:iJake}
\subsection{Entwicklung}
Anfänglich wurde \Jake als \emph{installer} konzipiert, der einfach nur die mühsehlige Installation
des Pakets abnehmen soll. Mittlerweile hat sich \Jake allerdings weiterentwickelt und
bietet das Potenzial für einiges mehr. Im Folgenden sei die Funktionsweise genauer erklärt. 
Zu beachten ist allerdings, dass \Jake bisher nur für Linux und MacOS einen Installer und somit
seine Funktionalität zur Verfügung stellt!

\subsection{Die Installation}
\Jake zu installieren sollte normalerweise einem Kinderspiel gleichen. Notwendig sind hierfür
auf allen bisher unterstützten Betriebssystemen (Debian-Basiertes Linux und MacOS) ein
\verb|C++14| fähiger \T{gcc}-Compiler und \T{make}.
Anschließend gilt es ins \verb|jake_source|-Verzeichnis zu navigieren. 
Es befindet sich hier: \T{Lilly/Jake/jake\_source}. 
In diesem Verzeichnis kann man nun \T{make} ausführen. Dies sorgt dafür,
dass nicht nur \T{jake.cpp} zu einer ausführbaren Datei wird, sondern auch,
dass \LJake systemweit zur Verfügung steht (sofern die verwendeten Konsole
bash, zsh oder iTerm ist, bzw. im allgemeinen auf eine der folgenden Dateien
zugreift: \T{.bashrc}, \T{.zshrc}, \T{.bash\_profile}).\newline
Damit gilt \Jake als \emph{installiert}.

\subsection{Die Schnittstelle}
Mit \Jake interagiert es sich über die Konsole ganz einfach. 
Die Eingabe von \LJake zeigt eine Hilfe mit allen nötigen Optionen an.
Nutzt man \T{bash} oder \T{zsh} so wird \Jake übrigens bereits automatisch
Vervollständigt.\bigskip\newline

\textbf{Du hast die aktuelle Grenze der offnene Dokumentationswelt erreicht. \say{Still under construction}, wie man so schön sagt. Für aktuelle Informationen sollten aber die im Repository existierenden Readmes ausreichen.}