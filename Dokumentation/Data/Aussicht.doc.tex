\chapter{Aussicht}
\TitleSUB{Das Wunder der Schöpf\ldots Evolution \Smiley}

\section{Todos}
\paragraph{Visuals}
Es wäre schön (auch auf Basis von tcolorbox) einige Umgebungen zu haben, mit denen sich Grafiken oder Textabschnitte einfach positionieren lassen. So ist es lästig hierfür jedesmal minipages und unsicher hierfür jedesmal floatings zu verwenden.
\paragraph{Fehler}
Das Paket sollte Befehle wie \T{\textbackslash PackageInfo}/\T{Error}/\T{Warning} unterstützen und auch ausgeben - zudem sollte die komplette Dateistruktur robuster werden und auf Fehler reagieren können

\paragraph{Hoverover tooltips}
\definestyle{icpopup}{author={Eagle},subject={Hyperlinkinfo},markup=Highlight,color=white}

Eine Idee war es bei Hyperlinks Kommentare mithilfe von Tooltips zu realisieren. Somit wäre es möglich auf den meisten Geräten schnell Informationen zu liefern mithilfe von: $\pdfmarkupcomment[style=icpopup]{\text{Ich bin ein toller Hyperlink}}{Diese Information kann wichtig im Rahmen des Hyperlinks sein}$.

\paragraph{Weitere}
Siehe hier für weitere Todos: \url{https://github.com/EagleoutIce/LILLY/issues}.

\section{Geplant für \small\LILLYxBOXxVersion{2.1.0}}

\paragraph{Persistence}
Geplant ist das \T{Persistence}-Paket, welches es ermöglicht Befehle wie die von \LILLYxNOTExLibrary{LILLYxBOXES} automatisch persistiert.

\paragraph{Weg von \T{xSemester}}
Die Semesterstruktur soll abgeschafft werden, lediglich der Name der Vorlesung solll das entsprechende Ergebnis liefern. Weiter soll das Konzept der Vorlesung erweitert werden um so allgemeine Datenpakete zur Verfügung zu stellen. Diese sollen weiter auch vom Nutzer in seine Nutzerkonfiguration eingegliedert werden. Allgemein soll es möglich sein, mehrere Pfade zum suchen anzugeben.

\paragraph{Different Suffix}
Es soll für Konfigurationsdateien möglich sein, auch auf \T{.conf} zu enden.

\paragraph{Jake Projectfiles \& Analysis}
Neue \Jake-Module: Zur Verfügung stellen von Projektdateien, die Gesamtoperationen liefern sowie ausbauen von des Anylsis-Moduls. Weiter soll das autocomplete schneller werden und sich \Jake bei der Installation an einen gewissen Pfad kopieren um Ort-Unabhängig zu sein!
%% !!!!! : % add documentations for the boxes (aufgabe-box etc.....)

\paragraph{Renormal aufgabe}
normalisiere die Parameter für die Aufgabenbox :D