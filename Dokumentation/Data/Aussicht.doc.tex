\chapter{Aussicht}
\TitleSUB{Das Wunder der Schöpf\ldots Evolution \Smiley}

\section{Todos}
\subsection{Visuals}
Es wäre schön (auch auf Basis von tcolorbox) einige Umgebungen zu haben, mit denen sich Grafiken oder Textabschnitte einfach positionieren lassen. So ist es lästig hierfür jedesmal minipages und unsicher hierfür jedesmal floatings zu verwenden.
\subsection{Fehler}
Das Paket sollte Befehle wie \T{\textbackslash PackageInfo}/\T{Error}/\T{Warning} unterstützen und auch ausgeben - zudem sollte die komplette Dateistruktur robuster werden und auf Fehler reagieren können
\subsection{Dateiaufteilung}
Die Aufteilung von LILLY in verschiedene Dateien war zum Beibehalt der Übersicht unabdinglich, allerdings sollte diese Aufteilung einigen Kontrollblicken und Korrekturen unterzogen werden - zudem sollte in dem Rahmen das Implementieren neuer Designs/Codes vereinfacht werden - hierfür würde sich ein einfaches Skript anbieten, was neue Dateien (je nach Typ) automatisch an die richtige Stelle bringt. Weiter wäre es gut, wenn die Dateiendungen nicht nur \T{.tex} o.ä. lauten würden
\subsection{Road to CTAN}
Es sollten die notwendigen Installationsdateien und Dokumentationen generiert und eingebracht werden - sodass Lilly automatisiert verwaltet werden kann.

\subsection{Hoverover tooltips}
\definestyle{icpopup}{author={Eagle},subject={Hyperlinkinfo},markup=Highlight,color=white}

Eine Idee war es bei Hyperlinks Kommentare mithilfe von Tooltips zu realisieren. Somit wäre es möglich auf den meisten Geräten schnell Informationen zu liefern mithilfe von: $\pdfmarkupcomment[style=icpopup]{\text{Ich bin ein toller Hyperlink}}{Diese Information kann wichtig im Rahmen des Hyperlinks sein}$.

\subsection{Weitere}
Siehe hier für weitere Todos: \url{https://github.com/EagleoutIce/LILLY/issues}