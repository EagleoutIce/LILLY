\chapter[Mathe \LILLYxBOXxVersion{\small 1.0.0}]{Mathe}
\TitleSUB{Einzelne Variationen und eine Menge Abkürzungen \hfill \LILLYxBOXxVersion{\small 1.0.0}}
\smallskip\renewcommand{\arraystretch}{1.5}
\reversemarginpar\par
An sich ändert LILLY nicht viel an der normalen Implementation der Matheumgebung. Die verwendete Matheumgebung lässt sich mithilfe des Befehls \CMDpreview{LILLYxMathxMode} frei einstellen. Standardmäßig wird dieser Wert auf \emph{normal} gesetzt.

\section{Weitere Befehle}

\subsection[Operatoren \LILLYxBOXxVersion{\small 1.0.3}]{Operatoren}
{\centering \framebox{Diese Definitionen befinden sich in der Datei: \T{Maths/\_LILLY\_MATHS\_OPERATORS}}\vspace*{0.5\baselineskip}\par}
Lilly liefert den Befehl \CMDpreview[(1)]{overbar} dieser wird auf Basis von \T{mkern} so definiert, dass er direkt Abstände zwischen den Overlines definiert. So ergibt sich: 
\[\begin{tabular}{!{\VRule[1pt]}@{\hspace{1em}}l@{\hspace{1em}}|@{\hspace{1em}}c@{\hspace{1em}}!{\VRule[1pt]}}
    \specialrule{1pt}{0pt}{0pt}
    \T{\CMDshow{overbar}\{a\_1\} \CMDshow{overbar}\{a\_2\}} & \(\overbar{a_1}\overbar{a_2}\)\\\hline
    \T{\CMDshow{overline}\{a\_1\} \CMDshow{overline}\{a\_2\}} & \(\overline{a_1}\overline{a_2}\)\\
    \specialrule{1pt}{0pt}{0pt}
\end{tabular}\]
Für Definitionen gibt es die Befehle \CMDpreview{das} ($\das$), \CMDpreview{sad} ($\sad$), \CMDpreview{daseq} ($\daseq$), \CMDpreview{qesad} ($\qesad$) sowie \CMDpreview{shouldeq} ($\shouldeq$). All diese Befehle funktionieren lediglich in einer Matheumgebung und werden nicht mit \verb|\ensuremath| abgesichert!\par
Bis auf den letzten werden zudem alle Befehle mithilfe von \verb|\vcentcolon| realisiert.\bigskip\newline
Weiter wurde das Aussehen der Wurzel verändert, so liefert nun der Befehl \CMDpreview[{[1](1)}]{sqrt} folgendes: 
\[\begin{tabular}{!{\VRule[1pt]}@{\hspace{1em}}l@{\hspace{1em}}|@{\hspace{1em}}c@{\hspace{1em}}!{\VRule[1pt]}}
    \specialrule{1pt}{0pt}{0pt}
    \T{\CMDshow{sqrt}[3]\{42\}} & \(\sqrt[3]{42}\)\\\hline
    \T{\CMDshow{oldsqrt}[3]\{42\}} & \(\oldsqrt[3]{42}\)\\
    \specialrule{1pt}{0pt}{0pt}
\end{tabular}\smallskip\]
Zudem gibt es einige Vereinfachungen für etliche typischen mathematischen Operatoren: \newline 
\CMDpreview{det} ($\det$), \CMDpreview{adj} ($\adj$), \CMDpreview{LH} ($\LH$), \CMDpreview{eig} ($\eig$), \CMDpreview{Dim} ($\Dim$), \CMDpreview{sel} ($\sel$), \CMDpreview{sign} ($\sign$), \CMDpreview{diag} ($\diag$), \CMDpreview{LK} ($\LK$), \CMDpreview{rg} ($\rg$), \CMDpreview{KER} ($\KER$), \CMDpreview{Eig} ($\Eig$), \CMDpreview{cd} {$\cd$}.
Auch wurde das Aussehen von \verb|\mod|, \verb|\Im| und \verb|\Re| modifiziert:
\[\begin{tabular}{!{\VRule[1pt]}@{\hspace{1em}}c@{\hspace{1em}}|@{\hspace{1em}}c@{\hspace{1em}}|@{\hspace{1em}}c@{\hspace{1em}}!{\VRule[1pt]}}
    \specialrule{1pt}{0pt}{0pt}
    \(\mod\) & \(\Im\) & \(\Re\)\\\hline
    \CMDshow{mod} & \CMDshow{Im} & \CMDshow{Re}\\
    \specialrule{1pt}{0pt}{0pt}
\end{tabular}\]
Des Weiteren wurde noch die Matrixumgebung (\verb|\env@matrix|) so erweitert, dass sie als optionales Argument eine gültige Array-Spaltendefinition entgegennimmt:
\[\begin{tabular}{!{\VRule[1pt]}@{\hspace{1em}}c@{\hspace{1em}}|@{\hspace{1em}}c@{\hspace{1em}}!{\VRule[1pt]}}
\specialrule{1pt}{0pt}{0pt}
{\begin{lstlisting}[style=latex,frame=none]
$\begin{pmatrix}[cc|c]
    1 & 2 & 3 \\
    4 & 5 & 6
\end{pmatrix}$
\end{lstlisting} }&  {$\begin{pmatrix}[cc|c]
    1 & 2 & 3 \\
    4 & 5 & 6
\end{pmatrix}$}\\
\specialrule{1pt}{0pt}{0pt}
\end{tabular}\]

\subsection[Symbole \LILLYxBOXxVersion{\small 1.0.3}]{Symbole}
{\centering \framebox{Diese Definitionen befinden sich in der Datei: \T{Maths/\_LILLY\_MATHS\_SYMBOLS}}\vspace*{0.5\baselineskip}\par}
%Weitere (nicht nennenswert und vermutlich \LILLYxNOTExWarning{Veraltet}) sind die Befehle: \CMDpreview{x} (\(\x\)), \CMDpreview{y} (\(\y\)) und \CMDpreview{z} (\(\z\))
%\normalmarginpar
Für die einzelnen Zahlenräume werden einige Befehle zur Verfügung gestellt, die alle über \verb|\ensuremath| abgesichert sind: 
\CMDpreview{N} ($\N$), \CMDpreview{Z} ($\Z$), \CMDpreview{Q} ($\Q$), \CMDpreview{R} ($\R$), \CMDpreview{C} ($\C$). Sie werden mithilfe von \verb|\mathbb| generiert. Auch die komplexe Einheit \i{} wird mit \CMDpreview{i} zur Verfügung gestellt. \medskip\newline
Weiter wurden die griechischen Buchstaben Epsilon und Phi modifiziert: 
\[\begin{tabular}{!{\VRule[1pt]}@{\hspace{1em}}l@{\hspace{1em}}|@{\hspace{1em}}c@{\hspace{1em}}!{\VRule[1pt]}l@{\hspace{1em}}|@{\hspace{1em}}c@{\hspace{1em}}!{\VRule[1pt]}}
    \specialrule{1pt}{0pt}{0pt}
    \T{\CMDshow{oldepsilon}} & \(\oldepsilon\) & \T{\CMDshow{epsilon}} & \(\epsilon\) \\\hline
    \T{\CMDshow{oldphi}} & \(\oldphi\)& \T{\CMDshow{phi}} & \(\phi\) \\
    \specialrule{1pt}{0pt}{0pt}
\end{tabular}\]
Zudem wird zum Beispiel die Menge der Binärzahlen über \CMDpreview{B} ($\B$), die Chromatische Zahl über \CMDpreview{X} ($\X$) und der generelle Körper mit \CMDpreview{K} ($\K$) zur Verfügung gestellt. Für die Potenzmenge liefert LILLY \CMDpreview{P} ($\P$), für die Menge der Funktionen \CMDpreview{F} ($\F$) und für die Groß-O-Notation \CMDpreview{O} ($\O$).\medskip\newline
Weiter bindet LILLY das \T{\href{http://ctan.math.utah.edu/ctan/tex-archive/macros/latex/required/psnfss/psnfss2e.pdf}{pifont}} Paket ein und liefert so zum Beispiel \verb|\ding{51}| (\ding{51}) und \verb|\ding{55}| (\ding{55}).

\subsection[Kompatibilität \LILLYxBOXxVersion{\small 1.0.3}]{Kompatibilität}
{\centering \framebox{Diese Definitionen befinden sich in der Datei: \T{Maths/\_LILLY\_MATHS\_COMPATIBILITIES}}\vspace*{0.5\baselineskip}\par}
Hier werden einige Befehle eingerichtet, die entweder noch nicht zugeordnet wurden\({}^{\tiny \LILLYxBOXxVersion{\tiny 1.0.3}}\) oder während der Vorlesung (im Überlebenskampf :P) ins \T{eagleStudiPackage} eingebaut worden sind. 
Darunter vor allem für die \la kreierten: \CMDpreview[(1)]{enum} (\verb|enumerate| mit \CMDshow{narrowitems} (TODO: LINK)) und \CMDpreview[(1)]{liste} (\verb|enumerate| mit römischen Zahlen und \CMDshow{narrowitems}). \medskip\newline
Weiter existieren die Befehle \CMDpreview{xa} ($\xa$), \CMDpreview{xb} ($\xb$), \CMDpreview{xc} ($\xc$). \medskip\newline
Für Ti\textit{k}Z gibt es noch die Befehle \CMDpreview[(1)]{crossAT} (\raisebox{-0.35\baselineskip}{\tikz{\crossAT{(0,0)};}}\footnote{\T{\textbackslash tikz\{\CMDshow{crossAT}\{(0,0)\};\}} -- Zum Erhalt der Textzeile vertikal um \T{-0.35\textbackslash baselineskip} verschoben.}) und analog \CMDpreview[(1)]{circAT} (\tikz{\circAT{(0,0)};} \footnote{\T{\textbackslash tikz\{\CMDshow{circAT}\{(0,0)\};\}}}), sowie \CMDpreview[(2)]{bblock} (\raisebox{-0.2\baselineskip}{\tikz{\bblock{(0,0)}{42};}} \footnote{\T{\textbackslash tikz\{\CMDshow{bblock}\{(0,0)\}\{42\};\}} -- Wieder vertikal um \T{-0.2\textbackslash baselineskip} verschoben.}). Diese sollen auf jedenfall noch in ein geeignetes Ti\textit{k}Z-Dokument übertragen werden (TODO:)!\newline\marginpar{}\ENVpar{nstabbing}\ENVpar{centered}\ENVpar{sqcases}%%bufferpar
Weiter werden drei (mittlerweile obsolete) Umgebungen definiert: \begin{itemize}[label=$\diamond$]\narrowitems
    \item \T{nstabbing}:  \verb|tabbing|-Umgebung, ohne Abstände
    \item \T{centered}: \verb|center|-Umgebung, ohne Abstände
    \item \T{sqcases}: Ähnelt \verb|cases| - nur mit '\T{]}'.
\end{itemize}
Zudem definiert sich noch für Tabellen der Befehl \CMDpreview[{[1]}]{VRule}, welcher eine Spalte variabler Größe für Tabellen zur Verfügung stellt. Eine exemplarische Einbindung findet sich hier:
\[\begin{tabular}{!{\VRule[1pt]}@{\hspace{1em}}c@{\hspace{1em}}|@{\hspace{1em}}c@{\hspace{1em}}!{\VRule[1pt]}}
\specialrule{1pt}{0pt}{0pt}
{\begin{lstlisting}[style=latex,frame=none]
\begin{tabular}{c!{\VRule[6pt]}c}
    \specialrule{2pt}{0pt}{0pt}
    You!*'*!re my & Wonder Wall\\
    \specialrule{2pt}{0pt}{0pt}
\end{tabular}
\end{lstlisting} }&  {\begin{tabular}{c!{\VRule[6pt]}c}
    \specialrule{2pt}{0pt}{0pt}
    You're my & Wonder Wall\\
    \specialrule{2pt}{0pt}{0pt}\end{tabular}}\\
\specialrule{1pt}{0pt}{0pt}
\end{tabular}\]
Weiter gibt es noch einige verschiedene Tabellen-Spalten, deren Kurzbezeichner den Anschein erwecken wild zusammengewürfelt zu sein: \begin{multicols}{3}
    \begin{itemize}[label=$\diamond$]\narrowitems
        \item \verb|b|: Fettgedruckt zentriert
        \item \verb|u|: Mathematisch zentriert
        \item \verb|g|: Fußnotengröße linksbündig
        \item \verb|w|: Fußnotengröße linksbündig(X)
        \item \verb|L|(1): Forciert links mit Breite \#1
        \item \verb|C|(1): Forciert zentriert mit Breite \#1
        \item \verb|R|(1): Forciert rechts mit Breite \#1 
    \end{itemize}
\end{multicols}
\normalmarginpar


\section{Plots \tiny\LILLYxBOXxVersion{1.0.8}}
\begin{center}
    \framebox{Für die Spezifikationen siehe hier: \jmark[klick mich]{jmp:PLOTSSPEC}!}
\end{center}

\subsection{graph-Environment}
Es existiert die folgende Implementation der Graph-Umgebung:
\[\begin{tabular}{!{\VRule[1pt]}@{\hspace{0.5em}}C{0.45\textwidth}@{\hspace{0.5em}}|@{\hspace{0.5em}}C{0.45\textwidth}@{\hspace{0.5em}}!{\VRule[1pt]}}
    \specialrule{1pt}{0pt}{0pt}
    {\footnotesize\begin{lstlisting}[style=latex,frame=none,breaklines=true]
\begin{graph}[scale=1.25,maxY=3]
    \plotline[purple]{sqrt(\x+2.5)};
    \plotline[Ao][\y]{(\y*\y)/1.5};
\end{graph}
    \end{lstlisting}} &  \begin{graph}[scale=1.25,maxY=3]
        \plotline[purple]{sqrt(\x+2.5)};
        \plotline[Ao][\y]{(\y*\y)/1.5};
    \end{graph}\\
    \specialrule{1pt}{0pt}{0pt}
    \end{tabular}\]
Die Signatur des Environments \ENVpar{graph} lautet: \T{[Argumente][tikz-Argumente]}.
Letzere sind einfach zusätzliche Argumente die der unterliegenden \T{tikzpicture}
beigefügt werden. Für die Argumente gibt es folgende Möglichkeiten:\newline
\begin{center}
    \begin{tabular}{>{\LILLYxlstTypeWriter}l>{\em}lcl}
        \toprule
        Bezeichner & \normalfont Typ & Standard & Beschreibung\\\midrule
        scale & Zahl & 1 & Skalierungsfaktor \\
        minX & Zahl & -2 & X-Achse Start \\
        maxX & Zahl & 2 & X-Achse Ende \\
        minY & Zahl & 0 & Y-Achse Start \\
        maxY & Zahl & 4 & Y-Achse Ende \\
        offset & Zahl & 0.4 & Zusatzlänge Achsen \\
        loffset & Zahl & 0.1 & Unbeachteter Zusatz Achsen\\
        labelX & String & \$x\$ & Bezeichner X-Achse \\
        labelY & String & \$y\$ & Bezeichner Y-Achse \\
        samples & Zahl & 250 & Anzahl an Kalkulationen \\
        \bottomrule
    \end{tabular}    
\end{center}

\subsection{wgraph-Environment}
Um die Graph-Umgebung noch vielfälter zu Gestalten wurde \ENVpar{wgraph} geschaffen.
Nach reichlicher Überlegung wurde ein neuer Befehl etabliert anstelle es in das
normale \T{graph}-Environment einzubetten. Er funktioniert mit der Syntax:
\begin{lstlisting}[style=latex,frame=none,breaklines=true]
    \begin{wgraph}{l}[][][0pt][\caption{Wichtiger Graph}]
        \plotline{\x*\x}
    \end{wgraph}
\end{lstlisting}
Die Signatur hierbei lautet: \T{\{ORIENTATION\}[Argumente][tikz-Argumente][width][wrapfig-zusatz]}.

\section{3D-Plots \LILLYxNOTExWarning{Ausstehend}}
Ich bin freier Platz :D
\renewcommand{\arraystretch}{1}
