\documentclass[Zusammenfassung, Vorlesung=ANA1]{Lilly}

\begin{document}

\TOP{Ich bin ein Titel}{Und das ist Lustig}
Wenn hier direkt Text kommt, sowas wie zum Beispiel ein wichtiger \kw{Begriff}, muss nnichts weiter gemacht werden!
Auch Definitionen sind kein Problem:
\begin{definition}[Wichtig]
    Dies ist die wichtigste Definition die du je gelesen haben wirst \aLink{mrk:Wichtig}.
\end{definition}
Eine \T{description}:
\begin{smalldesc}
    \item[Alphabet] Malphaset
    \item[Betabet] Wetaled
\end{smalldesc}
Oder auch eine \T{ditemize}-Umgebung:
\begin{smalldite}
    \item Punkt 1
    \item Punkt 2
    \item Punkt 7, tihihi
\end{smalldite}
\TOP{Ich bin ein besserer Titel}{Und das ist wirklich Lustig}\TOPskip
\begin{definition}[Heyho]
    Kommt eine Definition direkt nach einem neuen Thema, so bevorzuge ich das verringern des Abstandes!
\end{definition}
\customex{Dies war aber nun auch wirklch eine tolle Erklärung, finden sie nicht auch Mister Mister?}

\startAppendix

\TOP{Anhang, jippieh ajeah}{Wuuup wuup}
\elable{mrk:Wichtig}Im Anhang jetzt noch ein tolles \emph{showcase}:
\begin{center}
\showcase{Tolle Information}{%
    Ist das nicht wundervoll? Ich finde es eine tolle Box!
}
\end{center}
Ich bin auch ein wichtiger \sw[Begriff]{Super Begriff}, und der hier ist ein \sr[Begriff]{Super Begriff}{Noch besserer Begriff}, ist das nicht \kw{wundervoll}?

\begin{definition}[Anhangsdefinition]
Hallo Günthäääär.
\end{definition}
\end{document}
