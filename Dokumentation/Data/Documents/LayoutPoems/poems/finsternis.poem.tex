\begin{poem}{Finsternis}{26.07.2014} % Nachbearbeitung 20.12.2014
Die Finsternis erdrückend schwer,
Des alten Frühlings Sommerpracht.
Schwärzt die Farben immer mehr,
Erstrebt im Schatten große Macht.

Entwandelt still der Zeiten Glück,
Wo das Elend steigt, zieht die Freude sich zurück.
Und wo das Dunkel bleibt,
Wird das Licht gar schnell entzweit.

So verfällt des Landes Sommertraum,
Dem leeren Firmament.
Wo einst ward des Mondes silbrig Saum,
Mann nunmehr schwach das Licht erkennt.

Das Licht? Welch güldener Gedank,
Und da! Am Horizont,
Dem einzeln Blick fast unerkannt,
Ist ein kleines Licht entflammt.

Und so färbt sich feurig Rot,
des Nachtes äußrer Ring.
Und so erlebt sich Nachtes Tod,
In ewigem Freiheitssinn.

Und obwohl das Dunkel nun besiegt,
bleibt das ewge' Rad nicht stehn.
Beide Mächte Einklang zwei'n
Wie wirs jeden Tage sehn'
\end{poem}