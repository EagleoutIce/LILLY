\documentclass[Paper]{Lilly}

\usepackage{blindtext} % Naja... warum wohl?

\begin{filecontents}{books.bib}
@book{Buch,
    author      = { Sihler, F },
    title       = { Toiletten und ihre tiefgründige Bedeutung für die Gesellschaft },
    year        = { 2019 },
    address     = { Ulm, Deutschland }
}
\end{filecontents}

\def\TITLE{Klopapier, (nur) ein Paper?}
\def\BRIEF{Informatik Student,\\Universit\"{a}t Ulm -- \heute}
\def\ABSTRACT{In dieser Ausarbeitung werden verschiedene Toilettenpapiere auf- und gegeneinander abgewogen. Es wird sich modernster Verfahren der Klopapieranalysie bedient, was sich im Kontext der unabstreibtaren Relevanz dieses Themas auch nicht vermeiden lässt.}

\begin{document}

\printHeader


\section{Motivation}
\intro{Dieser erste Abschnitt zeigt nicht nur den Entstehungsgrund auf, sondern geht auch auf das Ziel der Ausarbeitung ein.}
\subsection{Das Warum}
Diese Ausarbeitung wurde im Anschluss des Buches \cite{Buch} getätigt und versucht die hier fundierten Thesen weiter auszubauen.

\subsection{Das Ziel}
Am Ende soll dem Leser ein tiefgreifender Ein-, Aus- und Durchblick vermittelt worden sein, was es mit dem Klopapier denn so alles auf sich hat und ob es wirklich nur ein Paper ist, wie jedes andere auch\ldots

\section{Grundlagen}
\intro{Dieser Abschnitt bereitet die mathematischen Grundlagen auf und erklärt einige Bezeichner, die in den folgenden Abschnitten von Bedeutung sind.}
\subsection{Bezeichner}
Auch wenn einige Bezeichner redundant erklärt werden, hier eine Auflistung der wichtigsten Bezeichner:
\begin{table}[H]
  \centering\begin{tabular}{^c^l+}
    \toprule
        \headerrow Bezeichner & Bedeutung\\
    \midrule
      \(T\) & Menge von Toiletten\\
      \(D\) & Menge aller Datenpunkte \(d\)\\
      \(G\) & Menge aller Toilettengänger\\
      \(dist(a,b)\) & Abstand zweier Toiletten\\
    \bottomrule
  \end{tabular}\par
  \caption{Verwendete Bezeichner}
\end{table}
\blindmathtrue
\blindtext

\subsection{Wichtig!}
\blindtext

\section{Kloanalyse}
\intro{Dieser Abschnitt behandelt einfache Algorithmen zur Kloanalyse und vergleicht ihre Vor- und Nachteile sowie eine exemplarische Präsentation in Python (2.7.16) mithilfe von \T{Scikit-Learn} (0.20.2) (\url{https://scikit-learn.org/stable/index.html}).}

\subsection{Der Klospüler}
Erst ein Algorithmus der relativ einfach zu verstehen ist:
\begin{python}
def spuelen():
    print "swuuuush"

spuelen()
\end{python}
\blindtext

\subsection{Die Klotarantel}
\blindtext

\subsection{Fazit}
Hier noch ein tolles Fazit:

\getGraphics[7cm]{Automat/AutomatDFA}

\Blindtext


\medskip\startAppendix
% \subsection{Abbildungsverzeichnis}
% \listoffigures
\subsection{Tabellenverzeichnis}
\listoftables
\subsection{Sonstiges}
\Blindtext

\end{document}
